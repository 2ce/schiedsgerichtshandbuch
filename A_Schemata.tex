\chapterpreamble{\enquote{Die Definition des Wahnsinns ist, immer das selbe zu tun, und ein anderes Ergebnis zu erwarten.} -- u.a. Albert Einstein, Benjamin Franklin oder Mark Twain zugeschrieben.}

\chapter{Prüfungsschemata}
\label{Schemata}

Die folgenden Schemata stellen zum einen eine Liste der Prüfungspunkte für den jeweiligen Fall dar, die zu einer vollumfassenden Prüfung durchdacht werden müssen.
Zum Anderen geben sie auch einen Anhalt für eine sinnvolle Prüfreihenfolge:
Ebenso, wie die Zulässigkeit sinnvollerweise vor der Begründetheit geprüft werden sollte, sind einige Prüfungspunkte ihrerseits Voraussetzungen für andere Fragen.
Nicht immer allerdings sind die Schemata, wie sie hier präsentiert werden, auf den konkreten Fall vollständig abbildbar.
Selbst bei ihrer Verwendung sollte daher stets überprüft werden, ob nicht im konkreten Fall eine Abweichung sinnvoll sein könnte.

\section{Anrufung}
\label{Schemata:Anrufung}
Hierzu ausführlich: Kapitel~\ref{Anrufung}.

\begin{enumerate}[label=\Roman*.]
\item Zuständigkeit des Schiedsgerichts
\item Prüfung der durch Satzung vermuteten Befangenheit (\S~5 Abs.~1~SGO)\indexPar{SGO!5@\S~5!1@Abs.~1}
\item Antragsteller
	\begin{enumerate}
	\item Name
	\item Anschrift
	\item Kontaktmöglichkeiten
	\item Parteifähigkeit
	\end{enumerate}
\item Antragsgegner
	\begin{enumerate}
	\item Name
	\item Anschrit
	\item Kontaktmöglichkeiten
	\item Parteifähigkeit
	\end{enumerate}
\item Antragschrift
	\begin{enumerate}
	\item Klageanträge
	\item Klagebefugnis
	\end{enumerate}
\item Form
\item Frist
\item Schlichtung
\end{enumerate}

\section{Zulässigkeit}
\begin{enumerate}
\item Erfolgreiche Anrufung (vgl. Schema~\ref{Schemata:Anrufung})
\item Weiterbestehen der Anrufungskriterien
\end{enumerate}

\section{Mündliche Verhandlung}
\begin{enumerate}
\item Feststellung der Beschlussfähigkeit des Gerichts
\item Hinweis zur Ordnung der Verhandlung
\item Hinweise zur Protokollierung, ggf. zu Aufzeichnungen
\item Feststellung des Erscheinens oder Nichterscheinens der Parteien und ihrer Vertretungen
\item Einführung in den Verfahrensgegenstand (Sachverhalt, ggf. vorläufige Rechtsauffassung)
\item Anhörung der Parteien
\item Anhörung und ggf. Befragung von Zeugen
\item (im Falle von Ordnungsmaßnahmen) Letztes Wort des Betroffenen
\item Urteilsverkündung bzw. Ankündigung des weiteren Verfahrens
\end{enumerate}

\section{Begründetheit}
Die Prüfung der Begründetheit einer Klage richtet sich vor allem nach der Klageart.
