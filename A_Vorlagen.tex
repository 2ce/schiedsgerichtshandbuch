\chapterpreamble{\enquote{Das haben wir noch nie so gemacht. -- Wo kommen wir denn da hin? -- Da könnt ja jede*r kommen!} -- Der Dreisatz der Verwaltung.}

\chapter{Vorlagen}
\label{Vorlage}
Dieser Anhang enthält Formulierungsvorschläge für übliche Vorgänge.
Der Übersichtlichkeit halber sind die Rechtsbehelfsbelehrungen separat gelistet.

\section{Nachbesserung einer Anrufung}
\label{Vorlage:Anrufungsnachbesserung}
wir haben deine Anrufung erhalten und bestätigen hiermit den Eingang.
Allerdings können wir uns noch nicht mit deinem Antrag befassen.
Wir müssen dich zuerst um Nachbesserung bitten.

(Modul, streichen, wenn unzutreffend)\\
Die angehängten Dateien sind für uns nicht lesbar.
Bitte sende uns diese erneut mit klarer, den Inhalt kenntlich machenden Bennenung in einem allgemein üblichen Format zu.
Wir bevorzugen aus Archivierungsgründen PDF-Dateien, aber auch andere übliche Formate werden akzeptiert.

(Modul, Streichen, wenn unzutreffend)\\
Leider fehlt das Urteil, gegen das du Berufung eingelegt hast.
Bitte sende uns das Urteil, gegen welches du Berufung einlegst in der dir zugestellten Fassung zu.

(Modul,  Streichen, wenn unzutreffend)\\
Du hast im Auftrag von (eigentlicher Kläger) geklagt.
Leider hast du uns keine Vollmacht vorgelegt, die belegt, dass du (eigentlicher Kläger) in diesem Verfahren gemäß § 9 Abs. 2 SGO vertrittst.
Bitte sende uns eine solche Vollmacht zu.

(Modul,  Streichen, wenn unzutreffend)\\
Du hast im Auftrag von (klagende Gliederung) geklagt.
Leider hast du uns keine Vollmacht vorgelegt, die belegt, dass du (klagende Gliederung) in diesem Verfahren gemäß § 9 Abs. 3 SGO vertrittst.
Bitte sende uns eine solche Vollmacht bzw. die Dokumentation des entsprechenden Vorstandsbeschlusses zu.

Wir bitten dich zur Nachbesserung mit Frist zum (Datum).

\section{Rubrum}
\label{Vorlage:Rubrum}
Aktenzeichen\\
Datum\\
Beschluss zu (Aktenzeichen)\\

In Sachen der Anrufung vom (Datum), (Aktenzeichen)

(Aufzählung der Antragsteller*innen)\\
vertreten durch (Vertretung)

gegen

(Aufzählung der Antragsgegner*innen)\\
vertreten durch (Vertretung)

wegen (Prozessgrund)

\section{Eröffnungsablehnung}
\label{Vorlage:Eröffnungsablehnung}
(Rubrum)

wird das Verfahren gemäß (Umlauf-)beschluss vom (Datum) nicht eröffnet.

I.\\
(Sachverhalt)

II.\\
(Entscheidungsgründe)

(Rechtsbehelfsbelehrung: vgl.~Schema~\ref{Vorlage:RMB_Eröffnungsbeschwerde})

\section{Eröffnungsbeschluss}
\label{Vorlage:Eröffnungsbeschluss}
(Rubrum)

wird das Verfahren gemäß (Umlauf-)beschlusses vom (Datum) mit dem Aktenzeichen (Aktenzeichen) eröffnet.
Die Antragsschrift ist angehängt.

Folgende Richter*innen sind dem Verfahren zugeordnet:\\
(Aufzählung der zugeordneten Richter*innen)

Das Landesschiedsgericht hat weiterin beschlossen:
\begin{enumerate}
\item Nach §~11~Abs.~9~S.~3~SGO i.V.m. dem aktuellen Geschäftsverteilungsplan wird (Berichterstattung) zum/zur Berichterstatter*in bestimmt.
\item Nach §~5~Abs.~5~S.~1~SGO steht es den Verfahrensbeteiligten frei, binnen \textbf{2~Wochen} die Ablehnung von Richtern wegen der Besorgnis der Befangenheit zu beantragen. Die Möglichkeit einer Ablehnung aus Gründen, die im Verfahren offenbar werden, bleibt davon unbenommen.
\item Jeder Verfahrensbeteiligte kann eine Vertretung benennen. Der (Gliederungs)vorstand wird aufgefordert nach §~9~Abs.~3~S.~1~SGO einen Vertreter zu benennen.
\item Dem Antragsgegner wird aufgegeben, auf das Klagevorbringen innerhalb von \textbf{2~Wochen} in Textform zu erwidern, wenn er sich gegen die Klage verteidigen will.
\item (Im Fall von OM/PAV) Der (Antragsteller bei OM, Antragsgegner bei PAV) wird gebeten mitzuteilen, ob ein nichtöffentliches Verfahren geführt werden soll, \S~9 Abs.~4~SGO.\indexPar{SGO!9@\S~9!4@Abs.~4}
\item Der Antragsteller erhält nach Eingang der Erwiderung \textbf{2~Wochen} zur Erwiderung.
\item Die Kommunikation mit dem Schiedsgericht erfolgt grundsätzlich per E-Mail. Alle weitere Korrespondenz soll im Betreff das Aktenzeichen (Aktenzeichen) tragen.
\item Das Landesschiedsgericht hat als Termin für die mündliche Verhandlung gemäß \S~10 Abs.~4 ff.~SGO\indexPar{SGO!10@\S~10!4@Abs.~4} den (Datum der geplanten mündlichen Verhandlung) vorgemerkt und bittet bereits jetzt um Stellungnahme zum Termin um baldestmöglich laden zu können.
\end{enumerate}

%\section{Besetzungsänderung} % ???
%\label{Vorlage:Besetzungsänderung}

\section{Ladung zur Verhandlung}
\label{Vorlage:Verhandlungsladung}
(Rubrum)

hat das Schiedsgericht durch die Richter*innen (Aufzählung der an der Beschlussfassung beteiligten Richter*innen) am (Datum) beschlossen:\\
\begin{enumerate}
\item	Die mündliche Verhandlung findet als mündliche Präsenzsitzung am (Verhandlungsdatum) um (Uhrzeit) in (Verhandlungsort mit Postanschrift und Raum) statt.
\item	Die Verhandlung ist öffentlich.
	Die Öffentlichkeit kann nur auf Antrag ausgeschlossen werden, \SSS~9 Abs.~4, 10 Abs.~7~SGO.\indexPar{SGO!9@\S~9!4@Abs.~4}\indexPar{SGO!10@\S~10!4@Abs.~4}
\item	Den Verfahrensparteien wird aufgegeben, folgende Fragen bis zum (Vorbereitungsdatum) zu beantworten:
	\begin{enumerate}
	\item (Fragen)
	\end{enumerate}
\item	Die Beteiligten mögen unverzüglich, spätestens jedoch bis zum (Vorbereitungsdatum) begründen, zu welchen Beweisthemen Zeug*innen benannt werden.
	Sofern kein Beweisthema konkret benannt wird oder dieses mit dem Streitgegenstand nicht in ausreichendem Zusammenhang steht, wird eine Ladung oder Anhörung der Zeug*innen unterbleiben.
\item	Für Anträge, Beweise etc. gilt Frist bis zum (Vorbereitungsdatum).
\end{enumerate}

Vorsorglich weist das (Schiedsgericht) darauf hin, dass ein Antrag auf Terminverlegung wegen Verhinderung begründet werden muss.
Das (Schiedsgericht) kann auch in Abwesenheit verhandeln, \S~10 Abs.~5~SGO.\indexPar{SGO!10@\S~10!5@Abs.~5}

Fristversäumnis, verspäteter Vortrag und verspätete Benennung von Zeug*innen kann dazu führen, dass der Vortrag nicht berücksichtigt und/oder die Zeug*innen nicht gehört werden.

\section{Einstweilige Anordnung}
\label{Vorlage:Einstweilige}
(Rubrum)

hat das Schiedsgericht durch die Richter*innen (Aufzählung der an der Beschlussfassung beteiligten Richter*innen) aufgrund ([fern-]mündlicher Verhandlung vom [Verhandlungsdatum] / schriftlichen Verfahrens) am (Datum) beschlossen:\\
(Tenor der einstweiligen Anordnung)

I.\\
(Sachverhalt)

II.\\
(Entscheidungsgründe)

(Rechtsbehelfsbelehrung: vgl.~Schema~\ref{Vorlage:RMB_einstweilige})

\section{Urteil}
\label{Vorlage:Urteil}
(Rubrum)

hat das Schiedsgericht durch die Richter*innen (Aufzählung der an der Beschlussfassung beteiligten Richter*innen) aufgrund ([fern-]mündlicher Verhandlung vom [Verhandlungsdatum] / schriftlichen Verfahrens) am (Datum) beschlossen:\\
(Tenor)

I.\\
(Sachverhalt)

II.\\
Entscheidungsgründe

1.\\
(Zulässigkeit)

2.\\
(Begründetheit)

(Rechtsbehelfsbelehrung: vgl.~Schema~\ref{Vorlage:RMB_Berufung})

\section{Ordnungsmaßnahme}
\label{Vorlage:OM}
Der Vorstand (der Piratenpartei [Gliederung]) verhängt gegen (Name), Mitgliedsnummer (Mitgliedsnummer) eine Verwarnung.
Er/Sie hat am 01.01.1970 seine Uhr nicht auf UNIX-Zeit umgestellt und damit gegen die Ordnung der Partei verstoßen.
Dadurch wurden von seinem Computer falsche Daten zur Einberufung des Parteitages der Auslandsgruppe Salzburg verschickt.

\section{Rechtsbehelfsbelehrungen}
\subsection{Sofortige Beschwerde: Nichteröffnung}
\label{Vorlage:RMB_Eröffnungsbeschwerde}
Gegen die Ablehnung der Verfahrenseröffnung ist die sofortige Beschwerde mit einer Frist von 14~Tagen beim (Berufungsgericht mit Post- und E-Mail-Anschrift) zulässig, \S~8 Abs.~6 S.~3~SGO.\indexPar{SGO!8@\S~8!6@Abs.~6}

\subsection{Berufung}
\label{Vorlage:RMB_Berufung}
Gegen dieses Urteil kann binnen (Berufungsfrist) beim (Berufungsgericht mit Post- und E-Mail-Anschrift) Berufung eingelegt werden.
Die Berufung ist zu begründen.
Der Berufungsschrit ist die angefochtene Entscheidung samt erstinstanzlichem Aktenzeichen (Aktenzeichen) beizufügen.

\subsection{Widerspruch gegen einstweilige Anordnung)}
\label{Vorlage:RMB_einstweilige}
Gegen die einstweilige Anordnung kann innerhalb von 14~Tagen nach Bekanntgabe und Erhalt der Begründung beim erlassenden Schiedsgericht ([eigenes Schiedsgericht mit Post- und E-Mail-Anschrift]) Widerspruch erhoben werden, \S~11 Abs.~4 S.~1~SGO.\indexPar{SGO!11@\S~11!4@Abs.~4}
Ein Widerspruch hat keine aufschiebende Wirkung, \S~11 Abs.~4 S.~2~SGO.\indexPar{SGO!11@\S~11!4@Abs.~4}

\subsection{Befangenheitsentscheidung}
\label{Vorlage:RMB_na}
Vorsorglich weist das Gericht darauf hin, dass der Beschluss über den Ausschluss von Richter*innen wegen Befangenheit unanfechtbar ist, \S~5 Abs.~6 S.~1~SGO.\indexPar{SGO!5@\S~5!6@Abs.~6}

\subsection{Ordnungsmaßnahme}
Gegen diese Ordnungsmaßnahme ist gemäß \S~8 Abs.~1 S.~2~SGO\indexPar{SGO!8@\S~8!1@Abs.~1} der Einspruch vor dem (Landesschiedsgericht des Landesverbands des Mitglieds) zulässig.
Gegen die Ablehnung ist die sofortige Beschwerde mit einer Frist von 14~Tagen zum nächsthöheren Schiedsgericht möglich.
