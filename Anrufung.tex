% \begin{refsection}
\chapterpreamble{\enquote{\enquote{Gestatten Sie das?} -- \textit{(nickt)} -- \enquote{Ohne Papiere?} -- \enquote{Ohne Papiere!}} --- Jagd auf Roter Oktober (US, 1990).}

\chapter{Die Anrufung}
\label{Anrufung}
Ein Schiedsgericht wird nur auf Anrufung tätig schreibt die SGO in \S~8 Abs.~1 S.~1 vor.
Damit ist die Anrufung der zentrale Punkt für ein Schiedsgericht, an dem jede Tätigkeitkeit anknüpft.

Aber auch der Umgang mit einer Anrufung ist streng formal geregelt.
Das soll es den Gerichten erleichtern, gewissen Standardfrage nicht immer noch im Nachhinein stellen zu müssen, weil sie immer auftauchen.

Grundlegend sieht der Ablauf der zu prüfenden Punkte dabei wie folgt aus:
\begin{enumerate}[label=\Roman*.]
% \itemsep0em 
\item \textbf{Antragsteller}: Ist klar genannt, \emph{wer} die Klage erheben will? (\S~8 Abs.~3 Nr. 1 SGO\index[paridx]{SGO!8@\S~8!3@Abs.~3})
\item \textbf{Antragsgegner}: Ist klar genannt, gegen \emph{wen} sich die Klage richten soll? (\S~8 Abs.~3 Nr.~2 SGO\index[paridx]{SGO!8@\S~8!3@Abs.~3})
\item Sind Antragsteller und Antragsgegner \textbf{parteifähig}? (\S~8 Abs.~1 S.~1 SGO\index[paridx]{SGO!8@\S~8!1@Abs.~1})
\item Ist jeweils eine zulässige \emph{Anschrift} angegeben? (\S~8 Abs.~3 Nr.~1, 2 SGO\index[paridx]{SGO!8@\S~8!3@Abs.~3})
\item Sagt der Antragsteller klar, \emph{was} genau er erreichen will? (\S~8 Abs.~3 Nr.~3 SGO\index[paridx]{SGO!8@\S~8!3@Abs.~3})
\item Ist das, was der Antragsteller will, etwas, das ihm nach Satzung, Parteiengesetz oder sonstigen mitgliedschaftlichen oder organschaftlichen Rechten \emph{möglicherweise zustehen könnte}? (\S~8 Abs.~1 S.~1 SGO\index[paridx]{SGO!8@\S~8!1@Abs.~1})
\item Nennt der Antragsteller \textbf{Gründe}, warum das gewünschte ihm zustehen sollte? (\S~8 Abs.~3 Nr.~4 SGO\index[paridx]{SGO!8@\S~8!3@Abs.~3})
\item Ist die \textbf{Form} eingehalten? (\S~8 Abs.~3 SGO\index[paridx]{SGO!8@\S~8!3@Abs.~3})
\item Ist die \textbf{Frist} eingehalten worden? (\S~8 Abs.~4 SGO\index[paridx]{SGO!8@\S~8!4@Abs.~4})
\item Ist das angerufene Schiedsgericht \textbf{zuständig}? (\S~8 Abs.~5 SGO\index[paridx]{SGO!8@\S~8!5@Abs.~5})
\item Hat eine \textbf{erfolglose Schlichtung} stattgefunden oder ist diese \textbf{entbehrlich}? (\S~7 SGO\index[paridx]{SGO!7@\S~7})
\end{enumerate}

\section{Statthaftigkeit}
\label{Anrufung:Statthaftigkeit}
Die grundlegendsten Anforderungen an eine Anrufung werden unter dem Begriff Statthaftigkeit zusammengefasst.
Eine Klage ist grundsätzlich nur statthaft, wenn diese Anforderungen erfüllt sind.

\subsection{Antragsteller}
\label{Anrufung:Statthaftigkeit:Antragsteller}
Das allererste dieser Kriterien ist die Benennung eines \index[idx]{Antragsteller|textbf} Antragstelllers bzw. einer Antragstellerin, andernorts auch Kläger bzw. Klägerin genannt.
Die SGO kennt allerdings den Begriff Kläger bzw. Klägerin nicht, sondern nur den Begriff Antragsteller bzw. Antragstellerin, entsprechend wird auch dieser Begriff hier verwendet.
Antragsteller bzw. Antragstellerin ist typischerweise der oder die Anrufende selbst.
In einigen Fällen, typischerweise im Falle einer Anrufung durch einen Vorstand oder ein anderes Organen, kann der oder die Anrufende auch Vertreter bzw. Vertreterin des Antragstellers bzw. der Antragstellerin sein.
In jedem Fall muss aus der Anrufung klar identifizierbar hervorgehen, wer nun mit der Anrufung Rechte bzw. Ansprüche vor dem Schiedsgericht gegen eine andere Person gelten machen will.
Dieser jemand ist der Antragsteller bzw. die Antragstellerin.

In den meisten Fällen dürfte dies ein Mitglied sein, in seltenen Fällen kann es auch ein einzelner Amtsträger oder eine einzelne Amtsträgerin sein, noch häufiger dürfte es ein Organ als Gesamtes sein.
Typischerweise gibt es lediglich vier Fälle, in denen die Anrufung durch ein Organ erfolgt:
\begin{enumerate}[label=\arabic*.)]
% \itemsep0em 
\item Parteiausschlussverfahren\index[idx]{Parteiausschluss}
\item Gliederungsordnungsmaßnahmeneinsprüche\index[idx]{Ordnungsmassnahme@Ordnungsma""snahme!Gliederungs-}
\item Gliederungskompetenzstreitigkeiten
\item Organhandlungsfähigkeitsstreitigkeiten
\end{enumerate}
Der letzte Fall, die Organhandlungsfähigkeitsstreitigkeiten sind strenggenommen sogar nur ein Unterfall der Gliederungskompetenzstreitigkeiten, aber dazu an entsprechender Stelle mehr.

Aus dem Rahmen fallen natürlich auch immer die Berufungs- und Beschwerdeanrufungen im Instanzenzug und die Widerspruchsanrufung im einstweiligen Rechtsschutz, weil hier die Anrufung typischerweise von der in der ursächlichen Schiedsgerichtsentscheidung unterlegenen Partei ausgeht.

\subsection{Antragsgegner}
\label{Anrufung:Statthaftigkeit:Antragsgegner}
Der Antragsgegner\index[idx]{Antragsgegner|textbf} oder die Antragsgegnerin ist ebenso notwendig und muss ebenso klar und eindeutig aus der Anrufung entnehmbar sein. Bezüglich der Vertretung gilt hier dasselbe wie bei dem \index[idx]{Antragsteller} Antragsteller bzw. der Antragstellerin.

Bei allen Verfahren nach SGO handelt es sicht um sogenannte kontradiktorische Verfahren.
Das bedeutet, dass das Verfahren Streit von zwei Parteien um gegenläufige Interessen ausgestaltet ist.
In den meisten Fällen ist der Antragsgegner bzw. die Antragsgegnerin daher ein Organ, in wenigen Fällen -- primär bei gerichtlich verhängten Ordnungsmaßnahmen wie dem Parteiausschluss -- aber auch ein einzelnes Mitglied.
Wichtig dabei ist aber: Es gibt nach SGO keine Verfahren von Mitgliedern gegen andere Mitglieder.
Schon fraglich ist, ob ein einzelner Amtsträger bzw. eine Amträgerin gegen einen anderen Amträgräger bzw. eine andere
Amtsträgerin einzeln vorgehen kann, dazu gab es bisher noch nie einen Anlass, das zu entscheiden.
Theoretisch ist es aber nicht komplett ausgeschlossen.
Möglich ist in jedemfall das vorgehen einer Amtsträgerin bzw. eines Amtsträgers gegen das restliche Organ, eines Organs  gegen einzelne Amtsträger und Amtsträgerinnen sowie das vorgehen von Mitgliedern gegen Organe und andersherum.
Wichtig ist daher der Grundsatz: Aus dem Antragsteller oder der Antragstellerin ergibt sich zwangsweise ein eingeschränkter Kreis möglicher Antragsgegner und Antragsgegnerinnen.

\subsection{Parteifähigkeit}
\label{Anrufung:Statthaftigkeit:Parteifaehigkeit}
Gerade wude es schon angesprochen: Antragsteller bzw. Antragstellerin und Antragsgegner bzw. Antragsgegnerin sind die \index[idx]{Streitpartei}Parteien des Prozesses.
Nach SGO sind Mitglieder und Organe explizit dazu berechtigt, ein Schiedsgericht anzurufen, wenn sie sich in ihren Rechten verletzt fühlen, \S~8 Abs.~1 S.~2 SGO\index[paridx]{SGO!8@\S~8!1@Abs.~1}.
Das bedeutet, dass diese auch zwingend Partei in einem Schiedsgerichtsverfahren sein können.
Ferner muss die Parteifähigkeit\index[idx]{Parteifähigkeit} auch Amtsträgern als spezieller Art von Mitgliedern zugestanden werden.
Ob man dabei davon ausgeht, dass Amtsträger eine eigene Kategorie von Partei darstellen oder lediglich Mitglider sind, deren Rechte aufgrund einer Wahl zeitlich beschränkt verändert wurden, ist eigentlich eine akademische Debatte.
Angesichts dessen, dass Amtsträger aber in die Rechte anderer Amtsträger eingreifen können und dann im Sinne eines effektiven Rechtsschutzes entweder das Recht des anderen Amtsträgers zu einem Organrecht erklärt werden muss oder aber eine Klage von Mitglied gegen Mitglied zugelassen werden muss, spricht doch einiges dafür, Amtrsträger als eigene, nicht explizit genannte,  aber doch erfasste Kategorie innerhalbt der Parteifähigkeit zu sehen.

Brisanz gewinnt die Einordnung der Amtsträger als eigene parteifähige Kategorie bei der Klage von Mitgliedern gegen Amtsträgern, was dadurch nicht mehr von der nicht erlaubten Klage von Mitglied gegen Mitglied vor dem Schiedsgericht ausgeschlossen ist.
Allerdings müsste dann eine Klagebefugnis vom Mitglied vorliegen, typischerweise ist die Rechtsverletzung aber nicht dem  einzelnen Amtsträger sondern dem ganzen Organ zuzurechnen genauso wie sich Ansprüche aus der Mitgliedschaft typischerweise gegen das zuständige Organ richten. 
Daher kann fast davon ausgegangen werden, dass es nicht zu einer solchen Klagekonstellation kommen kann bzw. falls doch, dass diese dann gerade nötig ist, um einen effektien Rechtsschutz für Mitglieder zu gewähren.

Fraglich ist zudem ob die Partei bzw. die einzelnen Gliederungen Parteifähig im Sinne der SGO sind.
Während dies für Prozesse vor staatlichen Gerichten aufgrund der dortigen Regelungen zur Parteifähigkeit\footnote{Vgl. etwa \S~50 ZPO\nomenclature{ZPO}{Zivilprozessordnung}\index[paridx]{ZPO!50@\S~50}.} grundsätzlich anzunehmen ist, spielt die Partei bzw. Gliederung als solche im internen Rechtsstreit eine untergeordnete Rolle, da die Handlung immer zumindest einem Organ oder zumindest Amtsträger zugeordnet werden kann.
Daher ist auch das Bundesschiedsgericht in seiner Rechtsprechung dazu übergegangen, die Parteifähigkeit der Partei und ihrer Gliederungen nur zu bejahen, wenn eine derartige Zurechnung nicht möglich ist.\footnote{Vgl. \cite[S. 4]{BSG1614HS}; Diese Rechsprechung fortführend \cite[S. 2]{BSG3815HS} m.w.N.}
Die Partei und die Gliederungen kommen also nur als subsidiärer Antragsgegner bzw. Antragsgegnerin in Betracht.

\subsection{Anschrift}
\label{Anrufung:Statthaftigkeit:Anschrift}
Der Antragsteller muss seine eigene Anschrift sowie die des Antragsgegners angeben.
Die Anschrift ist notwendig, damit das Schiedsgericht im Verfahren mit den Parteien in Kontakt treten kann.
Auch wenn die Kommunikation per E-Mail im Verfahren üblich ist, ist es hier notwendig, dass eine postalische Adresse genannt wird.\footnote{Ausführlich dazu \cite[S. 3]{BSG20130715} m.w.N.}
So gibt es zwar keinen Anspruch mehr in der SGO auf eine Urteilszusendung in Schriftform, die die Kenntnis einer postalischen Adresse früher zwingend machte, jedoch ist die Anschrift nach wie vor zur Individualisierung und damit zur Mitgliedsverifikation durch das Schiedsgericht erforderlich. Auch im Fall, dass eine Person mittels einer lediglich vom Verfahrensgegner genannten E-Mailadresse nicht erreichbar ist, ist eine zusätzlicher Kontaktversuch per Brief erforderlich, um die Rechte der nicht erreichbaren Verfahrenspartei zu sichern.

Da es allerdings nicht die Erforderlichkeit, den Ansprüchen einer ladungsfähigen Adresse genügen zu müssen, ist eine Postfachanschrift ausreichend und erfüllt die Anforderungen der Schiedsgerichtsordnung.\footnote{Mit Begründung etwa \cites[3]{BSG20130715}[S.~4 Rn~16]{LSGBB145}{LSGBB134}. Lediglich bestätigend etwa \cites[7]{LSGBB133}[S.~10~f. Rn~48]{LSGBB146}.}
Dies gilt nicht nur für den Fall, dass die Anrufung durch ein Mitglied erfolgt,\footnote{So aber \cites[S.~1]{LSGBB134}.} sondern gerade auch für den Fall, dass die Anrufung durch ein Organ oder eine Gliederung erfolgt.\footnote{\cites[S.~2~f.]{BSG20131230}.}

Gerade aber der Grund der Mitgliedsverifikation könnte auch gegen die Zulässigkeit einer Postfachanschrift zur Erfüllung von \S~8 Abs.~3 Nr.~1, 2 SGO\index[paridx]{SGO!8@\S~8!3@Abs.~3} sprechen.
Wie das Landesschiedsgericht Brandenburg zutreffend ausführt, hängt das davon ab, ob im Mitgliederverzeichnis ebenfalls Postfachanschriften geführt werden können.\footnote{\cites[S.~4 Rn~16]{LSGBB145}.}
Dies sollte aber gerade nicht der Fall sein, da die Anschrift im Mitgliederverzeichnis regelmäßig zur Feststellung der Stimmberechtigung bei Aufstellungsversammlungen nach den Wahlgesetzen benutzt wird und dafür auch notwendig ist.
Ist aber die Angabe eine Postfachanschrift im Mitgliederverzeichnis unzulässig, kann darüber gerade keine Identitätsverifikation erfolgen.
Die Postfachanschrift erfüllt ihren Zweck nach \S~8 Abs.~3 Nr.~1, 2 SGO\index[paridx]{SGO!8@\S~8!3@Abs.~3} dann gerade nicht mehr und erfüllt daher diese Anforderung nicht mehr.

Die weiteren Kontaktdaten gemäß \S~8 Abs.~3 Nr.~1 SGO\index[paridx]{SGO!8@\S~8!3@Abs.~3} umfassen Daten wie eine E-Mailadresse. Diese ist aus prozessökonomischen Gründen sinnvoll, da ein Information per E-Mail die Verfahrensparteien schneller erreicht und diese auch schneller reagieren können. Das ganz das Verfahren insgesamt erheblich beschleunigen und macht die direkte Versendung in Kopie zudem einfacher.

Dass die Anschrift unbekannt ist, kann auch nie ein Problem darstellen. Verfahren von Mitglied gegen Mitglied sind schon gar nicht vorgesehen (siehe \ref{Anrufung:Statthaftigkeit:Antragsgegner}), der Partei ist die Anschrift eines Mitglieds aus der Mitgliederdatenbank bekannt, die Anschrift der Partei und der Gliederungen sind allgemein bekannt, die Organe haben ihre Anschrift für gewöhnlich bei der Gliederung, oder, falls diese abweicht, ist sie vom Organ bekannt gemacht.

\subsection{Anträge}
\label{Anrufung:Statthaftigkeit:Antraege}
Ebenfalls zwingend erforderlich ist es, dass der Antragssteller mit der Anrufung bereits Anträge stellt.
Diese Anträge müssen nach \S~8 Abs.~3 Nr.~3 SGO\index[paridx]{SGO!8@\S~8!3@Abs.~3} klar und eindeutig sein.
Diese innere Dopplung der Formulierung zeigt schon, dass es dem Satzungsgeber darauf ankam, dass der Antragsteller explizit sagen muss, was exakt er mit der Anrufung erreichen will.
Das Schiedgericht ist daher nicht verpflichtet, zu spekulieren und Vermutungen darüber anzustellen, was der Antragsteller gewollt haben könnte. 
Wenn der Antragsteller das nicht in seiner Anrufung klar erkennbar niederschreibt, dann ist die Anrufung unvollständig und daher nicht statthaft.

Die Erfordernis der genauen Anträge dient auch dazu, die genaue Klageart festzustellen, da diese Einfluss auf das weitere Verfahren haben kann.
Zudem dienst diese Anforderung dem Zweck, sicherzustellen, dass das Schiedsgericht weiß, in welche Richtung es seine Amtsermittlungspflicht gemäß \S~10 Abs.~1 S.~1 Hs.~1 SGO\index[paridx]{SGO!10@\S~10!1@Abs.~1} ausüben muss, um alle relevanten Fragen beantworten zu können, die sich für eine Entscheidung in der Sache stellen.

Bei der Prüfung, ob Anträge gestellt wurden, ist also eine sehr strenger Maßstab anzulegen. Wird nicht klar und eindeutig spezifiziert, was der Antragsteller vom Antragsgegner will, liegt schon gar kein Antrag vor.
Ob der Antrag erfüllbar ist und rechtmäßig gestellt wurde, wird allerdings hier noch nicht entschieden.
Insofern ist es auch nur eine sehr formale und eingschränkte Prüfung, ob einer oder mehrere Anträge gestellt wurden.

Und natürlich reicht entgegen der Pluralformulierung von \S~8 Abs.~3 Nr.~3 SGO\index[paridx]{SGO!8@\S~8!3@Abs.~3} ein einzelner Antrag auch aus. Es ist nicht Sinn und Zweck der Regelung, dass ein Antragsteller, der nur eine einzelne Sache erreichen will, einen absolut unnötigen zusätzlichen Antrag stellt und das Schiedsgericht so mit Mehrarbeit belastet oder gar schon überhaupt nicht klagen darf, nur weil er nur eine Sache erreichen will.

\subsection{Begründung und Umstände}
\label{Anrufung:Statthaftigkeit:Gruende}
Die letzte Anforderung des \S~8 Abs.~3 SGO\index[paridx]{SGO!8@\S~8!3@Abs.~3} sind eine Darstellung der Umstände und Schilderung einer Begründung der Anträge.
Damit soll dem Schiedsgericht ein Anhaltspunkt für seine Amtsermittlung des Sachverhaltes nach \S~10 Abs.~1 S.~1 Hs.~1 SGO\index[paridx]{SGO!10@\S~10!1@Abs.~1} gegeben werden und dem Antragsgegner eine Möglichkeit, auf die Klage zu erwiedern.
Der Sinn und Zweck dieser Anforderung ist also, einen Startpunkt für das Verfahren haben, und den Streit der Parteien einsortieren zu können.
Entsprechend sind hieran keine hohen inhaltlichen Anforderungen zu stellen.
Solange ein Geschehen geschildert wird und irgendwelche Argumente angeführt werden, warum diese oder jene rechtliche Bewertung dafür zu gelten habe, ist regelmäßig anzunehmen, dass diese Punkt erfüllt ist.
Andernfalls würde \S~8 Abs.~3 Nr.~4 SGO\index[paridx]{SGO!8@\S~8!3@Abs.~3} eine Vorwegnahme der eigentlichen Beurteilung des Verfahrens darstellen, dass das Gericht erst nach ausführlicher Beschäftigung mit der Sache selbst durch die Durchführung des Verfahrens tätigen soll.

\section{Weitere Kriterien}
\label{Anrufung:Kriterien}
Neben der Statthaftigkeit gibt es noch weitere Kriterien, die für eine erfolgreiche Anrufung erfüllt sein müssen.
Dies sind unechte Zulässigkeitskriterien, da sie die Klage unzulässig machen, aber nicht erst im Verfahren mit dem Urteil beschieden werden, sondern schon mit Eröffnung.
Die Satzung nennt diese Kriterien auch Zulässigkeitskriterien der Eröffnung, vgl. \S~8 Abs.~6 Satz~4 SGO\index[paridx]{SGO!8@\S~8!6@Abs.~6}.

\subsection{Antragsbefugnis}
\label{Anrufung:Kriterien:Antragsbefugnis}
Die Notwendigkeit der Antragsbefugnis geht aus \S~8 Abs.~1 SGO\index[paridx]{SGO!8@\S~8!1@Abs.~1} hervor und soll sogenannte \emph{Popularklagen} verhinden.
Eine Antragsbefugnis liegt dann vor, wenn der Antragsteller selbst in seinen eigenen Rechten aus der Mitgliedschaft oder seinem sonstigen innerparteilichen Status verletzt ist oder ein Anspruch aus dieser Position nicht erfüllt wird.
Die Abgrenzung von eigenen Rechten und eigenen Ansprüchen ist nicht immer einfach, aber auch nicht zwingend nötig, da Alternativ eines von beides ausreicht.
Entsprechend den anderen nicht rein förmlichen Anrufungskriterien ist hier kein hoher inhaltlicher Anspruch zu stellen.
Daher muss für die Eröffnung lediglich dargelegt werden, dass ein solches Recht oder ein solcher Anspruch verletzt sein könnte.
Eine tiefere Prüfung erfolgt bei der Entscheidung über Eröffnung nicht, sondern im Zweifelsfall immer erst im laufenden Verfahren, da auch ergänzender Parteivortrag und Ermittlungen des Gerichts Einfluss auf diese Entscheidung haben und es daher einer ausführlichen Wertung durch das Schiedsgericht im Laufe des Verfahrens bedarf.
Hat das Schiedgericht von Anfang an Zweifel an der in der Anrufung dargeleten Begründung der Antragsbefugnis, ist das Verfahren dennoch zu eröffnen. Es bietet sich dann aber an, der anrufenden Partei einen richterlichen Hinweis zu geben, dass diese ihre Begründung diesbezüglich erweitern sollte.
Das geht auch aus \S~8 Abs.~5, 6 Satz~1 SGO\index[paridx]{SGO!8@\S~8!5@Abs.~5} hervor. Dort wird bestimmt, wann das Verfahren zu eröffnen ist. Nämlich immer dann, wenn das angerufene Schiedsgericht zuständig ist und die Anrufung korrekt erfolgte.
Korrekte Anrufung kann aber nur formelle Kriterien meinen, eine weitergehende, inhaltliche Prüfung zu diesem Zeitpunkt ist daher unzulässig.
Die Antragsbefugnis ist damit ein echtes Zulässigkeitskriterium und erst nach durchgeführtem Verfahren im Endurteil zu bescheiden, an die Anrufung selbst darf daher nur die Anforderung gestellt werden, dass die Möglichkeit einer solchen Antragsbefugnis in zumindest nicht gänzlich unplausibler Weise geltend gemacht wird.

\subsection{Form}
\label{Anrufung:Kriterien:Form}
Die Anforderungen an die Form der Anrufung sind sehr gering.
Es wird in \S~8 Abs.~3 SGO\index[paridx]{SGO!8@\S~8!3@Abs.~3} lediglich die Textform gefordert.
Im Gegensatz zur Schriftform, deren Erfüllung und Nichterfüllung bei untergesetzlich vorgeschriebener Formerfordernis sehr strittig ist und auch in der Schiedsgerichtsbarkeit der Piratenpartei schon zu Entscheidungen führte,\footcite{LSGBB146} ist die Textform ausreichend klar und eindeutig in \S~126b BGB\index[paridx]{BGB!126b@\S~126b} definiert und eine Formlockerung in \S~127 BGB\index[paridx]{BGB!127@\S~127} nicht vorgesehen.
Es reicht daher ein Schreiben, das mit dem Namen unterzeichnet ist und das in einer Form, die zur visuellen Widergabe des Inhalts geeignet ist (also auch z.B. eine E-Mail).

\subsection{Frist}
\label{Anrufung:Kriterien:Frist}
Die Frist für eine Anrufung soll sicherstellen, dass Streitigkeiten schnell geklärt werden.
Damit soll verhindert werden, dass auf Basis einer möglicherweise rechtswidrigen Handlung weitere Dinge passieren und eine Klage, die erst nach langer Zeit eingereicht wird, unerwartet alles mitreißen kann.
Kurz gesagt: Nach einer gewissen Zeit sollen alle möglicherweise Betroffenen auf einen eingetretenen Zustand vertrauen können.
Daher darf nach Ablauf dieser Frist nicht mehr geklagt werden und eine Anrufung, die bei Fristende noch nicht vollständig beim angerufenen Schiedsgericht eingegangen ist, ist grundsätzlich abzuweisen.\footnote{So die ständige Rechtsprechung des Bundesschiedsgerichts, statt vieler: \cites{BSG2315HS}{BSG20130227}.}

Dabei kennt die SGO unterschiedliche Anrufungsfristen.
\begin{enumerate}
\item Die Standardanrufungsfrist von 2 Monaten, \S~8 Abs.~4 Satz~1 SGO,\index[paridx]{SGO!8@\S~8!4@Abs.~4}
\item Die Anrufungsfrist in Einspruchsverfahren gegen Ordnungsmaßnahmen von 14 Tagen, \S~8 Abs.~4 Satz~2 SGO,\index[paridx]{SGO!8@\S~8!4@Abs.~4}
\item Die Anrufungsfrist in Parteiausschlussverfahren, \S~8 Abs.~4 Satz~3 SGO,\index[paridx]{SGO!8@\S~8!4@Abs.~4}
\item Die Berufungsfrist gegen ein Urteil mit korrekter Rechtsmittelbelehrung, \S~13 Abs.~2 Satz~1 SGO,\index[paridx]{SGO!13@\S~13!2@Abs.~2}
\item Die Berufungfrist gegen ein Urteil ohne korrekte Rechtsmittelbelehrung, \S~13 Abs.~2 Satz~4 SGO.\index[paridx]{SGO!13@\S~13!2@Abs.~2}
\end{enumerate}

Die Ausnahme stellt hier die Anrufungsfrist im Parteiausschlussverfahren dar.
Diese ist keine formelle Frist wie die anderen Fristen dar, sondern ist eine wertende, materielle Präklusionsfrist.
Das bedeutet, dass über die fristgerechte Einbringung der Vorwürfe nicht schon mit Eröffnung zu bescheiden ist, sondern erst im Urteil nach durchlaufenem Verfahren.
Dies soll einerseits das Parteimitglied dagegen schützen, dass es sich gegen alte Vorwürfe schützen muss und andererseits ein Parteiausschlussverfahren auch dann noch ermöglichen, wenn einzelne Handlungen schon länger zurück liegen, aber Teil eines Großen und Ganzen sind, das einen einzigen, fortlaufende Gesamtverstoß gegen Satzung, Grundsätze oder Ordnung der Partei mit schwerem Schaden darstellt.\footnote{Zur Zulässigkeit des Parteiausschlusses aufgrund eines solchen Gesamtverstoßes \cite[S.~19]{LSGHE20130624}\nomenclature{LSG~HE}{Landesschiedsgericht Hessen} m.w.N.\nomenclature{m.w.N.}{mit weiteren Nachweisen} und später bestätigend \cite{BSG20131028}.}
Alle anderen genannten Fristen sind dagegen echte formelle Fristen die schon die Eröffnung eines Verfahrens verhindern sollen.

Die Fristberechnung folgt dabei den \SSS~186 ff. BGB\index[paridx]{BGB!186@\S~186}.
Der Fristanfang wird nach \S~187 Abs.~1~BGB\index[paridx]{BGB!187@\S~187 Abs.~1} bestimmt, das Fristende nach \S~188 BGB\index[paridx]{BGB!188@\S~188}.
Das ist auch der Grund, warum in der Schiedsgerichtsordnung eine Ladungsfrist für (fern-)mündliche Verhandlungen in \S~10 Abs.~5 Satz~2 SGO\index[paridx]{SGO!10@\S~10!5@Abs.~5} auf 13 Tage gesetzt ist und nicht etwa 14 Tage oder zwei Wochen.
So kann das Schiedsgericht in seiner Sitzung einen Verhandlungstermin beschließen und wenn es am selben Tag noch zur Verhandlung lädt, kann diese am selben Wochentag in der zweiten Woche danach stattfinden und nicht erst in der dritten Woche danach.

Besonders geachtet werden sollte dabei vor allem auf \S~193 BGB\index[paridx]{BGB!193@\S~193} geachtet werden: Endet die Frist an einem Samstag, Sonntag oder gesetzlichen Feiertag, dann verlängert sich die Frist entsprechend.
Eine Fristende an einem Samstag um 24 Uhr wird somit zu einem Fristende am Montag um 24 Uhr (vorausgesetzt, dass der Montag kein Feiertag ist, dann wäre das Ende am Dienstag um 23 Uhr usw.).
Dabei müssen vor allem das Bundesschiedsgericht, aber auch für Landesschiedsgerichte, die verwiesene Verfahren behandeln, im Auge behalten, dass jedes Bundesland eigene Feiertagsgesetze hat.

\subsection{Zuständigkeit}
\label{Anrufung:Kriterien:Zustaendigkeit}
Über die Zuständigkeit hat das Gericht bereits mit Anrufung zu entscheiden, \S~8 Abs.~5 SGO\index[paridx]{SGO!8@\S~8!5@Abs.~5}.
Dies hat den Hintergrund, dass die SGO anders als die staatliche Gerichtsbarkeit, keine Verweisung vom unzuständigen, angerufenen Gericht an das zuständige Gericht vorsieht.
Auch sind die Zuständigkeitskriterien der SGO nach \S~6 SGO\index[paridx]{SGO!6@\S~6} wesentlich einfacher als etwa die der ZPO.
So ist es bereits aus den Daten der Anrufung ersichtlich, ob ein Gericht zuständig ist.

\subsubsection{Grundlegendes}
\label{Anrufung:Kriterien:Zustaendigkeit:Grundlegendes}
Dabei wird instanziell bei den untersten Gerichten begonnen, \S~6 Abs.~1 SGO\index[paridx]{SGO!6@\S~6!1@Abs.~1}.
Das sind grundsätzlich die Landesschiedsgerichte.
Allerdings kann der Satzungsgeber auf Landesebene die Einrichtung von niedrigeren Schiedsgerichten in der Landessatzung für den jeweiligen Landesverband erlauben, \S~2 Abs.~1 Satz~2 SGO\index[paridx]{SGO!2@\S~2!1@Abs.~1}.
Ist das passiert, ist das niederigste Schiedsgericht der entsprechenden Regelung zu entnehmen.
Bisher ist das allerdings in keinem Landesverband passiert.

\S~6 Abs.~2 SGO\index[paridx]{SGO!6@\S~6!2@Abs.~2} bestimmt dann noch, dass sie die Zuständigkeit nach der Verbandszugehörigkeit des Antragsgegners richtet.
Verbandszugehörige sind die Mitglieder eines Verbandes, das können nach \S~2 Abs.~1 Satz~2 PartG\index[paridx]{PartG!2@\S~2!1@Abs.~1} nur natürliche Personen sein.
Dabei wird in \S~6 Abs.~2 SGO nochmal klargestellt, was sich aus \S~8 Abs.~5 SGO,\index[paridx]{SGO!8@\S~8!5@Abs.~5} der die Entscheidung über die Zuständigkeit mit der Eröffnung als Entscheidung über die Korrektheit der Anrufung zusammenlegt, ergibt: Der relevante Zeitpunkt für die Entscheidung ist immer die mitgliedschaftliche Zuordnung zum Zeitpunkt der Anrufung.
Das wiederrum hängt auch damit zusammen, dass es keine Verweisung zwischen den einzelnen Gerichten gibt und es der anrufenden Streitpartei nicht zugemutet werden soll, dass ein etwaiger Gliederungswechsel oder z.B. eine Verschmelzung von Gliederungen das schon angelaufene Verfahren ungültig macht, weil es im Nachinein unzulässig wurde.

\subsubsection{Zuständigkeit bei Verfahren gegen ein Organ}
\label{Anrufung:Kriterien:Zustaendigkeit:Organ}
Organe haben -- juristisch streng genommen -- keine Mitgliedschaft in einem Verband, sondern sind Teil eines Verbandes und handeln für diesen.
Man könnte zwar eine Mitgliedschaft eines Organs in einem Verband im allgemeinen Sprachgebrauch konstruieren für den Verband, dessen Teil sie sind.
Aber auch in diesem Fall leitet sich daraus keine Mitgliedschaft in den höheren Verbänden ab, da \S~2 Abs.~1 Satz~2 PartG\index[paridx]{PartG!2@\S~2!1@Abs.~1} ganz eindeutig die Mitgliedschaft auf natürliche Personen limitiert.
Daher lässt würde die Sonderzuständigkeitsregelung nach \S~6 Abs.~3 SGO\index[paridx]{SGO!6@\S~6!3@Abs.~3}, die die Zuständigkeit des Bundesschiedsgerichts für Verfahren gegen Bundesorgane sowie die Zuständigkeit der Landesschiedsgerichte für Verfahren gegen Landesorgane anordnet, die Zuständigkeit für Verfahren gegen Organe von Verbänden unterhalb der Landesebene offen.
Das ist eine offensichtliche planwidrige Regelungslücke, da es nicht gewollt sein kann, dass Verfahren gegen untergeordnete Gliederungen unmöglich wird, nurweil das zuständige Gericht nicht determiniert werden kann, zumal die Einrichtung von Schiedsgerichten unterhalb der Landesebene in \S~2 Abs.~1 Satz~2 SGO\index[paridx]{SGO!2@\S~2!1@Abs.~1} explizit erlaubt wird.
Deswegen drängt sich eine analoge Anwendung der Zuständigkeitsregelungen für Organe nach \S~6 Abs.~3 SGO\index[paridx]{SGO!6@\S~6!3@Abs.~3} auf: Das niedrigste Schiedsgericht auf gleicher oder höherer Verbandsebene wie das Organ, das Antragsgegner ist, ist nach SGO zuständig für Verfahren gegen ebendieses Organ.

\subsubsection{Zuständigkeit bei Disziplinarverfahren}
\label{Anrufung:Kriterien:Zustaendigkeit:Disziplinarverfahren}
Für die Verfahren mit Disziplinarcharakter, also Verfahren über Einsprüche gegen Ordnungsmaßnahmen und Parteiausschlussverfahren, ist die Zuständigkeit nochmal gesondert in \S~6 Abs.~4 SGO\index[paridx]{SGO!6@\S~6!4@Abs.~4} geregelt.
Die etwaige Einrichtung von Schiedsgerichten unterhalb der Landesebene soll nichts am Verfahrensablauf ändern und das Verfahren etwa durch eine Instanz mehr im Zweifel in die Länge ziehen und gleichzeitig soll in jedem Fall der für Parteiausschlussverfahren in \S~10 Abs.~5 Satz~2 PartG\index[paridx]{PartG!10@\S~10!5@Abs.~5} vorgeschriebene zweizügige Instanzenzug für alle Mitglieder gleich aussehen. Im selben Zug wird eine nicht notwendige zweite Instanz auch für Einsprüche gegen Ordnungsmaßnahmen garantiert.

Durch die exklusive Formulierung sperrt \S~6 Abs.~4 SGO\index[paridx]{SGO!6@\S~6!4@Abs.~4}.
Dadurch könnte man aus der Präsensformulierung des Teilsatzes \enquote{Landesverband […], bei dem der Betroffene Mitglied ist.} schließen, dass ein anderer Zeitpunkt für die Entscheidung über die Zuständigkeit relevant wird.
In Betracht kommen hier sowohl die frühere Zeitpunkt des Beschlusses des Organs über die Verhängung der Ordnungsmaßnahme oder der Einreichung eines Parteiausschlussantrages und der frühere Zeitpunkt der vorgerichtlichen Anhörung des Mitglieds im Disziplinarverfahren als auch der spätere Zeitpunkt der tatsächlichen Entscheidung über die Eröffnung.
Ein späterer Zeitpunkt käme wegen \S~8 Abs.~5 SGO\index[paridx]{SGO!8@\S~8!5@Abs.~5} definitiv nicht mehr in Betracht, eine Flucht aus einem laufenden Parteiausschlussverfahren in einen anderen Verband ist daher definitiv nicht möglich.
Grundsätzlich bietet es sich aus den schon oben geschilderten, sich aus \S~8 Abs.~5 SGO\index[paridx]{SGO!8@\S~8!5@Abs.~5}

Problematisch wird dies allerdings in den Fällen, in denen zwischen Start des Disziplinarverfahrens durch Anhörung, dem Beschluss darüber und der Schiedsgerichtsanrufung ein Verbandswechsel des Betroffenen erfolgt.
Im Falle einer Ordnungsmaßnahme würde dann ein verbandsfremdes Schiedsgericht über die Handlung eines Verbandsorgans entscheiden und auch Ermessenskontrolle über dieses Organ ausüben, im Falle eines Parteiausschlusses wäre dann auch noch die Frage, ob ein schon beschlossener und zum Beschlusszeitpunkt zulässiger, aber noch nicht einreichter Parteiausschlussantrag seine Zulässigkeit verliert, zu entscheiden.

Für die Frage der verbandsfremden Ermessenskontrolle muss allerdins bedacht werden, dass in beiden Fällen das letzinstanzlich zuständige Bundesschiedsgericht das letzte Wort hätte.
Daher scheint ein außeinanderfallen von Verbandszugehörigkeit des ersintanzlich zuständigen Schiedsgerichts und des verhängenden Organs schlussendlich doch unproblematisch, zumal die gesonderte Regelung durch \S~6 Abs.~4 SGO\index[paridx]{SGO!6@\S~6!4@Abs.~4} wohl auch das mitbezweckt hat.

Für den Parteiausschluss wird das wohl davon abhängen, wie man ihn im Vergleich zu den anderen Ordnungsmaßnahmen einsortiert. Wird er einfach nur als stärkste Ordnungsmaßnahme einsortiert, auf die das Prozedere der Ordnungsmaßnahmen nach \SSS~14, 6 Abs.~1 BS\index[paridx]{BS!14@\S~14}\index[paridx]{BS!6@\S~6!1@Abs.~1} voll anwendbar ist\footnote{So wohl die jüngste Ansicht des Bundesschiedsgerichtes in \cite{BSG3615HS} und das Sondervotum des Bundesschiedsrichters Markus Gerstel in \cite{BSG20131005}.}, muss sich der Betroffene wohl damit abfinden, dass er ab Anhörung  bzw. spätestens ab dem Beschluss des Organs über die Einreichung einen Verbandswechsel der Zulässigkeit des Verfahrens nicht mehr entgegenhalten kann, um auch hier eine Gleichheit mit den anderen Ordnungsmaßnahmen zu erreichen.
Denn wenn das betroffene Mitglied  von dem Plan für Parteiaausschlussverfahren wusste und deswegen den Verband wechselt, ist es nicht schutzwürdig, ganz unabhängig davon, ob der Wechsel im Innenverhältnis der Gliederungen nach \S~3 Abs.~2a BS\index[paridx]{SGO!3@\S~3!2@Abs.~2}
Folgt man hingegen der in \cite{BSG20131005} aufgezeigten Rechtssprechungslinie und sieht die Anhörung durch das einleitende Organ im Parteiausschlussverfahren als nicht verpflichtend an, so ergibt sich kein Grund, warum das Mitglied einen solchen Beschluss gegen sich gelten lassen muss, von dem es im Zweifel noch nicht einmal etwas wusste.
Das Vertrauen darauf, dass grundsätzlich nur die Organe der Gliederungen, in denen man tatsächlich aktuell Mitglied ist,  Disziplinarmaßnahmen gegen das Mitglied unternehmen, ist dann schutzwürdig und der relevante Zeitpunkt für die Entscheidung über die Zuständigkeit dürfte wieder mit der Anrufung beim Schiedsgericht zusammenfallen.\footnote{Vgl. wie schon oben \cite[S. 9]{BSG115HS}, unter II.1.c.c).}

\subsection{Schlichtung}
\label{Anrufung:Kriterien:Schlichtung}
Grundsätzlich ist nach \S~7 Abs.~1 SGO\index[paridx]{SGO!7@\S~7!1@Abs.~1} ein Schlichtungsversuch vor der Anrufung der Schiedsgerichtsbarkeit erforderlich.
Damit ist die Schlichtung wie auch die Zuständigkeit (siehe \ref{Anrufung:Kriterien:Zustaendigkeit}) ein besonderes \emph{unechtes Zulässigkeitskriterium}: Diese Kriterien würden in den staatlichen Gerichtsordnungen als \emph{echte Zulässigkeitskriterien} von nicht bloß formaler Natur zu einem Unzulässigkeitsurteil führen. Nach der SGO aber führen sie zu einer Nichteröffnung und sind entsprechend zum Zeitpunkt der Eröffnung abschließend zu beurteilen.

Fehlt es an einem erfolglosen Schlichtungsversuch und ist dieser auch nicht entbehrlich, fehlt der anrufenden Streitpartei das Rechtsschutzinteresse und die Klage ist als unzulässig mit einer Nichteröffnung abzuweisen.

\subsubsection{Ablauf}
\label{Anrufung:Kriterien:Schlichtung:Ablauf}
Die Schlichtung wird von den Parteien eigenständig durchgeführt, \S~7 Abs.~2 Satz~1 SGO\index[paridx]{SGO!7@\S~7!2@Abs.~2}.
Daher ist der Ablauf nicht im Verantwortungs- oder Tätigkeitsbereich der Parteischiedsgerichte.
Bei tieferer Betrachtung macht das auch Sinn:
Ein Schiedsgericht, dass schon einen Schlichtungsversuch durchführt, muss im Zuge ebendieser Tätigkeit Vorschläge zur Einigung machen und sich so zumindest schon teilweise zum konkreten Sachverhalt positionieren.
Dadurch könnte die gewünschte Neutralität eines Schiedsgerichtes verloren gehen.
Gerade deswegen dürfen auch Richter, die schon als Schlichter aktiv waren, nicht als Richter im Folgeverfahren teilnehmen (vgl. auch \ref{Zusammensetzung:Spruchkoerper:Befangenheitsvermutung:Nr8}).

Es wäre auch mit der Funktion eines mediativen Schlichters, den \S~7 SGO\index[paridx]{SGO!7@\S~7} vorsieht, nicht vereinbar, wenn der Schlichter zuerst unverbindliche Vorschläge machen würde und diese im späteren Verfahren den Parteien als Richter per Urteil aufzwingen könnte.

\subsubsection{Auswirkungen}
\label{Anrufung:Kriterien:Schlichtung:Auswirkungen}
Die erfolglose Schlichtung ist Anrufungsvoraussetzung.
Da eine Schlichtung auch Zeit in Anspruch nehmen kann, wird die Anrufungsfrist (vgl. \ref{Anrufung:Kriterien:Frist}) in der Zeit, in der eine Schlichtung versucht wird, gehemmt, \S~8 Abs.~4 Satz~4 SGO\index[paridx]{SGO!8@\S~8!4@Abs.~4}.

Das eröffnende Schiedsgericht muss also feststellen, ob ein erfolgloser Schlichtungsversuch unternommen wurde.
Eine Schlichtung ist das ernsthafte Bemühen, auf konstruktivem Wege eine gütliche Einigung in einem Streit zu erreichen.\footnote{Vgl. Kommentierung des \S~7 SGO von Benjamin Siggel, \href{https://wiki.piratenpartei.de/Bundesschiedsgericht/Schlichtung}{https://wiki.piratenpartei.de/Bundesschiedsgericht/Schlichtung}.}
Daher kann eine fehlgeschlagene Schlichtung bereits dann gegeben sein, wenn sich eine Partei der Schlichtung konsequent verweigert oder diese immer wieder verzögert und so einen effektiven Rechtsschutz der Gegenpartei verhindert.
Das Schlichtungsverfahren soll kein Hindernis für den Rechtsschutz sein sondern ein weniger invasives Verfahren, in dem die Details zur Beilegung des Streits den Streitparteien überlassen werden und diese nicht von Dritten, dem Schiedsgericht, aufgezwungen werden.

\subsubsection{Entbehrlickeit}
\label{Anrufung:Kriterien:Schlichtung:Entbehrlickeit}
Die Schlichtung ist nach \S~7 Abs.~3 Satz 1 SGO\index[paridx]{SGO!7@\S~7!3@Abs.~3} entbehrlich, wenn es sich um Disziplinarverfahren (Einspruch gegen eine Ordnungsmaßnahme, Parteiausschluss) handelt, es ein Rechsmittelverfahren ist oder ein Fall der Eilbedürftigkeit des Verfahrens, die Aussichtslosigkeit oder des Scheiterns der Schlichtung vorliegt.

Für Disziplinarverfahren ist anzunehmen, dass eine Schlichtung zwingend scheitert. Daher macht eine Pflicht zum Schlichtungsversuch hier ebensowenig Sinn wie in Rechtsmittelverfahren, die sich auf ein Schiedsgerichtsverfahren beziehen. 

Für Anrufungen im einstweiligen Rechtsschutz entfällt die Schlichtungserfordernis grundsätzlich nach  \S~7 Abs.~3 Satz 1 Var.~4 SGO\index[paridx]{SGO!7@\S~7!3@Abs.~3}.\footnote{\cites[3]{BSG20131210}{BSG3014HS}.}
Andere Fälle der Eilbedürftigkeit dürften kaum eintreten.

Die Aussichtslosigkeit oder das Scheitern der Schlichtung unterscheiden sich lediglich dadurch, ob bereits ein Schlichtungsversuch unternommen wurde, oder aber eben dieser gar nicht unternommen wurde, weil er schon gar keine Aussicht auf Erfolg hat. Dabei unterliegt die Wertung des Einzelfalls dem Ermessen der Richter. Aussichtlos dürfte eine Schlichtung immer dann sein, wenn eine Partei in Bezug auf einen konkreten Streit schon von vorn herein ankündigt, für eine Schlichtung nicht zur Verfügung zu stehen, oder wenn es um reine Rechtsfragen wie im Falle des \S~14 Abs.~1 Satz~1 Var.~2 PartG\index[paridx]{PartG!14@\S~14!1@Abs.~1} geht.

Die Entscheidung eines Schiedsgerichts, ob ein solcher Fall vorliegt, ist nach \S~7 Abs.~3 Satz 2 SGO\index[paridx]{SGO!7@\S~7!3@Abs.~3} nicht anfechtbar.
Grundsätzlich sind hier zwei Auslegungen denkbar.

Erstens eine weite Auslegung:
Jede Entscheidung über Satz~1 ist unanfechtbar.
Eine Nichteröffnungsbeschwerde könnte sich dann nicht gegen die von einem erstinstanzlichen Schiedsgericht festgestellte Erforderlichkeit der Schlichtung in diesem Verfahren richten.

Und zweitens eine enge Auslegung:
Nur die positive festgestellte Entbehrlichkeit der Schlichtung ist unanfechtbar.
Eine Berufung gegen das Urteil im eröffneten Verfahren könnte sich dann nicht mehr darauf berufen, dass das Verfahren wegen eines erforderlichen, aber nicht erfolgten Schlichtungsversuches gar nicht hätte eröffnet werden dürfen.
Das Bundesschiedsgericht hat in seiner Rechtsprechung bisher die enge Auslegung angewendet,\footnote{Vgl. etwa \cites{BSG914H1}{BSG3014HS}.} ohne dies näher zu begründen.

Das ist auch korrekt:
Die Schlichtungserfordernis soll lediglich in den Fällen, in denen eine Schlichtung besser geeignet ist, den Rechtsfrieden wiederherzustellen, dieser den Vorrang geben und nicht jeglichen Rechtsschutz verhindern.
Daher ist die Hürde nicht zu hoch zu setzen, wenn das Eingangsgericht fälschlich das Vorliegen dieses Verfahrenserfordernisses verneint, ist dagegen Rechtsschutz zu gewähren.
Der Sinn und Zweck des Vorrangs des Schlichtungserfordernisses ist nicht gefährdet, wenn das Rechtsmittelgericht dessen Entbehrlichkeit feststellt und das Verfahren eröffnet.
Das Gebot des effektiven Rechtsschutzes stützt also diese enge Auslegung. Der Sinn und Zweck der Nichtanfechtbarkeit besteht also lediglich in dem Vertrauensschutz in das Bestehen des Verfahrenserfordernisses nach Eröffnung, damit nicht nachträglich ein Verfahrenserfordernis entzogen werden kann.

\section{Bescheidung der Eröffnung}
\label{Anrufung:Beschluss}
Nach der Prüfung aller dargelegten Kriterien ist über die Eröffnung zu bescheiden.
Dabei ist relevant, sich für jedes Kritrium und seine Bestimmungen einzeln anzusehen, um festzustellen, welche Anforderungen für die Eröffnung vorliegen müssen und welche Kriterien erst für das Endurteil vorliegen müssen.
Grundsätzlich gilt dabei:
Die Fakten müssen vollständig zutreffen, aber die Informationen über diese Fakten können in einigen Fällen noch während des Verfahrens nachgeliefert werden.
Und zwar ist ein solcher Nachtrag von Fakten immer möglich, wenn es Teil der Verhandlung ist, ebendiese Informationen zu ermitteln.
Das ist damit auch das Unterscheidungskriterium zwischen unechten Zulässigkeitskriterien, die bereits mit Eröffnung zu entscheiden sind, und echten Zulässigkeitskriterien, die erst im Prozessurteil beschieden werden müssen.\footnote{Vergleiche zu echten Zulässigkeitskriterien die Ausführungen im Abschnitt Zulässigkeit, S.~\pageref{Zulaessigkeit}.}
Das Musterbeispiel für einen solchen Nachtrag dürften Informationen über Sachverhalt und Begründung der Klage sein, deren Ermittlung ja gerade Haupttätigkeit im Verfahren ist.
Mit anderen Worten:
Alle Entscheidungen, die zwangsläufig auf Ermittlung und Wertungen des Gerichts und dem Vortrag der Streitparteien beruhen, sind im Zweifel echte Zulässigkeitsvorraussetzungen und erst mit dem Endurteil zu bescheiden.

\subsection{Nachbesserung}
\label{Anrufung:Beschluss:Nachbesserung}
Ist mindestens ein Kriterium nicht erfüllt, weil eine odere mehrere notwendige Angaben ganz oder teilweise fehlen, sollte der anrufenden Partei zunächst die Möglichkeit der Nachbesserung gegeben werden.
Die anrufende Partei sollte dabei möglichst genau auf die fehlenden Angaben hingewiesen werden.
Allerdings ist besonders auf die Frist zu achten.
Eine im Zeitpunkt der Verfristung unvollständige Anrufung ist unheilbar verfristet.
Daher ist eine Anrufung, die fristgerecht, aber unvollständig einging, und deren Anrufungsfrist vor einer Antwort durch das Gericht abgelaufen ist, nicht mit einer Nachbesserungsaufforderung zu beantworten, sondern mit einem Nichteröffnungsbeschluss.
Eine Ausnahme besteht dann, wenn sich die Frist aus dem Vorgetragenen nicht berechnen lässt.
Das ist typischerweise der Fall, wenn eine schlichtungspflichtige Anrufung vorliegt, aber Angaben zur Schlichtung und ihrem Umfang v.a. in zeitlicher Hinsicht fehlen.
In diesem Fall kann wegen der Hemmungswirkung der Schlichtung die Frist nicht berechnet werden, daher bietet sich hier zunächst eine Nachbesserungsaufforderung an.

Die Nachbesserung ist in der Schiedsgerichtsordnung nicht erwähnt.
Dennoch sollten Schiedsgerichte der anrufenden Partei die Möglichkeit dazu geben, sofern die Klage noch nicht verfristet ist.
Ob es gar eine Pflicht für die Schiedsgerichte gibt, Gelegenheit zur Nachbesserung zu geben, ist unklar.
Das Landesschiedsgericht Bayern nimmmt dies jedenfalls dann an, wenn innerhalb der regulären Anrufungsfrist noch genügend Zeit verblieben ist,\footnote{So in \cite[S.~2~f.]{LSGBYB413U},\nomenclature{LSG~BY}{Landesschiedsgericht Bayern (Bayerisches Landesschiedsgericht)} in der Bestätigung der Entscheidung durch das Bundesschiedsgericht mit \cite{BSG314HA} weiter thematisiert.} das Landesschiedsgericht Brandenburg nimmt dies wohl selbst dann an, wenn nur noch wenig Restfrist verblieben ist und setzt dann eine die Anrufungsfrist verlängernde Nachfrist.\footnote{So wohl jedenfalls in \cites[S.~7]{LSGBB133}{LSGBB134}.}
Das Bundesschiedsgericht hat zumindest die Erfordernis einer die Anrufungsfrist verlängernden Nachfristsetzung verneint,\footnote{\cites[S.~2]{BSG2315HS}[S.~2]{BSG20130227}.} und auch ansonsten die verpflichtende Erfordernis einer Nachbesserungsaufforderung  nie angenommen und zumindest dann verneint, wenn eine Fristberechnung dem Schiedsgericht nicht möglich ist und es durch die Nachbesserungsaufforderung, die eine die Anrufungsfrist ersetzende neue Frist setzt, einer ansonsten verfristeten Anrufung zum Erfolg verhelfen könnte.\footnote{\cites{BSG3915HS}[S.~2]{BSG215HS}}

\subsection{Nichteröffnungsbeschluss}
\label{Anrufung:Beschluss:Nichteroeffnung}
Wenn eine Partei nicht erfolgreich nachbessert bleibt keine andere Option als die Anrufung abzuweisen.
Die Abweisung muss begründet erfolgen, \S~8 Abs.~8 Satz~2 SGO\index[paridx]{SGO!8@\S~8!6@Abs.~6}

Es sollten also genau erläutert werden, warum das Verfahren nicht eröffnet wird.
Gab es mehrere Gründe, sollten auch alle alternativen Gründe angeführt werden.
Einerseits hilft das dem Rechtsschutzschuchenden, beim nächsten Mal etwas nicht zu übersehen, und zudem hilft das für den Fall, dass eine Nichteröffnungsbeschwerde beim Rechtsmittelgericht eingelegt wird, allen Beteiligten, das Verfahren schnell zu bearbeiten.

\subsection{Eröffnungsbeschluss}
\label{Anrufung:Beschluss:Eroeffnung}
Sind alle Kriterien erfüllt und die Anrufung daher korrekt im Sinne des \S~8 Abs.~5 SGO\index[paridx]{SGO!8@\S~8!5@Abs.~5}, ist das Verfahren zu eröffnen.

Im Eröffnungsbeschluss sind nach \S~9 SGO\index[paridx]{SGO!9@\S~9} einige Mitteilungen an die Parteien zu machen.
Den Parteien soll mitgeteilt werden:
\begin{enumerate}
\item Das Datum des Eröffnungsbeschlusses.
\item Das Aktenzeichen.
\item Die komplette Besetzung des Spruchkörpers.
\item Bei Änderungen gegenüber der Standardbesetzung die Gründe dafür.
\item Eine Kopie des Anrufungsschreibens.
\item Die Information, das ein Verfahrensvertreter bestellt werden kann, bzw. im Falle, dass eine Verfahrenspartei ein Organ ist, dass sie dies tun muss.
\item Eine Aufforderung zur Stellungnahme an beide Verfahrensparteien zum Verfahren.
\item Die Ladung zur (fern-)mündlichen Verhandlung, sofern sie schon steht mit dem Hinweis, dass auch in Abwesenheit Verhandelt werden kann, \S~10 Abs.~5 Satz~4 SGO\index[paridx]{SGO!10@\S~10!5@Abs.~5}.
\item Der Hinweis, dass die Parteien das Recht haben Richter abzulehen mit Hinweis auf die Präklusion nach \S~5 Abs.~2 Satz~4 SGO\index[paridx]{SGO!5@\S~5!2@Abs.~2}.
\item Gründe, die Richter nach \S~5 Abs.~2 Satz~3 SGO\index[paridx]{SGO!5@\S~5!2@Abs.~2} anzeigen müssen.
\item Nur bei Verfahren über eine Ordnungsmaßnahme oder einen Parteiausschluss: Frage an das Mitglied, ob es ein nichtöffentliches Verfahren wünscht.
\end{enumerate}

\subsection{Folgenden eines Eröffnungsbeschlusses}
\label{Anrufung:Beschluss:EroeffnungFolgen}
Die Schiedsgerichtsordnung startet das Verfahren mit dem Eröffnungsbeschluss.
Das heißt, ab diesem Moment entsteht ein Prozessrechtsverhältnis zwischen dem eröffnenden Schiedsgericht und den Verfahrensparteien nach den Verfahrensregeln der Schiedsgerichtsordnung.
Dieses Verfahren kann grundsätzlich nur durch ein Urteil abgeschlossen werden, \S~12 Abs.~1 SGO\index[paridx]{SGO!12@\S~12!1@Abs.~1}.\footnote{So explizit auch schon \cite[S.~2]{BSG20131204}.}
Das heißt auch, dass eine andere Beendigung dieses Prozessrechtsverhältnisses nicht zulässig ist.
Ausnahmen stellen hier selbständige Verfahren im einstweiligen Rechtsschutz vor,\footnote{Ausführlich zu Eigenständigkeit der Verfahren im einstweiligen Rechtsschutz \cite[S.~3]{LSGBB145} mit Verweis auf \cites[S.~4]{LSGHE20140423II} und die in \cites{BSG41114ES}{BSG3314EA} offensichtlich zum Ausdruck kommende Praxis des Bundesschiedsgerichtes, bestätigt von \cite[S.~2~f.]{BSG4214ESWiderspruch}.} die mit der einstweiligen Anordnung begründet werden, da es dort keine Eröffnung gibt, und mit dem Widerspruchsurteil oder der Verfristung seiner Beantragung beendet werden.

Allerdings gab es in der Rechtsprechung bisher schon Abweichungen von diesem Schema.
Das Bundesschiedsgericht ist in jüngster Zeit von Rücknehmbarkeit eines Eröffnungsbeschlusses ausgegangen, wenn seine Anforderungen nicht mehr erfüllt sind oder es noch nie waren und wendet dahingehend wohl \S~10 Abs.~1 SGO analog an.\footnote{So etwa \cite{BSGPP100127862}.}
Gerade im Prozessrecht mit seinen feinen Regelungen des Verfahrensablaufs ist es äußerst problematisch und begründungsbedürftig, wenn aus dem Vorhandensein der Regelung eine Kompetenz zu einem exakt gegenlaufenden Beschluss abgeleitet wird.
Zudem hat das Bundesschiedsgericht hier die tatsächliche Antragsbefugnis fehlerhaft für ein Statthaftigkeitskriterium gehalten und sie nicht, wie es korrekt wäre, als echtes Zulässigkeitskriterium behandelt (vgl. \ref{Anrufung:Kriterien:Antragsbefugnis}).

In älterer Rechtsprechung war das Bundesschiedgericht daher der Überzeugung, dass eine solche Rücknahme nicht möglich ist, es sei denn, dass der Eröffnungsbeschluss schon explizit unter einem Vorbehalt bestehender aus einer mit einer Frist versehenen Nachbesserungsaufforderung getroffen wurde und damit selbst bedingt war.\footnote{Vgl. \cite[S.~2]{BSG20130715} zu \cite{LSGNRW2013011},\nomenclature{LSG~NRW}{Landesschiedsgericht Nordrhein-Westfalen} in welchem das Bundesschiedsgericht einen solchen Vorbehalt erfüllt sah.}

Unter Anbetracht des Wortlauts der Schiedsgerichtsordnung ist die neuere Ansicht des Bundesschiedsgerichtes nicht haltbar.
Es bedarf schon gar keiner Analogie, da es schlicht an der notwendigen planwidrigen Regelungslücke fehlt, die aber Vorraussetzung jeder Analogie ist.
Es gibt auch bei einem Eröffnungsbeschluss, der trotz Nichterfüllung oder bei Wegfall der Anrufungskriterien oder in Unzuständigkeit getroffen wurde, die Möglichkeit, das Verfahren und damit das Prozessrechtsverhältnis wegen Unzulässigkeit mittels Urteil zu beenden.

\section{Sonderfall Verfahrensverweisung}
\label{Anrufung:Verweisung}
Einen Sonderfall der Anrufung stellt die Verfahrensverweisung nach \S~6 Abs.~5 SGO\index[paridx]{SGO!6@\S~6!5@Abs.~5} dar. Hier kommt es auf den Zustand des Verfahrens vor Verweisung an.

Ist das Verfahren vor Verweisung noch nicht eröffnet gewesen, muss das Zielgericht über die Eröffnung entscheiden, als wäre es selbst angerufen.
Dabei muss so vorgegangen werden, als wären die Anrufung statt an das Ursprungsgericht direkt an das Zielgericht gegangen.
Dies ist insbesondere für die Berechnung der Anrufungsfristen relevant.
Allerdings ist eine Verzögerung durch das Ursprungsgericht nicht der anrufenden Streitpartei zulasten zu legen.

Ist das Verfahren bereits vor Verweisung eröffnet gewesen, ist das Zielgericht an diese Entscheidung gebunden und muss das Verfahren entsprechend fortsetzen.
Es sollte trotzdem analog zum Eröffnungsbeschluss ein Übernahmebeschluss getroffen werden und die Verfahrensübernahme den Streitparteien mitgeteilt werden.
In dieser Übernahmemitteilung sollten den Parteien alle Mitteilungen gemacht werden, die sonst im Eröffnungsbeschluss getätigt werden (vgl. \ref{Anrufung:Beschluss:Eroeffnung}), um ein ordentliches Verfahren zu gewähren
Die Vorschriften über die Eröffnung sind dabei mangels eigener Regelung für die Verweisung analog anzuwenden.

Eine besondere Aufmerksamkeit gilt fehlerhaften Verweisungsbeschlüssen:
Diese sind gültig und mangels Rechtsmittelfähigkeit\footnote{\S~13 Abs.~6 Satz~1 SGO\index[paridx]{SGO!13@\S~13!6@Abs.~6} erlaubt nur in den Fällen, in denen es explizit durch die SGO vorgesehen ist, eine Beschwerde. Vgl. hierzu \cites{BSG201305151}{BSG201306071}.}  auch nicht im Instanzenzug \textbf{isoliert angreifbar}.
Ein fehlerhafter Verweisungsbeschluss ist daher bestandskräftig und somit rechtswirksam und begründet die Zuständigkeit des Zielgerichtes.\footnote{Vgl. dazu etwa \cite[5]{LSGBB147}.}
Allerdings dürfte diese Rechtsfrage auch nie zur Entscheidung vor der ordentlichen Gerichtsbarkeit landen, da das Bundesschiedsgericht immer letzinstanzlich zu entscheiden hat.
Da das Bundesschiedsgericht aber eine vollständige Tatsacheninstanz ist und eine eigene Entscheidung trifft, ist ein derartiger Fehler in der ersten Instanz für die ordentliche Gerichtsbarkeit nicht mehr relevant, da diese nur die ausschlaggebende Entscheidung des Bundesschiedsgerichtes zu berücksichtigen hat, das als solches auch die Instanzengarantie im Sinne des \S~10 Abs.~5 Satz~2 SGO\index[paridx]{SGO!10@\S~10!5@Abs.~5} erfüllt.
Die vor der ordentlichen Gerichtsbarkeit gerügte Beschwer bezüglich der Fehlerhaftigkeit des gesetzlichen Richters kann sich daher nur auf die Besetzung des Bundesschiedsgerichtes beziehen und nicht auf einen wegen falscher Verweisung fehlerhaften gesetzlichen Richter in der ersten Instanz.


\chapterbib
% \end{refsection}
