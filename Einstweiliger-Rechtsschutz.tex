% \begin{refsection}
\chapterpreamble{Der einstweilige Rechtsschutz hat den Zweck, Rechte des Antragstellers vorläufig zu sichern. Zentral sind hier das sog. „Eilbedürfnis“ und das „Sicherungsinteresse“, §~11~Abs.~2~SGO. Obwohl einstweilige Anordnungen regelmäßig dazu dienen dürften, den status quo bis zur Entscheidung in der Hauptsache zu sichern, können sie auch ohne eine Klage in der Hauptsache beantragt werden.}

\chapter{Der einstweilige Rechtsschutz}
%\blindtext[1]

%\chapterbib
% \end{refsection}
