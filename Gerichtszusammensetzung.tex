\chapterpreamble{Innerhalb des Organs „Schiedsgericht“, das sich zusammensetzt aus den gewählten Richtern und Ersatzrichtern, stellen die mit dem jeweiligen Verfahren befassten Richter das (entscheidende) „Schiedsgericht“ dar. Die genaue Zusammensetzung ergibt sich aus der Eigenschaft als Schiedsrichter oder (im Fall von Kammersystemen) aus der Geschäftsordnung. Die Zusammensetzung muss von vorn herein feststehen und darf nicht willkürlich verändert werden. Ein einmal befasster Richter kann regelmäßig nur durch Ausschluss aus dem konkreten Verfahren (typischerweise wegen Besorgnis der Befangenheit), generelle Beurlaubung oder Rücktritt vom Amt aus einem Verfahren ausscheiden. Ein temporäres Ausscheiden und spätere Wiederbefassung kommt ebensowenig in Betracht wie eine Rücktritt nur für ein einzelnes Verfahren. Ablehnungsanträge (Anträge auf Feststellung der Besorgnis der Befangenheit) müssen sich immer gegen jeweils eine konkrete Person richten und müssen einzeln (im Zweifel in der zeitlichen Reihenfolge der Antragstellung) entschieden werden. Näheres legt \S~5~SGO\index[paridx]{SGO!\S~5} fest.}

\chapter{Zusammensetzung des Schiedsgerichts}
\label{Zusammensetzung}
%\section{Organ}
%\subsection{Mitglieder des Schiedsgerichts}
%Richter ./. Ersatzrichter
%Vorsitzender Richter (inkl. Wahl → aktives/passives Wahlrecht)
%\subsection{Geschäftsordnung} % GO ist für Kapitel "Innere Organisation" geplant – braucht es hier eine eigene Überschrift?
%Exkurs(?): Geschäftsordnung (hier Ersatzrichter mit Stimmrecht?)
Das Schiedsgericht ist zwar kein Organ\index[idx]{Organ} im Sinne des \S~8 Abs.~2 PartG\index[paridx]{PartG!8@\S~8!2@Abs.~2}\footnote{\cite[\S~14 Rn.~3]{ipsen2008parteiengesetz}}, da es nicht der originiären politischen Willensbildung der Partei dient, allerdings ist es grundsätzlich ein Organ im weiteren Sinne.
Dementsprechend haben die Satzungen der Piratenparteien und ihrer Untergliederungen das Schiedsgericht typischerweise auch als Organ bezeichnet.\footnote{MISSING: Satzungen}

\section{Das Schiedsgericht als Organ}
\label{Zusammensetzung:Organ}
Das Organ setzt sich zusammen aus allen gewählten Richtern und Ersatzrichtern.
Für das Organ selbst gibt es dabei noch keine Unterscheidung in der Titelqualität.
Das Gesamtorgan entscheidet dabei nie über konkrete Fälle, sondern nur über die allgemeinen Angelegenheiten des Organs.
Dazu gehören etwa der Beschluss der Geschäftsordnung\index[idx]{Geschäftsordnung} nach \S~2 Abs.~6 SGO\index[paridx]{SGO!2@\S~2!6@Abs.~6} empfiehlt.
Auch etwa gemeinsam formulierte Standardsschreiben, die Zuständigkeiten für die Verwaltung der Dokumentation nach \S~14 SGO\index[paridx]{SGO!\S~14} und sämtliche sonstigen, nicht auf konkrete Verfahren oder Anrufungen bezogenen Tätigkeiten werden von dem gesamten Organ entschieden.
Nicht dazu gehört allerdings die Wahl des Vorsitzenden Richters.
Die Schiedsgerichtsordnung hat diese in \S~3 Abs.~1 Satz~2 SGO\index[paridx]{SGO!3@\S~3!1@Abs.~1} allein in die Hände der Richter gelegt, die Ersatzrichter haben hierbei weder aktives noch passives Wahlrecht.

\section{Das Schiedsgericht als Spruchkörper}
\label{Zusammensetzung:Spruchkoerper}
Das Schiedgericht im Sinne des entscheidenen Spruchkörpers\index[idx]{Spruchkörper|textbf} für konkrete Verfahren setzt sich dahingegen nach \SSS~3--5 SGO\index[paridx]{SGO!\S~3}\index[paridx]{SGO!\S~4}\index[paridx]{SGO!\S~5} zusammen. Dabei ist im Folgenden mit Schiedsgericht nicht mehr das Organ, sondern der Spruchkörper für eine konkrete Anrufung gemeint.

\subsection{Besetzung}
\label{Zusammensetzung:Spruchkoerper:Besetzung}
Dieser besteht standardmäßig aus den gewählten drei Richtern.
Sobald einer dieser Richter sein Amt gemäß \S~3 Abs.~7, 8 SGO\index[paridx]{SGO!3@\S~3!7@Abs.~7} verliert, rückt ein Ersatzrichter dauerhaft nach und wird Teil des standardmäßigen Spruchkörpers.
Zur Unterscheidung von Ersatzrichtern, die weiterhin diese Rolle haben, ist er ab diesem Zeitpunkt so zu behandeln, als wäre er ursprünglich als Richter gewählt worden.
Er erhält damit passives und aktives Wahlrecht bei der Wahl des Vorsitzenden Richters\index[idx]{Vorsitzender Richter}.
Dies ist einerseits schon durch das Wort \enquote{dauerhaft} nahegelegt, dass es einen rechtlichen Unterschied zum nur temporären Einrücken in ein einzelnes Verfahren nach \S~4 Abs.~4 SGO\index[paridx]{SGO!4@\S~4!4@Abs.~4} erlaubten erhöhten Ersatzrichterzahl sogar zu einer Situation kommen könne, in der kein verbleibendes Mitglied des Schiedsgerichts aktives oder passives Wahlrecht für die Wahl des Vorsitzenden Richters hätte.
Auf die Besonderheiten der Besetzung des Bundesschiedsgerichtes aufgrund der erhöhten Richterzahl und der Möglichkeit eines Kammersystems wird hier nicht eingegangen.

\subsection{Änderung der Besetzung}
\label{Zusammensetzung:Spruchkoerper:Aenderung}
Die Besetzung\index[idx]{Besetzung|textbf} des Gerichts darf nur in den dafür vorgesehenen Fällen verändert werden.
Dies liegt daran, dass Art.~101 Abs.~2 Satz~2 GG\index[paridx]{GG!Art.~101!Abs.~2 Satz~2} jedermann das Recht auf den \emph{gesetzlichen Richter} garantiert.
Der gesetzliche Richter ist allerdings nur der, der aufgrund des geltenden Rechts zum Klageeinreichungszeitpunkt dazu bestimmt war.
Dies schließt jede Modifikation durch Einzelfallbeschluss aus.
Einzig und allein die vom Gesetz vorgesehen Gründe können daher zu einer Besetzungsänderung führen.
Allen diesen Gründen ist gemeinsam, dass sie den Konflikt der Garantie des gesetzlichen Richters mit dem Recht auf ein faires, neutrales Verfahren und einen effektiven Rechsschutz auflösen.

Insgesamt gibt folgende Gründe für eine Besetzungsänderung:
\begin{enumerate}
\item Ausschluss wegen Untätigkeit eines einzelnen Richters, \S~4 Abs.~1 SGO.\index[paridx]{SGO!4@\S~4!1@Abs.~1}
\item Ausschluss wegen Krankheit, \S~4 Abs.~3 Satz~1 Alt.~1 SGO.\index[paridx]{SGO!4@\S~4!3@Abs.~3}
\item Ausschluss wegen Beurlaubung, \S~4 Abs.~3 Satz~1 Alt.~2 SGO.\index[paridx]{SGO!4@\S~4!3@Abs.~3}
\item Ausschluss wegen Besorgnis der Befangenheit, \S~5 Abs.~2 Satz 1 Alt.~1, Satz~2 SGO.\index[paridx]{SGO!5@\S~5!2@Abs.~2}
\item Ausschluss wegen satzungsmäßig unwiderlegbar vermuteter Befangenheit, \S~5 Abs.~1 SGO.\index[paridx]{SGO!5@\S~5!1@Abs.~1}
\end{enumerate}

\subsection{Beschlussbesetzung im für den Beschluss über einen Richterausschluss}
\label{Zusammensetzung:Spruchkoerper:Besetzungausschlussbeschluss}
Die Satzung äußert sich nicht konkret zu der Frage, wer an der Entscheidung über den Ausschluss eines Richters mitabstimmen darf.
\S~4 Abs.~4 Satz~2 Alt.~2 SGO\index[paridx]{SGO!4@\S~4!4@Abs.~4} sagt lediglich aus, dass die Beschlussfähigkeit auch mit zwei Richtern gegeben ist.
Aus dem Grundsatz, dass niemand Richter in eigener Sache sein darf,\footnote{lat.: \emph{nemo iudex in sua causa}.} folgt schon, dass der betroffene Richter selbst nicht mitentscheiden darf. Für die Fälle des \S~5 SGO\index[paridx]{SGO!\S~5} ist das auch ausdrücklich in der Schiedsgerichtsordnung niedergelegt, aber auch in den anderen Fällen muss dieser Grundsatz gelten.
In Betracht kommen daher zwei denkbare Besetzungen: Die verbleibenden Richter der bisherigen Spruchkörperbesetzung entscheiden über den Ausschluss des Richters oder aber \S~4 Abs.~2 Satz~1 SGO\index[paridx]{SGO!4@\S~4!2@Abs.~2} nicht nur die schon per Beschluss ausgeschlossenen Richter umfasst, sondern auch die, über die konkret beschieden werden soll.
Die sonstigen Regeln, insbesondere über die Beschlussfähigkeit, sagen dazu nichts.
Allerdings deutet schon der Wortlaut \Zitat{Notbesetzung} an, dass es nicht der Normalfall sein soll.
Der Satzung lassen sich keine Indizien für absichtliche Redundanz der Betonung des Sonderfalls entnehmen.
Daher ist rückt schon für den Beschluss über den Richterausschluss der in der Rangfolge nächste Ersatzrichter temporär ein und entscheidet über den Ausschluss mit.
Dass dies geboten ist, ergibt sich auch schon zwingend aus der Möglichkeit, das zwei Richter nicht erreichbar sein könnten.
In diesem Fall wäre der verbleibende Richter allein nicht mehr als Spruchkörper beschlussfähig.
Ohne den Beschluss über den Ausschluss der nicht erreichbaren Richter könnte dann aber kein Ersatzrichter nachrücken, um eine beschlussfähige Notbesetzung zu erzeugen.
Daher müsste bei einer derartigen Verfahrensweise der verbleibende Richter sich bzw. den Spruchkörper für handlungsunfähig erklären und um Verfahrensverweisung bitten, obwohl doch gerade das Institut der Ersatzrichter unnötige vorschnelle Verfahrensverweisungen verhindern soll.
Auch lässt sich der Wortlaut \enquote{ein befangener oder ausgeschlossener Richter} dahingehend deuten, dass \enquote{ausgeschlossener Richter} diejenigen umfasst, die schon per Beschluss ausgeschlossen wurden und \enquote{befangener […] Richter} diejenigen umfasst, die selbst in einer Entscheidung tatsächlich und unzweifelhaft befangen sind. Letzteres würde aufgrund des Grundsatzes, dass niemand Richter in eigener Sache sein darf, wohl unzweifelhaft zutreffen, sodass auch eine derartige Auslegung des \S~4 Abs.~2 mit dem Wortlaut vereinbar ist.
Daher rückt der in der Reihenfolge nächste Ersatzrichter schon für den Beschluss über den Ausschluss eines Richters ein.

Diese Vorgehensweise, obschon nie ausführlich begründet, entspricht auch der jahrelangen Praxis des Bundesschiedsgerichtes.\footnote{MISSING: Entscheidungen}

\subsection{Änderung durch Ausschluss eines Richters wegen Untätigkeit}
\label{Zusammensetzung:Spruchkoerper:Untaetigkeit}
Der Ausschluss eines Richters wegen Untätigkeit kann dann beschlossen werden, wenn ein Richter auch auf Nachfrage nicht mitarbeitet und so die Beschlussfähigkeit des Gerichts in dem Verfahren gefährdet.
\S~4 Abs.~1 SGO\index[paridx]{SGO!4@\S~4!1@Abs.~1} fordert dafür, dass der Richter bereits an Zusammentreffen oder Beratungen in einem anderen Format, etwa per E-Mail-Verkehr im Bezug auf eine konkretes Verfahren nicht beteiligt hat, die übrigen Richter diesen Richter konkret zur Mitarbeit ermahnt haben und ihm dazu eine Frist von mindestens 13 Tagen gegeben haben, sich doch noch einzubringen.
Wenn der Richter trotz alldem nicht reagiert oder sich trotzdem aktiv weigert, mitzuarbeiten und so das Verfahren bzw. Beschlussfassung blockiert, damit die Verfahrensdauer ohne Grund die Länge treibt und den Rechtsschutzanspruch der Streitparteien vereitelt, soll dies nicht zu Lasten der Streitparteien gehen.
In diesem Fall blockiert der gesetzliche Richter gerade den Anspruch auf effektiven Rechtsschutz und entzieht so aktiv den gesetzlichen Richter, der gerade nicht entzogen werden darf.
Daher ist in diesem Fall das verfassungsrechtliche Verbot des Entzugs des gesetzlichen Richters nicht im Konflikt mit der Satzungsbestimmung.
Die Entscheidung über den Ausschluss ist eine Ermessensentscheidung des Spruchkörpers und keine zwingende Folge.
Sie ist dann geboten, wenn anders kein Fortschritt im Verfahren erzielt werden kann oder weiteres Abwarten die Einhaltung der Verfahrensmaximalsolldauer nach \S~12 Abs.~1 SGO\index[paridx]{SGO!12@\S~12!1@Abs.~1} gefährdet.

\subsection{Änderung durch Ausschluss eines Richters wegen Krankheit}
\label{Zusammensetzung:Spruchkoerper:Krankheit}
Der Ausschluss eines Richters wegen Krankheit gemäß \S~4 Abs.~3 Satz~1 Alt.~1 SGO\index[paridx]{SGO!4@\S~4!3@Abs.~3} ist etwas anders geartet.
Hier kommt regelmäßig eine Ablehnung schon gar nicht in Betracht.
Wenn ein Richter sich dem Gericht gegenüber krank meldet, ist dies hinzunehmen und den Parteien mitzuteilen.
Die Krankmeldung ist dabei zu den Akten zu nehmen.
Es ist somit regelmäßig gerade keinen Beschluss über den Ausschluss wegen Krankheit zu treffen.
Der Ausschluss ist auch insofern anders, als dass er für alle aktuell laufenden Verfahren gilt und abgesehen von der Mitteilung an die Parteien nicht für jedes Verfahrens ein eigenes Prozedere erfordert.
Die Meldung der Krankheit ist daher eine temporäre Amtsniederlegung.

Allerdings ist die Schiedsgerichtsordnung an dieser Stelle hochproblematisch, da sie von einem lediglich temporären Ausscheiden und der Möglichkeit der Rückkehr in das Verfahren ausgeht.
Dies würde Prozesstaktiken ermöglichen, die defacto einer parteigetriebenen Richterauswahl nahekommen, indem von Seite einer Streitpartei versucht wird, das Verfahren zu beschleunigen oder zu verzögern und so mit oder ohne den kranken Richter zum Abschluss zu bringen. 

Daher muss davon ausgegangen werden, dass die entsprechende Satzungsbestimmung über die Rückkehr eines ehemals kranken und deswegen ausgeschiedenen Richters in laufende Verfahren wegen Verstoß gegen das Verbot des Entzugs des gesetzlichen Richters unwirksam ist. Der gesetzliche Richter nach Art.~101 Abs.~1 Satz~2~GG\index[paridx]{GG!101@Art.~101!1@Abs.~1} soll der Gefahr vorbeugen, daß die Justiz durch eine Manipulierung der rechtsprechenden Organe sachfremden Einflüssen ausgesetzt wird, insbesondere daß im Einzelfall durch die Auswahl der zur Entscheidung berufenen Richter ad hoc das Ergebnis der Entscheidung beeinflußt wird, gleichgültig, von welcher Seite die Manipulierung ausgeht.\footnote{So schon \cite[299]{BVerfGE17294}.}

Auch angesichts typischer Verfahrensdauern im Rahmen der erlaubten Verfahrensmaximalsolldauer ist jeder Zeitraum, der überhaupt zu einem temporären Ausscheiden eines Richters führt, schon so relevant, dass das Verfahren vom eingerückten Ersatzrichter maßgeblich mitbeeinflusst werden konnte.
Daher muss auch dieser weiterhin dem Spruchkörper angehören.

Deswegen kann ein einmal ersetzter Richter nicht wieder in ein Verfahren zurückkehren, sobald den Parteien sein Ausscheiden und das Eintreten eines Ersatzrichters mitgeteilt wurde.\footnote{So auch etwa \cite[Eschelbach][\S~15 Rn~31]{BVerfGGMitarbeiterKommentar} entsprechend für das Bundesverfassungsgericht. Dieser geht davon aus, dass ein Rückkehr in die Richterolle im Spruchkörper für den Vorsitzenden nur dann möglich ist, wenn er lediglich in seiner Rolle als Vorsitzener etwa wegen Heiserkeit, nicht aber in der Rolle als Richter gänzlich wegfiel.}

\subsection{Änderung durch Ausschluss eines Richters wegen Beurlaubung}
\label{Zusammensetzung:Spruchkoerper:Urlaub}
In die selbe Kategorie wie der Ausschluss eines Richters wegen Krankheit fällt der Ausschluss eines Richters wegen Beurlaubung gemäß \S~4 Abs.~3 Satz~1 Alt.~2 SGO.\index[paridx]{SGO!4@\S~4!3@Abs.~3}
Hier gelten aber die Bedenken über die Prozessmanipulation in Richtung einer effektiv parteigetriebenen Richterauswahl umso mehr, da das Ende der voraussichtlichen Abwesenheit noch besser vorhersehbar ist.
Daher muss hier das gleiche gelten: Die Regelung ist bezüglich des Wiedereinstritts eines beurlaubten Richters in das Verfahren unwirksam und ein Wiedereinstritt ausgeschlossen, sobald den Parteien sein Ausscheiden und das Eintreten eines Ersatzrichters mitgeteilt wurde.

\subsection{Änderung durch Ausschluss eines Richters wegen Besorgnis der Befangenheit}
\label{Zusammensetzung:Spruchkoerper:Befangenheitsbesorgnis}
Das Parteiengesetz fordert in \S~14 Abs.~4\index[paridx]{PartG!14@\S~14!4@Abs.~4} lediglich, dass jede Streitpartei Richter wegen Befangenheit ablehnen kann.
Auch wenn hier der Wortlaut des Parteiengesetzes andeutet, dass es lediglich bei tatsächlicher Befangenheit ein solches Ablehnungsrecht geben muss, liegt es viel näher, die Anforderung soweit auszulegen, dass die Schiedsgerichtsordnung schon für die Besorgnis der Befangenheit ein Ablehnungsrecht garantieren muss.\footnote{\cites[Wißmann][\S~14 Rn~32]{kersten2007parteiengesetz}[Lenski][\S~14 Rn~23]{lenski2011parteiengesetz}}
Dieser akademische Streit um die Auslegung des \S~14 PartG\index[paridx]{PartG!\S~14} ist jedoch für Verfahren in der Schiedsgerichtsbarkeit der Piratenpartei Deutschland nicht relevant.
Dies Schiedsgerichtsordnung garantiert in \S~5 Abs.~2 SGO\index[paridx]{SGO!5@\S~5!2@Abs.~2} ein Recht auf Ablehnung eines jeden Richters bei Besorgnis seiner Befangenheit.


\subsubsection{Prozessusales}
\label{Zusammensetzung:Spruchkoerper:Befangenheitsbesorgnis:Prozessuales}
Einen Antrag auf Ablehnung eines Richters wegen Besorgnis eines Richters kann nur eine Verfahrenspartei stellen, nicht aber der Richter selber.

Der Antrag muss immer eine Begründung enthalten, \S~5 Abs.~3 Satz~1 SGO\index[paridx]{SGO!5@\S~5!3@Abs.~3}, und sich gegen einene einzelnen Richter richten, andernfalls ist er unzulässig.\footnote{\cites[S.~1]{BSG201305062BefangenheitI}{BGHIIARZ101}.}
Eine Ablehnung des ganzen Gerichts auf einmal ist also nicht möglich, allerdings können durchaus mehrere oder sogar alle Richter eines Gerichts jeweils einzeln nacheieinader abgelehnt werden.
Entscheidend für die Bescheidungsreihenfolge ist dann die Reihenfolge der Antragstellung.
In besonderen Fällen ist sogar der Verweis auf die gleiche Begründung für mehrere Richter zulässig, nämlich immer dann, wenn vorgetragen wird, das die die Besorgnis der Befangenheit tragende Tatsache auf mehrere Richter jeweils individuell gleichzeitig zutrifft.
In allen anderen Fällen ist der Ablehnungsantrag aber unzulässig und muss als solcher direkt abgewiesen werden.

Ein unzulässiger Ablehnungsantrag wird entgegen \S~5 Abs.~5 SGO\index[paridx]{SGO!5@\S~5!5@Abs.~5} in originaler, ungeänderter Besetzung beschieden, da schon gar kein formal korrekter Befangenheitsantrag im Sinne des \S~5 Abs.~5 SGO\index[paridx]{SGO!5@\S~5!5@Abs.~5} vorliegt.\footnote{\cites[S.~1]{BSG201305062BefangenheitI}{BSG201305062BefangenheitII}.}

Sobald ein zulässiger Ablehnungantrag gegen einen Richter gestellt ist, darf dieser nicht mehr an verfahrensleitenden Entscheidungen teilnehmen, bis der Antrag abgelehnt wurde oder angnommen wurde, \S~5 Abs.~4 SGO.\index[paridx]{SGO!5@\S~5!4@Abs.~4}
Im letzten Fall ist mit Beschluss aus dem Verfahren ausgeschieden und wird durch den nächsten Ersatzrichter ersetzt, \S~4 Abs.~2 SGO.\index[paridx]{SGO!4@\S~4!2@Abs.~2}

Vor der Entscheidung muss der Richter selbst sich zur Begründung des Befangenheitsgesuches äußern.
Dabei sollte sie diese dienstliche Stellungnahme auf die Tatsachen, die die Besorgnis der Befangenheit stützen sollen, beschränken und die rechtliche Wertung den Richterkollegen überalssen werden.
Beiden Parteien daraufhin noch eine Gelegenheit zur Stellungnahme zu geben, \S~5 Abs.~3 Satz~2 SGO\index[paridx]{SGO!5@\S~5!3@Abs.~3}, danach kann in geänderter Besetzung nach \SSS~5 Abs.~5, 4 Abs.~2 SGO\index[paridx]{SGO!5@\S~5!5@Abs.~5}\index[paridx]{SGO!4@\S~4!2@Abs.~2} über den Befangenheitsantrag beschieden werden.

\subsubsection{Tatbestand der Besorgnis der Befangenheit}
\label{Zusammensetzung:Spruchkoerper:Befangenheitsbesorgnis:Tatbestand}
Für den Beschluss, ob ein Richter im Verfahren bleibt, oder wegen Besorgnis der Befangenheit aus dem Verfahren ausscheidet, muss entschieden werden, ob ein neutraler, objektiver, fiktiver Dritter, der in Kenntnis der Tatsachen zu entscheiden hätte, einen gerechtfertigten Grund hätte, an dem Unparteilichkeit des Richters zu zweifeln, \S~5 Abs.~2 Satz~2.\index[paridx]{SGO!5@\S~5!2@Abs.~2}
Dies ist noch nicht erfüllt, wenn der Richter einer speziellen parteiinternen Organisation oder einem Flügel angehört.\footnote{\cite[S.~6]{BSG115HSBefangeheitIII}.}
Ebenfalls genügt dafür nicht, dass der Richter sich bereits früher mal über eine verfahrensrelevante Rechtsfrage geäußert hat,\footnote{\cite{BGHXIZR38801}.} die Partei muss dann damit leben, dass ihr gesetzlicher Richter in einem rechtlichen Meinungsstreit eine gewisse Position vertritt.
Insbesondere gilt dies in einer Partei, die ihre Richter aus dem Pool aktiver Mitglieder rektrutiert.\footnote{\cites[Lenski][\S~14 Rn~15]{lenski2011parteiengesetz}.}
Erst recht liegt keine Besorgnis der Befangenheit vor, weil eine Partei davon ausgeht, dass ein Richter anders als von ihr gewünscht entscheiden wird.
Dies auch dann nicht, wenn das Gericht bereits eine entsprechende vorläufige Rechtsauffassung geäußert hat.\footnote{\cites[S.~6]{BSG20131028}.}
Unerwünschte Entscheidungen sind bei Gerichten der Regelfall.\footnote{\cites[S.~2]{BSG201305062BefangenheitI}.}

\subsubsection{Rechtsmissbräuchlichkeit der Ablehnung}
\label{Zusammensetzung:Spruchkoerper:Befangenheitsbesorgnis:Rechtsmissbrauch}
Es kann auch sein, dass ein Antrag auf Richterablehnung eigentlich zulässig wäre, aber im speziellen Fall rechtsmissbräuchlich ist und daher als unzulässig abzulehnen ist.
Dies ist eine wertende Entscheidung und nicht eine bloße Formalentscheidung und muss daher in der geänderten Besetzung nach \SSS~5 Abs.~5, 4 Abs.~2 SGO\index[paridx]{SGO!5@\S~5!5@Abs.~5}\index[paridx]{SGO!4@\S~4!2@Abs.~2} beschieden werden.
Allerdings kann auf die Stellungnahme und die Anhörung der Parteien verzichtet werden, wenn die geänderte Besetzung direkt zur Überzeugung gelangt, dass der Antrag rechtsmissbräuchlich ist, da dann kein zulässiger Ablehnungsantrag vorliegt.
Im Zweifel gilt allerdings immer: \Zitat{Eine Anhörung zu viel schadet nicht, eine Anhörung zu wenig kann zur Aufhebung des Urteils führen.}

Rechtsmissbräuchlich ist der Antrag auf Richterablehnung, wenn er nur gestellt wird, um das Verfahren aufzuhalten,\footnote{\cites[S.~8]{BSG2815HS}.} das Gericht zu überlasten oder mit offensichtlich nicht aussichtsreichen Befangenheitsanträgen zu bombardieren,\footnote{\cites[S.~4]{BSG201305062BefangenheitII}.} oder einen nicht genehmen Richter allein wegen seiner Spruchtätigkeit oder Rechtsüberzeugung abzulehnen.\footnote{\cites[S.~5]{BSG115HSBefangeheitIV}{OLGNaumburg3WF7609}.}

\subsection{Änderung durch Ausschluss eines Richters wegen satzungsmäßig unwiderlegbar vermuteter Befangenheit}
\label{Zusammensetzung:Spruchkoerper:Befangenheitsvermutung}
Ein Richter kann durch eine Streitpartei nach \S~5 Abs.~2 Satz~1 Alt.~2, Abs.~1 SGO\index[paridx]{SGO!5@\S~5!2@Abs.~2} abgelehnt werden.
Die Möglichkeit, dass die Streitparteien einen solchen Antrag stellen können, darf aber nicht über die Natur des \S~5 Abs.~1 SGO\index[paridx]{SGO!5@\S~5!1@Abs.~1} täuschen.
Es handelt sich hier um eine Vorschrift, die von Amts wegen von jedem Gericht gerpüft werden muss, bevor ein Richter an einem Verfahren teilnehmen kann.\footnote{So auch schon \cite{BSGPP100127862}, daher die dortige Vorabprüfung.} \S~5 Abs.~1 Satz~1 SGO\index[paridx]{SGO!5@\S~5!1@Abs.~1} stellt verschiedene objektive Bedingungen auf, die ohne wertenden Ermessenspielraum verbieten, dass ein Richter sein Amt in einem Verfahren ausübt.

\subsubsection{Prozessual Grundlegendes}
\label{Zusammensetzung:Spruchkoerper:Befangenheitsvermutung:Prozessuales}
\S~5 Abs.~1 Satz~2 SGO\index[paridx]{SGO!5@\S~5!1@Abs.~1} fordert lediglich, dass ohne den betroffenen Richter beschieden wird.
Insofern rückt der in der Rangfolge nächste Ersatzrichter für diese Entscheidung in den Spruchkörper ein und die Parteien sind davon in Kenntnis zu setzen, \S~4 Abs.~2 S.~2 SGO.\index[paridx]{SGO!4@\S~4!2@Abs.~2}
Weitere Vorschriften macht die SGO nicht für den Fall, dass die von Amts wegen zu treffende Entscheidung, ob ein Richter von der Ausübung seines Amtes ausgeschlossen ist, beschieden wird.
Anders könnte es aussehen, wenn eine Streitpartei die Entscheidung darüber nach \S~5 Abs.~2 Satz~1 Alt.~2 SGO\index[paridx]{SGO!5@\S~5!2@Abs.~2} beantragt hat.
Denn dem Wortlaut der Schiedsgerichtsordnung nach treffen die prozessualen Vorschriften des \S~5 Abs.~2 ff. SGO\index[paridx]{SGO!5@\S~5!2@Abs.~2}
Demnach wäre das Verfahren unterschiedlich, je nachdem, ob es durch einen Antrag einer Streitpartei ausgelöst wird oder von den Richtern selbst.
Es gibt es allerdings keinen Grund, der das sachlich rechtfertigen würde.
Daher müssen die Prozessvorschriften des \S~5 Abs.~2 ff. SGO\index[paridx]{SGO!5@\S~5!2@Abs.~2} einzeln geprüft werden, ob sie auf die Eigenheiten der Richterablehnung wegen satzungsmäßigem Ausschlussgrund anwendbar sind oder nicht.
Für den Fall, dass sie anwendbar sind, müssen sie in beiden denkbaren Verfahrenswegen angewendet werden.

\subsubsection{Anwendung der richterlichen Anzeigepflicht}
\label{Zusammensetzung:Spruchkoerper:Befangenheitsvermutung:Anzeigepflicht}
Richter sind gemäß \S~5 Abs.~2 Satz~3 SGO\index[paridx]{SGO!5@\S~5!2@Abs.~2} verpflichtet, jeden Umstand dem Spruchkörper und den Streitparteien anzuzeigen, die einen Antrag auf Ablehnung begründen könnten.
Der Konjunktiv der Formulierung gebietet daher eine solche Anzeige auch schon, wenn der Richter überzeugt ist, dass kein Tatbestand des \S~5 Abs.~1 Satz~1 SGO\index[paridx]{SGO!5@\S~5!1@Abs.~1} erfüllt ist, eine solche Erfüllung aber denkbar wäre.
Diese Anzeigepflicht muss gerade und insbesondere für die abschließende Liste der klar umrissenen Tatbestände des \S~5 Abs.~1 Satz~1 SGO\index[paridx]{SGO!5@\S~5!1@Abs.~1} gelten.
Die Anzeige löst auch sofort eine Bescheidungspflicht durch die anderen Richter nach \S~5 Abs.~1 Satz~2 SGO\index[paridx]{SGO!5@\S~5!1@Abs.~1} aus, eines Antrags einer Partei bedarf es dann nicht mehr.

\subsubsection{Anwendung der Präklusionsregelung}
\label{Zusammensetzung:Spruchkoerper:Befangenheitsvermutung:Praeklusion}
\S~5 Abs.~2 Satz~4 SGO\index[paridx]{SGO!5@\S~5!2@Abs.~2} präkludiert Streitparteien davon, einen Richter abzulehnen, wenn der Grund für die Ablehnung schon ihr bekannt war und sie sich dennoch auf eine Verhandlung mit dem Richter eingelassen hat.
Auf den satzungsmäßigem Ausschluss angewendet würde dies bedeuten, dass ein Richter, dem die Satzung objektiv und ohne Wertungsspielraum die Ausübung seines Amtes im Verfahren versagt ist, dennoch teilnehmen kann, wenn über den vorgetragenen Ausschlussgrund wegen Präklusion nicht mehr entschieden werden müsste.
Dies kann jedoch keine gewollte Rechtsfolge sein, auch da der Richter diesen Grund schon hätte aufgrund seiner Anzeigepflicht mitteilen hätte müssen.
Dass der Ausschlussgrund für den Richter nicht erkennbar war und die Streitpartei daher prozesstaktisch Wissen zurückgehalten hat, ist angesichts der klaren und abschließenden Liste der möglichen Tatbestände nicht vorstellbar.
Auch müsste das Gericht, sobald es Kenntnis von der Möglichkeit eines satzungsmäßigen Ausschlussgrundes nach \S~5 Abs.~1 Satz~1 SGO\index[paridx]{SGO!5@\S~5!1@Abs.~1} hat, von Amts wegen über den Ausschluss entscheiden.
Daher würde die Anwendung der Präklusionsregelung zu der absurden Situation führen, dass der Antrag der Streitpartei abgelehnt werden müsste, aber inhaltlich dennoch beschieden werden müsste.
Dies kann keine gewollte Folge sein.
Daher ist die Präklusionsregelung nicht anwendbar.

\subsubsection{Anwendung des Begründungsgebotes}
\label{Zusammensetzung:Spruchkoerper:Befangenheitsvermutung:Begruendungsgebot}
Das Begründungsgebot aus \S~5 Abs.~3 Satz~1 SGO\index[paridx]{SGO!5@\S~5!3@Abs.~3} ist unproblematisch anwendbar.
Wenn eine Partei der Ansicht ist, dass einer der Tatbestände aus \S~5 Abs.~1 Satz~1 SGO\index[paridx]{SGO!5@\S~5!1@Abs.~1} zutrifft, muss sie auch ausführen, welcher das ist und warum er zutrifft.
Andernfalls könnte mit derartigen Anträgen das Gericht bis zum Rande der Arbeitsunfähigkeit blockiert werden, wenn die nicht betroffenen Richter im Wege der Amtsermittlung jeder Behauptung ohne den Ansatz einer Begründung nachgehen müssten.

\subsubsection{Anwendung der Regelung zur dienstlichen Stellungnahme und Parteistellungnahme}
\label{Zusammensetzung:Spruchkoerper:Befangenheitsvermutung:Stellungnahme}
Die Regelung über die dienstliche Stellungnahme und die abschließenden Parteistellungnahme nach \S~5 Abs.~2 Satz~2, 3 SGO.\index[paridx]{SGO!5@\S~5!2@Abs.~2} dient der Aufklärung des Tatbestandes und der Erfüllung des grundrechtsgleichen Rechts auf rechtlichen Gehör und sind daher anzuwenden.
Ebenfalls dient sie dazu, den Parteien ihre Stellungnahme jeweils erst zu ermöglichen.
Diese wiederrum ist aus dem von der Verfassung garantierten rechtlichen Gehörs nach Art.~103 Abs.~1 GG\index[paridx]{GG!103@Art.~103!1@Abs.~1} geboten.

\subsubsection{Anwendung der Besetzungsregelung}
\label{Zusammensetzung:Spruchkoerper:Befangenheitsvermutung:Besetzung}
Die Besetzungregelung wäre schon aufgrund der allgemeinen juristischen Prozessgrundsätze geboten (Siehe schon \ref{Zusammensetzung:Spruchkoerper:Besetzungausschlussbeschluss}), allerdings liegt die analoge Anwendung des \S~5 Abs.~1 Satz~2 SGO\index[paridx]{SGO!5@\S~5!1@Abs.~1} näher als die Anwendung allgemeiner juristischer Grundsätze, daher ist dieser hier die Grundlage.

\subsubsection{Anwendung der Rechtsmittel}
\label{Zusammensetzung:Spruchkoerper:Befangenheitsvermutung:Rechtsmittel}
Gegen die Anwendung der Rechtsmittelregelungen in \S~5 Abs.~6 SGO\index[paridx]{SGO!5@\S~5!6@Abs.~6} gibt es keine Einwände. Im Gegenteil, es ist nicht einzusehen, warum das der Wahrung einer einheitlichen Auslegung und Rechtsprechung im Falle eines satzungsmäßigen Richterausschlusses nicht anzuwenden sein sollte.
Eine derartige Verkürzung des Rechtsschutzes erscheint wegen der gleichlaufenden Interessen in beiden Fällen nicht angemessen.

\subsubsection{Grundegendes zu den Tatbeständen}
\label{Zusammensetzung:Spruchkoerper:Befangenheitsvermutung:Tatbestandsgrundsaetze}
Jeder der Tatbestände des \S~5 Abs.~1 Satz~1 SGO\index[paridx]{SGO!5@\S~5!1@Abs.~1} stellt eine satzungsmäßige, unwiderlegliche Vermutung der Befangenheit des Betroffenen Richters auf.
Es kommt somit für die einzelnen Tatbestände nicht mehr darauf an, ob ein neutraler objektiver, prozessfremder Beobachter nach verständiger Würdigung der Umstände berechtigte Zweifel an der Unparteilichkeit eines Richters haben kann.
Sobald der Tatbestand erfüllt ist, ist ein Richter von der Mitwirkung am Verfahren auszuschließen.

Die Regelung ist an dabei die staatlichen Prozessordnungen angelehnt.\footnote{Vgl. etwa \S~41 ZPO\index[paridx]{ZPO!\S~41},  \S~22 StPO\index[paridx]{StPO!\S~22}\nomenclature{StPO}{Strafprozessordnung}, \S~54 VwGO\index[paridx]{VwGO!\S~54}\nomenclature{VwGO}{Verwaltungsgerichtsordnung}, \S~51 FGO\index[paridx]{FGO!\S~51}\nomenclature{FGO}{Finanzgerichtsordnung}, \S~60 SGG\index[paridx]{SGG!\S~60}\nomenclature{SGG}{Sozialgerichtsgesetz} und \S~19 BVerfGG\index[paridx]{BVerfGG!\S~19}\nomenclature{BVerfGG}{Bundesverfassungsgerichtsgesetz}.}
Diese Regelungen kommen aufgrund der vom Satzungsgeber getätigten Regelung in \S~5 Abs.~1 SGO\index[paridx]{SGO!5@\S~5!1@Abs.~1} nicht in Betracht für eine analoge Anwendung.
\S~14 Abs.~4 PartG\index[paridx]{PartG!14@\S~14!4@Abs.~4} fordert explizit vom Satzungsgeber, dass er die Richterablehnung selbst regeln muss.
Damit hat der Gesetzgeber die Richterablehnung komplett in Hände des Satzungsgebers gelegt.
Daher ist die Richterablehnung tatsächlich ein Gebiet der SGO, in der von einer vollständigen Regelung des Satzungsgebers auszugehen ist und eine analoge Anwendung staatlicher Prozessordnungen grundsätzlich nicht in Betracht kommen.
Selbt jede konkretisierende Zuhilfenahme staatlicher Prozessordnungen zur Auslegung der Befangenheitsregelungen der SGO ist kritisch zu sehen, da der entsprechende Übernahmewille vom Satzungsgeber in der Satzung zum Ausdruck gebracht sein müsste.

Die Tatbestände des \S~5 Abs.~1 SGO\index[paridx]{SGO!5@\S~5!1@Abs.~1} gliedern sich in drei Gruppen (Nr.~1--5, Nr.~6--7, Nr.~8).

Die erste Gruppe von Tatbeständen (Nr.~1 -- Nr.~5) erfassen Konstellationen, in denen die Satzung dem Richter aufgrund seiner persönlichen Nähe zu den \textbf{Verfahrensparteien} nicht zutraut, einen Fall neutral bewerten zu können, ohne dass der Entscheidung der Makel dess Misstrauens der Unparteilichkeit des Richters anhaftet.
Die Gruppe lässt sich nochmal in zwei Untergruppen unterteilen:
Die erste Untergruppe (Nr.~1 -- Nr.~3) begründet die Nähe in der personenstandsrechtlichen Nähe zu den Verfahrensparteien und dem darin vermuteten Interessenkonflikt.
Die zweite Untergruppe (Nr.~4 -- Nr.~5) begründet die Nähe durch die Möglichkeit der Einflussnahme auf die Entscheidungsfindung in der konkreten Sache sowie generell.
Auch die Möglichkeit, etwa schon Prozessstrategien oder anderes Vorwissen zum Verfahrensgegenstand aus Sicht der Partei zu kennen, dürfte hier eine Rolle gespielt haben für den Satzungsgeber.
Insofern ist die Untergruppe auch verwandt mit der zweiten Gruppe.

Die zweite Gruppe von Tatbeständen (Nr.~6 -- Nr.~7) erfassen Konstellationen, in denden die Satzung dem Richter aufgrund seiner persönlichen Nähe zum \textbf{Verfahrensgegenstand} nicht zutraut, einen Fall neutral bewerten zu können, ohne dass der Entscheidung der Makel dess Misstrauens der Unparteilichkeit des Richters anhaftet.

Die dritte Gruppe umfasst nur noch einen Tatbestand (Nr.~8), und somit die Konstellation, in denden die Satzung dem Richter aufgrund seiner persönlichen Nähe zum \textbf{Vorverfahren} nicht zutraut, einen Fall neutral bewerten zu können, ohne dass der Entscheidung der Makel dess Misstrauens der Unparteilichkeit des Richters anhaftet.
An sich ist das sehr nah an der zweiten Gruppe, aber doch nochmal verschieden, da nicht das gerichtliche Verfahren selbst betroffen ist.

\subsubsection{Tatbestand \S~5 Abs.~1 Satz~1 Nr.~1 SGO\index[paridx]{SGO!5@\S~5!1@Abs.~1}}
\label{Zusammensetzung:Spruchkoerper:Befangenheitsvermutung:Nr1}
\S~5 Abs.~1 Satz~1 Nr.~1 SGO\index[paridx]{SGO!5@\S~5!1@Abs.~1}
Es versteht sich von selbst, dass ein Richter in Verfahren, denen er selbst Verfahrenspartei ist, nicht mitentscheiden soll.
Dies war auch schon vor Einführung des Tatbestandes aus dem allgemeinen Rechtsgrundsatz heraus automatisch Praxis der Parteirechtssprechung,\footnote{So hat das damalige Bundesschiedsgericht es in \cite{BSG3014HS}, \cite{BSG4414HS} und \cite{BSG3215HS} etwa schon gar nicht mehr für nötig gehalten, zur Erläutern, warum der zur ordentlichen Besetzung gehörende Richter Florian Zumkeller-Quast an der tenorierten Entscheidung nicht teilnahm.} wurde aber nun explizit in der Satzung festgehalten.

\subsubsection{Tatbestand \S~5 Abs.~1 Satz~1 Nr.~2 SGO\index[paridx]{SGO!5@\S~5!1@Abs.~1}}
\label{Zusammensetzung:Spruchkoerper:Befangenheitsvermutung:Nr2}
\S~5 Abs.~1 Satz~1 Nr.~2 SGO\index[paridx]{SGO!5@\S~5!1@Abs.~1} vermutet unwiderleglich, dass ein Richter, dessen Ehe- oder Lebenspartner Verfahrensrenspartei ist, befangen ist.
Lebenspartner ist hier als Rechtsbegriff nach Lebenspartnerschaftsgesetz zu verstehen.
Das ergibt sich einerseits aus der Alternativnennung zur Ehepartnerschaft und ihrer Ähnlichkeit als nicht nur soziale, sondern auch rechtlich-wirschaftliche Verknüpfung, andererseits aus Halbsatz 2, der festhält, dass der Ausschluss auch gilt bei einer nicht mehr fortbestehenden Lebenspartnerschaft, da daraus folgt, dass es sich um ein Rechtsverhältnis und kein reines Sozialverhältnis handeln muss.
Es wäre einem Richter, der verpflichtet ist, das Erfülltsein des Tatbestandes nach \S~5 Abs.~2 Satz~3 SGO\index[paridx]{SGO!5@\S~5!2@Abs.~2} offenzulegen, auch nicht zuzumuten, jegliche frühere Beziehung in seinem Leben in einem Verfahren offenzulegen, wenn keinerlei sonstiger Bezug zum konkreten Verfahren besteht.
Auch würde der Satzung offensichtlich eine Definition fehlen, wann eine Lebenspartnerschaft beginnt und endet.
Ein solch unbestimmter Begriff wäre angesichts der eingriffsintensiven Rechtsfolge der absolut unwiderleglichen Befangenheitsvermutung schon eng auszulegen.
All dies spricht letzliche dafür, dass der Satzungsgeber hier keine länger andauernde Beziehung gemeint hat, sondern nur die verrechtlichte Lebenspartnerschaft nach Lebenspartnerschaftsgesetz.

\subsubsection{Tatbestand \S~5 Abs.~1 Satz~1 Nr.~3 SGO\index[paridx]{SGO!5@\S~5!1@Abs.~1}}
\label{Zusammensetzung:Spruchkoerper:Befangenheitsvermutung:Nr3}
\S~5 Abs.~1 Satz~1 Nr.~3 SGO\index[paridx]{SGO!5@\S~5!1@Abs.~1} Verfahrenspartei ist, befangen ist.
Dabei kommt es der Satzung nicht darauf an, ob die Verwandschaft beendet wurde, etwa durch Vaterschaftsanfechtung oder Adoption.
Entsprechend müssen Richter ihre Verwandschaft zu Verfahrensparteien generell offenlegen.
Die Satzung macht im Gegensatz zu staatlichen Prozessordnungen vom Wortlaut her keine Vorraussetzungen an den Grad der Verwandschaft oder Verschwägerung, sodass rein vom Wortlaut her auch sehr entfernte Verwandschafts- und Verschwägerungsgrade eine unwiderlegliche Befangenheitsvermutung begründen würden.
Daher gebietet der gesetzliche Richter, dass eine Grenze spätestens dort zu ziehen ist, wo der Grad so entfernt ist, dass eine Befangenheitsvermutung im allgemeinen nicht mehr durch die bloße Verwandschaft getragen werden kann.
Bisher gibt es in der Partei keine Rechtsprechung zu diesem Tatbestand.
Hier bietet sich eine enge Wortlautauslegung an, so dass lediglich eine Verwandtschaft in gerade Linie sowie eine Verschwägerung ersten Grades den Tatbestand erfüllen.
Dies führt zwar dazu, dass schon Geschwister den Tatbestand nicht mehr erfüllen würden, allerdings lässt sich dies immernoch durch die Möglichkeit der Richterablehnung wegen Besorgnis der Befangenheit für die Fälle, in denen diese tatsächlich notwendig ist, abdecken, während jede andere Grenzziehung innerhalb der Verwandsschaftsgrade willkürlich wäre, ohne einen tatsächlichen Anker im Wortlaut zu haben.
Die theoretische Alternative, gar keine Granzziehung in der rechtlichen Verwandschaft zu ziehen, würde vermutlich so gut wie alle Parteimitglieder erfassen, da auch Verwandtschaften über die Ecke von 40 oder gar mehr Generationen rechtliche Verwandschaften sind, auch wenn sie seltenst von rechtlicher Erheblichkeit sind.\footnote{Eine Ausnahme stellt hier das unbegrenzte Verwandtenerbrecht dar, vgl. \S~1929 Abs.~1 BGB\index[paridx]{BGB!1929@\S~1929!1@Abs.~1}.}
Eine sehr enge Auslegung wird der Rechtsfolge und somit dem Normcharakter als absolutes richterliches Mitwirkungsverbot eher gerecht als eine weite Auslegung oder gar eine willkürliche Grenzziehung ohne Verankerung im Wortlaut.

\subsubsection{Tatbestand \S~5 Abs.~1 Satz~1 Nr.~4 SGO\index[paridx]{SGO!5@\S~5!1@Abs.~1}}
\label{Zusammensetzung:Spruchkoerper:Befangenheitsvermutung:Nr4}
\S~5 Abs.~1 Satz~1 Nr.~4 SGO\index[paridx]{SGO!5@\S~5!1@Abs.~1} vermutet unwiderleglich, dass ein Richter unwiderleglich als befangen gilt, wenn eine der Personen nach Nr.~1--3 einem Organ angehören, dass Streitpartei ist.
Organ ist hier jedes Organ des Bundesverbandes oder einer Parteigliederung, die grundsätzlich immer aktiv- und passivlegitimiert sind.\footnote{Dazu siehe \ref{Anrufung:Statthaftigkeit:Parteifaehigkeit}.}

Auch diese Vorschrift deutet im Übrigen an, dass der Satzungsgeber bei Nr.~3 keine große, unübersichtliche Personengruppe im Blickfeld hatte und daher die vorgeschlagene enge Auslegung dort geboten ist.

Problematisch ist diese Vorschrift dennoch.
Organe gibt es auf Bundesebene vier: Den Parteitag, den Vorstand, die Gründungsversammlung und das Schiedsgericht.
Soweit bekannt, haben die bundesweiten Gliederungen kaum weitere Organe eingeführt.
Mitgliederabstimmungen außerhalb der klassischen ortsgebundenen Parteitagen sind meist als besondere Parteitagstagung (SMV) oder Urabstimmung (SDMV, BEO) organisiert, Präsidien, Beiräte oder andere Organ gibt es in der Regel nicht.
Lediglich der Landesverband Brandenburg hat wohl noch zusätzliche Organe, und zwar seine Arbeitsgemeinschaften.
Dies dürfte aufgrund der Rarität dieser Regelung kaum im Blickfeld des Bundessatzungsgebers gewesen sein, jedenfalls findet sich weder in der Satzung oder sonst noch wo ein Anhaltspunkt dafür, daher sind derartige zusätzliche Organe für diese Betrachtung vorerst vernachlässigbar, auch wenn die Norm natürlich ebenso für diese gilt.

Schiedsgerichte können nach \S~8 Abs.~7 SGO\index[paridx]{SGO!8@\S~8!7@Abs.~7} schon gar keine Verfahrensbeteiligten sein, sodass sie nicht von diesem Tatbestand gemeint sein können.

Mitglieder von Vorständen können nach \S~3 Abs.~6 SGO\index[paridx]{SGO!3@\S~3!6@Abs.~6} nicht Mitglieder eines Schiedsgerichtes sein, da es aber für diesen Tatbestand aufgrund der Präsensformulierung nur auf gegenwärtige Organmitgliedschaften ankommt, kommen für diese Konstellation nur Nr.~2 und Nr.~3 in Betracht, Nr.~1 kann nie verwirklicht werden.

Die Gründungsversammlung ist von ihrer Natur her in der Piratenpartei dem Parteitag gleich und wird daher wie dieser betrachtet.
Das Bundesschiedsgericht hat in seiner Entscheidung\footnote{\cite{BSGPP100127862}.} zu ebendiesem Tatbestand entschieden, dass nur Organe der \enquote{Exekutive} erfasst seien, und damit gerade der Parteitag und somit auch die Grundsversammlung als satzungsgebendes (oder auch: rechtsetzendes) Organ und somit Organ der \enquote{Legislative} schon gar nicht erfasst wäre.
Diese Auslegung findet schon keinen Halt im Wortlaut, sondern würde eine Reduktion nach dem Sinn der Norm darstellen.
Warum diese geboten sein soll, hat das Bundesschiedsgericht nicht begründet.
Im Gegenteil, der Tatbestand von Nr.~1 in Verbindung mit Nr.~4 könnte in dieser Auslegung des Bundesschiedsgerichtes schon in gar keiner denkbaren Situation verwirklicht werden, wäre also eine leere Norm.
Nach der systematischen Auslegung darf es aber keine leeren Normen geben, da jede Regelung irgendeinen Sinn und Zweck hat.\footnote{Siehe zur systematischen Auslegung auch \ref{Normenauslegung:Auslegung:Methoden:Systematisch}.}
Daher ist die nicht weiter begründete Auslegung des Bundesschiedsgerichtes nicht haltbar, vielmehr sind Parteitage gerade nach dem dem Wortlaut entnehmenbaren Willen des Satzungsgebers auch erfasst.\footnote{Zu diesem Ergebnis kommt nach ausführlicher Betrachtung auch \cite[2 ff. mwN.]{LSGBB153}.}

Dies führt jedoch dazu, dass Landesschiedgerichte nie Anfechtungen gegen Entscheidungen ihres eigenen Landesparteitages verhandeln dürfen, da alle Richter zugleich Mitglieder des Landesverbandes sind und als solches Mitglieder des Organs Parteitag.
Das Organ darf hier nicht mit der Tagung des Organ verwechselt werden, da jedes Organ, also auch der Parteitag, außerhalb seiner Zusammentreffen weiter existiert.
Der Parteitag ist in der Piratenpartei typischerweise eine Mitgliederversammlung.
Dies hat zur Folge, das alle Mitglieder einer Gliederung Mitglieder des Organs Parteitags sind.
Um Sinn und Zweck der unwiderleglichen Befangenheitsvermutung zu genügen und gleichzeitig nicht zu umfangreich Mitglieder auszuschließen, muss der Wortlaut aber auch hier wohl teleologisch reduziert werden: Nur Mitglieder, die tatsächlich an der betreffenen Entscheidung beteiligt waren, können Befangen sein.
Allerdings ist jeder Redebeitrag wie auch jedes Abstimmverhalten schon eine Beteiligung. Das umfasst auch wieder die Enthaltung.
Es gibt auch keinen Grund, einen Unterschied zwischen aktiver und passiver Unterhaltung zu machen, zudem das typischerweise im Nachhinein nicht beweisbar sein wird.
Daher reicht letzlich doch wieder die Akkreditierung an dem Zusammentreffen des Organs für die Erfüllung des Tatbestands nach Nr.~4 in Verbindung mit Nr.~1.
(Dies gilt natürlich auch entsprechend für die Varianten Nr.~2 und Nr.~3).
Damit ist aber effektiv das eigene Landesschiedsgericht in jedem Verfahren, in dem der eigene Landesparteitag beteiligt ist handlungsunfähig und das Verfahren muss noch vor Eröffnung an ein anderes Landesschiedsgericht verwiesen werden.
Für das Bundesschiedsgericht heißt das sogar, das Verfahren mit Beteiligung des Bundesparteitages niemals zur Eröffnung oder gar Verhandlung kommen dürfen.
Diese Konsequenz, dass der Zugang zur Parteischiedsgerichtsbarkeit komplett entfällt, wiederspricht \S~14 Abs.~1 PartG, der fordert, dass gerade in solchen Fällen die Parteinterne Schiedergerichtsbarkeit zuständig ist.
Im Fall der Landesschiedgerichte entfällt durch die konsequente Umgehung der Zuständigkeitsregelungen die Bestimmbarkeit des entscheidenden Richters: Zuerst muss immer erst das Bundesschiedsgericht nach freiem Ermessen das Verfahren verweisen.
Das widerspricht Elementar dem Verfassungsprinzip des gesetzlichen Richters.
Damit verstößt \S~5 Abs.~1 Satz~1 Nr.~4, 1 SGO\index[paridx]{SGO!5@\S~5!1@Abs.~1} in diesen Konstellationen gegen höherrangiges Recht und darf zumindest in diesen Fällen nicht angewendet werden.
Sobald es um Parteitage niederer Gliederungen geht, und kein entsprechendes Schiedsgericht existiert, trifft diese geschilderte Konstellation nicht zu, in diesen Fällen ist \S~5 Abs.~1 Satz~1 Nr.~4, 1 SGO\index[paridx]{SGO!5@\S~5!1@Abs.~1} daher anwendbar.


\subsubsection{Tatbestand \S~5 Abs.~1 Satz~1 Nr.~5 SGO\index[paridx]{SGO!5@\S~5!1@Abs.~1}}
\label{Zusammensetzung:Spruchkoerper:Befangenheitsvermutung:Nr5}
\S~5 Abs.~1 Satz~1 Nr.~5 SGO\index[paridx]{SGO!5@\S~5!1@Abs.~1} vermutet unwiderleglich, dass ein Richter, der selbst eine Prozessvertreter oder gesetzlicher Vertreter einer Verfahrenspartei ist, befangen ist.
Letzlich bedeutet dies, dass jede Berechtigung, eine Partei in einem Verfahren zu vertreten, dazu führt, dass der betreffende Richter nicht im Gericht mitentscheiden darf.
Dies ist insofern wieder sehr nah an der Regelung von Nr.~1 und ist entsprechend auch begründet.
Die Bestellung zum Prozessvertreter ist einer einseitige, empfangsbedürtige Prozesserklärung.
Das heißt aber in der Konsequenz, dass der Bestellte für die wirksame Bestellung nicht zustimmen muss.
Es ist ohne Zustimmung zwar nicht der bestellenden Verfahrenspartei gegenüber verpflichtet, auch für sie zu handeln, gegenüber der anderen Verfahrenspartei und dem Gericht ist er aber trotzdem wirksam zum Vertreter bestellt.
Damit also eine Verfahrenspartei diese Vorschrift nicht ausnutzen kann, um etwa unliebsame Richter auszuschalten, in dem sie zum Verfahrensvertreter bestellt werden, muss auch hier eine teleologische Korrektur angewandt werden: Der Tatbestand der Norm ist dahingehend zu verstehen, dass der bestellte Vertreter auch plausibler Vertreter der Partei ist, also gerade im Fall eines eigentlich zuständigen Richters wird eine Zustimmung zur Bestellung oder zumindest aktives Handeln als Parteivertreter verlangen zu sein.
Tut er das nicht oder lehnt die Bestellung ab, kann keine Partei den Richter wegen diesem Tatbestand ablehnen und auch ein Ausschluss von Amts wegen ist abzulehnen.

\subsubsection{Tatbestand \S~5 Abs.~1 Satz~1 Nr.~6 SGO\index[paridx]{SGO!5@\S~5!1@Abs.~1}}
\label{Zusammensetzung:Spruchkoerper:Befangenheitsvermutung:Nr6}
\S~5 Abs.~1 Satz~1 Nr.~6 SGO\index[paridx]{SGO!5@\S~5!1@Abs.~1} vermutet unwiderleglich, dass ein Richter, der  Zeuge oder Sachverständiger in einem Verfahren ist, in diesem Verfahren befangen ist.
Wie schon Nr.~1 und Nr.~5 liegt ist hier der Interessenkonflikt, den die Satzung als Grundlage für ihre Befangenheitsvermutung nimmt, in der Person des Richters offensichtlich, auch wenn er diesemal nicht in seiner natürlichen Person liegt, sondern erst durch den Zusammenhang mit dem eigentlichen Verfahrensgegenstand aufkommt.

\subsubsection{Tatbestand \S~5 Abs.~1 Satz~1 Nr.~7 SGO\index[paridx]{SGO!5@\S~5!1@Abs.~1}}
\label{Zusammensetzung:Spruchkoerper:Befangenheitsvermutung:Nr7}
\S~5 Abs.~1 Satz~1 Nr.~7 SGO\index[paridx]{SGO!5@\S~5!1@Abs.~1} vermutet unwiderleglich, dass ein Richter, der  schon über das Verfahren oder den konkreten Verfahrensgegenstand im regulären vorprozessualen Geschehen oder einer unteren Instanz mitentschieden hat oder als Antragsteller oder Berater die Entscheidung eines anderen Organs, die nun Verfahrensgegenstand mitverursacht hat, im betreffenden Verfahren befangen ist.
Das derselbe Richter nicht in mehreren Instanzen über dasselbe Verfahren entscheiden soll, ist logisch.
Aber auch, wenn er sonst im vorgerichtlichen Verfahren, etwa die Ausarbeitung eines Antrags, gegen dessen Ablehnung bspw. ein Mitantragsteller klagt, relevant beteiligt war, soll dieser Grundsatz gelten, da der Satzungsgeber der Ansicht ist, dass der Richter zu nah an dem Verfahrensgegenstand dran ist, als dass die Entscheidung als objektiv unbefangen wahrgenommen werden würde.

\subsubsection{Tatbestand \S~5 Abs.~1 Satz~1 Nr.~8 SGO\index[paridx]{SGO!5@\S~5!1@Abs.~1}}
\label{Zusammensetzung:Spruchkoerper:Befangenheitsvermutung:Nr8}
\S~5 Abs.~1 Satz~1 Nr.~8 SGO\index[paridx]{SGO!5@\S~5!1@Abs.~1} vermutet unwiderleglich, dass ein Richter, der als Schlichter in einem Verfahren aktiv war, nicht auch Richter sein soll.
Dies untermauert die Trennung von richtender Funktion des Schiedsgerichtsverfahrens und schlichtender Funktion des vorgelagerten Schlichtungsverfahrens.
Nr.~8 ist somit sehr nah an den Tatbestand der Vorbefassung aus Nr.~7 dran, auch wenn es gerade um eine nicht entscheidende, außergerichtliche Instanz geht, die mangels Beschlussfassung nicht immer von Nr.~7 erfasst ist.


%\blindtext[1]
%\section{Spruchkörper}
%\subsection{Zusammensetzung, Kammern}
%\subsection{Ablehnung von Richtern}
%\blindtext[5]
%\subsection{Richterurlaub und -abwesenheit}
%\blindtext[5]

\chapterbib
% \end{refsection}
