\documentclass{sghandbuch}

\newcommand*{\documentversion}{Milestone 1}
\author{Simon Gauseweg \and Florian Zumkeller-Quast}
\title{Schiedsgerichtshandbuch\\
für die Piratenpartei Deutschland und Untgliederungen\\
\hfill\break
\hfill\break
\normalsize{\documentversion}}

\usepackage{blindtext}
\addbibresource{sghandbuch.bib}
\newindex[Urteilsverzeichnis]{jurisdiction}
\newindex[Stichwortverzeichnis]{idx}
\newindex[Paragraphenverzeichnis]{paridx}
\makenomenclature
% \usepackage{refsection}

\begin{document}

\maketitle

\onecolumn			% Einspaltenlayout anschalten
\thispagestyle{plain}		% Seitenummerierung eine Seite unterdrücken
\hfill				% Zentrierung 
\vfill				% Zentrierung
\begin{quote}
\centering			% Zentrierung
\itshape
\Huge
Eine Meinung zu haben gehört zum Berufsbild eines Richters.

\par
\begin{flushright}
\Large
\normalfont
-- (unbekannt, Hans-Jürgen Papier zugeschrieben)
\end{flushright}
\end{quote}
\hfill				% Zentrierung
\vfill				% Zentrierung
\clearpage			% Seitenende
\twocolumn			% Wieder ins Zweitspaltenlayout wechselns

\tableofcontents


% \begin{refsection}
\chapterpreamble{Ein Schiedsgerichtshandbuch ist geeignet, erforderlich und auch angemessen, die Qualität der Rechtsprechung innerhalb der Parteigerichte der Piratenpartei zu verbessern.}

\chapter{Einleitung}
Schon die Verfassung gibt den politischen Parteien in Art.~21\indexPar{GG!21@Art.~21} eine ganz besondere Stellung.
\Zitat{Ihre Gründung ist frei} heißt es dort in Absatz~1 Satz~2\indexPar{GG!21@Art.~21!1@Abs.~1}.
Doch nicht nur die Gründung der Parteien ist frei, auch die innere Struktur ist ihnen --- unter Beachtung gewisser demokratischer Mindeststandards gemäß Art.~21 Abs.~1 Satz~3 GG\indexPar{GG!21@Art.~21!1@Abs.~1} --- selbst überlassen.
Sie sollen unabhängig sein von staatlichem Einfluss.

Das hat einen ganz einfachen Grund:
Politische Parteien sind ein wichtiger Spieler im Machtgefüge der Bundesrepublik Deutschland.
Sie sind es, die Einfluss auf den Staat haben sollen und nicht umgekehrt.
Wo es aber um Macht geht, wird gestritten.
Und das nicht selten.
Das war selbst dem Gesetzgeber bewusst.

Um die \emph{Staatsferne der Parteien} zu sichern und die eigenen staatlichen Gerichte nicht mit den parteiinternen Streitigkeiten zu belasten, wurde das Institut der Parteischiedsgerichtsbarkeit geschaffen.
Parteischiedsgerichte handeln dabei nach ihrer von ihrer eigenen Partei erlassenen Verfahrensordnung.

Auch wenn der Gesetzgeber im Wissen um die ehrenamtliche Besetzung der Schiedsgerichte mitunter mit Laien und nicht berufserfahrenen Richtern den Parteischiedsgerichten nicht die gänzliche Komplexität der großen Prozessordnungen aufbürdet, so müssen sie doch die allgemeinen Grundsätze eines gerechten Gerichtsverfahrens einhalten (vgl. \S~14 Abs.~4 PartG)\indexPar{PartG!14@\S~14!4@Abs.~4} und sich strikt an die von der eigenen Partei erlassene Prozessordnung halten.

Da gerade in der Piratenpartei oft Unerfahrene und Laien die Richterämter übernehmen, haben wir dieses Handbuch verfasst, um all denen einen Leitfaden an die Hand zu geben, die sich in dieses Amt einarbeiten wollen.
Auch erfahrenen Richtern wird hier eine Übersicht über Regelungen der Schiedsgerichtsordnung und der zugehörigen Rechtsprechung an die Hand gegeben.
Das Handbuch kann so als Nachschlagewerk genutzt werden.

Die grobe Gliederung des Handbuchs ist an einem üblichen Verfahrenslauf bzw. einer üblichen Fallprüfung orientiert:
Die innere Organisation der Schiedsgerichte, z.B. unter Berücksichtigung der Vorschriften über deren Geschäftsordnungen, und Zusammensetzung des Gerichts werden vor der Anrufung durch eine Streitpartei behandelt; die Zulässigkeitsvoraussetzungen der Klage vor den Grundlagen der Satzungsauslegung erläutert.
Jedem dieser Kapitel folgt ein Verzeichnis der im jeweiligen Kapitel zitierten Urteile.
Auch die verwendete Literatur findet sich hier wieder, wobei wir darauf geachtet haben, möglichst elektronisch (und frei) verfügbare Werke zu verwenden und insgesamt einen ausufernden Fußnotenapparat zu vermeiden.

\Zitat{Zwei Juristen, drei Meinungen} lautet ein geflügeltes Wort.
Einen Anspruch auf Vollständigkeit der Darstellung erheben wir nicht.
Dem Anspruch einer umfassenden Widergabe aller Meinungen und Argumente zu sämtlichen Auslegungen der relevanten Normen könnten wir schon gar nicht gerecht werden.
Das vorliegende Handbuch soll keine Kommentierung sein, sondern ein praktischer Leitfaden.
Das reduziert den Umfang um tiefgehende juristische Auseinandersetzung mit dem einen oder anderen Problem unserer Satzung --- lässt aber dennoch beträchtlichen Raum für den Umfang des Werkes übrig.
Wir haben daher beschlossen, das Handbuch etappenweise zu vervollständigen und uns dabei an der Rechtsprechung des Bundesschiedsgerichtes und der Landesschiedsgerichte zu orientieren und so die typischen Probleme und Konstellationen dazustellen, mit denen Richter in der Parteienschiedsgerichtbarkeit konfrontiert sein werden.
Den noch nicht fertiggestellten Kapitel haben wir --- gewissermaßen als vorläufigen Platzhalter --- eine Kurzzusammenfassung der wichtigsten Dinge vorangestellt, die es unserer Meinung nach in diesem Bereich zu beachten gilt.

%Besonderer Dank gilt (Namen und was sie gemacht hätten).
%Ebenso gilt es den derzeitigen und ehemaligen Richterinnen und Richtern der Schiedsgerichte aller Gliederungen für ihr ehrenamtliches Engagement um die Piratenpartei zu danken.

\vspace{20mm}

Dipl. jur. Simon Gauseweg, LL.B.\\
ehem. Vorsitzender Richter am Landesschiedsgericht Brandenburg

\vspace{5mm}

Florian Zumkeller-Quast\\
Richter am Bundesschiedsgericht a.D.

%\chapterbib
% \end{refsection}

 


% \begin{refsection}
\chapterpreamble{Trifft eine Anrufung beim Schiedsgericht ein, ist zu prüfen ob der Antragsteller antragsberechtigt ist (\S~8 Abs.~1 SGO\nomenclature{SGO}{Schiedsgerichtsordnung}) und ob die Anrufung formell vollständig ist (\S~8 Abs.~3 SGO). Die fristgerechte Klageerhebung ist bereits im Rahmen der Anrufung zu prüfen. Nur bei Zweifeln, die sich nicht ohne Ermittlungen klären lassen, ist das Verfahren zu eröffnen. Die Anschrift i.S.d. \S~8 Abs.~3 S.~1 SGO kann auch aus einer Postfachadresse bestehen.}

\chapter{Schema für den Standardworkflow \enquote{Ich habe eine Anrufung erhalten}}

Ein Schiedsgericht wird nur auf Anrufung tätig schreibt die SGO in \S~8 Abs.~1 S.~1 vor.
Damit ist die Anrufung der zentrale Punkt für ein Schiedsgericht, an dem jede Tätigkeitkeit anknüpft.

Aber auch der Umgang mit einer Anrufung ist streng formal geregelt.
Das soll es den Gerichten erleichtern, gewissen Standardfrage nicht immer noch im Nachhinein stellen zu müssen, weil sie immer auftauchen.

Grundlegend sieht der Ablauf der zu prüfenden Punkte dabei wie folgt aus:
\begin{enumerate}[label=\Roman*.]
% \itemsep0em 
\item \textbf{Antragsteller}: Ist klar genannt, \emph{wer} die Klage erheben will? (\S~8 Abs.~3 Nr. 1 SGO\index[paridx]{SGO!8@\S~8!3@Abs.~3})
\item \textbf{Antragsgegner}: Ist klar genannt, gegen \emph{wen} sich die Klage richten soll? (\S~8 Abs.~3 Nr.~2 SGO\index[paridx]{SGO!8@\S~8!3@Abs.~3})
\item Sind Antragsteller und Antragsgegner \textbf{parteifähig}? (\S~8 Abs.~1 S.~1 SGO\index[paridx]{SGO!8@\S~8!1@Abs.~1})
\item Ist jeweils eine zulässige \emph{Anschrift} angegeben? (\S~8 Abs.~3 Nr.~1, 2 SGO\index[paridx]{SGO!8@\S~8!3@Abs.~3})
\item Sagt der Antragsteller klar, \emph{was} genau er erreichen will? (\S~8 Abs.~3 Nr.~3 SGO\index[paridx]{SGO!8@\S~8!3@Abs.~3})
\item Ist das, was der Antragsteller will, etwas, das ihm nach Satzung, Parteiengesetz oder sonstigen mitgliedschaftlichen oder organschaftlichen Rechten \emph{möglicherweise zustehen könnte}? (\S~8 Abs.~1 S.~1 SGO\index[paridx]{SGO!8@\S~8!1@Abs.~1})
\item Nennt der Antragsteller \textbf{Gründe}, warum das gewünschte ihm zustehen sollte? (\S~8 Abs.~3 Nr.~4 SGO\index[paridx]{SGO!8@\S~8!3@Abs.~3})
\item Ist die \textbf{Form} eingehalten? (\S~8 Abs.~3 SGO\index[paridx]{SGO!8@\S~8!3@Abs.~3})
\item Ist die \textbf{Frist} eingehalten worden? (\S~8 Abs.~4 SGO\index[paridx]{SGO!8@\S~8!4@Abs.~4})
\item Ist das angerufene Schiedsgericht \textbf{zuständig}? (\S~8 Abs.~5 SGO\index[paridx]{SGO!8@\S~8!5@Abs.~5})
\end{enumerate}

\section{Statthaftigkeit}
\label{Standardworkflow:Statthaftigkeit}
Die grundlegendsten Anforderungen an eine Anrufung werden unter dem Begriff Statthaftigkeit zusammengefasst.
Eine Klage ist grundsätzlich nur statthaft, wenn diese Anforderungen erfüllt sind.

\subsection{Antragsteller}
\label{Standardworkflow:Antragsteller}
Das allererste dieser Kriterien ist die Benennung eines \index[idx]{Antragsteller|textbf} Antragstelllers bzw. einer Antragstellerin, andernorts auch Kläger  bzw. Klägerin genannt.
Die SGO kennt alledings den Begriff Kläger bzw. Klägerin nicht sondern nur den Begriff Antragsteller bzw. Antragstellerin, entsprechend wird auch dieser Begriff hier verwendet.
Antragsteller bzw. Antragstellerin ist typischerweise der oder die Anrufende selbst.
In einigen Fällen, typischerweise im Falle einer Anrufung durch einen Vorstand oder ein anderes Organen, kann der oder die Anrufende auch Vertreter bzw. Vertreterin des Antragstellers bzw. der Antragstellerin sein.
In jedem Fall muss aus der Anrufung klar identifizierbar hervorgehen, wer nun mit der Anrufung Rechte bzw. Ansprüche vor dem Schiedsgericht gegen eine andere Person gelten machen will.
Dieser jemand ist der Antragsteller bzw. die Antragstellerin.

In den meisten Fällen dürfte dies ein Mitglied sein, in seltenen Fällen kann es auch ein einzelner Amtsträger oder eine einzelne Amtsträgerin sein, noch häufiger dürfte es ein Organ als gesamtes sein.
Typischerweise gibt es lediglich vier Fälle, in denen die Anrufung durch ein Organ erfolgt:
\begin{enumerate}[label=\arabic*.)]
% \itemsep0em 
\item Parteiausschlussverfahren\index[idx]{Parteiausschluss}
\item Gliederungsordnungsmaßnahmeneinsprüche\index[idx]{Ordnungsmassnahme@Ordnungsma""snahme!Gliederungs-}
\item Gliederungskompetenzstreitigkeiten
\item Organhandlungsfähigkeitsstreitigkeiten
\end{enumerate}
Der letzte Fall, die Organhandlungsfähigkeitsstreitigkeiten sind strenggenommen sogar nur ein Unterfall der Gliederungskompetenzstreitigkeiten, aber dazu an entsprechender Stelle mehr.

Aus dem Rahmen fallen natürlich auch immer die Berufungs- und Beschwerdeanrufungen im Instanzenzug und die Widerspruchsanrufung im einstweiligen Rechtsschutz, weil hier die Anrufung typischerweise von der in der ursächlichen Schiedsgerichtsentscheidung unterlegenen Partei ausgeht.

\subsection{Antragsgegner}
\label{Standardworkflow:Antragsgegner}
Der Antragsgegner\index[idx]{Antragsgegner|textbf} oder die Antragsgegnerin ist ebenso notwendig und muss ebenso klar und eindeutig aus der Anrufung entnehmbar sein. Bezüglich der Vertretung gilt hier dasselbe wie bei dem \index[idx]{Antragsteller} Antragsteller bzw. der Antragstellerin.

Bei allen Verfahren nach SGO\nomenclature{SGO}{Schiedsgerichtsordnung} handelt es sicht um sogenannte kontradiktorische Verfahren.
Das bedeutet, dass das Verfahren Streit von zwei Parteien um gegenläufige Interessen ausgestaltet ist.
In den meisten Fällen ist der Antragsgegner bzw. die Antragsgegnerin daher ein Organ, in wenigen Fällen -- primär bei gerichtlich verhängten Ordnungsmaßnahmen wie dem Parteiausschluss -- aber auch ein einzelnes Mitglied.
Wichtig dabei ist aber: Es gibt nach SGO keine Verfahren von Mitgliedern gegen andere Mitglieder.
Schon fraglich ist, ob ein einzelner Amtsträger bzw. eine Amträgerin gegen einen anderen Amträgräger bzw. eine andere
Amtsträgerin einzeln vorgehen kann, dazu gab es bisher noch nie einen Anlass, das zu entscheiden.
Theoretisch ist es aber nicht komplett ausgeschlossen.
Möglich ist in jedemfall das vorgehen einer Amtsträgerin bzw. eines Amtsträgers gegen das restliche Organ, eines Organs  gegen einzelne Amtsträger und Amtsträgerinnen sowie das vorgehen von Mitgliedern gegen Organe und andersherum.
Wichtig ist daher der Grundsatz: Aus dem Antragsteller oder der Antragstellerin ergibt sich zwangsweise ein eingeschränkter Kreis möglicher Antragsgegner und Antragsgegnerinnen.

\subsection{Parteifähigkeit}
\label{Standardworkflow:Parteifaehigkeit}
Gerade wude es schon angesprochen: Antragsteller bzw. Antragstellerin und Antragsgegner bzw. Antragsgegnerin sind die \index[idx]{Streitpartei}Parteien des Prozesses.
Nach SGO sind Mitglieder und Organe explizit dazu berechtigt, ein Schiedsgericht anzurufen, wenn sie sich in ihren Rechten verletzt fühlen, \S~8 Abs.~1 S.~2 SGO\index[paridx]{SGO!8@\S~8!1@Abs.~1}.
Das bedeutet, dass diese auch zwingend Partei in einem Schiedsgerichtsverfahren sein können.
Ferner muss die Parteifähigkeit\index[idx]{Parteifähigkeit} auch Amtsträgern als spezieller Art von Mitgliedern zugestanden werden.
Ob man dabei davon ausgeht, dass Amtsträger eine eigene Kategorie von Partei darstellen oder lediglich Mitglider sind, deren Rechte aufgrund einer Wahl zeitlich beschränkt verändert wurden, ist eigentlich eine akademische Debatte.
Angesichts dessen, dass Amtsträger aber in die Rechte anderer Amtsträger eingreifen können und dann im Sinne eines effektiven Rechtsschutzes entweder das Recht des anderen Amtsträgers zu einem Organrecht erklärt werden muss oder aber eine Klage von Mitglied gegen Mitglied zugelassen werden muss, spricht doch einiges dafür, Amtrsträger als eigene, nicht explizit genannte,  aber doch erfasste Kategorie innerhalbt der Parteifähigkeit zu sehen.

Brisanz gewinnt die Einordnung der Amtsträger als eigene parteifähige Kategorie bei der Klage von Mitgliedern gegen Amtsträgern, was dadurch nicht mehr von der nicht erlaubten Klage von Mitglied gegen Mitglied vor dem Schiedsgericht ausgeschlossen ist.
Allerdings müsste dann eine Klagebefugnis vom Mitglied vorliegen, typischerweise ist die Rechtsverletzung aber nicht dem  einzelnen Amtsträger sondern dem ganzen Organ zuzurechnen genauso wie sich Ansprüche aus der Mitgliedschaft typischerweise gegen das zuständige Organ richten. 
Daher kann fast davon ausgegangen werden, dass es nicht zu einer solchen Klagekonstellation kommen kann bzw. falls doch, dass diese dann gerade nötig ist, um einen effektien Rechtsschutz für Mitglieder zu gewähren.

Fraglich ist zudem ob die Partei bzw. die einzelnen Gliederungen Parteifähig im Sinne der SGO sind.
Während dies für Prozesse vor staatlichen Gerichten aufgrund der dortigen Regelungen zur Parteifähigkeit\footnote{Vgl. etwa \S~50 ZPO\nomenclature{ZPO}{Zivilprozessordnung}\index[paridx]{ZPO!\S~50}.} grundsätzlich anzunehmen ist, spielt die Partei bzw. Gliederung als solche im internen Rechtsstreit eine untergeordnete Rolle, da die Handlung immer zumindest einem Organ oder zumindest Amtsträger zugeordnet werden kann.
Daher ist auch das Bundesschiedsgericht in seiner Rechtsprechung dazu übergegangen, die Parteifähigkeit der Partei und ihrer Gliederungen nur zu bejahen, wenn eine derartige Zurechnung nicht möglich ist.\footnote{Vgl. \cite[S. 4]{BSG1614HS}; Diese Rechsprechung fortführend \cite[S. 2]{BSG3815HS} mwN.}
Die Partei und die Gliederungen kommen also nur als subsidiärer Antragsgegner bzw. Antragsgegnerin in Betracht.

\subsection{Anschrift}
\label{Standardworkflow:Anschrift}
Der Antragsteller muss seine eigene Anschrift sowie die des Antragsgegners angeben.
Die Anschrift ist notwendig, damit das Schiedsgericht im Verfahren mit den Parteien in Kontakt treten kann.
Auch wenn die Kommunikation per E-Mail im Verfahren üblich ist, ist es hier notwendig, dass eine postalische Adresse genannt wird.\footnote{Ausführlich dazu \cite[S. 3]{BSG20130715} mwN.}
So gibt es zwar keinen Anspruch mehr in der SGO auf eine Urteilszusendung in Schriftform, die die Kenntnis einer postalischen Adresse früher zwingend machte, jedoch ist die Anschrift nach wie vor zur Individualisierung und damit zur Mitgliedsverifikation durch das Schiedsgericht erforderlich. Auch im Fall, dass eine Person mittels einer lediglich vom Verfahrensgegner genannten E-Mailadresse nicht erreichbar ist, ist eine zusätzlicher Kontaktversuch per Brief erforderlich, um die Rechte der nicht erreichbaren Verfahrenspartei zu sichern.

Die weiteren Kontaktdaten gemäß \S~8 Abs.~3 Nr.~1 SGO\index[paridx]{SGO!8@\S~8!3@Abs.~3} umfassen Daten wie eine E-Mailadresse. Diese ist aus prozessökonomischen Gründen sinnvoll, da ein Information per E-Mail die Verfahrensparteien schneller erreicht und diese auch schneller reagieren können. Das ganz das Verfahren insgesamt erheblich beschleunigen und macht die direkte Versendung in Kopie zudem einfacher.

Dass die Anschrift unbekannt ist, kann auch nie ein Problem darstellen. Verfahren von Mitglied gegen Mitglied sind schon gar nicht vorgesehen (siehe \ref{Standardworkflow:Antragsgegner}), der Partei ist die Anschrift eines Mitglieds aus der Mitgliederdatenbank bekannt, die Anschrift der Partei und der Gliederungen sind allgemein bekannt, die Organe haben ihre Anschrift für gewöhnlich bei der Gliederung, oder, falls diese abweicht, ist sie vom Organ bekannt gemacht.

\subsection{Anträge}
\label{Standardworkflow:Antraege}
Ebenfalls zwingend erforderlich ist es, dass der Antragssteller mit der Anrufung bereits Anträge stellt.
Diese Anträge müssen nach \S~8 Abs.~3 Nr.~3 SGO\index[paridx]{SGO!8@\S~8!3@Abs.~3} klar und eindeutig sein.
Diese innere Dopplung der Formulierung zeigt schon, dass es dem Satzungsgeber darauf ankam, dass der Antragsteller explizit sagen muss, was exakt er mit der Anrufung erreichen will.
Das Schiedgericht ist daher nicht verpflichtet, zu spekulieren und Vermutungen darüber anzustellen, was der Antragsteller gewollt haben könnte. 
Wenn der Antragsteller das nicht in seiner Anrufung klar erkennbar niederschreibt, dann ist die Anrufung unvollständig und daher nicht statthaft.

Die Erfordernis der genauen Anträge dient auch dazu, die genaue Klageart festzustellen, da diese Einfluss auf das weitere Verfahren haben kann.
Zudem dienst diese Anforderung dem Zweck, sicherzustellen, dass das Schiedsgericht weiß, in welche Richtung es seine Amtsermittlungspflicht gemäß \S~10 Abs.~1 S.~1 Hs.~1 SGO\index[paridx]{SGO!10@\S~10!1@Abs.~1} ausüben muss, um alle relevanten Fragen beantworten zu können, die sich für eine Entscheidung in der Sache stellen.

Bei der Prüfung, ob Anträge gestellt wurden, ist also eine sehr strenger Maßstab anzulegen. Wird nicht klar und eindeutig spezifiziert, was der Antragsteller vom Antragsgegner will, liegt schon gar kein Antrag vor.
Ob der Antrag erfüllbar ist und rechtmäßig gestellt wurde, wird allerdings hier noch nicht entschieden.
Insofern ist es auch nur eine sehr formale und eingschränkte Prüfung, ob einer oder mehrere Anträge gestellt wurden.

Und natürlich reicht entgegen der Pluralformulierung von \S~8 Abs.~3 Nr.~3 SGO\index[paridx]{SGO!8@\S~8!3@Abs.~3} ein einzelner Antrag auch aus. Es ist nicht Sinn und Zwekc der Regelung, dass ein Antragsteller, der nur eine einzelne Sache erreichen will, einen absolut unnötigen zusätzlichen Antrag stellt und das Schiedsgericht so mit Mehrarbeit belastet oder gar schon überhaupt nicht klagen darf, nur weil er nur eine Sache erreichen will.

\subsection{Begründung und Umstände}
\label{Standardworkflow:Gruende}
Die letzte Anforderung des \S~8 Abs.~3 SGO\index[paridx]{SGO!8@\S~8!3@Abs.~3} sind eine Darstellung der Umstände und Schilderung einer Begründung der Anträge.
Damit soll dem Schiedsgericht ein Anhaltspunkt für seine Amtsermittlung des Sachverhaltes nach \S~10 Abs.~1 S.~1 Hs.~1 SGO\index[paridx]{SGO!10@\S~10!1@Abs.~1} gegeben werden und dem Antragsgegner eine Möglichkeit, auf die Klage zu erwiedern.
Der Sinn und Zweck dieser Anforderung ist also, einen Startpunkt für das Verfahren haben, und den Streit der Parteien einsortieren zu können.
Entsprechend sind an keine hohen inhaltlichen Anforderungen zu stellen.
Solange ein Geschehen geschildert wird und irgendwelche Argumente angeführt werden, warum diese oder jene rechtliche Bewertung dafür zu gelten habe, ist regelmäßig anzunehmen, dass diese Punkt erfüllt ist.
Andernfalls würde \S~8 Abs.~3 Nr.~4 SGO\index[paridx]{SGO!8@\S~8!3@Abs.~3} eine Vorwegnahme der eigentlichen Beurteilung des Verfahrens darstellen, dass das Gericht erst nach ausführlicher Beschäftigung mit der Sache selbst durch die Durchführung des Verfahrens tätigen soll.

\section{Weitere Kriterien}
\label{Standardworkflow:Kriterien}
Neben der Statthaftigkeit gibt es noch weitere Kriterien, die für eine erfolgreiche Anrufung erfüllt sein müssen.
Dies sind unechte Zulässigkeitskriterien, da sie die Klage unzulässig machen, aber nicht erst im Verfahren mit dem Urteil beschieden werden, sondern schon mit Eröffnung.
Die Satzung nennt diese Kriterien auch Zulässigkeitskriterien der Eröffnung, vgl. \S~8 Abs.~6 Satz~4 SGO\index[paridx]{SGO!8@\S~8!6@Abs.~6}.

\subsection{Antragsbefugnis}
\label{Standardworkflow:Antragsbefugnis}
Die Notwendigkeit der Antragsbefugnis geht aus \S~8 Abs.~1 SGO\index[paridx]{SGO!8@\S~8!1@Abs.~1} hervor und soll sogenannte Popularklagen verhinden.
Eine Antragsbefugnis liegt dann vor, wenn der Antragsteller selbst in seinen eigenen Rechten aus der Mitgliedschaft oder seinem sonstigen innerparteilichen Status verletzt ist oder ein Anspruch aus dieser Position nicht erfüllt wird.
Die Abgrenzung von eigenen Rechten und eigenen Ansprüchen ist nicht immer einfach, aber auch nicht zwingend nötig, da Alternativ eines von beides ausreicht.
Entsprechend den anderen nicht rein förmlichen Anrufungskriterien ist hier kein hoher inhaltlicher Anspruch zu stellen.
Daher muss für die Eröffnung lediglich dargelegt werden, dass ein solches Recht oder ein solcher Anspruch verletzt sein könnte.
Eine tiefere Prüfung erfolgt bei der Entscheidung über Eröffnung nicht, sondern im Zweifelsfall immer erst im laufenden Verfahren.
Hat das Schiedgericht von Anfang an Zweifel an der in der Anrufung dargeleten Begründung der Antragsbefugnis, ist das Verfahren dennoch zu eröffnen. Es bietet sich dann aber an, der anrufenden Partei einen richterlichen Hinweis zu geben, dass sie die Begründung diesbezüglich erweitern sollte.
Das geht auch aus \S~8 Abs.~5, 6 Satz~1 SGO\index[paridx]{SGO!8@\S~8!5@Abs.~5} hervor. Dort wird bestimmt, wann das Verfahren zu eröffnen ist. Nämlich immer dann, wenn das angerufene Schiedsgericht zuständig ist und die Anrufung korrekt erfolgte.
Korrekte Anrufung kann aber nur formelle Kriterien meinen, eine weitergehende, inhaltliche Prüfung zu diesem Zeitpunkt ist daher unzulässig.
Die Antragsbefugnis ist damit ein echtes Zulässigkeitskriterium und erst nach durchgeführtem Verfahren im Endurteil zu bescheiden.

\subsection{Form}
\label{Standardworkflow:Form}
Die Anforderungen an die Form der Anrufung sind sehr gering.
Es wird in \S~8 Abs.~3 SGO\index[paridx]{SGO!8@\S~8!3@Abs.~3} lediglich die Textform gefordert.
Im Gegensatz zur Schriftform, deren Erfüllung und Nichterfüllung bei untergesetzlicher vorgeschriebener Formerfordernis sehr strittig ist und auch in der Schiedsgerichtsbarkeit der Piratenpartei schon zu Entscheidungen führte\footcite{LSGBB146}, ist die Textform ausreicht klar und eindeutig in \S~126b BGB\index[paridx]{BGB!\S~126b} definiert und eine Formlockerung in \S~127 BGB\index[paridx]{BGB!\S~127} nicht vorgesehen.
Es reicht daher ein Schreiben, das mit dem Namen unterzeichnet ist und das in einer Form, die zur visuellen Widergabe des Inhalts geeignet ist.

\subsection{Frist}
\label{Standardworkflow:Frist}
Die Frist für eine Anrufung soll sicherstellen, dass Streitigkeiten schnell geklärt werden.
Damit soll verhindert werden, dass auf Basis einer möglicherweise rechtswidrigen Handlung weitere Dinge passieren und eine Klage, die erst nach langer Zeit eingereicht wird, unerwartet alles mitreißen kann.
Kurz gesagt: Nach einer gewissen Zeit sollen alle möglicherweise Betroffenen auf einen eingetretenen Zustand vertrauen können.
Daher darf nach Ablauf dieser Frist nicht mehr geklagt werden.
Dabei kennt die SGO unterschiedliche Anrufungsfristen.
\begin{enumerate}
\item Die Standardanrufungsfrist von 2 Monaten, \S~8 Abs.~4 Satz~1 SGO\index[paridx]{SGO!8@\S~8!4@Abs.~4}
\item Die Anrufungsfrist in Einspruchsverfahren gegen Ordnungsmaßnahmen von 14 Tagen, \S~8 Abs.~4 Satz~2 SGO\index[paridx]{SGO!8@\S~8!4@Abs.~4}
\item Die Anrufungsfrist in Parteiausschlussverfahren, \S~8 Abs.~4 Satz~3 SGO\index[paridx]{SGO!8@\S~8!4@Abs.~4}
\item Die Berufungsfrist gegen ein Urteil mit korrekter Rechtsmittelbelehrung, \S~13 Abs.~2 Satz~1 SGO\index[paridx]{SGO!13@\S~13!2@Abs.~2}
\item Die Berufungfrist gegen ein Urteil ohne korrekte Rechtsmittelbelehrung, \S~13 Abs.~2 Satz~4 SGO\index[paridx]{SGO!13@\S~13!2@Abs.~2}
\end{enumerate}

TODO

\subsection{Zuständigkeit}
\label{Standardworkflow:Zustaendigkeit}
Über die Zuständigkeit hat das Gericht bereits mit Anrufung zu entscheiden, \S~8 Abs.~5 SGO\index[paridx]{SGO!8@\S~8!5@Abs.~5}.
Dies hat den Hintergrund, dass die SGO anders als die staatliche Gerichtsbarkeit, keine Verweisung vom unzuständigen, angerufenen Gericht an das zuständige Gericht vorsieht.
Auch sind die Zuständigkeitskriterien der SGO nach \S~6 SGO\index[paridx]{SGO!\S~6} wesentlich einfacher als etwa die der ZPO.
So ist es bereits aus den Daten der Anrufung ersichtlich, ob ein Gericht zuständig ist.

\subsubsection{Grundlegendes}
\label{Standardworkflow:Zustaendigkeit:Grundlegendes}
Dabei wird instanziell bei den untersten Gerichten, also den Landesschiedsgerichten begonnen, \S~6 Abs.~1 SGO\index[paridx]{SGO!6@\S~6!1@Abs.~1}.
\S~6 Abs.~2 SGO\index[paridx]{SGO!6@\S~6!2@Abs.~2} bestimmt dann noch, dass sie die Zuständigkeit nach der Verbandszugehörigkeit des Antragsgegners richtet.
Verbandszugehörige sind die Mitglieder eines Verbandes, das können nach \S~2 Abs.~1 Satz~2 PartG\index[paridx]{PartG!2@\S~2!1@Abs.~1} nur natürliche Personen sein.
Dabei wird in \S~6 Abs.~2 SGO nochmal klargestellt, was sich aus \S~8 Abs.~5 SGO, der die Entscheidung über die Zuständigkeit mit der Eröffnung als Entscheidung über die Korrektheit der Anrufung zusammenlegt, ergibt: Der relevante Zeitpunkt für die Entscheidung ist immer die mitgliedschaftliche Zuordnung zum Zeitpunkt der Anrufung.
Das wiederrum hängt auch damit zusammen, dass es keine Verweisung zwischen den einzelnen Gerichten gibt und es der anrufenden Streitpartei nicht zugemutet werden soll, dass ein etwaiger Gliederungswechsel oder z.B. eine Verschmelzung von Gliederungen das schon angelaufene Verfahren ungültig macht, weil es im Nachinein unzulässig wurde.

\subsubsection{Zuständigkeit bei Verfahren gegen ein Organ}
\label{Standardworkflow:Zustaendigkeit:Organ}
Organe haben -- juristisch streng genommen -- keine Zugehörigkeit, sondern sind Teil eines Verbandes und handeln für diesen.
Man könnte zwar eine Zugehörigkeit im allgemeinen Sprachgebrauch konstruieren für den Verband, dessen Teil sie sind.
Aber auch in diesem Fall leitet sich daraus keine Zugehörigkeit zu den höheren Verbänden ab.
Daher würde die Sonderzuständigkeitsregelung nach \S~6 Abs.~3 SGO\index[paridx]{SGO!6@\S~6!3@Abs.~3} die Zuständigkeit des Bundesschiedsgerichts für Verfahren gegen Bundesorgane anordnet, die Zuständigkeit für Verfahren gegen Organ von Verbänden unterhalb der Landesebene offenlassen.
Diese wäre dann nicht definiert. Das ist eine offensichtliche planwidrige Regelungslücke, da es nicht gewollt sein kann, dass Verfahren gegen untergeordnete Gliederungen unmöglich sind, zumal die Schiedsgerichtsordnung selbst in \S~2 Abs.~1 Satz~2 SGO\index[paridx]{SGO!2@\S~2!1@Abs.~1} nicht passivlegitimiert ist.
Daher bietet sich eine analoge Anwendung der Zuständigkeitsregelungen für Organe nach \S~6 Abs.~3 SGO\index[paridx]{SGO!6@\S~6!3@Abs.~3} an: Das niedrigste Schiedsgericht auf gleicher oder höherer Verbandsebene wie das Organ, das Antragsgegner ist, ist zuständig.

\subsubsection{Zuständigkeit bei Disziplinarverfahren}
\label{Standardworkflow:Zustaendigkeit:Disziplinarverfahren}
Für die Verfahren mit Disziplinarcharakter, also Verfahren über Einsprüche gegen Ordnungsmaßnahmen und Parteiausschlussverfahren, ist die Zuständigkeit nochmal gesondert in \S~6 Abs.~4 SGO\index[paridx]{SGO!6@\S~6!4@Abs.~4} geregelt.
Die etwaige Einrichtung von Schiedsgerichten unterhalb der Landesebene soll nichts am Verfahrensablauf ändern und das Verfahren etwa durch eine Instanz mehr im Zweifel in die Länge ziehen und gleichzeitig soll in jedem Fall der für Parteiausschlussverfahren in \S~10 Abs.~5 Satz~2 PartG\index[paridx]{PartG!10@\S~10!5@Abs.~5} vorgeschriebene zweizügige Instanzenzug für alle Mitglieder gleich aussehen. Im selben Zug wird eine nicht notwendige zweite Instanz auch für Einsprüche gegen Ordnungsmaßnahmen garantiert.

Durch die exklusive Formulierung sperrt \S~6 Abs.~4 SGO\index[paridx]{SGO!6@\S~6!4@Abs.~4}.
Dadurch könnte man aus der Präsensformulierung des Teilsatzes \enquote{Landesverband […], bei dem der Betroffene Mitglied ist.} schließen, dass ein anderer Zeitpunkt für die Entscheidung über die Zuständigkeit relevant wird.
In Betracht kommen hier sowohl die frühere Zeitpunkt des Beschlusses des Organs über die Verhängung der Ordnungsmaßnahme oder der Einreichung eines Parteiausschlussantrages und der frühere Zeitpunkt der vorgerichtlichen Anhörung des Mitglieds im Disziplinarverfahren als auch der spätere Zeitpunkt der tatsächlichen Entscheidung über die Eröffnung.
Ein späterer Zeitpunkt käme wegen \S~8 Abs.~5 SGO\index[paridx]{SGO!8@\S~8!5@Abs.~5} definitiv nicht mehr in Betracht, eine Flucht aus einem laufenden Parteiausschlussverfahren in einen anderen Verband ist daher definitiv nicht möglich.
Grundsätzlich bietet es sich aus den schon oben geschilderten, sich aus \S~8 Abs.~5 SGO\index[paridx]{SGO!8@\S~8!5@Abs.~5}

Problematisch wird dies allerdings in den Fällen, in denen zwischen Start des Disziplinarverfahrens durch Anhörung, dem Beschluss darüber und der Schiedsgerichtsanrufung ein Verbandswechsel des Betroffenen erfolgt.
Im Falle einer Ordnungsmaßnahme würde dann ein verbandsfremdes Schiedsgericht über die Handlung eines Verbandsorgans entscheiden und auch Ermessenskontrolle über dieses Organ ausüben, im Falle eines Parteiausschlusses wäre dann auch noch die Frage, ob ein schon beschlossener und zum Beschlusszeitpunkt zulässiger, aber noch nicht einreichter Parteiausschlussantrag seine Zulässigkeit verliert, zu entscheiden.

Für die Frage der verbandsfremden Ermessenskontrolle muss allerdins bedacht werden, dass in beiden Fällen das letzinstanzlich zuständige Bundesschiedsgericht das letzte Wort hätte.
Daher scheint ein außeinanderfallen von Verbandszugehörigkeit des ersintanzlich zuständigen Schiedsgerichts und des verhängenden Organs schlussendlich doch unproblematisch, zumal die gesonderte Regelung durch \S~6 Abs.~4 SGO\index[paridx]{SGO!6@\S~6!4@Abs.~4} wohl auch das mitbezweckt hat.

Für den Parteiausschluss wird das wohl davon abhängen, wie man ihn im Vergleich zu den anderen Ordnungsmaßnahmen einsortiert. Wird er einfach nur als stärkste Ordnungsmaßnahme einsortiert, auf die das Prozedere der Ordnungsmaßnahmen nach \SSS~14, 6 Abs.~1 BS\index[paridx]{BS!\S~14}\index[paridx]{BS!6@\S~6!1@Abs.~1} voll anwendbar ist\footnote{So wohl die jüngste Ansicht des Bundesschiedsgerichtes in \cite{BSG3615HS} und die Mindermeinung des Bundesschiedsrichters a.D. Markus Gerstel in \cite{BSG20131005}.}, muss sich der Betroffene wohl damit abfinden, dass er ab Anhörung  bzw. spätestens ab dem Beschluss des Organs über die Einreichung einen Verbandswechsel der Zulässigkeit des Verfahrens nicht mehr entgegenhalten kann, um auch hier eine Gleichheit mit den anderen Ordnungsmaßnahmen zu erreichen.
Denn wenn das betroffene Mitglied  von dem Plan für Parteiaausschlussverfahren wusste und deswegen den Verband wechselt, ist es nicht schutzwürdig, ganz unabhängig davon, ob der Wechsel im Innenverhältnis der Gliederungen nach \S~3 Abs.~2a BS\index[paridx]{SGO!3@\S~3!2@Abs.~2}
Folgt man hingegen der in \cite{BSG20131005} aufgezeigten Rechtssprechungslinie und sieht die Anhörung durch das einleitende Organ im Parteiausschlussverfahren als nicht verpflichtend an, so ergibt sich kein Grund, warum das Mitglied einen solchen Beschluss gegen sich gelten lassen muss, von dem es im Zweifel noch nicht einmal etwas wusste.
Das Vertrauen darauf, dass grundsätzlich nur die Organe der Gliederungen, in denen man tatsächlich aktuell Mitglied ist,  Disziplinarmaßnahmen gegen das Mitglied unternehmen, ist dann schutzwürdig und der relevante Zeitpunkt für die Entscheidung über die Zuständigkeit dürfte wieder mit der Anrufung beim Schiedsgericht zusammenfallen.\footnote{Vgl.  wie schon oben \cite[S. 9]{BSG115HS}, unter II.1.c.c).}

\section{Bescheidung der Eröffnung}
\label{Standardworkflow:Beschluss}
Nach der Prüfung aller dargelegten Kriterien ist über die Eröffnung zu bescheiden.

\subsection{Nachbesserung}
\label{Standardworkflow:Beschluss:Nachbesserung}
Sind die Kriterien nicht erfüllt, weil eine odere mehrere notwendige Angaben ganz oder teilweise fehlen, sollte der anrufenden Partei zunächst die Möglichkeit der Nachbesserung gegeben werden.
Die anrufende Partei sollte dabei möglichst genau auf die fehlenden Angaben hingewiesen werden.
Allerdings ist besonders auf die Frist zu achten.
Eine im Zeitpunkt der Verfristung unvollständige Anrufung ist unheilbar verfristet.
Daher ist eine Anrufung, die fristgerecht, aber unvollständig einging, und deren Anrufungsfrist vor einer Antwort durch das Gericht abgelaufen ist, nicht mit einer Nachbesserungsaufforderung zu beantworten, sondern mit einem Nichteröffnungsbeschluss.
Eine Ausnahme besteht dann, wenn sich die Frist aus dem vorgetragenen nicht berechnen lässt.
Das ist typischerweise der Fall, wenn eine schlichtungspflichtige Anrufung vorliegt, aber Angaben zur Schlichtung und ihrem Umfang v.a. in zeitlicher Hinsicht fehlen.
In diesem Fall kann wegen der Hemmungswirkung der Schlichtung die Frist nicht berechnet werden, daher bietet sich hier zunächst eine Nachbesserungsaufforderung an.

Die Nachbesserungsaufforderung ist allerdings keine satzungsmäßige Pflicht der Schiedsgerichte.
Jedes Schiedsgericht kann eine unvollständige Anrufung auch sofort mit einem Nichteröffnungsbeschluss bescheiden.
Allerdings ist jahrelange Praxis der meisten Schiedsgerichte, zunächst um Nachbesserung zu bitten, da dies allen Beteiligten, auch dem Schiedsgericht selbst unnötige Arbeit ersparen kann.

\subsection{Nichtseröffnungsbeschluss}
\label{Standardworkflow:Beschluss:Nichteroeffnung}
Wenn eine Partei nicht erfolgreich nachbessert bleibt keine andere Option als die Anrufung abzuweisen.
Die Abweisung muss begründet erfolgen, \S~8 Abs.~8 Satz~2 SGO\index[paridx]{SGO!8@\S~8!6@Abs.~6}

Es sollten also genau erläutert werden, warum das Verfahren nicht eröffnet wird.
Gab es mehrere Gründe, sollten auch alle alternativen Gründe angeführt werden.
Einerseits hilft das dem Rechtsschutzschuchenden, beim nächsten Mal etwas nicht zu übersehen, und zudem hilft das für den Fall, dass eine Nichteröffnungsbeschwerde beim Rechtsmittelgericht eingelegt wird, allen Beteiligten, das Verfahren schnell zu bearbeiten.

\subsection{Eröffnungsbeschluss}
\label{Standardworkflow:Beschluss:Eroeffnung}
Sind alle Kriterien erfüllt und die Anrufung daher korrekt im Sinne des \S~8 Abs.~5 SGO\index[paridx]{SGO!8@\S~8!5@Abs.~5}, ist das Verfahren zu eröffnen.

Im Eröffnungsbeschluss sind nach \S~9 SGO\index[paridx]{SGO!9@\S~9} einige Mitteilungen an die Parteien zu machen.
Den Parteien soll mitgeteilt werden:
\begin{enumerate}
\item Das Datum des Eröffnungsbeschlusses.
\item Das Aktenzeichen.
\item Die komplette Besetzung des Spruchkörpers.
\item Bei Änderungen gegenüber der Standardbesetzung die Gründe dafür.
\item Eine Kopie des Anrufungsschreibens.
\item Die Information, das ein Verfahrensvertreter bestellt werden kann, bzw. im Falle, dass eine Verfahrenspartei ein Organ ist, dass sie dies tun muss.
\item Eine Aufforderung zur Stellungnahme an beide Verfahrensparteien zum Verfahren.
\item Die Ladung zur (fern-)mündlichen Verhandlung, sofern sie schon steht mit dem Hinweis, dass auch in Abwesenheit Verhandelt werden kann, \S~10 Abs.~5 Satz~4 SGO\index[paridx]{SGO!10@\S~10!5@Abs.~5}.
\item Der Hinweis, dass die Parteien das Recht haben Richter abzulehen mit Hinweis auf die Präklusion nach \S~5 Abs.~2 Satz~4 SGO\index[paridx]{SGO!5@\S~5!2@Abs.~2}.
\item Gründe, die Richter nach \S~5 Abs.~2 Satz~3 SGO\index[paridx]{SGO!5@\S~5!2@Abs.~2} anzeigen müssen.
\item Nur bei Verfahren über eine Ordnungsmaßnahme oder einen Parteiausschluss: Frage an das Mitglied, ob es ein nichtöffentliches Verfahren wünscht.
\end{enumerate}

\chapterbib
% \end{refsection}

\chapterpreamble{Innerhalb des Organs „Schiedsgericht“, das sich zusammensetzt aus den gewählten Richtern und Ersatzrichtern, stellen die mit dem jeweiligen Verfahren befassten Richter das (entscheidende) „Schiedsgericht“ dar. Die genaue Zusammensetzung ergibt sich aus der Eigenschaft als Schiedsrichter oder (im Fall von Kammersystemen) aus der Geschäftsordnung. Die Zusammensetzung muss von vorn herein feststehen und darf nicht willkürlich verändert werden. Ein einmal befasster Richter kann regelmäßig nur durch Ausschluss aus dem konkreten Verfahren (typischerweise wegen Besorgnis der Befangenheit), generelle Beurlaubung oder Rücktritt vom Amt aus einem Verfahren ausscheiden. Ein temporäres Ausscheiden und spätere Wiederbefassung kommt ebensowenig in Betracht wie eine Rücktritt nur für ein einzelnes Verfahren. Ablehnungsanträge (Anträge auf Feststellung der Besorgnis der Befangenheit) müssen sich immer gegen jeweils eine konkrete Person richten und müssen einzeln (im Zweifel in der zeitlichen Reihenfolge der Antragstellung) entschieden werden. Näheres legt \S~5~SGO\index[paridx]{SGO!\S~5} fest.}

\chapter{Zusammensetzung des Schiedsgerichts}
%\section{Organ}
%\subsection{Mitglieder des Schiedsgerichts}
%Richter ./. Ersatzrichter
%Vorsitzender Richter (inkl. Wahl → aktives/passives Wahlrecht)
%\subsection{Geschäftsordnung} % GO ist für Kapitel "Innere Organisation" geplant – braucht es hier eine eigene Überschrift?
%Exkurs(?): Geschäftsordnung (hier Ersatzrichter mit Stimmrecht?)
Das Schiedsgericht ist zwar kein Organ\index[idx]{Organ} im Sinne des \S~8 Abs.~2 PartG\index[paridx]{PartG!8@\S~8!2@Abs.~2}\footnote{MISSING: Kommentare}, da es nicht der originiären politischen Willensbildung der Partei dient, allerdings ist es grundsätzlich ein Organ im weiteren Sinne.
Dementsprechend haben die Satzungen der Piratenparteien und ihrer Untergliederungen das Schiedsgericht typischerweise auch als Organ bezeichnet.\footnote{MISSING: Satzungen}

\section{Das Organ Schiedsgericht und seine Zusammentzung}
\label{Zusammensetzung:Organ}
Das Organ setzt sich zusammen aus allen gewählten Richtern und Ersatzrichtern.
Für das Organ selbst gibt es dabei noch keine Unterscheidung in der Titelqualität.
Das Gesamtorgan entscheidet dabei nie über konkrete Fälle, sondern nur über die allgemeinen Angelegenheiten des Organs.
Dazu gehören etwa der Beschluss der Geschäftsordnung\index[idx]{Geschäftsordnung} nach \S~2 Abs.~6 SGO\index[paridx]{SGO!2@\S~2!6@Abs.~6} empfiehlt.
Auch etwa gemeinsam formulierte Standardsschreiben, die Zuständigkeiten für die Verwaltung der Dokumentation nach \S~14 SGO\index[paridx]{SGO!\S~14} und sämtliche sonstigen, nicht auf konkrete Verfahren oder Anrufungen bezogenen Tätigkeiten werden von dem gesamten Organ entschieden.
Nicht dazu gehört allerdings die Wahl des Vorsitzenden Richters.
Die Schiedsgerichtsordnung hat diese in \S~3 Abs.~1 Satz~2 SGO\index[paridx]{SGO!3@\S~3!1@Abs.~1} allein in die Hände der Richter gelegt, die Ersatzrichter haben hierbei weder aktives noch passives Wahlrecht.

\section{Der Spruchkörper Schiedsgericht und seine Zusammensetzung}
\label{Zusammensetzung:Spruchkoerper}
Das Schiedgericht im Sinne des entscheidenen Spruchkörpers\index[idx]{Spruchkörper|textbf} für konkrete Verfahren setzt sich dahingegen nach \S~3 -- \S~5 SGO\index[paridx]{SGO!\S~3}\index[paridx]{SGO!\S~4}\index[paridx]{SGO!\S~5} zusammen. Dabei ist im Folgenden mit Schiedsgericht nicht mehr das Organ, sondern der Spruchkörper für eine konkrete Anrufung gemeint.

\subsection{Besetzung}
\label{Zusammensetzung:Spruchkoerper:Besetzung}
Dieser besteht standardmäßig aus den gewählten drei Richtern.
Sobald einer dieser Richter sein Amt gemäß \S~3 Abs.~7, 8 SGO\index[paridx]{SGO!3@\S~3!7@Abs.~7} dauerhaft nach und wird Teil des standardmäßigen Spruchkörpers.
Zur Unterscheidung von Ersatzrichtern, die weiterhin diese Rolle haben, ist er ab diesem Zeitpunkt so zu behandeln, als wäre er ursprünglich als Richter gewählt worden.
Er erhält damit passives und aktives Wahlrecht bei der Wahl des Vorsitzenden Richters\index[idx]{Vorsitzender Richter}.
Dies ist einerseits schon durch das Wort \enquote{dauerhaft} nahegelegt, dass es einen rechtlichen Unterschied zum nur temporären Einrücken in ein einzelnes Verfahren nach \S~4 Abs.~4 SGO\index[paridx]{SGO!4@\S~4!4@Abs.~4} erlaubten erhöhten Ersatzrichterzahl sogar zu einer Situation kommen könne, in der kein verbleibendes Mitglied des Schiedsgerichts aktives oder passives Wahlrecht für die Wahl des Vorsitzenden Richters hätte.
Auf die Besonderheiten der Besetzung des Bundesschiedsgerichtes aufgrund der erhöhten Richterzahl und der Möglichkeit eines Kammersystems wird hier nicht eingegangen.

\subsection{Änderung der Besetzung}
\label{Zusammensetzung:Spruchkoerper:Aenderung}
Die Besetzung\index[idx]{Besetzung|textbf} des Gerichts darf nur in den dafür vorgesehenen Fällen verändert werden.
Dies liegt daran, dass Art.~101 Abs.~2 Satz~2 GG\index[paridx]{GG!Art.~101!Abs.~2 Satz~2} jedermann das Recht auf den gesetzlichen Richter garantiert.
Der gesetzliche Richter ist allerdings nur der, der aufgrund des geltenden Rechts zum Klageeinreichungszeitpunkt dazu bestimmt war.
Dies schließt jede Modifikation durch Einzelfallbeschluss aus.
Einzig und allein die vom Gesetz vorgesehen Gründe können daher zu einer Besetzungsänderung führen.
Allen diesen Gründen ist gemeinsam, dass sie den Konflikt der Garantie des gesetzlichen Richters mit dem Recht auf ein faires, neutrales Verfahren und einen effektiven Rechsschutz auflösen.

Insgesamt gibt folgende Gründe für eine Besetzungsänderung:
\begin{enumerate}
\item Ausschluss wegen Untätigkeit eines einzelnen Richters, \S~4 Abs.~1 SGO.\index[paridx]{SGO!4@\S~4!1@Abs.~1}
\item Ausschluss wegen Krankheit, \S~4 Abs.~3 Satz~1 Alt.~1 SGO.\index[paridx]{SGO!4@\S~4!3@Abs.~3}
\item Ausschluss wegen Beurlaubung, \S~4 Abs.~3 Satz~1 Alt.~2 SGO.\index[paridx]{SGO!4@\S~4!3@Abs.~3}
\item Ausschluss wegen Besorgnis der Befangenheit, \S~5 Abs.~2 Satz 1 Alt.~1, Satz~2 SGO.\index[paridx]{SGO!5@\S~5!2@Abs.~2}
\item Ausschluss wegen satzungsmäßig unwiderlegbar vermuteter Befangenheit, \S~5 Abs.~1 SGO.\index[paridx]{SGO!5@\S~5!1@Abs.~1}
\end{enumerate}

\subsection{Beschlussbesetzung im für den Beschluss über einen Richterausschluss}
\label{Zusammensetzung:Spruchkoerper:Besetzungausschlussbeschluss}
Die Satzung äußert sich nicht konkret zu der Frage, wer an der Entscheidung über den Ausschluss eines Richters mitabstimmen darf.
\S~4 Abs.~4 Satz~2 Alt.~2 SGO\index[paridx]{SGO!4@\S~4!4@Abs.~4} sagt lediglich aus, dass die Beschlussfähigkeit auch mit zwei Richtern gegeben ist.
Aus dem Grundsatz, dass niemand Richter in eigener Sache sein darf\footnote{lat.: nemo iudex in sua causa.}, folgt schon, dass der betroffene Richter selbst nicht mitentscheiden darf. Für die Fälle des \S~5 SGO\index[paridx]{SGO!\S~5} ist das auch ausdrücklich in der Schiedsgerichtsordnung niedergelegt, aber auch in den anderen Fällen muss dieser Grundsatz gelten.
In Betracht kommen daher zwei denkbare Besetzungen: Die verbleibenden Richter der bisherigen Spruchkörperbesetzung entscheiden über den Ausschluss des Richters oder aber \S~4 Abs.~2 Satz~1 SGO\index[paridx]{SGO!4@\S~4!2@Abs.~2} nicht nur die schon per Beschluss ausgeschlossenen Richter umfasst, sondern auch die, über die konkret beschieden werden soll.
Die sonstigen Regeln, insbesondere über die Beschlussfähigkeit, sagen dazu nichts.
Allerdings deutet schon der Wortlaut \enquote{Notbesetzung} an, dass es nicht der Normalfall sein soll.
Der Satzung lassen sich keine Indizien für absichtliche Redundanz der Betonung des Sonderfalls entnehmen.
Daher ist rückt schon für den Beschluss über den Richterausschluss der in der Rangfolge nächste Ersatzrichter temporär ein und entscheidet über den Ausschluss mit.
Dass dies geboten ist, ergibt sich auch schon zwingend aus der Möglichkeit, das zwei Richter nicht erreichbar sein könnten.
In diesem Fall wäre der verbleibende Richter allein nicht mehr als Spruchkörper beschlussfähig.
Ohne den Beschluss über den Ausschluss der nicht erreichbaren Richter könnte dann aber kein Ersatzrichter nachrücken, um eine beschlussfähige Notbesetzung zu erzeugen.
Daher müsste bei einer derartigen Verfahrensweise der verbleibende Richter sich bzw. den Spruchkörper für handlungsunfähig erklären und um Verfahrensverweisung bitten, obwohl doch gerade das Institut der Ersatzrichter unnötige vorschnelle Verfahrensverweisungen verhindern soll.
Auch lässt sich der Wortlaut \enquote{ein befangener oder ausgeschlossener Richter} dahingehend deuten, dass \enquote{ausgeschlossener Richter} diejenigen umfasst, die schon per Beschluss ausgeschlossen wurden und \enquote{befangener […] Richter} diejenigen umfasst, die selbst in einer Entscheidung tatsächlich und unzweifelhaft befangen sind. Letzteres würde aufgrund des Grundsatzes, dass niemand Richter in eigener Sache sein darf, wohl unzweifelhaft zutreffen, sodass auch eine derartige Auslegung des \S~4 Abs.~2 mit dem Wortlaut vereinbar ist.
Daher rückt der in der Reihenfolge nächste Ersatzrichter schon für den Beschluss über den Ausschluss eines Richters ein.

Diese Vorgehensweise, obschon nie ausführlich begründet, entspricht auch der jahrelangen Praxis des Bundesschiedsgerichtes.\footnote{MISSING DERP}

\subsection{Änderung durch Ausschluss eines Richters wegen Untätigkeit}
\label{Zusammensetzung:Spruchkoerper:Untaetigkeit}
Der Ausschluss eines Richters wegen Untätigkeit kann dann Beschlossen werden, wenn ein Richter auch auf Nachfrage nicht mitarbeitet und so die Beschlussfähigkeit des Gerichts in dem Verfahren gefährdet.
\S~4 Abs.~1 SGO\index[paridx]{SGO!4@\S~4!1@Abs.~1} fordert dafür, dass der Richter bereits an Zusammentreffen oder Beratungen in einem anderen Format, etwa per E-Mail-Verkehr im Bezug auf eine konkretes Verfahren nicht beteiligt hat, die übrigen Richter diesen Richter konkret zur Mitarbeit ermahnt haben und ihm dazu eine Frist von mindestens 13 Tagen gegeben haben, sich doch noch einzubringen.
Wenn der Richter trotz alldem nicht reagiert oder sich trotzdem aktiv weigert, mitzuarbeiten und so das Verfahren bzw. Beschlussfassung blockiert, damit die Verfahrensdauer ohne Grund die Länge treibt und den Rechtsschutzanspruch der Streitparteien vereitelt, soll dies nicht zu Lasten der Streitparteien gehen.
In diesem Fall blockiert der gesetzliche Richter gerade den Anspruch auf effektiven Rechtsschutz und entzieht so aktiv den gesetzlichen Richter, der gerade nicht entzogen werden darf.
Daher ist in diesem Fall das verfassungsrechtliche Verbot des Entzugs des gesetzlichen Richters nicht im Konflikt mit der Satzungsbestimmung.
Die Entscheidung über den Ausschluss ist eine Ermessensentscheidung des Spruchkörpers und keine zwingende Folge.
Sie ist dann geboten, wenn anders kein Fortschritt im Verfahren erzielt werden kann oder weiteres Abwarten die Einhaltung der Verfahrensmaximalsolldauer nach \S~12 Abs.~1 SGO\index[paridx]{SGO!12@\S~12!1@Abs.~1} gefährdet.

\subsection{Änderung durch Ausschluss eines Richters wegen Krankheit}
\label{Zusammensetzung:Spruchkoerper:Krankheit}
Der Ausschluss eines Richters wegen Krankheit gemäß \S~4 Abs.~3 Satz~1 Alt.~1 SGO.\index[paridx]{SGO!4@\S~4!3@Abs.~3} ist etwas anders geartet.
Hier kommt regelmäßig eine Ablehnung schon gar nicht in Betracht.
Wenn ein Richter sich dem Gericht gegenüber krank meldet, ist dies hinzunehmen und den Parteien mitzuteilen.
Die Krankmeldung ist dabei zu den Akten zu nehmen.
Es erfordert somit regelmäßig gerade keinen Beschluss über die Krankheit.
Der Ausschluss ist auch insofern anders, als dass er für alle aktuell laufenden Verfahren gilt und abgesehen von der Mitteilung an die Parteien nicht für jedes Verfahrens ein eigenes Prozedere erfordert.
Die Meldung der Krankheit ist daher eine temporäre Amtsniederlegung.
Allerdings ist die Schiedsgerichtsordnung an dieser Stelle hochproblematisch, da sie von einem lediglich temporären Ausscheiden und der Möglichkeit der Rückkehr in das Verfahren ausgeht.
Dies würde Prozesstaktiken ermöglichen, die defacto einer parteigetriebenen Richterauswahl nahekommen, indem das Verfahren von Seite der Streitpartei versucht wird, zu beschleunigen oder zu verzögern.
Daher muss davon ausgegangen werden, dass die entsprechende Satzungsbestimmung wegen Verstoß gegen das Verbot des Entzugs des gesetzlichen Richters unwirksam ist.\footnote{So schon \cite[299]{BVerfGE17294}: \enquote{Art. 101 Abs. 1 Satz 2 GG soll der Gefahr vorbeugen, daß die Justiz durch eine Manipulierung der rechtsprechenden Organe sachfremden Einflüssen ausgesetzt wird, insbesondere daß im Einzelfall durch die Auswahl der zur Entscheidung berufenen Richter ad hoc das Ergebnis der Entscheidung beeinflußt wird, gleichgültig, von welcher Seite die Manipulierung ausgeht}.}
Auch angesichts typischer Verfahrensdauern im Rahmen der erlaubten Verfahrensmaximalsolldauer ist jeder Zeitraum, der überhaupt zu einem temporären Ausscheiden eines Richters führt, schon so relevant, dass das Verfahren vom eingerückten Ersatzrichter maßgeblich mitbeeinflusst werden konnte.
Daher muss auch dieser weiterhin dem Spruchkörper angehören.
Entsprechend kann ein einmal ersetzter Richter nicht wieder in ein Verfahren zurückkehren, sobald den Parteien sein Ausscheiden und das Eintreten eines Ersatzrichters mitgeteilt wurde.\footnote{So auch etwa \cite[Eschelbach][\S~15 Rn~31]{BVerfGGMitarbeiterKommentar} entsprechend für das Bundesverfassungsgericht. Dieser geht davon aus, dass ein Rückkehr in die Richterolle im Spruchkörper für den Vorsitzenden nur dann möglich ist, wenn er lediglich in seiner Rolle als Vorsitzener etwa wegen Heiserkeit, nicht aber in der Rolle als Richter gänzlich wegfiel.}

\subsection{Änderung durch Ausschluss eines Richters wegen Beurlaubung}
\label{Zusammensetzung:Spruchkoerper:Urlaub}
In die selbe Kategorie wie der Ausschluss eines Richters wegen Krankheit fällt der Ausschluss eines Richters wegen Beurlaubung gemäß \S~4 Abs.~3 Satz~1 Alt.~2 SGO.\index[paridx]{SGO!4@\S~4!3@Abs.~3}.
Hier gelten aber die Bedenken über die Prozessmanipulation in Richtung einer effektiv parteigetriebenen Richterauswahl umso mehr, da das Ende der voraussichtlichen Abwesenheit noch besser vorhersehbar ist.
Daher muss hier das gleiche gelten: Die Regelung ist bezüglich des Widereinstritts eines beurlaubten Richters in das Verfahren unwirksam und ein Widereinstritt ausgeschlossen, sobald den Parteien sein Ausscheiden und das Eintreten eines Ersatzrichters mitgeteilt wurde.

\subsection{Änderung durch Ausschluss eines Richters wegen Besorgnis der Befangenheit}
\label{Zusammensetzung:Spruchkoerper:Befangenheitsbesorgnis}
Das Parteiengesetz fordert in \S~14 Abs.~4\index[paridx]{PartG!14@\S~14!4@Abs.~4} lediglich, dass jede Streitpartei Richter wegen Befangenheit ablehnen kann.
Auch wenn hier der Wortlaut des Parteiengesetzes andeutet, dass es lediglich bei tatsächlicher Befangenheit ein solches Ablehnungsrecht geben muss, liegt es viel näher, die Anforderung soweit auszulegen, dass die Schiedsgerichtsordnung schon für die Besorgnis der Befangenheit ein Ablehnungsrecht garantieren muss.\footnote{MISSING: Lenski}
Dieser akademische Streit um die Auslegung des \S~14 PartG\index[paridx]{PartG!\S~14} ist jedoch für Verfahren in der Schiedsgerichtsbarkeit der Piratenpartei Deutschland nicht relevant.
Dies Schiedsgerichtsordnung garantiert in \S~5 Abs.~2 SGO\index[paridx]{SGO!5@\S~5!2@Abs.~2} ein Recht auf Ablehnung eines jeden Richters bei Besorgnis seiner Befangenheit.


\subsubsection{Prozessusales}
\label{Zusammensetzung:Spruchkoerper:Befangenheitsbesorgnis:Prozessuales}
Einen Antrag auf Ablehnung eines Richters wegen Besorgnis eines Richters kann nur eine Verfahrenspartei stellen, nicht aber der Richter selber.

Der Antrag muss immer begründet sein, \S~5 Abs.~3 Satz~1 SGO\index[paridx]{SGO!5@\S~5!3@Abs.~3}, und sich gegen einene einzelnen Richter richten, andernfalls ist er unzulässig.\footnote{\cites[S.~1]{BSG201305062BefangenheitI}{BGHIIARZ101}.}
Eine Ablehnung des ganzen Gerichts auf einmal ist also nicht möglich, allerdings können durchaus mehrere oder sogar alle Richter eines Gerichts jeweils einzeln nacheieinader abgelehnt werden.
Entscheidend für die Bescheidungsreihenfolge ist ann die Reihenfolge der Antragstellung.
In besonderen Fällen ist sogar der Verweis auf die gleiche Begründung für mehrere Richter zulässig, nämlich immer dann, wenn vorgetragen wird, das die die Besorgnis der Befangenheit tragende Tatsache auf mehrere Richter jeweils individuell gleichzeitig zutrifft.
In allen anderen Fällen ist der Ablehnungsantrag aber unzulässig und muss als solcher direkt abgewiesen werden.

Ein unzulässiger Ablehnungsantrag wird entgegen \S~5 Abs.~5 SGO\index[paridx]{SGO!5@\S~5!5@Abs.~5} in originaler, ungeänderter Besetzung beschieden, da schon gar kein formal korrekter Befangenheitsantrag im Sinne des \S~5 Abs.~5 SGO\index[paridx]{SGO!5@\S~5!5@Abs.~5} vorliegt.\footnote{\cites[S.~1]{BSG201305062BefangenheitI}{BSG201305062BefangenheitII}.}

Sobald ein zulässiger Ablehnungantrag gegen einen Richter gestellt ist, darf dieser nicht mehr an verfahrensleitenden Entscheidungen teilnehmen, bis der Antrag abgelehnt wurde oder angnommen wurde, \S~5 Abs.~4 SGO.\index[paridx]{SGO!5@\S~5!4@Abs.~4}
Im letzten Fall ist mit Beschluss aus dem Verfahren ausgeschieden und wird durch den nächsten Ersatzrichter ersetzt, \S~4 Abs.~2 SGO.\index[paridx]{SGO!4@\S~4!2@Abs.~2}

Vor der Entscheidung muss der Richter selbst sich zur Begründung des Befangenheitsgesuches äußern.
Dabei sollte sie diese dienstliche Stellungnahme auf die Tatsachen, die die Besorgnis der Befangenheit stützen sollen, beschränken und die rechtliche Wertung den Richterkollegen überalssen werden.
Beiden Parteien daraufhin noch eine Gelegenheit zur Stellungnahme zu geben, \S~5 Abs.~3 Satz~2 SGO\index[paridx]{SGO!5@\S~5!3@Abs.~3}, danach kann in geänderter Besetzung nach \SSS~5 Abs.~5, 4 Abs.~2 SGO\index[paridx]{SGO!5@\S~5!5@Abs.~5}\index[paridx]{SGO!4@\S~4!2@Abs.~2} über den Befangenheitsantrag beschieden werden.

\subsubsection{Tatbestand der Besorgnis der Befangenheit}
\label{Zusammensetzung:Spruchkoerper:Befangenheitsbesorgnis:Tatbestand}
Für den Beschluss, ob ein Richter im Verfahren bleibt, oder wegen Besorgnis der Befangenheit aus dem Verfahren ausscheidet, muss entschieden werden, ob ein neutraler, objektiver fiktiver Dritter, der in Kenntnis der Tatsachen zu entscheiden hätte, einen gerechtfertigten Grund hätte, an dem Unparteilichkeit des Richters zu zweifeln, \S~5 Abs.~2 Satz~2.\index[paridx]{SGO!5@\S~5!2@Abs.~2}
Dies ist noch nicht erfüllt, wenn der Richter einer speziellen parteiinternen Organisation oder einem Flügel angehört.\footnote{\cite[S.~6]{BSG115HSBefangeheitIII}.}
Ebenfalls genügt dafür nicht, dass der Richter sich bereits früher mal über eine verfahrensrelevante Rechtsfrage geäußert hat,\footnote{\cite{BGHXIZR38801}.} die Partei muss dann damit leben, dass ihr gesetzlicher Richter in einem rechtlichen Meinungsstreit eine gewisse Position vertritt.
Erst Recht gilt dies in einer Partei, die ihre Richter aus dem Pool aktiver Mitglieder rektrutiert.\footnote{\cites[Lenski][\S~14 Rn~15]{lenski2011parteiengesetz}.}
Erst recht liegt keine Besorgnis der Befangenheit vor, weil eine Partei davon ausgeht, dass ein Richter anders als von ihr gewünscht entscheiden wird. Unerwünschte Entscheidungen sind bei Gerichten der Regelfall.\footnote{\cites[S.~2]{BSG201305062BefangenheitI}.}

\subsubsection{Rechtsmissbräuchlichkeit der Ablehnung}
\label{Zusammensetzung:Spruchkoerper:Befangenheitsbesorgnis:Rechtsmissbrauch}
Es kann auch sein, dass ein Antrag auf Richterablehnung eigentlich zulässig wäre, aber im speziellen Fall rechtsmissbräuchlich ist und daher als unzulässig abzulehnen ist.
Dies ist eine wertende Entscheidung und nicht eine bloße Formalentscheidung und muss daher in der geänderten Besetzung nach \SSS~5 Abs.~5, 4 Abs.~2 SGO\index[paridx]{SGO!5@\S~5!5@Abs.~5}\index[paridx]{SGO!4@\S~4!2@Abs.~2} beschieden werden.
Allerdings kann auf die Stellungnahme und die Anhörung der Parteien verzichtet werden, wenn die geänderte Besetzung direkt zur Überzeugung gelangt, dass der Antrag rechtsmissbräuchlich ist, da dann kein zulässiger Ablehnungsantrag vorliegt.
Im Zweifel gilt allerdings immer: Eine Anhörung zu viel schadet nicht, eine Anhörung zu wenig kann zur Aufhebung des Urteils führen.

Rechtsmissbräuchlich ist der Antrag auf Richterablehnung, wenn er nur gestellt wird, um das Verfahren aufzuhalten\footnote{\cites[S.~8]{BSG2815HS}.}, das Gericht zu überlasten oder mit offensichtlich nicht aussichtsreichen Befangenheitsanträgen zu bombardieren\footnote{\cites[S.~4]{BSG201305062BefangenheitII}.} oder einen nicht genehmen Richter allein wegen seiner Spruchtätigkeit oder Rechtsüberzeugung abzulehnen\footnote{\cites[S.~5]{BSG115HSBefangeheitIV}{OLGNaumburg3WF7609}.}

\subsection{Änderung durch Ausschluss eines Richters wegen satzungsmäßig unwiderlegbar vermuteter Befangenheit}
\label{Zusammensetzung:Spruchkoerper:Befangenheitsvermutung}
Ein Richter kann durch eine Streitpartei nach \S~5 Abs.~2 Satz~1 Alt.~2, Abs.~1 SGO\index[paridx]{SGO!5@\S~5!2@Abs.~2} abgelehnt werden.
Die Möglichkeit, dass die Streitparteien einen solchen Antrag stellen können, darf aber nicht über die Natur des \S~5 Abs.~1 SGO\index[paridx]{SGO!5@\S~5!1@Abs.~1} täuschen.
Es geht dabei um eine Vorschrift, die von Amts wegen von jedem Gericht gerpüft werden muss, bevor ein Richter an einem Verfahren teilnehmen kann\footnote{So auch schon \cite{BSGPP100127862}, daher die dortige Vorabprüfung.}. \S~5 Abs.~1 Satz~1 SGO\index[paridx]{SGO!5@\S~5!1@Abs.~1} stellt verschiedene objektive Bedingungen auf, die ohne wertenen Ermessenspielraum verbieten, dass ein Richter sein Amt in einem Verfahren ausübt.

\subsubsection{Prozessual Grundlegendes}
\label{Zusammensetzung:Spruchkoerper:Befangenheitsvermutung:Prozessuales}
\S~5 Abs.~1 Satz~2 SGO\index[paridx]{SGO!5@\S~5!1@Abs.~1} fordert lediglich, dass ohne den betroffenen Richter beschieden wird.
Insofern rückt der in der Rangfolge nächste Ersatzrichter für diese Entscheidung in den Spruchkörper ein und die Parteien sind davon in Kenntnis zu setzen, \S~4 Abs.~2 Satz~2 SGO.\index[paridx]{SGO!4@\S~4!2@Abs.~2}
Weitere Vorschriften macht die SGO nicht für den Fall, dass die von Amts wegen zu treffende Entscheidung, ob ein Richter von der Ausübung seines Amtes ausgeschlossen ist, beschieden wird.
Anders könnte es aussehen, wenn eine Streitpartei die Entscheidung darüber nach \S~5 Abs.~2 Satz~1 Alt.~2 SGO\index[paridx]{SGO!5@\S~5!2@Abs.~2} beantragt hat.
Denn dem Wortlaut der Schiedsgerichtsordnung nach treffen die prozessualen Vorschriften des \S~5 Abs.~2 ff. SGO\index[paridx]{SGO!5@\S~5!2@Abs.~2}
Demnach wäre das Verfahren unterschiedlich, je nachdem, ob es durch einen Antrag einer Streitpartei ausgelöst wird oder von den Richtern selbst.
Dafür gibt es allerdings keinen Grund, der das sachlich rechtfertigen würde.
Daher müssen die Prozessvorschriften des \S~5 Abs.~2 ff. SGO\index[paridx]{SGO!5@\S~5!2@Abs.~2} einzeln geprüft werden, ob sie auf die Eigenheiten der Richterablehnung wegen satzungsmäßigem Ausschlussgrund anwendbar sind oder nicht.
Für den Fall, dass sie anwendbar sind, müssen sie in beiden denkbaren Verfahrenswegen angewendet werden.

\subsubsection{Anwendung der richterlichen Anzeigepflicht}
\label{Zusammensetzung:Spruchkoerper:Befangenheitsvermutung:Anzeigepflicht}
Richter sind gemäß \S~5 Abs.~2 Satz~3 SGO\index[paridx]{SGO!5@\S~5!2@Abs.~2} verpflichtet, jeden Umstand dem Spruchkörper und den Streitparteien anzuzeigen, die einen Antrag auf Ablehnung begründen könnten.
Der Konjunktiv der Formulierung gebietet daher eine solche Anzeige auch schon, wenn der Richter überzeugt ist, dass kein Tatbestand des \S~5 Abs.~1 Satz~1 SGO\index[paridx]{SGO!5@\S~5!1@Abs.~1} erfüllt ist, eine solche Erfüllung aber denkbar wäre.
Diese Anzeigepflicht muss gerade und insbesondere für die abschließende Liste der klar umrissenen Tatbestände des \S~5 Abs.~1 Satz~1 SGO\index[paridx]{SGO!5@\S~5!1@Abs.~1} gelten.
Die Anzeige löst auch sofort eine Bescheidungspflicht durch die anderen Richter nach \S~5 Abs.~1 Satz~2 SGO\index[paridx]{SGO!5@\S~5!1@Abs.~1} aus, eines Antrags einer Partei bedarf es dann nicht mehr.

\subsubsection{Anwendung der Präklusionsregelung}
\label{Zusammensetzung:Spruchkoerper:Befangenheitsvermutung:Praeklusion}
\S~5 Abs.~2 Satz~4 SGO\index[paridx]{SGO!5@\S~5!2@Abs.~2} präkludiert Streitparteien davon, einen Richter abzulehnen, wenn der Grund für die Ablehnung schon ihr bekannt war und sie sich dennoch auf eine Verhandlung mit dem Richter eingelassen hat.
Auf den satzungsmäßigem Ausschluss angewendet würde dies bedeuten, dass ein Richter, dem die Satzung objektiv und ohne Wertungsspielraum die Ausübung seines Amtes im Verfahren versagt ist, dennoch teilnehmen kann, wenn über den vorgetragenen Ausschlussgrund wegen Präklusion nicht mehr entschieden werden müsste.
Dies kann jedoch keine gewollte Rechtsfolge sein, auch da der Richter diesen Grund schon hätte aufgrund seiner Anzeigepflicht mitteilen hätte müssen.
Das der Ausschlussgrund für den Richter nicht erkennbar war und die Streitpartei daher prozesstaktisch Wissen zurückgehalten hat, ist angesichts der klaren und abschließenden Liste der möglichen Tatbestände nicht vorstellbar.
Auch müsste das Gericht, sobald es Kenntnis von der Möglichkeit eines satzungsmäßigen Ausschlussgrundes nach \S~5 Abs.~1 Satz~1 SGO\index[paridx]{SGO!5@\S~5!1@Abs.~1} hat, von Amts wegen über den Ausschluss entscheiden.
Daher würde die Anwendung der Präklusionsregelung zu der absurden Situation führen, dass der Antrag der Streitpartei abgelehnt werden müsste, aber inhaltlich dennoch beschieden werden müsste.
Dies kann keine gewollte Folge sein.
Daher ist die Präklusionsregelung nicht anwendbar.

\subsubsection{Anwendung des Begründungsgebotes}
\label{Zusammensetzung:Spruchkoerper:Befangenheitsvermutung:Begruendungsgebot}
Das Begründungsgebot aus \S~5 Abs.~3 Satz~1 SGO\index[paridx]{SGO!5@\S~5!3@Abs.~3} ist unproblematisch anwendbar.
Wenn eine Partei der Ansicht ist, dass einer der Tatbestände aus \S~5 Abs.~1 Satz~1 SGO\index[paridx]{SGO!5@\S~5!1@Abs.~1} zutrifft, muss sie auch ausführen, welcher das ist und warum er zutrifft.
Andernfalls könnte mit derartigen Anträgen das Gericht bis zum Rande der Arbeitsunfähigkeit blockiert werden, wenn die nicht betroffenen Richter im Wege der Amtsermittlung jeder Behauptung ohne den Ansatz einer Begründung nachgehen müssten.

\subsubsection{Anwendung des Regelung zur dienstlichen Stellungnahme und Parteistellungnahme}
\label{Zusammensetzung:Spruchkoerper:Befangenheitsvermutung:Stellungnahme}
Die dienstliche Stellungnahme selbst dient der Aufklärung des Tatbestandes und ist daher anzuwenden.
Ebenfalls dient sie dazu, den Parteien ihre Stellungnahme jeweils erst zu ermöglichen.
Diese wiederrum ist aus dem von der Verfassung garantierten Recht auf rechtliches Gehör nach Art.~103 Abs.~1 GG\index[paridx]{GG!Art.~103!Abs.~1} geboten.
Es wird über eine Änderung ihrer Gerichtsbesetzung entschieden, daher sind die Parteien anzuhören.
Die Regelung über die dienstliche Stellungnahme und die abschließenden Parteistellungnahme nach \S~5 Abs.~2 Satz~2, 3 SGO.\index[paridx]{SGO!5@\S~5!2@Abs.~2}

\subsubsection{Anwendung der Besetzungsregelung}
\label{Zusammensetzung:Spruchkoerper:Befangenheitsvermutung:Besetzung}
Die Besetzungregelung wäre schon aufgrund der allgemeinen juristischen Prozessgrundsätze geboten (Siehe schon \ref{Zusammensetzung:Spruchkoerper:Besetzungausschlussbeschluss}), allerdings ist die Grundlage hier \S~5 Abs.~1 Satz~2 SGO\index[paridx]{SGO!5@\S~5!1@Abs.~1}.

\subsubsection{Anwendung der Rechtsmittel}
\label{Zusammensetzung:Spruchkoerper:Befangenheitsvermutung:Rechtsmittel}
Gegen die Anwendung der Rechtsmittelregelungen in \S~5 Abs.~6 SGO\index[paridx]{SGO!5@\S~5!6@Abs.~6} anzuwenden.

\subsubsection{Grundegendes zu den Tatbeständen}
\label{Zusammensetzung:Spruchkoerper:Befangenheitsvermutung:Tatbestandsgrundsaetze}
Jeder der Tatbestände des \S~5 Abs.~1 Satz~1 SGO\index[paridx]{SGO!5@\S~5!1@Abs.~1} stellt eine satzungsmäßige, unwiderlegliche Vermutung der Befangenheit des Betroffenen Richters auf.
Es kommt somit für die einzelnen Tatbestände nicht mehr darauf an, ob ein neutraler objektiver Prozessfremder Beobachter nach verständiger Würdigung der Details Zweifel an der Unparteilichkeit eines Richters haben kann.
Sobald der Tatbestand erfüllt ist, ist ein Richter von der Mitwirkung ausgeschlossen.

Die Regelung ist an die staatlichen Prozessordnungen angelehnt.\footnote{Vgl. etwa \S~41 ZPO\index[paridx]{ZPO!\S~41},  \S~22 StPO\index[paridx]{StPO!\S~22}\nomenclature{StPO}{Strafprozessordnung}, \S~54 VwGO\index[paridx]{VwGO!\S~54}\nomenclature{VwGO}{Verwaltungsgerichtsordnung}, \S~51 FGO\index[paridx]{FGO!\S~51}\nomenclature{FGO}{Finanzgerichtsordnung}, \S~60 SGG\index[paridx]{SGG!\S~60}\nomenclature{SGG}{Sozialgerichtsgesetz} und \S~19 BVerfGG\index[paridx]{BVerfGG!\S~19}\nomenclature{BVerfGG}{Bundesverfassungsgerichtsgesetz}.}
Diese Regelungen kommen aufgrund der vom Satzungsgeber getätigten Regelung in \S~5 Abs.~1 SGO\index[paridx]{SGO!5@\S~5!1@Abs.~1}.
Grundsätzlich lässt sich sagen, dass eine Richterablehnung, die nicht durch einen Antrag auf Festsstellung der Besorgnis der Befangenheit nach \S~5 Abs.~2 Satz 1 Alt.~1, Satz~2 SGO\index[paridx]{SGO!5@\S~5!2@Abs.~2} gefüllt werden darf.
Daher ist die Richterablehnung tatsächlich ein Gebiet der SGO, in der von einer vollständigen Regelung des Satzungsgebers auszugehen ist und Analogien grundsätzlich nicht in Betracht kommen.

Die Tatbestände des \S~5 Abs.~1 SGO\index[paridx]{SGO!5@\S~5!1@Abs.~1} gliedern sich in drei Gruppen.

Die erste Gruppe von Tatbeständen (Nr.~1 -- Nr.~5) erfassen Konstellationen, in denen die Satzung dem Richter aufgrund seiner persönlichen Nähe zu den \textbf{Verfahrensparteien} nicht zutraut, einen Fall neutral bewerten zu können, ohne dass der Entscheidung der Makel dess Misstrauens der Unparteilichkeit des Richters anhaftet.
Die Gruppe lässt sich nochmal in zwei Untergruppen unterteilen:
Die erste Untergruppe (Nr.~1 -- Nr.~3) begründet die Nähe in der personenstandsrechtlichen Nähe zu den Verfahrensparteien und dem darin vermuteten Interessenkonflikt.
Die zweite Untergruppe (Nr.~4 -- Nr.~4) begründet die Nähe durch die Möglichkeit der Einflussnahme auf die Entscheidungsfindung in der konkreten Sache sowie generell. Auch die Möglichkeit, etwa schon Prozessstrategien oder anderes Vorwissen zum Verfahrensgegenstand aus Sicht der Partei zu kennen, dürfte hier eine Rolle gespielte haben für den Satzungsgeber. Insofern ist die Untergruppe auch verwandt mit der zweiten Gruppe.

Die zweite Gruppe von Tatbeständen (Nr.~6 -- Nr.~7) erfassen Konstellationen, in denden die Satzung dem Richter aufgrund seiner persönlichen Nähe zum \textbf{Verfahrensgegenstand} nicht zutraut, einen Fall neutral bewerten zu können, ohne dass der Entscheidung der Makel dess Misstrauens der Unparteilichkeit des Richters anhaftet.

Die dritte Gruppe umfasst nur noch einen Tatbestand (Nr.~8), und somit die Konstellation, in denden die Satzung dem Richter aufgrund seiner persönlichen Nähe zum \textbf{Vorverfahren} nicht zutraut, einen Fall neutral bewerten zu können, ohne dass der Entscheidung der Makel dess Misstrauens der Unparteilichkeit des Richters anhaftet. An sich ist das sehr nah an der zweiten Gruppe, aber doch nochmal verschieden.

\subsubsection{Tatbestand \S~5 Abs.~1 Satz~1 Nr.~1 SGO\index[paridx]{SGO!5@\S~5!1@Abs.~1}}
\label{Zusammensetzung:Spruchkoerper:Befangenheitsvermutung:Nr1}
\S~5 Abs.~1 Satz~1 Nr.~1 SGO\index[paridx]{SGO!5@\S~5!1@Abs.~1}
Es versteht sich von selbst, dass ein Richter in Verfahren, denen er selbst Verfahrenspartei ist, nicht mitentscheiden soll.
Dies war auch schon vor Einführung des Tatbestandes aus dem allgemeinen Rechtsgrundsatz automatisches Praxis der Parteirechtssprechung\footnote{So hat das damalige Bundesschiedsgericht es in \cite{BSG3014HS}, \cite{BSG4414HS} und \cite{BSG3215HS} etwa schon gar nicht mehr für nötig gehalten, zur Erläutern, warum der zur ordentlichen Besetzung gehörende Richter Florian Zumkeller-Quast an der tenorierten Entscheidung nicht teilnahm.}, wurde aber nun explizit in der Satzung festgehalten.

\subsubsection{Tatbestand \S~5 Abs.~1 Satz~1 Nr.~2 SGO\index[paridx]{SGO!5@\S~5!1@Abs.~1}}
\label{Zusammensetzung:Spruchkoerper:Befangenheitsvermutung:Nr2}
\S~5 Abs.~1 Satz~1 Nr.~2 SGO\index[paridx]{SGO!5@\S~5!1@Abs.~1} vermutet unwiderleglich, dass ein Richter, dessen Ehe- oder Lebenspartner Verfahrensrenspartei ist, befangen ist.
Lebenspartner ist hier als Rechtsbegriff nach Lebenspartnerschaftsgesetz zu verstehen.
Das ergibt sich einerseits aus der Alternativnennung zur Ehepartnerschaft und ihrer Ähnlichkeit als nicht nur soziale, sondern auch rechtlich-wirschaftliche Verknüpfung, andererseits aus Halbsatz 2, der festhält, dass der Ausschluss auch gilt bei einer nicht mehr fortbestehenden Lebenspartnerschaft, da daraus folgt, dass es sich um ein Rechtsverhältnis und kein reines Sozialverhältnis handeln muss.
Es wäre einem Richter, der verpflichtet ist, das Erfülltsein des Tatbestandes nach \S~5 Abs.~2 Satz~3 SGO\index[paridx]{SGO!5@\S~5!2@Abs.~2} offenzulegen, auch nicht zuzumuten, jegliche frühere Beziehung in seinem Leben in einem Verfahren offenzulegen, wenn keinerlei sonstiger Bezug zum konkreten Verfahren besteht.
Auch würde der Satzung offensichtlich eine Definition fehlen, wann eine Lebenspartnerschaft beginnt und endet.
Ein solch unbestimmter Begriff wäre angesichts der eingriffsintensiven Rechtsfolge der absolut unwiderleglichen Befangenheitsvermutung schon eng auszulegen.
All dies spricht letzliche dafür, dass der Satzungsgeber hier keine länge andauernde Beziehung gemeint ist, sondern nur die verrechtlichte Lebenspartnerschaft nach Lebenspartnerschaftsgesetz.

\subsubsection{Tatbestand \S~5 Abs.~1 Satz~1 Nr.~3 SGO\index[paridx]{SGO!5@\S~5!1@Abs.~1}}
\label{Zusammensetzung:Spruchkoerper:Befangenheitsvermutung:Nr3}
\S~5 Abs.~1 Satz~1 Nr.~3 SGO\index[paridx]{SGO!5@\S~5!1@Abs.~1} Verfahrenspartei ist, befangen ist.
Dabei kommt es der Satzung nicht darauf an, ob die Verwandschaft beendet wurde, etwa durch Vaterschaftsanfechtung oder Adoption.
Entsprechend müssen Richter ihre Verwandschaft zu Verfahrensparteien offenlegen.
Die Satzung macht keine Vorraussetzungen an den Grad, sodass rein vom Wortlaut her auch sehr entfernte Verwandschafts- und Verschwägerungsgrade eine unwiderlegliche Befangenheitsvermutung begründen würden.
Daher gebietet der gesetzliche Richter, dass eine Grenze dort zu ziehen ist, wo der Grad so entfernt ist, dass eine Befangenheitsvermutung im allgemeinen nicht mehr durch die bloße Verwandschaft getragen werden kann.
Bisher gibt es in der Partei keine Rechtsprechung dazu.
Hier bietet sich daher eine enge Wortlautauslegung an, dass lediglich eine Verwandtschaft in gerade Linie sowie eine Verschwägerung ersten Grades den Tatbestand erfüllen.
Dies führt zwar dazu, dass schon Geschwister den Tatbestand nicht mehr erfüllen, allerdings lässt sich dies immernoch durch die Möglichkeit der Richterablehnung wegen Besorgnis der Befangenheit für die Fälle, in denen diese tatsächlich notwendig ist, abdecken, während jede andere Wortlautdeutung willkürlich wäre, ohne einen tatsächlichen Anker im Wortlaut, oder aber vermutlich so gut wie alle Parteimitglieder erfassen würde, da auch Verwandtschaften über die Ecke von 40 Generationen und mehr erfasst würden.
Eine enge Auslegung wird der Rechtsfolge und somit dem Normcharakter eher gerecht, als eine weite Auslegung oder gar ein willkürliche Grenzziehung ohne Verankerung im Wortlaut.

\subsubsection{Tatbestand \S~5 Abs.~1 Satz~1 Nr.~4 SGO\index[paridx]{SGO!5@\S~5!1@Abs.~1}}
\label{Zusammensetzung:Spruchkoerper:Befangenheitsvermutung:Nr4}
\S~5 Abs.~1 Satz~1 Nr.~4 SGO\index[paridx]{SGO!5@\S~5!1@Abs.~1} vermutet unwiderleglich, dass ein Richter unwiderleglich als befangen gilt, wenn eine der Personen nach Nr.1 -- 3 einem Organ angehören, dass Streitpartei ist.
Organ ist hier jedes Organ des Bundesverbandes oder einer Parteigliederung, die grundsätzlich immer aktiv- und passivlegitimiert sind.\footnote{Dazu siehe \ref{Standardworkflow:Parteifaehigkeit}.}

Auch diese Vorschrift deutet im Übrigen an, dass der Satzungsgeber bei Nr.~3 keine große, unübersichtliche Personengruppe im Blickfeld hatte und daher die vorgeschlagene enge Auslegung dort geboten ist.

Problematisch ist diese Vorschrift dennoch.
Organe gibt es auf Bundesebene vier: Den Parteitag, den Vorstand, die Gründungsversammlung und das Schiedsgericht.
Soweit bekannt, haben die bundesweiten Gliederungen kaum weitere Organe eingeführt.
Mitgliederabstimmungen außerhalb der klassischen ortsgebundenen Parteitagen sind meist als besondere Parteitagstagung (SMV) oder Urabstimmung (BEO) organisiert, Präsidien, Beiräte oder andere Organ gibt es nicht.
Lediglich der Landesverband Brandenburg hat wohl noch zusätzliche Organe, und zwar seine Arbeitsgemeinschaften.
Dies dürfte aufgrund der Rarität dieser Regelung kaum im Blickfeld des Bundessatzungsgebers gewesen sein, jedenfalls findet sich weder in der Satzung oder sonst noch wo ein Anhaltspunkt dafür, daher sind derartige zusätzliche Organe für diese Betrachtung vorerst vernachlässigbar, auch wenn die Norm natürlich ebenso für diese gilt.

Schiedsgerichte können nach \S~8 Abs.~7 SGO\index[paridx]{SGO!8@\S~8!7@Abs.~7} schon gar keine Verfahrensbeteiligten sein, sodass sie nicht von diesem Tatbestand gemeint sein können.

Mitglieder von Vorständen können nach \S~3 Abs.~6 SGO\index[paridx]{SGO!3@\S~3!6@Abs.~6} nicht Mitglieder eines Schiedsgerichtes sein, da es aber für diesen Tatbestand aufgrund der Präsensformulierung nur auf gegenwärtige Organmitgliedschaften ankommt, kommen für diese Konstellation nur Nr.~2 und Nr.~3 in Betracht, Nr.~1 kann nie verwirklicht werden.

Die Gründungsversammlung ist von ihrer Natur her in der Piratenpartei dem Parteitag gleich und wird daher wie dieser betrachtet. Das Bundesschiedsgericht hat in seiner Entscheidung\footnote{\cite{BSGPP100127862}.} zu ebendiesem Tatbestand entschieden, dass nur Organe der \enquote{Exekutive} erfasst seien, und damit gerade der Parteitag und somit auch die Grundsversammlung als satzungsgebendes (oder auch: rechtsetzendes) Organ und somit Organ der \enquote{Legislative} schon gar nicht erfasst wäre.
Diese Auslegung findet schon keinen Halt im Wortlaut, sondern würde eine Reduktion nach dem Sinn der Norm darstellen.
Warum diese geboten sein soll, hat das Bundesschiedsgericht nicht begründet.
Im Gegenteil, der Tatbestand von Nr.~1 in Verbindung mit Nr.~4 könnte in dieser Auslegung des Bundesschiedsgerichtes schon in gar keiner denkbaren Situation verwirklicht werden, wäre also eine leere Norm.
Nach der systematischen Auslegung darf es aber keine leeren Normen geben, da jede Regelung irgendeinen Sinn und Zweck haben muss.\footnote{Siehe zur systematischen Auslegung auch \ref{Normenauslegung:Auslegung:Methoden:Systematisch}.}
Daher ist die nicht weiter begründete Auslegung des Bundesschiedsgerichtes nicht haltbar, vielmehr sind Parteitage gerade nach dem dem Wortlaut entnehmenbaren Willen des Satzungsgebers auch erfasst.

Dies führt jedoch dazu, dass Landesschiedgerichte nie Anfechtungen gegen Entscheidungen ihres eigenen Landesparteitages verhandeln dürfen, da alle Richter zugleich Mitglieder des Landesverbandes sind und als solches Mitglieder des Organs Parteitag.
Das Organ darf hier nicht mit der Tagung des Organ verwechselt werden, da jedes Organ, also auch der Parteitag, außerhalb seiner Zusammentreffen weiter existiert.
Der Parteitag ist in der Piratenpartei typischerweise eine Mitgliederversammlung.
Dies hat zur Folge, das alle Mitglieder einer Gliederung Mitglieder des Organs Parteitags sind.
Um Sinn und Zweck der unwiderleglichen Befangenheitsvermutung zu genügen und gleichzeitig nicht zu umfangreich Mitglieder auszuschließen, muss der Wortlaut aber auch hier wohl teleologisch reduziert werden: Nur Mitglieder, die tatsächlich an der betreffenen Entscheidung beteiligt waren, können Befangen sein.
Allerdings ist jeder Redebeitrag wie auch jedes Abstimmverhalten schon eine Beteiligung. Das umfasst auch wieder die Enthaltung.
Es gibt auch keinen Grund, einen Unterschied zwischen aktiver und passiver Unterhaltung zu machen, zudem das typischerweise im Nachhinein nicht beweisbar sein wird.
Daher reicht letzlich doch wieder die Akkreditierung an dem Zusammentreffen des Organs für die Erfüllung des Tatbestands nach Nr.~4 in Verbindung mit Nr.~1.
(Dies gilt natürlich auch entsprechend für die Varianten Nr.~2 und Nr.~3).
Damit ist aber effektiv das eigene Landesschiedsgericht in jedem Verfahren, in dem der eigene Landesparteitag beteiligt ist handlungsunfähig und das Verfahren muss noch vor Eröffnung an ein anderes Landesschiedsgericht verwiesen werden.
Für das Bundesschiedsgericht heißt das sogar, das Verfahren mit Beteiligung des Bundesparteitages niemals zur Eröffnung oder gar Verhandlung kommen dürfen.
Diese Konsequenz, dass der Zugang zur Parteischiedsgerichtsbarkeit komplett entfällt, wiederspricht \S~14 Abs.~1 PartG, der fordert, dass gerade in solchen Fällen die Parteinterne Schiedergerichtsbarkeit zuständig ist.
Im Fall der Landesschiedgerichte entfällt durch die konsequente Umgehung der Zuständigkeitsregelungen die Bestimmbarkeit des entscheidenden Richters: Zuerst muss immer erst das Bundesschiedsgericht nach freiem Ermessen das Verfahren verweisen.
Das widerspricht Elementar dem Verfassungsprinzip des gesetzlichen Richters.
Damit verstößt \S~5 Abs.~1 Satz~1 Nr.~4, 1 SGO\index[paridx]{SGO!5@\S~5!1@Abs.~1} in diesen Konstellationen gegen höherrangiges Recht und darf zumindest in diesen Fällen nicht angewendet werden.
Sobald es um Parteitage niederer Gliederungen geht, und kein entsprechendes Schiedsgericht existiert, trifft diese geschilderte Konstellation nicht zu, in diesen Fällen ist \S~5 Abs.~1 Satz~1 Nr.~4, 1 SGO\index[paridx]{SGO!5@\S~5!1@Abs.~1} daher anwendbar.


\subsubsection{Tatbestand \S~5 Abs.~1 Satz~1 Nr.~5 SGO\index[paridx]{SGO!5@\S~5!1@Abs.~1}}
\label{Zusammensetzung:Spruchkoerper:Befangenheitsvermutung:Nr5}
\S~5 Abs.~1 Satz~1 Nr.~5 SGO\index[paridx]{SGO!5@\S~5!1@Abs.~1} vermutet unwiderleglich, dass ein Richter, der selbst eine Prozessvertreter oder gesetzlicher Vertreter einer Verfahrenspartei ist, befangen ist.
Letzlich bedeutet dies, dass jede Berechtigung, eine Partei in einem Verfahren zu vertreten, dazu führt, dass der betreffende Richter nicht im Gericht mitentscheiden darf.
Dies ist insofern wieder sehr nah an der Regelung von Nr.~1 und ist entsprechend auch begründet.
Die Bestellung zum Prozessvertreter ist einer einseitige, empfangsbedürtige Prozesserklärung.
Das heißt aber in der Konsequenz, dass der Bestellte für die wirksame Bestellung nicht zustimmen muss.
Es ist ohne Zustimmung zwar nicht der bestellenden Verfahrenspartei gegenüber verpflichtet, auch für sie zu handeln, gegenüber der anderen Verfahrenspartei und dem Gericht ist er aber trotzdem wirksam zum Vertreter bestellt.
Damit also eine Verfahrenspartei diese Vorschrift nicht ausnutzen kann, um etwa unliebsame Richter auszuschalten, in dem sie zum Verfahrensvertreter bestellt werden, muss auch hier eine teleologische Korrektur angewandt werden: Der Tatbestand der Norm ist dahingehend zu verstehen, dass der bestellte Vertreter auch plausibler Vertreter der Partei ist, also gerade im Fall eines eigentlich zuständigen Richters wird eine Zustimmung zur Bestellung oder zumindest aktives Handeln als Parteivertreter verlangen zu sein.
Tut er das nicht oder lehnt die Bestellung ab, kann keine Partei den Richter wegen diesem Tatbestand ablehnen und auch ein Ausschluss von Amts wegen ist abzulehnen.

\subsubsection{Tatbestand \S~5 Abs.~1 Satz~1 Nr.~6 SGO\index[paridx]{SGO!5@\S~5!1@Abs.~1}}
\label{Zusammensetzung:Spruchkoerper:Befangenheitsvermutung:Nr6}
\S~5 Abs.~1 Satz~1 Nr.~6 SGO\index[paridx]{SGO!5@\S~5!1@Abs.~1} vermutet unwiderleglich, dass ein Richter, der  Zeuge oder Sachverständiger in einem Verfahren ist, in diesem Verfahren befangen ist.
Wie schon Nr.~1 und Nr.~5 liegt ist hier der Interessenkonflikt, den die Satzung als Grundlage für ihre Befangenheitsvermutung nimmt, in der Person des Richters offensichtlich, auch wenn er diesemal nicht in seiner natürlichen Person liegt, sondern erst durch den Zusammenhang mit dem eigentlichen Verfahrensgegenstand aufkommt.

\subsubsection{Tatbestand \S~5 Abs.~1 Satz~1 Nr.~7 SGO\index[paridx]{SGO!5@\S~5!1@Abs.~1}}
\label{Zusammensetzung:Spruchkoerper:Befangenheitsvermutung:Nr7}
\S~5 Abs.~1 Satz~1 Nr.~7 SGO\index[paridx]{SGO!5@\S~5!1@Abs.~1} vermutet unwiderleglich, dass ein Richter, der  schon über das Verfahren oder den konkreten Verfahrensgegenstand im regulären vorprozessualen Geschehen oder einer unteren Instanz mitentschieden hat oder als Antragsteller oder Berater die Entscheidung eines anderen Organs, die nun Verfahrensgegenstand mitverursacht hat, im betreffenden Verfahren befangen ist.
Das derselbe Richter nicht in mehreren Instanzen über dasselbe Verfahren entscheiden soll, ist logisch.
Aber auch, wenn er sonst im vorgerichtlichen Verfahren, etwa die Ausarbeitung eines Antrags, gegen dessen Ablehnung bspw. ein Mitantragsteller klagt, relevant beteiligt war, soll dieser Grundsatz gelten, da der Satzungsgeber der Ansicht ist, dass der Richter zu nah an dem Verfahrensgegenstand dran ist, als dass die Entscheidung als objektiv unbefangen wahrgenommen werden würde.

\subsubsection{Tatbestand \S~5 Abs.~1 Satz~1 Nr.~8 SGO\index[paridx]{SGO!5@\S~5!1@Abs.~1}}
\label{Zusammensetzung:Spruchkoerper:Befangenheitsvermutung:Nr8}
\S~5 Abs.~1 Satz~1 Nr.~8 SGO\index[paridx]{SGO!5@\S~5!1@Abs.~1} vermutet unwiderleglich, dass ein Richter, der als Schlichter in einem Verfahren aktiv war, nicht auch Richter sein soll.
Dies untermauert die Trennung von richtender Funktion des Schiedsgerichtsverfahrens und schlichtender Funktion des vorgelagerten Schlichtungsverfahrens.
Nr.~8 ist somit sehr nah an den Tatbestand der Vorbefassung aus Nr.~7 dran, auch wenn es gerade um eine nicht entscheidende, außergerichtliche Instanz geht, die mangels Beschlussfassung nicht immer von Nr.~7 erfasst ist.


%\blindtext[1]
%\section{Spruchkörper}
%\subsection{Zusammensetzung, Kammern}
%\subsection{Ablehnung von Richtern}
%\blindtext[5]
%\subsection{Richterurlaub und -abwesenheit}
%\blindtext[5]

\chapterbib
% \end{refsection}

% \begin{refsection}
\chapterpreamble{Zentral im Verfahren ist die Gewähr rechtlichen Gehörs: Jede Partei muss Gelegenheit haben, zu allen Einzelheiten des Verfahrens Stellung zu nehmen. Üblicherweise setzt das Gericht mit Verfahrenseröffnung eine Frist zur Klageerwiderung und gleichzeitig eine weitere, so die Antragstellerin darauf noch antworten möchte. Weitere Sachverhaltserklärung erfolgt durch Fragen des Gerichts an die Parteien oder an Zeugen, die gebeten werden können, auszusagen. Eine Pflicht, dem Gericht Informationen zu geben, haben nur Organe der Partei. Die Parteien dürfen sich vertreten lassen; im Falle von Gliederungen oder Organen als Streitpartei ist eine Vertretung Pflicht. Die Vertretung muss immer eindeutig sein, d.h. bei mehreren Vertretern muss zumindest eine Rangfolge festgelegt werden. Jede Partei kann ihre eigenen Vertreter nach belieben bestellen, ändern oder entpflichten; dies wird mit Zugang bei Gericht wirksam. Andere Verfahrensordnungen (z.B. die Zivilprozessordnung [ZPO] oder die Verwaltungsgerichtsordnung [VwGO]) sind nur im Ausnahmefall anwendbar; die Notwendigkeit ihrer Anwendung muss im Urteil begründet werden.}

\chapter{Verfahrensablauf und Verfahrensführung}
%\blindtext[1]
%\section{Gerichtsgrundsätze und Gerichtsgrundrechte}
%rechtliches Gehör, gesetzlicher Richter etc.?
%\section{Grundsätze zu unterschiedlichen Klage- bzw Verfahrensarten}
%\blindtext[5]
%\section{Details zu unterschiedlichen Klage- bzw Verfahrensarten}
%\blindtext[5]
%\section{Vertretung im Verfahren}
%\blindtext[5]
%\section{Analoge Anwendung anderer Verfahrensordnungen}
%\blindtext[5]
%\section{Fristberechnung}
%\section{Die Öffentlichkeit des Verfahrens}
%Grundsatz der Öffentlichkeit
%Nichtöffentliche Verfahren (auf Antrag; gebundene Entscheidung in Disziplinarsachen, freie Entscheidung bei sonstigen Sachen)
%Wiederherstellung der Öffentlichkeit (z.B. bei Verwirkung des Schutzrechtes)

%\chapterbib
% \end{refsection}


% \begin{refsection}
\chapterpreamble{Der einstweilige Rechtsschutz hat den Zweck, Rechte des Antragstellers vorläufig zu sichern. Zentral sind hier das sog. „Eilbedürfnis“ und das „Sicherungsinteresse“, §~11~Abs.~2~SGO. Obwohl einstweilige Anordnungen regelmäßig dazu dienen dürften, den status quo bis zur Entscheidung in der Hauptsache zu sichern, können sie auch ohne eine Klage in der Hauptsache beantragt werden.}

\chapter{Der einstweilige Rechtsschutz}
%\blindtext[1]

%\chapterbib
% \end{refsection}


% \begin{refsection}
\chapterpreamble{Zulässig ist eine Klage nur dann, wenn die Anrufung erfolgreich war.
Form und Frist der Anrufung, Zuständigkeit für das Verfahren und Erfolglosigkeit oder Entbehrlichkeit der Schlichtung sind bereits vor Eröffnung zu prüfen, sind im Urteil jedoch noch einmal darzulegen.
Die Klage ist grundsätzlich gegen das jeweilige Organ zu richten.
Klagen gegen Einzelmitglieder, losgelöst von ihrer Funktion, sieht die SGO nicht vor.}

\chapter{Zulässigkeit der Klagebehren}
%\blindtext[1]
%\section{Klagegegner (Richtige Auswahl, Vielzahl von Gegnern etc.)}
%\blindtext[5]
%\section{Grundlagen Beweiswürdigung}
%\blindtext[5]

%\chapterbib
% \end{refsection}


%\begin{refsection}
\chapterpreamble{\Zitat{Verträge sind so auszulegen, wie Treu und Glauben mit Rücksicht auf die Verkehrssitte es erfordern.} – \S~157~\nomenclature{BGB}{Bürgerliches Gesetzbuch}\index[paridx]{BGB!\S~157}}

\chapter{Grundlagen der Normenauslegung}
% Ein paar einleitende, warme Worte.(?)

\section{Die Struktur von Rechtssätzen}
Rechtssätze sind Handlungsanweisungen. Gleichgültig, ob die fragliche \emph{Norm} dem Grundgesetz oder einer Vereinssatzung entstammt, schreibt sie eine bestimmte Verfahrensweise vor.\footnote{Ausnahmen hiervon bilden nur sog. \emph{Legaldefinitionen}, d.h. Rechtssätze, die keine Handlung sondern einen Zustand oder einen Begriff beschreiben.} Der Aufbau einer Norm ist dabei schematisch, ihre Struktur folgt wiederkehrenden Mustern.

\subsection{Formale Strukturen}
Gliederung schafft Übersicht. Daher sind unterschiedliche Rechtsbereiche in unterschiedlichen Gesetzen geregelt. Die wichtigste Gliederungsebene in Gesetzen (wie auch in den Satzungen der Partei) sind die \emph{Paragraphen},\footnote{Seltener, z.B. im Grundgesetz, die Artikel.} die zur einfacheren Auffindbarkeit und Zitierbarkeit von Normen fortlaufend nummeriert sind.

Umfassendere Regelwerke (wie bspw. das BGB\nomenclature{BGB}{Bürgerliches Gesetzbuch}) sind darüber teilweise in Bücher, Teile, Abschnitte, Unterabschnitte, Titel, Untertitel, Kapitel usw. gegliedert. Eine solche Gliederung nimmt die Satzung der Piratenpartei Deutschland insofern vor, als dass sie in drei \enquote{Abschnitte} unterteilt ist. Da die Abschnitte allerdings jeweils unterschiedlich zitiert werden, nämlich die Grundlagensatzung (BS\nomenclature{BS}{Bundessatzung (Abschnitt~A der Bundessatzung, Grundlagensatzung)}, Abschnitt~A)\footnote{Wird \enquote{die Bundessatzung} zitiert, bezieht sich das Zitat in der Regel auf die Grundlagensatzung.}, die Finanzordnung (FO\nomenclature{FO}{Finanzordnung (Abschnitt~B der Bundessatzung)}, Abschnitt~B) und die Schiedsgerichtsordnung (SGO\nomenclature{SGO}{Schiedsgerichtsordnung (Abschnitt~C der Bundessatzung)}, Abschnitt~C) und auch die Paragraphenzählung jeweils bei 1 beginnt, ist es leichter, sie sich als eigenständige Regelwerke zu vergegenwärtigen. So untergliedert letztlich nur die Finanzordnung oberhalb der Ebene der Paragraphen. Diese Gliederungsebene oberhalb der Paragraphen in der Finanzordnung ist nicht benannt. Da \enquote{Abschnitte} aber bereits, wie geschildert, durch die Gesamtsatzung verwendet werden, empfiehlt sich die Verwendung der neutralen Bezeichnung \enquote{Teil}.\footnote{Insb. \enquote{Unterabschnitt} ist ungeeignet, da die Gliederung in Abschnitte, wie geschildert, kaum argumentativ dargestellt wird.}

Paragraphen gliedern sich zunächst in Absätze. Auch Absätze sind nummeriert, allerdings nur eindeutig in Bezug auf den Paragraphen, nicht mehr in Bezug auf das \enquote{Gesamtwerk}. Absätze können einen oder mehrere Sätze und/oder Nummern (d.h. Aufzählungen, teilweise ihrerseits verschachtelt) enthalten. Teilweise bietet es sich an, Sätze als Halbsätze, Alternativen oder Varianten zu zitieren, wenn sich in einem einzelnen Satz mehrere Kombinationen von Tatbeständen und Rechtsfolgen verbergen (s.u. \emph{Tatbestände und Rechtsfolgen}).

\subsubsection{Das Normzitat}
Da der Paragraph (Artikel) die wichtigste Gliederungsebene ist, ist er der Ausgangspunkt eines jeden Normzitats. Von ihm ausgehend wird \enquote{nach unten}, d.h. nach Absätzen, Sätzen, Nummern usw. zitiert. Je genauer der betreffende Abschnitt durch das Zitat benannt wird, desto leichter lässt er sich durch den Leser wiederfinden – es gilt: \Zitat{Viel hilft viel}! Erst am Ende wird das jeweilige Regelwerk benannt, dem der Paragraph entstammt.\footnote{Das \enquote{Quellregelwerk} erst am Ende zu benennen, mag der Ordnung des Zitats – \Zitat{vom Großen ins Kleine} – widersprechen, findet jedoch im juristischen Gebrauch absolute Verbreitung.} Die Gliederungsebenen zwischen der Ebene \enquote{Regelwerk} und der Ebene \enquote{Paragraph} werden nicht mitzitiert, da sie lediglich der systematischen Übersicht dienen; die Nennung von Regelwerk und nummeriertem Paragraphen reicht zur Auffindung bereits aus.\footnote{Es gibt daher auch keine vereinheitlichen Regeln über Verwendung oder gar Rangfolge dieser reinen \enquote{Gliederungsebenen}; sie können vom jeweiligen Gesetz- bzw. Satzungsgeber völlig frei verwendet werden.}

\textbf{Beispiel:} \S~9a Abs.~9 S.~2 Hs.~2~BS\index[paridx]{BS!\S~9a!Abs.~9~S.~2} (Paragraph~9a Absatz~9 Satz~2 Halbsatz~2 der Bundessatzung) schreibt vor, dass jedes Bundesvorstandsmitglied bei der Abfassung des Tätigkeitsberichts die Verantwortung für seine eigenen Tätigkeitsbereiche trägt. 

\subsection{Tatbestände und Rechtsfolgen}
Wichtiger als die formalen, äußeren Strukturen ist der semantische, innere Aufbau von Normen: Sie gliedern sich in \emph{Tatbestände} und \emph{Rechtsfolgen}. Dem zu Grunde liegt ein simples \emph{Wenn-Dann}-Prinzip: Wenn der Tatbestand erfüllt ist, tritt die Rechtsfolge ein.

Tatbestände sind demnach reelle Sachverhalte, also tatsächliche Geschehnisse oder Handlungen. Die Rechtsfolgen sind die Handlungsanweisungen (s.o.), die sich ergeben, wenn ein solcher Sachverhalt eintritt, ein Tatbestand also erfüllt wird. Zuweilen führen mehrere Tatbestände zur selben Rechtsfolge, ebenso können an einen Tatbestand mehrere Rechtsfolgen geknüpft werden (in dieser Konstellation oft mit einer Auswahl durch die Rechtsanwenderin oder den Rechtsanwender). In einem Satz können sich dadurch durchaus mehrere Rechtssätze verbergen.

\textbf{Beispiel:} \S~6 Abs.~1 S.~1~BS\index[paridx]{BS!\S~6!Abs.~1 S.~1} (\Zitat{Verstößt ein Pirat gegen die Satzung oder gegen Grundsätze oder Ordnung der Piratenpartei Deutschland und fügt Ihr damit Schaden zu, so kann der Bundesvorstand folgende Ordnungsmaßnahmen anordnen: Verwarnung, Verweis, Enthebung von einem Parteiamt, Aberkennung der Fähigkeit ein Parteiamt zu bekleiden, Ausschluss aus der Piratenpartei Deutschland.} enthält insg. 15 Rechtssätze: An die drei Tatbestände Satzungsverstoß, Grundsatzverstoß oder Ordnungsverstoß können nach Auswahl des Bundesvorstands jeweils fünf Rechtsfolgen (Verwarnung, Verweis, Enthebung, Wählbarkeitsverlust, Ausschlussantrag beim Schiedsgericht\footnote{Vgl. \S~6 Abs.~2~BS,\index[paridx]{BS!\S~6!Abs.~2} ebenso \S~10 Abs.~5 S.~1~PartG.\nomenclature{PartG}{Gesetz über die politischen Parteien (Parteiengesetz)}\index[paridx]{PartG!\S~10!Abs.~4 S.~1}} geknüpft werden.

%\subsection{Verweisungen}

\section{Die Subsumtion}
Juristerei ist der Streit um Begrifflichkeiten. Aus der Frage, ob ein bestimmter, realer Sachverhalt unter einen juristischen Begriff zu fassen (\emph{subsumieren}, von lat. \emph{sub-sumere} – „darunter fassen“) ist, oder nicht, leitet sich letztlich die Anwendung eines Rechtssatzes ab: \S~14~BS\index[paridx]{BS!\S~14} bspw. schreibt vor, dass \emph{\enquote{Die Satzungen der Landesverbände und ihrer Untergliederungen […] mit den grundsätzlichen Regelungen dieser Satzung übereinstimmen}} müssen. Nur, wenn klar ist, ob eine Regelung der Bundessatzung eine \enquote{grundsätzliche Regel} ist, oder nicht, ist klar, ob eine Landessatzung davon abweichen darf, oder nicht. Da \S~14~BS\index[paridx]{BS!\S~14} die grundsätzlichen Regeln nicht selbst nennt, müssen sie erst ermittelt werden. Möglich wäre hier, die Aufzählung selbst vorzunehmen, d.h. alle grundsätzlichen Regeln herauszufinden. Das aber ist zeitaufwändig und regelmäßig zu umfangreich, da ja nur die Frage zu klären ist, ob eine bestimmte Regel der Satzung eine grundsätzliche Regel ist und daher alle anderen Regeln außer Betracht bleiben können. Sinnvoller ist, für den Begriff der \enquote{grundsätzlichen Regel} einen anderen Begriff zu finden, der diesen anhand von prüfbaren Merkmalen beschreibt. Gesucht wird also eine \emph{Definition}, die durch Auslegung der Bedeutung des zu definierenden Begriffs zu finden ist.

Methodisch am besten zu vergegenwärtigen ist ein ergebnisorientierter Subsumtionsvorgang, indem er nach dem folgenden Schema bearbeitet wird:
\begin{enumerate}
\item Obersatz
\item Definition
\item Subsumtion
\item Ergebnis
\end{enumerate}

Diese Methodik ist Grundlage jeden juristischen Studiums und wird dort als sog. \enquote{Gutachtenstil}\index[idx]{Gutachtenstil} gelehrt. Tatsächlich eignet sich dieser Stil hervorragend, auch komplexe Sachverhalte vollumfänglich juristisch zu prüfen. Es bietet sich daher an, dieses Schema bei der Prüfung eines Sachverhalts einzuhalten, auch wenn das Urteil letztlich etwas anders zu formulieren ist (s.a. das entsprechende Kapitel).

Der \textbf{Obersatz} stellt die \enquote{Fallfrage} dar, die zu prüfen ist. Im bereits oben verwendeten Beispiel der \enquote{grundsätzlichen Bestimmung} lautete die konkrete Frage also: \Zitat{Ist die fragliche Satzungsbestimmung eine grundlegende Bestimmung im Sinne von \S~14~BS?}

Die \textbf{Definition} gibt Antwort auf die im Obersatz implizit gestellte Frage \Zitat{Was ist überhaupt eine grundsätzliche Bestimmung?}. Wenn nicht auf bestehende Rechtsprechung zurückgegriffen werden kann\footnote{Vgl. hierzu ausführlich \cite[S.~8]{LSGBB135}; \cite[S.~5]{BSG1215HS}.} muss hier durch Auslegung eine Definition bestimmt werden.

Die \textbf{Subsumtion} ist der Schritt, in dem geprüft wird, ob der zu prüfende Sachverhalt die zuvor gefundene Definition erfüllt. Anstatt also zu entscheiden, ob eine Satzungsbestimmung \enquote{grundlegend} ist, wird nun entschieden, ob sie die Bedingungen erfüllt, die eine grundlegende Satzungsbestimmung ausmachen. Dieser Schritt beantwortet im Endeffekt die Fallfrage; anhand der Definition aber wird begründet, warum die Antwort \enquote{Ja} oder \enquote{Nein} lautet. Während sich also an der Frage nichts ändert, bietet die Subsumtion unter einen zuvor definierten Begriff eine nachvollziehbare und begründete Antwort.

Das \textbf{Ergebnis} schließlich ist mehr als die Antwort der Subsumtion: Je nachdem, wie es aussieht, kommt die Rechtsfolgenseite einer Rechtsnorm zur Anwendung oder nicht. Das Ergebnis bestimmt nun über den weiteren Verlauf des Verfahrens oder der Prüfung. Es zumindest gedanklich zu formulieren dient der eigenen Kontrolle: \Zitat{Passt meine Antwort zur Frage?}

\section{Auslegung einer einzelnen Norm}
\subsection{Auslegungsmethoden}
Die juristische Hermeneutik\footnote{Theorie des Textverständnisses, Begriff von altgriechisch \enquote{ἑρμηνεύειν} (hermēneúein), \enquote{erklären}, \enquote{auslegen}, \enquote{übersetzen}.} hat vier Methoden der Auslegung von Rechtssetzen entwickelt:
\begin{enumerate}
\item Wortlautauslegung,
\item Systematische Auslegung,
\item Historische Auslegung,
\item Teleologische Interpretation.
\end{enumerate}

Diese Methoden werden im Folgenden kurz und in ihren Grundzügen dargestellt.

\subsubsection{Wortlautauslegung}
Diese Methode wird auch als grammatische oder semantische Auslegung bezeichnet. Sie setzt an der Bedeutung der verwendeten Begriffe an und ist damit der Ausgangspunkt einer jeden Auslegung. Dem Wortlaut nach auszulegen bedeutet, zu prüfen, ob ein bestimmter Begriff des Sachverhalts unmittelbar unter einen Begriff der fraglichen Norm zu fassen ist. Die Begriffe müssen daher jeweils ihrer eigenen Bedeutung und dem Kontext ihrer Verwendung nach zueinander passen. Dabei wird es Fälle geben, in denen diese Frage eindeutig mit \enquote{Ja} zu beantworten ist, andere, in denen die Antwort eindeutig \enquote{Nein} lautet. Schließlich wird es Fälle geben, die nicht eindeutig sind. Je näher der Begriff des Sachverhalts dem Begriff der Norm ist und je besser er auch ihren Sinn erfüllt, desto eher spricht das für eine \emph{weite} Auslegung des Norm-Begriffs. Umgekehrt kann auch \emph{eng} ausgelegt werden, d.h. die \enquote{Zweifelsfälle} werden gerade nicht zugelassen.

\subsubsection{Systematische Auslegung}
\label{Normenauslegung:Auslegung:Methoden:Systematisch}
Die Systematische Auslegung betrachtet nicht die einzelne Norm, sondern ihre Stellung im Gesamtgefüge des \enquote{Systems} ihres Rechtsgebiets (bspw. einer Satzung). Dabei gelten die folgenden vier Postulate:\footnote{Vgl. umfassend: \cite[S.~66~ff.]{Puppe2008}}
\begin{enumerate}
\item \textbf{Das Postulat der Widerspruchsfreiheit:} Das Gesetz widerspricht sich nicht selbst.
\item \textbf{Das Postulat der Nichtredundanz:} Das Gesetz sagt nichts Überflüssiges.
\item \textbf{Das Postulat der Vollständigkeit:} Das Gesetz lässt keine Regelungslücken.
\item \textbf{Das Postulat der systematischen Ordnung:} Die Vorschriften des Gesetzes sind sinnvoll geordnet.
\end{enumerate}

Widerspruchsfreiheit bedeutet, dass die Handlungsanweisungen, die bspw. eine Satzung beinhalten, für jede Situation eindeutig sind. Häufig jedoch fällt ein Sachverhalt unter mehrere Tatbestände, die jeweils unterschiedliche Rechtsfolgen setzen. Durch systematische Auslegung wird bestimmt, welche Norm anwendbar ist (dazu ausführlich weiter unten: \emph{Das Zusammenspiel von Normen}). Ebenso können die umgebenden Normen oder auch das Gesamtregelwerk, der eine Norm zugeordnet ist, Hinweise auf deren Interpretation bieten.

Nichtredundanz bedeutet, dass jede Norm einen Anwendungsbereich besitzt. Tatbestände dürfen daher nicht so weit ausgelegt werden, dass sie andere Tatbestände vollständig umfassen. Umgekehrt dürfen sie auch nicht so eng ausgelegt werden, dass sie keinen eigenen Anwendungsbereich mehr besitzen, sondern von einer anderen Norm völlig eingeschlossen sind.

Vollständigkeit bedeutet, dass aus ungeregelte Sachverhalten keine verpflichtenden Normen entstehen. Eine \enquote{Lücke} in der Satzung bedeutet nicht automatisch einen Fehler, besser: das Fehlen einer Regelung. Vielmehr gilt zunächst die Annahme, dass der Parteitag als Satzungsgeber davon ausging, diesen Sachverhalt nicht regeln zu müssen oder aber er wollte es nicht.

Systematische Ordnung bedeutet, dass die Anordnung von Einzelnormen im Gesamtregelwerk einen Sinn hat. Je näher sich bspw. einzelne Normen in der formalen Aufzählung sind, desto größer ist die Wahrscheinlichkeit, dass sie auch inhaltlich im Zusammenhang stehen. Davon unabhängig ist zu beachten, dass Recht in der Regel nach der sog. \enquote{Klammertechnik} verfasst ist. Der Begriff ist an die Verklammerung mathematischer Operationen in einer Gleichung angelehnt: \enquote{Vor die Klammer} wird ein \emph{allgemeiner Teil} gezogen, der auf alle \emph{besonderen Teile} \enquote{in der Klammer} Anwendung findet. Dadurch kann es passieren, dass auch Regeln, die in der Paragraphenzählung weit auseinander liegen, in engem Zusammenhang stehen. Zu beachten ist schließlich, dass schon die Gliederung in allgemeinen Teil am Anfang und besonderen Teil im weiteren Verlauf eindeutig dagegen spricht, die Bedeutung einer Norm anhand ihrer Paragraphennummer ablesen zu wollen: Zu behaupten, eine Norm sei \enquote{wichtiger} als eine andere, nur weil ihre Ordnungsnummer niedriger sei, ist schlicht falsch. 

\subsubsection{Historische Auslegung}
Die historische Auslegung befasst sich mit der Entstehungs- und Änderungsgeschichte einer Norm. Bspw. aus Diskussionsprotokollen kann hervorgehen, ob nach dem Willen der Autorinnen und Autoren ein Sachverhalt vom Tatbestand einer Norm erfasst werden sollte, oder nicht. Ebenso ist eine Änderung, die dazu führte, dass ein einstmals eindeutig erfasster Sachverhalt nun zweifelhaft erscheint, ein klares Indiz für eine enge Auslegung der jeweiligen Norm.

Für die Satzungen privatrechtlicher Organisationen (so auch Parteien) gilt jedoch allgemein, dass ihre Satzungen aus sich heraus verständlich sein sollen. Weder soll Recherche zu (ohnehin selten in der notwendigen Ausführlichkeit bestehenden) Protokollen, noch zu alten Fassungen der Satzung notwendig sein. Mit der \enquote{Vertragstheorie}, die davon ausgeht, dass die Satzung als Grundlagendokument einer privatrechtlichen Vereinigung letztlich ein privatrechtlicher Vertrag ist, der ein \enquote{Eigenleben} entwickelt hat, kann man zudem argumentieren, dass \S~157~BGB\index[paridx]{BGB!\S~157} insb. durch den Verweis auf die (aktuelle) \Zitat{Verkehrssitte} einen starken Gegenwartsbezug aufweist. Aus diesen Argumenten wird abgeleitet, dass sich die historische Auslegung allgemein verbietet.\footnote{\cite[Rn.~36]{sauter2010}.} Ausnahmen hiervon sind daher stets gut zu begründen!

\subsubsection{Teleologische Interpretation}
Die teleologische Interpretation leitet sich von \enquote{τέλος} (gr. \enquote{Ziel}) ab. Der Schwerpunkt liegt hier also auf dem Zweck der Norm, nicht auf ihrem Wortlaut. In die Nähe zur historischen Auslegung rückt dabei die Fragestellung, was der ursprüngliche Zweck der Norm war, besser: ist. In die Nähe der systematischen Auslegung rückt die Frage, was mit einer solchen Norm \enquote{objektiv sinnvoll} bezweckt werden kann; hier wird nach dem Sinn und Zweck der Norm im aktuellen Gefüge gefragt. Die Antworten auf diese Fragen geben regelmäßig den Ausschlag, wenn bloße Wortlaut- und systematische Auslegung nicht weiter geholfen haben.

In sehr seltenen Fällen kann der Sinngehalt einer Norm dazu führen, dass ein Begriff dahingehend uminterpretiert wird, dass er nun auch Begriffe umfasst, die vom Wortlaut eigentlich nicht gedeckt sind (s.o.); hier spricht man von einer \emph{teleologischen Erweiterung}. Umgekehrt kann ein Begriff auch \emph{teleologisch reduziert} werden, wenn aus Zweckmäßigkeitserwägungen auch Sachverhalte außen vor bleiben sollen, die eigentlich vom Wortlaut klar umfasst sind.\footnote{Ein Beispiel einer solchen teleologischen Korrektur ist die Auslegung des \S~8 Abs.~3 Nr.~3~SGO\index[paridx]{SGO!\S~8!Abs.~3 Nr.~3}, der von \enquote{Anträgen} im Plural spricht, wohingegen auch die Erhebung einer Klage unter Stellung nur eines einzelnen Antrags unzweifelhaft zulässig sein soll!} Diese Methoden sind aber Ausnahmen, nicht die Regel!

\subsection{Anwendbarkeit der Auslegungsmethoden für Schiedsgerichte}
Als einzige Aussage der Satzung zu anwendbarem Recht und auch der Methodik, dieses zu erschließen, schreibt \S~2 Abs.~2 S.~2~SGO\index[paridx]{SGO!\S~2!Abs.~2 S.~2} vor, dass Entscheidungen \Zitat{nach bestem Wissen und Gewissen} und \Zitat{auf Grundlage der Satzungen und gesetzlichen Vorgaben} zu fällen sind. Diese Anweisung begrenzt die Wahl der Auslegungsmethoden nicht. Mit der teilweisen Ausnahme der historischen Auslegung, deren Grenzen auf der Grundlage gesetzlicher Vorgaben bereits dargestellt wurden, sind diese vier Methoden in der Schiedsgerichtsbarkeit der Piratenpartei anwendbar. Eine frühere Fassung dieses Paragraphen hatte die auf die Satzung und die Schiedsgerichtsordnung anwendbaren Methoden auf Wortlautauslegung und teleologische Interpretation begrenzt, ist jedoch inzwischen weggefallen.

\subsection{Die Rangfolge der Auslegungsmethoden}
Es gibt keine starre Rangfolge der Auslegungsmethoden. Allerdings ist der Wortlaut ebenso Ausgangspunkt wie Grenze einer jeden Auslegung, die semantische Auslegung ist daher die wichtigste Methode. Die teleologische Interpretation bietet den meisten Spielraum und bedarf für ein klares Ergebnis der meisten Begründung, sie ist praktischerweise dann zu Rate zu ziehen, wenn die anderen Methoden nicht allein zu einem Ergebnis kommen. Im Begründungsaufwand dazwischen steht die systematische Auslegung; historisch ausgelegt wird eine Satzung nur im Ausnahmefall.

Obgleich es keine Rangfolge unter den Auslegungsmethoden gibt, kann die obige Reihenfolge als Prüfungsreihenfolge in der Auslegung dienen: Schematisch bedeutet das, dass zunächst zu prüfen ist, ob der Sachverhalt unter den Wortlaut der Norm fassbar ist (Wortlautauslegung). Daraufhin ist zu prüfen, ob die Norm anwendbar ist (Systematische Auslegung) und ggf. ob sich aus der Stellung der Norm im Gesamtgefüge weitere Hinweise auf die Subsumtion ergeben (etwa, ob die Norm eher eng oder eher weit auszulegen ist). Schließlich ist die Frage zu stellen, ob die Anwendung der Norm im konkreten Fall nicht zu auffälligen Widersprüchen zu ihrem Sinn und Zweck führt. Letzteres darf aber nicht zur völligen Uminterpretation des Wortlautes führen!

\section{Das Zusammenspiel von Normen}
% Dieser Abschnitt ist ein Dreizeiler, der Rest kommt mit Milestone 5
Behandeln zwei anwendbare Bestimmungen den selben Regelungsgegenstand und setzen sie unterschiedliche Rechtsfolgen (\emph{Kollision}), so gelten die folgenden \enquote{Vorfahrtsregeln}: Die neuere Regel überschreibt die ältere Regel, die speziellere Regel drängt sich vor die allgemeinere Regel und die höhere Regel steht über der niedrigeren Regel.

%\subsection{Der Stufenbau der Rechtsordnung}
%\subsubsection{Direkter Einfluss höherrangigen Rechts}
%lex superior
%\subsubsection{Indirekter Einfluss höherrangigen Rechts}
%Verfassungskonforme Auslegung?
%Konventionskonforme Auslegung?
%Beachte bei xy-konformer Auslegung: einheitliche Auslegung der Satzung aus sich heraus (Sauter)!
%\subsubsection{Höherrangiges Satzungsrecht}
% § 6 Abs. 1 PartG, § 14 BS
%\subsection{Kollisionsregeln}
%\subsection{Analogieschlüsse}
%\section{} %Irgendwas mit Schlüssen?

%\chapterbib
%\end{refsection}


% \begin{refsection}
\chapterpreamble{Grundsätzlich kann eine Klage darauf gerichtet sein, etwas außer Kraft zu setzen (Anfechtungsklage), jemanden zu verpflichten (Verpflichtungsklage), oder das Bestehen oder Nichtbestehen eines Rechtsverhältnisses festzustellen (Feststellungsklage). Die Feststellungsklage tritt hinter eine zulässige Anfechtung oder Verpflichtung immer zurück, sie ist nur subsidiär zulässig. Ist eine Sache erledigt, d.h. hat sich der reale Sachverhalt so verändert, dass die Klage keine Änderung in der Sache mehr herbeiführen kann, ist das Verfahren in der Hauptsache erledigt. Nur in Fällen, in denen ein besonderes Fortsetzungsinteresse besteht (etwa zur Rehabilitierung) kann das Verfahren als Fortsetzungsfeststellungsklage weitergeführt werden.}

\chapter{Verfahrensarten}
%\section{Anfechtung von Mitgliederversammlungen}
%\blindtext[1]
%\section{Gliederungsstreitigkeiten}
%\blindtext[1]
%\section{Grundlagen Ordnungsmaßnahmen}
%\blindtext[1]
%\subsection{Gliederungsordnungsmaßnahmen}
%\blindtext[5]
%\subsection{Individualordnungsmaßnahmen}
%\blindtext[5]
%\section{Normenkontrolle}
%\blindtext[1]

%\chapterbib
% \end{refsection}

% \begin{refsection}
\chapterpreamble{Das Urteil ist das Ergebnis des Prozesses. Es sollte follguht™ sein und Rechtsfrieden schaffen.}

\chapter{Urteilsaufbau}

\section{Grundlagen}
Gem. \S~12 Abs.~3 S.~1~SGO\index[paridx]{SGO!12@\S~12!3@Abs.~3} enthält ein Urteil \Zitat{einen Tenor, eine Sachverhaltsdarstellung und eine Begründung der Sach- und Rechtslage.}
Es empfiehlt sich, dieser Aufzählung notwendiger Inhalte auch in der Reihenfolge zu folgen.
Ein Urteil beginnt daher, nach einigen Formalien im \emph{Rubrum}, mit der letztendlichen Entscheidung des Gerichts.
Sie wird im \emph{Tenor} aufgeführt.
Erst danach wird der \emph{Sachverhalt} dargestellt und die im Tenor verkündeten Entscheidungen des Gerichts in den \emph{Entscheidungsgründen} begründet.
Sofern gegen das Urteil Rechtsmittel eingelegt werden können, wird es mit einer \emph{Rechtsmittelbelehrung} abgeschlossen.
Während gem. \S~11 Abs.~7~SGO\index[paridx]{SGO!11@\S~11!7@Abs.~7} lediglich die Dokumentationsvorschriften explizit auch für einstweilige Anordnungen gelten, ist es gängige Praxis, sämtliche Vorschriften über Urteile auf einstweilige Anordnungen sinngemäß anzuwenden.

Anders als bei Urteilen des Europäischen Gerichtshofes oder z.B. französischen Gerichten oder Gerichten aus dem angelsächsischen Rechtsraum werden Urteile in Deutschland regelmäßig in \enquote{ganz normalem Fließtext} verfasst, d.h. als mehrere Sätze, in Absätze unterteilt, regelmäßig mit Zwischenüberschriften bzw. Nummerierung gegliedert.
Die Parteien werden im Rahmen der Schiedsgerichtsordnung schlicht als \emph{Antragstellerinnen} und \emph{Antragsgegnerinnen} bezeichnet (nicht etwa als Klägerinnen und Beklagte); \enquote{Angeklagte}\footnote{Der Begriff stammt aus dem Strafrecht; die Verwendung innerhalb eines Schiedsgerichtsverfahrens verbietet sich.} kennt die SGO auch im Falle von Ordnungsmaßnahmen nicht.
Auch sonst sollte die Sprache möglichst einfach gehalten sein:
Ein gutes Urteil kommt ohne komplexe Schachtelsätze aus.

\section{Das Rubrum}
\index[idx]{Rubrum}
Unter der Überschrift \Zitat{Urteil zu (Aktenzeichen)} beginnt das Urteil mit der Aufzählung (inkl. Anschriften) der Streitparteien, ihrer jeweiligen Vertretung und der Streitsache.
Dieser erste Teil eines Urteils wurde früher einmal rot gedruckt und wird daher traditionell als \enquote{Rubrum} (lat. \enquote{rot}) bezeichnet.

Das Rubrum wird in der Schiedsgerichtsordnung nicht explizit verlangt.
Es hat sich dennoch in der Praxis sämtlicher Schiedsgerichte eingebürgert.
Auf eine Formel vergleichbar zum bei staatlichen Gerichten verwendeten \Zitat{Im Namen des Volkes} wird dabei verzichtet und stattdessen schlicht mit den Worten \Zitat{In dem Verfahren (Aktenzeichen)} begonnen, woran sich die folgenden Angaben anschließen:
\begin{itemize}
\item Name und Anschrift der Antragstellerin (vgl. auch \S~8 Abs.~3~SGO\index[paridx]{SGO!8@\S~8!3@Abs.~3}),
\item Name und Kontakt der Antragstellervertretung (\Zitat{vertreten durch…}),
\item Bezeichnung der vorstehenden Partei als \Zitat{– Antragsteller(in) –},
\item \Zitat{gegen},
\item Name und Anschrift der Antragsgegnerin (vgl. auch \S~8 Abs.~3~SGO\index[paridx]{SGO!8@\S~8!3@Abs.~3}),
\item Name und Kontakt der Antragsgegnervertretung (\Zitat{vertreten durch…}),
\item Bezeichnung der vorstehenden Partei als \Zitat{– Antragsgegner(in) –},
\item Bezeichnung des Streitgegenstandes (z.B. \Zitat{wegen Anfechtung von Parteitagsbeschlüssen}).
\end{itemize}

Es folgt die namentliche Nennung der beschließenden Richterinnen und Richter sowie das Datum des Urteils, etwa \Zitat{haben die Richterinnen und Richter (Namen) am (Datum) entschieden:}.
Darauf folgt der \emph{Tenor}, der streng genommen zum Rubrum gehört, aufgrund seiner besonderen Stellung hier aber gesondert behandelt wird.

Sind auf Seiten einer Partei (oder beiden) mehrere Personen aufzuführen (Streitgenossenschaft), so werden diese als Antragstellerinnen fortlaufend nummeriert aufgeführt.
Hierfür bieten sich arabische Nummerierung und ein Neubeginn jeweils bei 1 bei Antragstellern bzw. Antragsgegnern an.

In einem Berufungsverfahren wird die Bezeichnung als \enquote{Antragstellerin} oder \enquote{Antragsgegnerin} jeweils um die Bezeichnung als \enquote{Berufungsführerin} bzw. \enquote{Berufungsgegnerin} erweitert (bspw. \Zitat{Antragstellerin und Berufungsführerin} oder \Zitat{Antragstellerin und Berufungsgegnerin}).

\section{Tenor}
\index[idx]{Tenor}
\index[idx]{Urteilstenor|see{Tenor}}
Der Tenor enthält die Entscheidung (oder die Entscheidungen) in der Hauptsache.

Da die Verfahren vor den Schiedsgerichten der Piratenpartei immer kostenfrei sind (\S~16 Abs.~1 S.~1~SGO),\index[paridx]{SGO!16@\S~16!1@Abs.~1} kann eine Kostenentscheidung\index[idx]{Kosten!Entscheidung} entfallen.\footnote{Eine Kostenentscheidung muss auch auf expliziten Antrag einer Partei nicht getroffen werden, \cite{LSGNRW2016001HStreitwert}.}
Auch einer Entscheidung über vorläufige Vollstreckbarkeit\index[idx]{Vollstreckung!vorläufige} bedarf es nicht, da die SGO weder Vollstreckungsvorschriften vorsieht, noch Maßnahmen zur Zwangsvollstreckung kennt.
Ebenso ist nicht notwendig, ein mögliches Rechtsmittel im Tenor aufzuführen, da über die Möglichkeit von Rechtsmitteln nicht durch das Gericht entschieden wird:
Die Entscheidung, ob Rechtsmittel möglich sind, oder nicht, wird von der SGO getroffen.
Da sie dem Gericht nicht überlassen ist, ist sie auch im Tenor nicht zu erwähnen.
Sind Rechtsmittel möglich, sind die Parteien darüber gesondert zu belehren (\S~13 Abs.~2 S.~3~SGO\index[paridx]{SGO!13@\S~13!2@Abs.~2}, siehe unten).

Die Entscheidungsformel ist der wichtigste Teil des Urteils.
Sie ist besonders sorgfältig zu formulieren.
Unterlaufen Fehler, kann das darin münden, dass keine konkreten Rechtsfolgen und/oder Handlungsanweisungen für die Parteien abgeleitet werden können, dass für etwas keine Rechtskraft erwächst oder aber etwas unbeabsichtigt nicht oder falsch gestaltet wird.

Obwohl die Schiedsgerichtsordnung keine Regeln zur Vollstreckung\index[idx]{Vollstreckung} kennt (s.o.) und es daher regelmäßig den zum Tun oder Unterlassen verpflichteten Organen obliegt, die Urteile umzusetzen, sollte bei der Tenorierung unbedingt darauf geachtet werden, dass der Tenor \emph{vollstreckbar} ist.
In der staatlichen Gerichtsbarkeit bedeutet Vollstreckung die zwangsweise Durchsetzung des Urteils.
Sollen aber Zwangsmittel (wie bspw. Zwangsgelder, die eingetrieben werden, bis eine Verpflichtung aus einem Urteil erfüllt wurde) verhängt werden, soll also aus dem Tenor vollstreckt werden, so muss aus dem Tenor klar abzuleiten sein, welches Tun oder Unterlassen im Einzelnen erwartet wird.
Die \enquote{Vollstreckbarkeit} eines Tenors bedeutet also, dass auch hohen Ansprüchen an Klarheit und Eindeutigkeit genügt wird:
Die prinzipielle \enquote{Vollstreckbarkeit} liegt dann vor, wenn stets eindeutig feststellbar ist, ob die durch das Urteil Verpflichteten ihren jeweiligen Verpflichtungen nachkommen bzw. nachgekommen sind oder nicht.
Dies dient einerseits der Verständlichkeit des Urteils, andererseits kann es ggf. auch die Durchsetzung vor einem staatlichen Gericht ermöglichen oder vereinfachen.

Der einzige Tenor, der nicht der Vollstreckbarkeit zugänglich ist, ist die \emph{Feststellung}.\index[idx]{Feststellungsklage!Tenor}
Anstatt durch Urteil Rechte oder Pflichten aufzuerlegen, wird hier lediglich das Bestehen oder Nichtbestehen eines bestimmten Rechtsverhältnisses festgestellt (bspw. die Nichtigkeit einer Satzungsänderung oder einer Wahl).
Diese Feststellung ist so nicht vollstreckbar; die Existenz oder Nichtexistenz einer Rechtsbeziehung ist nicht erzwingbar.
Lediglich die Akzeptanz einer solchen Feststellung könnte erzwungen werden; in diesem Falle jedoch ist nicht \emph{festzustellen}, sondern zu \emph{verpflichten}.
Aus diesem Grunde ist eine Feststellungsklage nur subsidiär\index[idx]{Feststellungsklage!Subsidiarität} zuständig, wenn eine Anfechtungs- oder Verpflichtungsklage nicht möglich ist (dazu ausführlich unter \enquote{Zulässigkeit}).

\section{Sachverhalt}
\index[idx]{Sachverhalt}
Gemäß \S~12 Abs.~3 S.~1~SGO\index[paridx]{SGO!12@\S~12!3@Abs.~3} enthält ein Urteil unter anderem eine Sachverhaltsdarstellung.
In diesem Abschnitt sind die tatsächlichen Feststellungen, die der Entscheidung des Gerichts zu Grunde liegen, objektiv und neutral aufzuführen.
Er muss aus sich heraus verständlich sein, Verweise sollten nur ausnahmsweise verwendet werden; insbesondere muss hier sichergestellt sein, dass die entsprechende Quelle auch nach geraumer Zeit noch verfügbar sein wird.
Sind Teile von Dokumenten für die Entscheidung maßgeblich, sollten sie im entsprechenden Umfang zitiert werden, anstatt lediglich zu verweisen.

Der Sachverhalt ermöglicht es Außenstehenden, die zum Streit und letztlich zur Entscheidung führende Sachlage zu verstehen.
Den Parteien dient die Sachverhaltsdarstellung als Kontrollmittel, nämlich dahingehend, ob das Gericht ihr Vorbringen zu den Tatsachen zur Kenntnis genommen und richtig verstanden hat.

Inhaltlich bietet sich eine Gliederung nach Anträgen an, sowie danach, ob die Tatsachen unstreitig (also von allen Parteien anerkannt) oder streitig sind.
Vorbringen der Parteien sollte als Prozessgeschichte chronologisch und so wie von ihnen geäußert widergegeben werden.
Insbesondere sind Schlussfolgerungen zu vermeiden.
Dabei sollten Schriftsätze und auch mündliche Vorträge – schon aus Platzgründen – nicht vollständig zitiert werden, sondern nur ihrem Sinngehalt und ihren Schwerpunkten nach.

Im Normalfall beginnt ein Urteil mit der Prozessgeschichte und den unstreitigen Tatsachen.
Darauf folgt der streitige Vortrag der Antragstellerinnen, gefolgt von ihren Anträgen.
Die Anträge der Antragsgegnerinnen schließen sich an, gefolgt von ihrem Vorbringen.
Sprachlich ist darauf zu achten, dass sich das Gericht das jeweilige Vorbringen der Parteien nicht durch Formulierungen im Indikativ zu Eigen macht.
Stattdessen soll es indirekte Rede verwenden.

Der Sachverhalt sollte so kurz wie möglich gefasst sein, Überflüssiges ist wegzulassen.
Zur Kontrolle gilt:
Jede Begründung benötigt eine korrespondierende Darstellung der Fakten im Sachverhalt – Fakten, auf die in der Begründung nicht Bezug genommen werden, sind überflüssig.
Rechtliche Bewertungen durch das Gericht finden in der Sachverhaltsdarstellung keinen Platz.

\section{Entscheidungsgründe}
Das Urteil ist das Ergebnis der juristischen Prüfung des Gerichts.
Während die Prüfung selbst in der Regel nach dem weiter oben beschriebenen \enquote{Gutachtenstil} folgt, der zum Ergebnis hinführt, wird im Urteil vom Ergebnis ausgehend begründet (\enquote{Urteilsstil})\index[idx]{Urteilsstil}.
In den Entscheidungsgründen\footnote{Diese Formel ist für Urteile in der Zivilgerichtsbarkeit üblich; bei Beschlüssen der Zivilgerichtsbarkeit sowie Urteilen im Strafprozess wird schlicht \enquote{Gründe} verwendet.} legt das Gericht dar, warum die gefällte Entscheidung so – und nicht anders – gefällt werden musste.

Die streitentscheidenden Normen sind im ersten Satz aufzuführen und im Ergebnis zu bewerten.%Wäre hier ein Beispiel angebracht?
Danach folgt die juristische Würdigung der Rechtslage, d.h. bei einer erfolgreichen Klage eine Prüfung aller relevanten Tatbestandsmerkmale in geeigneter Reihenfolge; bei einer abweisenden Klage genügt die Schilderung des (in der Prüfreihenfolge zuerst auftauchenden) Grundes ihres Scheiterns.

\emph{Geeignete Reihenfolge} ist in der Regel die Reihenfolge, in der der Fall insgesamt geprüft wurde:
Beginnend mit der Statthaftigkeit der Anrufung über die Zulässigkeit hin zur Begründetheit der Klage.
Dabei reicht es bei Statthaftigkeit und Zulässigkeit aus, wenn lediglich problematische Aspekte diskutiert und begründet werden.
Unproblematisches, d.h.\nomenclature{d.h.}{das heißt} für jeden ersichtlich erfüllte Kriterien können hier weggelassen oder in kurzen Sätzen festgestellt werden (bspw. \Zitat{M ist Mitglied der Piratenpartei. Er hat das zuständige Gericht form- und fristgerecht angerufen.}).
Ausführlicher ist mit der Begründetheit zu verfahren:
Hier sollten alle relevanten Tatbestandsmerkmale zumindest erwähnt werden.
Ausführlich begründet werden sollte jede Entscheidung des Gerichts, die sich gegen eine Rechtsansicht einer der Parteien wendet:
Widerspricht das Gericht der Rechtsauffassung einer der Parteien (oder gar beiden, wenn diese sich in diesem Falle einig waren oder gar eine dritte Lösung gefunden wurde), so dürfen diese erwarten, dass das Gericht hierzu Stellung bezieht.

Da im Rahmen der Schiedsgerichtsbarkeit der Piratenpartei keine Kostenentscheidung getroffen werden muss (s.o.), erübrigt sich auch die Begründung einer solchen.

\subsection{Obiter Dicta}
\index[idx]{obiter dictum}
Ein \emph{obiter dictum} (lat. \emph{das nebenbei Gesagte}) ist eine rechtliche Ausführung zur Urteilsfindung, die über das Erforderliche hinausgeht und auf der das Urteil dementsprechend nicht beruht.\footnote{\url{http://www.duden.de/rechtschreibung/Obiter_Dictum}.}
\emph{Obiter dicta} müssen sich dementsprechend stets entgegenhalten lassen, dass sie im Urteilstext eigentlich überflüssig und damit wegzulassen wären.
Auf der anderen Seite kann ein \emph{obiter dictum} zur Rechtsfortbildung beitragen, etwa indem eine streitentscheidende Norm über ihren Beitrag zum vorliegenden Streit hinaus ausgelegt oder kommentiert wird.
Es kann auch genutzt werden, um eine Ansicht zu vergleichbaren Fällen darzulegen, eine stehende Rechtsprechung zu bestätigen oder aufzuheben.
Gerade aufgrund der vermittelnden, streitschlichtenden Aufgabe von Schiedsgerichten kann eine ergänzende Anmerkung sinnvoll sein, um die streitenden Parteien zu befrieden oder weitere Streitigkeiten \Zitat{im Umfeld des verfahrensgegenständlichen Streits} zu vermeiden.

Gefahren bestehen bei \emph{obiter dicta} allerdings (auch) darin, dass sie spätere Fälle zumindest teilweise vorwegnehmen, ohne aber die Möglichkeit zu haben, konkrete Gesichtspunkte einer bestimmten, nicht vorhersehbaren Konstellation zu würdigen.
Sie können (bspw. bei geänderter Rechtsauffassung eines Obergerichts) spätere Klagen ermöglichen, da die Betroffenen aus dem \emph{obiter dictum} ersehen können, dass das Gericht ihren Argumenten (wieder) zugänglich ist.
Die sich im Umkehrschluss ergebende Gefahr besteht darin, dass spätere Verfahren, die im Kern vielleicht legitim gewesen wären, aufgrund eines unachtsam formulierten oder auf einen ganz anderen Fall bezogenen \emph{obiter dictum} unterbleiben.

Wenngleich die Verwendung von \emph{obiter dicta} innerhalb der Schiedsgerichtsbarkeit der Piratenpartei häufiger und ausführlicher vorkommen dürfte – und darf – als das bei staatlichen Gerichten der Fall ist, ist bei ihrer Verwendung äußerste Vorsicht geboten.

\subsection{Abweichende Meinungen}
\index[idx]{Sondervotum}
\index[idx]{Abweichende Meinung|see{Sondervotum}}
Gemäß \S~12 Abs.~4 S.~1~SGO\index[paridx]{SGO!12@\S~12!4@Abs.~4} haben die Richter \Zitat{das Recht, in der Urteilsbegründung eine abweichende Meinung zu äußern.}
Dieses Recht besitzt unter den staatlichen Gerichten der Bundesrepublik Deutschland lediglich die Verfassungsgerichtsbarkeit (vgl. z.B. \S~30 Abs.~2~BVerfGG\index[paridx]{BVerfGG!30@\S~30!2@Abs.~2}).
In der Piratenpartei steht es allen Richterinnen und Richtern zu, unabhängig von der Ebene ihres Schiedsgerichts.
Die Bestimmung zu solchen abweichenden Meinungen (auch als \enquote{Sondervotum}, plural \enquote{Sondervoten}, bezeichnet) stellt damit eine der eben dort genannten Ausnahmen zur ansonsten geltenden Regel, über \enquote{dienstliche} Vorgänge Stillschweigen zu bewahren (\S~2 Abs.~4~SGO\index[paridx]{SGO!2@\S~2!4@Abs.~4}), dar.\index[idx]{Verschwiegenheitspflicht}

Obwohl der Wortlaut darauf hindeutet, dass die abweichende Meinung innerhalb der Urteilsbegründung formuliert werden soll, ist die Vorschrift in der bisherigen Praxis so ausgelegt worden, dass sie außerhalb der eigentlichen Urteilsbegründung, aber innerhalb des Urteils verfasst wird.
Auf diese Weise bleibt die mehrheitlich beschlossene Begründung in sich konsistent und wird nicht durch Diskussionen innerhalb des Spruchkörpers unterbrochen.
Es empfiehlt sich stattdessen, die abweichende(n) Meinung(en) zusammenhängend ans Ende des Urteilstext als eigenen Abschnitt anzuhängen und sie gesondert kenntlich zu machen, anstatt sie über die Urteilsbegründungen zu verteilen.
Das erleichtert insbesondere den Parteien, aber auch der Parteiöffentlichkeit und anderen Schiedsgerichten das Lesen.
Sofern eine Rechtsmittelbelehrung (s.u.) erfolgt, sollte sie nach die abweichenden Meinungen gesetzt werden, um die abweichenden Meinungen weiter \Zitat{in der Urteilsbegründung} zu halten.

Näheres zur abweichenden Meinung sollen die Schiedsgerichte in ihrer Geschäftsordnung regeln, \S~12 Abs.~4 S.~2~SGO\index[paridx]{SGO!12@\S~12!4@Abs.~4}.

\section{Die Rechtsmittelbelehrung}
\index[idx]{Rechtsmittelbelehrung}
Eine Rechtsmittelbelehrung soll gemäß \S~12 Abs.~5~SGO\index[paridx]{SGO!12@\S~12!5@Abs.~5} zumindest dann erfolgen, wenn gegen das Urteil die Berufung möglich ist.
Die Vorschrift ist allerdings dahin auszulegen, dass die Parteien stets über mögliche Rechtsmittel belehrt werden sollen.

Wann ein Rechtsmittel zulässig ist und wie in diesen Fällen zu verfahren ist, wird im Kapitel \enquote{Instanzenzug} beschrieben.

\subsection{Inhalt einer Rechtsmittelbelehrung}
Die Rechtsmittelbelehrung soll die Parteien darüber in Kenntnis setzen, welches Rechtsmittel bei welchem Gericht in welcher Form und in welcher Frist einzulegen ist.
Das Rechtsmittel ist unter Zitat der entsprechenden Norm anzugeben (bspw. die Berufung gemäß \S~13 Abs.~1~SGO\index[paridx]{SGO!13@\S~13!1@Abs.~1}).
Zusätzlich zur (eindeutigen) Benennung des Gerichts sind Einreichungsmöglichkeiten anzugeben, also zumindest die Postanschrift, ggf. E-Mail Adresse und Faxnummer des Gerichts.
Die Rechtsmittelbelehrung muss den Hinweis enthalten, dass die Einreichung in Textform (d.h. auch per E-Mail) ausreichend ist.
Ebenso zu den Formvorschriften gehört der Hinweis, was die Berufung enthalten muss (nämlich das Aktenzeichen und die angegriffene Entscheidung des Gerichts).
Ebenso ist die Frist zu nennen, wobei sie nicht berechnet werden muss:
Es reicht aus, die Parteien darüber zu belehren, dass die Berufungsfrist gemäß \S~13 Abs.~2 S.~1~SGO\index[paridx]{SGO!13@\S~13!2@Abs.~2} 14~Tage beträgt, sie muss nicht in einem genauen Datum angegeben werden.
Diese Berechnung führt im Zweifel das Berufungsgericht aus, das an die Entscheidung der unteren Instanz ohnehin nicht gebunden wäre; die Gefahr, dass die Rechtsmittelbelehrung durch eine falsche Fristberechnung unrichtig wird, kann so umgangen werden.

Die Rechtsmittelbelehrung ist nur für Berufungen normiert.\index[idx]{Rechtsmittel!Berufung}
Außer der Berufung kennt die Schiedsgerichtsordnung allerdings noch den \emph{Widerspruch}\index[idx]{Rechtsmittel!Widerspruch} gegen einstweilige Anordnung (\S~11 Abs.~4~SGO)\index[paridx]{SGO!11@\S~11!4@Abs.~4} und die \emph{sofortige Beschwerde}\index[idx]{Rechtsmittel!sofortige Beschwerde} gegen die Ablehnung einer einstweiligen Anordnung (\S~11 Abs.~6~SGO)\index[paridx]{SGO!11@\S~11!6@Abs.~6}, die Nichteröffnung (\S~8 Abs.~6 S.~3~SGO)\index[paridx]{SGO!8@\S~8!6@Abs.~6} oder Verzögerung eines Verfahrens (\S~10 Abs.~8 S.~1,~2~SGO)\index[paridx]{SGO!10@\S~10!9@Abs.~9}, sowie den Nichtausschluss eines Richters vom Verfahren nach erfolgter Ablehnung (\S~5 Abs.~6 S.~2~SGO)\index[paridx]{SGO!5@\S~5!6@Abs.~6}.
Im Gegensatz zu den Dokumentationsvorschriften ist eine Anwendbarkeit der Vorschriften über die Rechtsmittelbelehrung auf diese Institute in der Schiedsgerichtsordnung nicht explizit vorgesehen.
Die Abhängigkeit einer Belehrung über Verfahrensschritte im Rechtsschutz sollte allerdings von der konkreten Verfahrensart unabhängig sein.
Es ist daher eher anzunehmen, dass die Vorschriften über die Belehrung zum Rechtsmittel \emph{Berufung} auf die anderen Rechtsmittel der SGO analog anzuwenden sind.
Insbesondere eine korrekte Belehrung schadet in keinem Falle.

\subsection{Unrichtige oder unterbliebene Rechtsmittelbelehrung}
\index[idx]{Rechtsmittelbelehrung!unrichtige}
\index[idx]{Rechtsmittelbelehrung!fehlende}
Ist die Rechtsmittelbelehrung unterblieben, so beginnt die Rechtsmittelfrist nicht zu laufen, \S~13 Abs.~2 S.~3~SGO\index[paridx]{SGO!13@\S~13!2@Abs.~2}.\footnote{Anderer Ansicht ist neuerdings das Bundesschiedsgericht, nach dem eine solche, dem §~58~VwGO entsprechende Regelung, in der SGO nicht enthalten ist, \cite{BSGPP100132493}.}
Dies ändert nichts an der Ausschlussfrist von 3~Monaten nach Zugang, nach der eine Entscheidung unanfechtbar ist.
Effektiv verlängert sich bei einer fehlenden Rechtsmittelbelehrung also die Berufungsfrist von 14~Tagen auf 3~Monate nach Zustellung des (unvollständigen) Urteils.

Wenngleich die Schiedsgerichtsordnung zu einer bloß unrichtigen, d.h. prinzipiell nicht unterbliebenen, aber nicht vollständigen Rechtsmittelbelehrung schweigt, ist anzunehmen, dass dies zur gleichen Rechtsfolge führt:
Sinn der \enquote{Fristverlängerung} ist, dass ein Fehler des Gerichts den Parteien nicht schaden soll.
Zweck der Rechtsmittelbelehrung ist es, den Parteien den Zugang zu effektivem Rechtsschutz zu ermöglichen, ohne dass sie die Schiedsgerichtsordnung in all ihren Facetten kennen müssen.
Da jeder Fehler eines Rechtsmittels dem Rechtsmittel gleichermaßen schadet, besteht kaum ein Unterschied zwischen einer fehlenden und einer unrichtigen Rechtsmittelbelehrung.
Im Gegenteil können die Parteien auf den Inhalt der Rechtsmittelbelehrung vertrauen, der Schaden einer unrichtigen Belehrung könnte daher auch als größer angesehen werden als der einer unterbliebenen Belehrung.
Auch bei einer fehlerhaften Rechtsmittelbelehrung \enquote{verlängert} sich die Frist also auf 3~Monate.

% Eine \section zum Vergleich? Zulässigkeit, Unterschiede zu einem regulären Urteil in der Tenorierung und Begründung

\chapterbib
% \end{refsection}


% \begin{refsection}
\chapterpreamble{Dokumentiert (vgl. \S~14~SGO\index[paridx]{SGO!14@\S~14}) werden Verfahrensakten und Urteile.
Die Verfahrensakte umfasst sämtliche verfahrensrelevante Kommunikation des Gerichts mit den Parteien (und umgekehrt).
Das umfasst auch Aktennotizen, die sich auf den Verfahrensverlauf beziehen (z.B. bei telefonisch gestellten Anträgen o.ä.).
Interne Kommunikation des Gerichts, auch wertende Notizen einzelner Mitglieder des Gremiums, gehören nicht zur Verfahrensakte.
Werden während Verfahren Tonaufzeichnungen angefertigt, sind diese zu löschen, sobald die Parteien das daraus angefertigte Protokoll erhalten und einen Monat keinen Widerspruch erhoben haben, \S~14 Abs.~3~SGO\index[paridx]{SGO!14@\S~14!3@Abs.~3}.
Die Verfahrensakte ist fünf Jahre nach Abschluss des Verfahrens aufzubewahren, Urteile unbefristet, \S~14~Abs.~5~SGO\index[paridx]{SGO!14@\S~14!5@Abs.~5}.}

\chapter{Dokumentation und Rechenschaftslegung}
\label{Dokumentation}
Die Dokumentationspflichten der Schiedsgerichte sind in \S~14~SGO\index[paridx]{SGO!14@\S~14} geregelt.
Diese Dokumentation ist für den Gebrauch im Verfahren bestimmt; bezieht sich also auf die Verfahrensakte.
Dies dient einerseits dazu, das Verfahren zu erleichtern (vgl.\nomenclature{vgl.}{vergleiche} auch \S~10 Abs.~1 S.~3~SGO),\index[paridx]{SGO!10@\S~10!1@Abs.~1} andererseits auch zur Verwendung durch eine Rechtsmittelinstanz.
Darüber hinaus dient die Dokumentation auch dem längerfristigen Nachweis, \S~14 Abs.~5~SGO\index[paridx]{SGO!14@\S~14!5@Abs.~5}.

In \S~15~SGO\index[paridx]{SGO!15@\S~15} schließlich ist die Rechenschaftslegung nach außen geregelt.
Die Veröffentlichungsrechte und -pflichten sind die Begrenzung der ansonsten für die Mitglieder von Schiedsgerichten geltenden Verschwiegenheitspflicht (\S~2 Abs.~4~SGO).\index[paridx]{SGO!2@\S~2!4@Abs.~4}\index[idx]{Verschwiegenheitspflicht}
Da Dokumentation für den internen Gebrauch und die Rechenschaftslegung zur Veröffentlichung eng miteinander verflochten sind, werden sie hier gemeinsam behandelt.

\section{Verfahrensakte}
\label{Dokumentation:Akte}
\index[idx]{Akte!Inhalt}
\index[idx]{Verfahrensakte|see{Akte}}
Der Umfang der Verfahrensakte wird von \S~14 Abs.~2~SGO\index[paridx]{SGO!14@\S~14!2@Abs.~2} definiert.
Da es sich hierbei ausschließlich um Texte handelt -- von Tonaufzeichnungen sind Protokolle anzufertigen, \S~14 Abs.~3 S.~2~SGO\index[paridx]{SGO!14@\S~14!3@Abs.~3} -- empfiehlt es sich, die verschiedenen Quellen in einer Datei (bspw.\nomenclature{bspw.}{beispielsweise} im PDF\nomenclature{PDF}{Portable Document Format}) zusammenzufassen.
Soweit die Akte aus einzelnen Dateien besteht, empfiehlt sich der Übersicht halber die Benennung nach eindeutigen Zeitstempeln\footnote{Bspw. im Format \emph{JJMMDD}.} als Präfix.

Als \enquote{Schriftstücke} i.S.d.\nomenclature{i.S.d.}{im Sinne des} \S~14 Abs.~2~SGO sind auch sämtliche Aktennotizen (z.B.\nomenclature{z.B.}{zum Beispiel} zu (fern-) mündlich gestellten Anträgen etc.\nomenclature{etc.}{et cetera} zu verstehen.
Nicht zur Akte gehören hingegen die Beratungen der Richter:
Da das Abstimmverhalten nicht mitzuteilen ist, \S~12 Abs.~3 S.~4~SGO\index[paridx]{SGO!12@\S~12!3@Abs.~3} und die Richter allgemein Verschwiegenheit zu wahren haben, \S~2 Abs.~4~SGO,\index[paridx]{SGO!2@\S~2!4@Abs.~4} sind auch Beratungen der Richter untereinander nicht Teil der Akte.
Stattdessen umfasst der Begriff der \enquote{relevanten Schriftstücke} (u.a.)\nomenclature{u.a.}{unter anderem} jedweden (Schrift-) Verkehr zwischen Gericht und den Parteien, sowie verfahrensleitende Beschlüsse, dienstliche Stellungnahmen der Richter (bspw. zur Besorgnis der Befangenheit) und die Notizen über Spruchkörperveränderungen.

Die Verfahrensakte ist nach Abschluss des Verfahrens 5~Jahre aufzubewahren, \S~14 Abs.~5 S.~1~SGO.\index[paridx]{SGO!14@\S~14!5@Abs.~5}
Das ist auch elektronisch möglich, bspw. im ohnehin verwendeten Ticket-System oder als konsolidiertes PDF.
Die Akte ist gegen unbefugten Zugriff zu sichern; neben dem jeweils amtierenden Gericht haben nur die jeweiligen Verfahrensparteien das Recht auf Akteneinsicht, \S~14 Abs.~4~SGO.\index[paridx]{SGO!14@\S~14!4@Abs.~4}
Nach Ablauf der Aufbewahrungsfrist ist die Akte (mit Ausnahme des Urteils, \S~14 Abs.~5 S.~2~SGO),\index[paridx]{SGO!14@\S~14!5@Abs.~5} zu vernichten bzw. zu löschen.

\section{Protokolle}
\label{Dokumentation:Protokolle}
In Bezug auf Protokolle fallen die Dokumentations- und Berichtspflichten der Schiedsgerichte unterschiedlich aus:
Es kommt darauf an, ob es sich um Protokolle von (fern-) mündlichen Verhandlungen handelt (Verhandlungsprotokolle), oder um solche von den übrigen Sitzungen des Gerichts (Sitzungsprotokolle).

\subsection{Verhandlungsprotokolle}
\label{Dokumentation:Protokolle:Verhandlungsprotokolle}
Die Protokolle von Verhandlungen sind Teil der Verfahrensakte, \S~14 Abs.~2~SGO.\index[paridx]{SGO!14@\S~14!2@Abs.~2}
Die SGO sieht dabei explizit \emph{Verlaufsprotokolle} vor:
Solche sind umfangreicher als bloße \emph{Ergebnisprotokolle}, da sie auch den Inhalt des jeweiligen Vorbringens, d.h. ggf.\nomenclature{ggf.}{gegebenenfalls} auch der rechtlichen Diskussion widergeben.
Sie sind allerdings weniger umfangreich als \emph{Wortprotokolle} und geben im Gegensatz zu diesen das Vorbringen der Beteiligten nur zusammenfassend und in indirekter Rede wider, anstatt den genauen Wortlaut aufzuführen.

Auch die Verschriftlichung von Tonaufzeichnungen muss nur als Verlaufsprotokoll erfolgen.
Soweit das Gericht wörtliche Widergabe der Aufzeichnung im Protokoll wünscht, ist es zulässig, eine solche anzufertigen und damit über die Anforderungen der Satzung hinauszugehen.
Ein Anspruch der Parteien darauf besteht indes nicht.

\subsection{Sitzungsprotokolle}
\label{Dokumentation:Protokolle:Sitzungsprotokolle}
Die Protokollierung von Sitzungen des Schiedsgerichts, in denen nicht mit den Beteiligten verhandelt, sondern lediglich das Verfahren -- oder Administratives -- beraten wird, ist in der SGO ungeregelt.
Da keine Pflicht besteht, kann eine Protokollierung nicht verbindlich verlangt werden.
Es besteht allerdings auch kein Verbot.
Genauer:
Die Verschwiegenheitspflicht verbietet eine Protokollierung nicht:
Es steht dem Gericht im Rahmen seiner inneren Organisationsfreiheit frei, diese Sitzungen nach eigenem Ermessen zu protokollieren.
Lediglich die Veröffentlichung ist reglementiert.

Hier bietet sich eine Veröffentlichung im Rahmen der regelmäßigen Berichtspflicht des Schiedsgerichts nach \S~15 Abs.~1~SGO\index[paridx]{SGO!15@\S~15!1@Abs.~1} und bzw.\nomenclature{bzw.}{beziehungsweise} oder im Arbeitsbericht an.
In beiden Fällen sieht die SGO jeweils nur Mindestanforderungen für die Veröffentlichung vor, über die die Schiedsgerichte in gewissem Umfang hinausgehen dürfen.

\section{Beschlüsse}
\label{Dokumentation:Beschlüsse}
Im Gegensatz zum \emph{Urteil} ist der Beschluss in der SGO nicht eigens geregelt.
Dabei ist bereits der Begriff selbst nicht eindeutig belegt:
Einerseits ist ein Beschluss die formelle Entschließung eines Organs durch seine Mitglieder; im Rahmen der (staatlichen) Gerichtsbarkeit werden als \emph{Beschluss} auch Entscheidungen bezeichnet, die -- im Gegensatz zu \emph{Urteilen} -- nicht nach mündlicher Verhandlung ergehen.
Dieser Nomenklatur folgend wäre die Mehrzahl der \enquote{Urteile} im Rahmen der Schiedsgerichtsbarkeit der Piratenpartei wohl als \enquote{Beschluss} zu bezeichnen.%Verweis auf Urteilskapitel; dort noch nachbessern!

Tatsächlich kennt auch die SGO Beschlüsse, die ein Verfahren abschließen, ebenso allerdings auch Beschlüsse, die das Verfahren lediglich vorantreiben und ggf. in eine bestimmte Richtung lenken (sog. \emph{verfahrensleitende} Beschlüsse).
Alle Beschlüsse müssen jedenfalls als Teil der jeweiligen Verfahrensakte dokumentiert werden.
Unterschiedlich ausfallen muss jedoch die Bewertung bezüglich der Veröffentlichung (und damit auch Aufbewahrung) je nach Kategorie, der der Beschluss angehört.

\subsection{Verfahrensleitende Beschlüsse}
\label{Dokumentation:Beschlüsse:Verfahrensleitend}
Verfahrensleitende Beschlüsse sind Entscheidungen des Gerichts, die sich auf den Verfahrensfortgang beziehen, ohne das Verfahren zu beenden.
Diese lassen sich wiederum untergliedern in rechtsmittelfähige und unanfechtbare Beschlüsse.

\subsubsection{Rechtsmittelfähige Beschlüsse}
\label{Dokumentation:Beschlüsse:Verfahrensleitend:Rechtsmittelfähig}
Gegen einen rechtsmittelfähigen Beschluss können die Beteiligten auch aus dem Verfahren heraus ein Rechtsmittel zur nächsten Instanz erheben (vgl. auch \S~13 Abs.~6~SGO).\index[paridx]{SGO!13@\S~13!6@Abs.~6}
Der einzige rechtsmittelfähige, aber gleichzeitig nicht auch das Verfahren (zmd. am jeweiligen  Gericht) beendende Beschluss ist der Beschluss, der die Ablehnung eines Richters durch eine Streitpartei für unbegründet erklärt, \S~5 Abs.~6 S.~2, 3~SGO.\index[paridx]{SGO!5@\S~5!6@Abs.~6}
Da das Verfahren am ursprünglichen Gericht weiterläuft und die Entscheidung im Urteil ohnehin im Rahmen der Prozessgeschichte zumindest zu erwähnen ist, kann eine Veröffentlichung hier unterbleiben.
Zudem besteht für die Beteiligten die Möglichkeit des Rechtsmittelgebrauchs, die ihrerseits eine zu veröffentlichende Entscheidung (dann des Obergerichts) nach sich zöge. 

\subsubsection{Unanfechtbare Beschlüsse}
\label{Dokumentation:Beschlüsse:Verfahrensleitend:Unanfechtbar}
\index[idx]{Beschluss!Unanfechtbarkeit}
Nicht anfechtbar sind bspw. die Eröffnung eines Verfahrens gemäß \SSS~8 Abs.~6 S.~1, 9 Abs.~1 S.~1~SGO,\index[paridx]{SGO!8@\S~8!6@Abs.~6}\index[paridx]{SGO!9@\S~9!1@Abs.~1} der Ausschluss eines Mitglieds des Spruchkörpers wegen der Besorgnis der Befangenheit (\S~5 Abs.~6 S.~1~SGO\index[paridx]{SGO!5@\S~5!6@Abs.~6}), sowie die Anordnungen der Schiedsgerichte zu Fristen, über Beweisanträge (vgl. \S~10 Abs.~1 S.~2~SGO)\index[paridx]{SGO!10@\S~10!2@Abs.~2} und alle sonstigen Beschlüsse, gegen die nicht explizit ein Rechtsmittel vorgesehen ist.\footnote{Vgl. \S~13 Abs.~6~SGO\index[paridx]{SGO!13@\S~13!6@Abs.~6}, ebenso \cite[5]{LSGBB147}.}

Unanfechtbare Beschlüsse dürften in der Regel eindeutig der Verfahrensakte zuzuordnen sein.
Eine Veröffentlichung muss daher nicht erfolgen.
Soweit eine Veröffentlichung nicht erfolgt, gilt aber auch die fünfjährige Aufbewahrungsfrist für Akten, anstatt der unbegrenzten Aufbewahrungspflicht für Urteile.

% Sonderfall Verweisung an den Senat des BSG → oder ist das verfahrensbeendend? Mal BSG-Praxis angucken!

\subsection{Verfahrensabschließende Beschlüsse}
\label{Dokumentation:Beschlüsse:Verfahrensabschließend}
Aus der SGO ergeben sich allerdings eine erhebliche Anzahl an Beschlüssen, die ein Verfahren beenden (können).
Dies sind:
\begin{enumerate}
\item Die Übertragung eines Verfahrens an den Senat des Bundesschiedsgerichts, \S~3 Abs.~11 S.~7~SGO,\index[paridx]{SGO!3@\S~3!11@Abs.~11}
\item die Verweisung eines Verfahrens wegen Handlungsunfähigkeit des Gerichts, \S~6 Abs.~5~SGO,\index[paridx]{SGO!6@\S~6!5@Abs.~5}
\item die Übernahme eines Verfahrens durch das Obergericht wegen Verfahrensverzögerung, \S~10 Abs.~9 S.~5~SGO,\index[paridx]{SGO!10@\S~10!9@Abs.~9}
\item die Ablehnung der Verfahrenseröffnung (Nichteröffnung),\index[idx]{Nichteröffnung} \S~8 Abs.~6~SGO,\index[paridx]{SGO!8@\S~8!6@Abs.~6}
\item Erlass oder Ablehnung einer einstweiligen Anordnung (da hiermit das Verfahren im einstweiligen Rechtsschutz zunächst beendet wird), \S~11 Abs.~1, 7~SGO.\index[paridx]{SGO!11@\S~11!1@Abs.~1}\index[paridx]{SGO!11@\S~11!7@Abs.~7}
\end{enumerate}

Es ist naheliegend, diese Beschlüsse aufgrund ihrer verfahrensbeendenden Wirkung wie Urteile zu behandeln und die auf Urteile anzuwendenden Vorschriften ebenfalls (ggf. analog) anzuwenden.\footnote{In Bezug auf einstweilige Anordnungen ist dies bereits durch die SGO vorgeschrieben, \S~11 Abs.~7~SGO.\index[paridx]{SGO!11@\S~11!7@Abs.~7}}

\section{Urteile}
\label{Dokumentation:Urteile}
Der Aufbau der Urteile ist bereits ausführlich behandelt worden (vgl. Kapitel~\ref{Urteilsaufbau} ab S.~\pageref{Urteilsaufbau}).
Diese Originalfassung muss in Papierform vorliegen (\Zitat{schriftlich}, \Zitat{von allen Richtern unterschrieben}, \S~12 Abs.~7~SGO\index[paridx]{SGO!12@\S~12!7@Abs.~7}) und darf keine Schwärzungen enthalten.
Für sie gelten daher besondere Aufbewahrungsbestimmungen (vgl.~\ref{Dokumentation:Aufbewahrung}).

Anders verhält es sich mit der für die Veröffentlichung bestimmte (\S~12 Abs.~8~SGO)\index[paridx]{SGO!12@\S~12!8@Abs.~8} Fassung.
Diese Fassung ist einerseits vom Schiedsgericht selbst zu veröffentlichen, andererseits ist eine Kopie dem Bundesschiedsgericht zuzuleiten, \S~12 Abs.~9 S.~1~SGO.
Die zur Veröffentlichung bzw. zur Weiterleitung an das Bundesschiedsgericht bestimmte Fassung muss in unterschiedlichem Umfang geschwärzt werden.

\subsection{Pseudonymisierung}
\label{Dokumentation:Urteile:Pseudonymisierung}
\index[idx]{Pseudonymisierung}
Ist das Verfahren öffentlich, so wird das Urteil insgesamt veröffentlicht, \S~12 Abs.~8 S.~1~SGO.\index[paridx]{SGO!12@\S~12!8@Abs.~8}
Zum Schutz der Persönlichkeitsrechte aller Beteiligten soll allerdings der Rückschluss auf ihre Identität erschwert werden.
Daher sind ihre Namen zu pseudonymisieren.

Pseudonymisieren ist das Ersetzen des Namens und anderer Identifikationsmerkmale durch ein Kennzeichen zu dem Zweck, die Bestimmung des Betroffenen auszuschließen oder wesentlich zu erschweren (\S~3 Abs.~6a~BDSG).\index[paridx]{BDSG!3@\S~3!6a@Abs.~6a}\nomenclature{BDSG}{Bundesdatenschutzgesetz}
Zu beachten ist, dass laut Satzung eine Pflicht zur Pseudonymisierung lediglich für die Namen besteht.
Davon sind zwar auch sämtliche Arten Nicknames umfasst, aber eben -- im Gegensatz zur Definition aus dem BDSG -- keine sonstigen Merkmale der Person.
Die Bestimmung der Identität der Beteiligten soll also erschwert werden; verpflichtet, sie tatsächlich unmöglich zu machen, ist das Schiedsgericht nicht.
Im Interesse der Beteiligten können aber auch andere Merkmale, die zur einfachen Identifizierung einer Person dienen können, pseudonymisiert werden.
Insbesondere gilt dies für Anschrift und Kontaktdaten von Individuen, die zwar nicht von Satzung wegen, aber aus Gründen des Datenschutzes zu schwärzen sind.
Aus der Praxis der Schiedsgerichte hat sich ergeben, dass sie gleich dem Namen zu behandeln sind. 

Ausdrücklich ausgenommen von der Pseudonymisierungspflicht sind gemäß \S~12 Abs.~8 S.~2~SGO\index[paridx]{SGO!12@\S~12!8Abs.~8} lediglich die Namen von Gliederungen und die Namen der Richter in ihrer Funktion.
Es soll also aus den Urteilen auch für die Öffentlichkeit stets hervorgehen, welche Gliederung in welcher Weise beteiligt war.
Auch die Bezeichnung von Organen (z.B. Vorstand, Kreisparteitag, etc.) darf nicht geschwärzt werden, da es sich hierbei nicht um Personen, sondern eben um Organe handelt.
Soweit aber Personen für die Gliederungen (oder deren Organe) handeln, sind diese wiederum zu pseudonymisieren -- selbst wenn sie in Funktion (bspw. als Vorstand oder als Prozessvertretung) handeln.
Kein Recht auf Pseudonymisierung in einem Urteil haben lediglich Richter, soweit sie als Richter auftreten und handeln.
Sobald ein Richter lediglich als Mitglied der Partei oder sonst außerhalb seiner Funktion als Richter im Urteil benannt wird, wäre sein Name ebenfalls zu pseudonymisieren.

Die Vorschrift wurde in der Vergangenheit von den Schiedsgerichten sehr eng ausgelegt bzw. fast lax gehandhabt:
So wurden bspw. Anwälte, die als Prozessvertreter auftreten, namentlich und mit Kanzleianschrift benannt oder vom streitgegenständlichen Geschehen betroffene Bundesvorstandsmitglieder namentlich im Sachverhalt aufgeführt.\footnote{Die Urteile liegen den Verfassern vor; auf einen Nachweis wurde aus offensichtlichen Gründen bewusst verzichtet.}

Da die Personennamen nicht anonymisiert, sondern lediglich pseudonymisiert werden müssen, ist die Verwendung von personenbezogenen Schlüsseln zulässig.
Das bedeutet, dass innerhalb eines Urteils (aber nicht darüber hinaus!) die Personen jeweils wiedererkennbar sein dürfen.
Hier bietet sich an, die Personen alphabetisch fortlaufend mit Buchstaben zu benennen (bspw. Zeugen A, B und C), oder aber Kürzel ihrer Funktion nach (bspw. Landesvorstandsmitglied~L, Prozessvertreter~V, Zeuge~Z) einzuführen.
Eine bloße Abkürzung von Namen auf den ersten Buchstaben hingegen sollte unterbleiben, da hierbei die Identifizierung zumindest innerhalb der Piratenpartei zu einfach möglich wäre.

\subsection{Nichtöffentliche Verfahren}
\label{Dokumentation:Urteile:Nichtöffentlich}
\index[idx]{Verschlusssachen}
Ist das Verfahren nichtöffentlich, so wird lediglich der Tenor veröffentlicht, \S~12 Abs.~8 S.~3~SGO.\index[paridx]{SGO!12@\S~12!8@Abs.~8}
Die Bestimmung ist -- auch in Ansehung der Formulierung des \S~12 Abs.~3 S.~1~SGO\index[paridx]{SGO!12@\S~12!Abs.~3} -- dergestalt auszulegen, dass das Rubrum zum \enquote{Tenor} gehört (vgl.~\ref{Urteilsaufbau:Rubrum}~f.).\index[idx]{Tenor}

Effektiv wird das Urteil wie ein Urteil eines öffentlichen Verfahrens geschwärzt bzw. pseudonymisiert (s.o. \ref{Dokumentation:Urteile:Pseudonymisierung}), allerdings vor Schilderung des Sachverhalts und der Entscheidungsgründe \enquote{abgeschnitten}.

Diese Bestimmung dient dem Schutz der Persönlichkeitsrechte des \enquote{disziplinierten} Mitglieds.
Sie birgt allerdings das Problem, dass in der Praxis nur ein kleiner Teil der (Individual-) Ordnungsmaßnahmen veröffentlicht werden:
Der Beschluss zum Nichtöffentlichen Verfahren ist in diesen Fällen eine gebundene Entscheidung, \S~10 Abs.~7 S.~2~SGO.\index[paridx]{SGO!10@\S~10!7@Abs.~7}
Insbesondere, da auf diese Möglichkeit im Eröffnungsbeschluss hinzuweisen ist, \S~9 Abs.~4 S.~1~SGO,\index[paridx]{SGO!9@\S~9!4@Abs.~4} sind viele Ordnungsmaßnahmeverfahren nichtöffentlich.
Für die Parteiöffentlichkeit, insbesondere aber auch Vorstände niedrigerer Untergliederungen mit geringerem Aufkommen an Ordnungsmaßnahmen und auch ortsfremde Schiedsgerichte besteht daher das Problem, dass insbesondere bestätigte Ordnungsmaßnahmen und erfolgreiche Anträge auf Parteiausschlussverfahren inhaltlich nicht nachvollziehbar sind.
Dies erschwert die Etablierung von Beurteilungsmaßstäben, welches Verhalten disziplinarwürdig ist, und welches nicht.

Eine Möglichkeit, die in solchen Verfahren aufgeworfenen Rechtsfragen dennoch öffentlich zu diskutieren, besteht im Rahmen des Arbeitsberichts (vgl.~\ref{Dokumentation:Rechenschaftslegung:Arbeitsbericht:nichtöffentlicheVerfahren}).

\section{Öffentliche Mitteilungen}
\label{Dokumentation:Veröffentlichungen}
Abgesehen von den zu veröffentlichenden Urteilen und Protokollen treten die Schiedsgerichte eher selten in Kommunikation mit anderen Organen oder der Gesamtpartei.
Ausnahmen hiervon sind vor allem die Bekanntmachung von Beeinflussungsversuchen, sowie die Stellungnahmen zu laufenden Verfahren.

\subsection{Bekanntmachung von Beeinflussungsversuchen}
\label{Dokumentation:Veröffentlichungen:Beeinflussungen}
\index[idx]{Beeinflussungsversuch}
Gemäß \S~2 Abs.~5~SGO\index[paridx]{SGO!2@\S~2!5@Abs.~5} sind die Schiedsgerichte verpflichtet, Versuche der Beeinflussung eines Verfahrens öffentlich bekannt zu machen.
Was eine Beeinflussung ist, liegt letztlich im Ermessen des Schiedsgerichts.
Insbesondere der Versuch, dem Schiedsgericht Weisungen zu erteilen (im Innenverhältnis der Partei durch \S~2 Abs.~2~SGO,\index[paridx]{SGO!2@\S~2!2@Abs.~2} im Außenverhältnis durch \S~14 Abs.~2 S.~4~PartG\index[paridx]{PartG!14@\S~14!2@Abs.~2} verboten) stellt eine solche veröffentlichungspflichtige Tatsache dar.
Obwohl der Wortlaut nahelegt, dass sich die Vorschrift ausschließlich auf die Beeinflussung von bestimmten Verfahren bezieht, ist aufgrund der besonderen (auch gesetzlich geschützten) Bedeutung der Unabhängigkeit der Schiedsgerichte anzunehmen, dass sich die Norm auf jedwede Beeinflussung des Organs Schiedsgericht oder der Richter in ihrer Funktion bezieht.
Es ist dabei prinzipiell nicht maßgeblich, ob die Quelle der Beeinflussung inner- oder außerhalb der Partei liegt.

Die Öffentlichkeit, der der Beeinflussungsversuch bekannt gemacht werden soll, ist die Parteiöffentlichkeit.
Eine Verlautbarung über eine geeignete Mailingliste (bspw. die Aktiven-Liste des Landes, ggf. sogar des Bundesverbandes) ist daher zur Bekanntmachung prinzipiell ausreichend.

Eine explizite Aufbewahrungsfplichtfür die Bekanntmachung von Beeinflussungsversuchen ergibt sich aus der SGO nicht.
Um allerdings die Unabhängigkeit der Schiedsgerichte zu schützen und aus dem Prinzip der Transparenz heraus bietet es sich an, die Verlautbarung auch im Rahmen der allgemeinen Dokumentation des Schiedsgerichts abrufbar zu halten.

Bezüglich der Dokumentation und Aufbewahrung bietet sich daher eine Behandlung entsprechend den Urteilen an (wobei die Aufbewahrung einer unterschriebenen Fassung wohl unterbleiben kann).
Zum Schutze der Persönlichkeitsrechte der Beteiligten sollte eine angemessene Pseudonymisierung stets in Betracht gezogen werden.

\subsection{Stellungnahmen zu laufenden Verfahren}
\label{Dokumentation:Veröffentlichungen:Stellungnahmen}
\index[idx]{Stellungnahme}
Das Gericht kann zu laufenden Verfahren öffentliche Stellungnahmen abgeben, \S~15 Abs.~2 S.~1~SGO.\index[paridx]{SGO!15@\S~15!2@Abs.~2}
Voraussetzung dafür ist, dass das Verfahren öffentlich ist (\S~15 Abs.2 S.~2~SGO)\index[paridx]{SGO!15@\S~15!2@Abs.~2} und dass das Schiedsgericht ein erhebliches parteiöffentliches Interesse feststellt.

Die \emph{Parteiöffentlichkeit}\index[idx]{Parteiöffentlichkeit} beschreibt den öffentlichen Raum, den die Mitglieder der Partei gemeinsam bilden.
Für ein \enquote{parteiöffentliches Interesse} sind Umstände außerhalb der Piratenpartei also nicht von Belang.
Ein solches Interesse kann einerseits \enquote{von Seiten} der Parteiöffentlichkeit bestehen, oder aber \enquote{für} sie:
Im ersteren Fall wird das Verfahren bereits von einem (erheblichen) Teil der Parteiöffentlichkeit verfolgt.
Dies lässt sich bspw. an Diskussionen über das Verfahren ablesen, die an für die Parteiöffentlichkeit exponiert wahrnehmbarer Stelle stattfinden, oder auch an direkten Nachfragen an das Gericht.
Im zweiten Fall ist eine tatsächliche Kenntnisnahme durch die Parteiöffentlichkeit unerheblich.
Hier ist nicht maßgeblich, dass von Seiten eines erheblichen Teils der Parteiöffentlichkeit ein Interesse an Verfahrensdetails besteht, sondern, dass dieses Interesse bestehen sollte.

Das Gericht kann zu Stellungnahmen nicht verpflichtet werden.
Ebenso liegt der Umfang der Stellungnahme ausschließlich im Ermessen des Schiedsgerichts.
Entsprechend den Vorschriften zum Urteil sollte pseudonymisiert werden.
Das Gericht sollte besonderes Augenmerk darauf richten, die Veröffentlichung neutral zu halten, um keiner Partei einen Anlass zur Besorgnis der Befangenheit (vgl.~\ref{Zusammensetzung:Spruchkoerper:Befangenheitsbesorgnis}) zu geben.

Da Stellungnahmen nur zu laufenden Verfahren zulässig sind, erlischt das Recht prinzipiell mit Abschluss des Verfahrens.
In der entsprechenden Satzungsbestimmung aber ein Verbot von Korrekturen oder sachgerechten Ergänzungen bereits veröffentlichter Stellungnahmen zu sehen, widerspräche jedoch dem Sinn der Vorschrift, die Parteiöffentlichkeit sachgerecht zu unterrichten.
Korrekturen müssen daher immer, Ergänzungen in engen Grenzen ebenfalls zulässig sein, soweit das Gericht eine Stellungnahme veröffentlicht hat.
Zu beachten ist:
Das Urteil darf in solchen, nachträglichen Veröffentlichungen, nicht berührt werden.

Bezüglich der Dokumentation und Aufbewahrung bietet sich eine Behandlung entsprechend der Bekanntmachungen von Beeinflussungsversuchen an:
Stellungnahmen sind als \enquote{relevantes Schriftstück} i.S.d. \S~14 Abs.~2~SGO\index[paridx]{SGO!14@\S~14!2@Abs.~2} zugleich Teil der Verfahrensakte.\index[idx]{Akte!Inhalt}
Während der Aufbewahrungsfrist bietet sich daher eine Aufbewahrung mit der Verfahrensakte an.
Da die Stellungnahme aber auch veröffentlicht wurde, sollte sie unbegrenzt verfügbar gehalten werden; ein Interesse an einer Depublizierung ist der SGO nicht zu entnehmen.

\section{Aufbewahrung der Akten}
\label{Dokumentation:Aufbewahrung}
Die Aufbewahrung der Akten bestimmt sich nach der Geschäftsordnung des Schiedsgerichts, \S~2 Abs.~6 S.~2~SGO.\index[paridx]{SGO!2@\S~2!6@Abs.~6}
Sie sollte möglichst zweckmäßig erfolgen.
Eine zentrale, elektronische Verwaltung, auf die das gesamte Gericht zugreifen kann, ist daher sinnvoll.
Hinsichtlich der Erstellung von Backups etc. ist ein Austausch mit den Technikverantwortlichen der Gliederung empfehlenswert.

Auf Anfrage können möglicherweise die Datenschutzbeauftragten in der Piratenpartei Hilfestellungen zum Datenschutz geben.

\subsection{Laufende Verfahren}
Für laufende Verfahren ist bedeutsam, dass der gesamte Spruchkörper einfachen Zugriff auf die gesamte Verfahrensakte nehmen kann.
Ebenso ist wichtig, dass sie zu jedem Zeitpunkt insgesamt an die Beteiligten gesendet werden kann, um deren Recht auf Akteneinsicht gewährleisten zu können.

Am einfachsten lässt sich dies durch ein geeignetes Fallbearbeitungssystem\footnote{Auch \emph{Issue-Tracking-System}, \href{https://de.wikipedia.org/wiki/Issue-Tracking-System}{Wikipedia (de): Issue-Tracking-System}.} gewährleisten.
Auch (in nicht abschließender Aufzählung) \enquote{Cloud}-Systeme, ein gemeinsam genutzer FTP\nomenclature{FTP}{File Transfer Protocol}-Server oder Etherpad bzw. Piratenpad Teampads sind hierzu geeignet.
Unabhängig von der Software bzw. dem genutzten Protokoll ist wichtig, dass die administrativen Rechte auf dem jeweiligen Server in der Hand der Piratenpartei liegen.
Dienste Dritter dürfen nicht genutzt werden.
Dass die Nutzung der in den Akten enthaltenen Daten ausschließlich beim Schiedsgericht liegen darf, versteht sich von selbst.

\subsection{Abgeschlossene Verfahren}
Mit Abschluss des Verfahrens gelten für die Verfahrensakte und das Urteil (und ggf. weitere Beschlüsse, s.o.~\nomenclature{s.o.}{siehe oben}\ref{Dokumentation:Beschlüsse}) unterschiedliche Bestimmungen.

\subsubsection{Verfahrensakten}
Soweit ein elektronisches Aufbewahrungssystem genutzt wird, können die Akten über die Aufbewahrungsfrist darin verbleiben.

In dem Falle, dass die Geschäftsordnung des Gerichts eine sofortige Löschung der elektronischen Akte festlegt, ist sie auf einem Datenträger oder in gedruckter Form (in jedem Falle aber vollständig) beim Gericht zu hinterlegen.
Hierfür bietet sich eine gegen unbefugten Zugriff gesicherte Aufbewahrung in der Geschäftsstelle der jeweiligen Gliederung an.
Zweckmäßig ist dann die Aufbewahrung im verschlossenen Umschlag mit außen angebrachtem Verfallsdatum.

Soweit es während des Verfahrens mehrere Speicherorte für die Akte gab (bspw. jeweils bei den mit dem Verfahren befassten Mitgliedern des Spruchkörpers), sollten diese nach Verfahrensabschluss auf ein einzelnes Archiv der Akte reduziert werden.
Nach Ablauf der fünfjährigen Aufbewahrungsfrist (\S~14 Abs.~5~SGO)\index[paridx]{SGO!14@\S~14!5@Abs.~5} ist die Akte an all ihren Speicherorten zu vernichten.

\subsubsection{Urteile}
Urteile sind unbefristet aufzubewahren, \S~14 Abs.~5 S.~2~SGO.\index[paridx]{SGO!14@\S~14!5@Abs.~5}
Dies bezieht sich insbesondere auf die schriftliche, von allen (am Verfahren beteiligten) Richtern unterschriebene Originalfassung nach \S~12 Abs.~7~SGO.\index[paridx]{SGO!12@\S~12!7@Abs.~7}
Diese sollten zentral, bspw. in der Geschäftsstelle der Gliederung, aufbewahrt werden.
Sie müssen gegen den Zugriff Unbefugter gesichert sein (bspw. Verschluss in einem Schrank o.ä.\nomenclature{o.ä.}{oder ähnliche(r/s)}).

Eine getrennte Aufbewahrung der Urteile ist hingegen nicht notwendig.
Einfaches Abheften (ggf. in intransparenter Hülle) in einem Ordner ist ausreichend.
Zugriffsrechte bestehen nur für die Beteiligten und für das Gericht.

\subsection{Sonstige Akten}
Die Aufbewahrung sonstiger Akten liegt im Ermessen des Gerichts.
Sie regelt sich nach der Geschäftsordnung oder aber der im Gericht gängigen Praxis.
Es gelten lediglich die allgemein üblichen Vorschriften (Datenschutz etc.).

Auch der Zugang zu sonstigen Akten des Gerichts ist grundsätzlich zu beschränken, da sie von der Verschwiegenheitspflicht aus \S~2 Abs.~4~SGO\index[paridx]{SGO!2@\S~2!4@Abs.~4} umfasst sind.
Ausnahmen hiervon sind möglich.
Als Faustformel kann gelten:
\Zitat{Was veröffentlicht ist, bleibt veröffentlicht.}
Insbesondere alle im Rahmen der Rechenschaftslegung erfolgenden Veröffentlichungen, sowie Bekanntmachungen und Stellungnahmen (s.o.~\ref{Dokumentation:Veröffentlichungen}) sollten daher nicht depubliziert werden.

\section{Rechenschaftslegung}
\label{Dokumentation:Rechenschaftslegung}
Das Gericht legt für seine Arbeit öffentlich Rechenschaft.\index[idx]{Rechenschaftspflicht}
Es unterliegt dabei einerseits einer laufenden Berichts- und Veröffentlichungspflicht, andererseits einer (zusammenfassenden) Berichtspflicht dem das Gericht wählenden Parteitag gegenüber.

\subsection{Laufende Berichtspflicht}
\label{Dokumentation:Rechenschaftslegung:Laufend}
Das Gericht soll während der Amtszeit regelmäßig berichten; gemäß \S~15 Abs.~1~SGO\index[paridx]{SGO!15@\S~15!1@Abs.~1} insbesondere über die Zahl der Fälle.
Allerdings hat das Gericht auch Urteile zu veröffentlichen, \S~12 Abs.~8 S.~1~SGO.\index[paridx]{SGO!12@\S~12!8@Abs.~8}
Als Praxis aller Schiedsgerichte hat sich daher eingebürgert, dieser Berichtspflicht durch die Führung eines regelmäßig aktualisierten, öffentlichen Verfahrensverzeichnisses nachzukommen.
Diese sind häufig im Wiki der jeweiligen Gliederung angelegt.

Der Satzungsbestimmung genügt dabei dem Wortlaut nach die Nennung der Anzahl der anhängigen, sowie der der (in der laufenden Amtsperiode) abgeschlossenen Verfahren.
Die Praxis der Gerichte geht darüber hinaus und listet die einzelnen Verfahren zumindest mit Aktenzeichen, dem Datum der Verfahrenseröffnung (ggf. auch der Anrufung), den veröffentlichten Beschlüssen (s.o.~\ref{Dokumentation:Beschlüsse}), und einer kurzen Zusammenfassung von Sachverhalt und ggf. der Verfahrensgeschichte auf.

Technisch ist die Veröffentlichung innerhalb eines elektronischen Systems sinnvoll, das sowohl für Menschen, als auch für Maschinen lesbar ist.
Insbesondere die Urteile sollten als eigenständige Dokumente verfasst und nicht lediglich in einer Datenbank abgelegt sein.
Dies erleichtert die Nachvollziehbarkeit der Dateiintegrität, bspw. über Prüfsummen oder sogar Signaturen.\footnote{Das Bundesschiedsgerichts bspw. hatte zeitweise die Urteile im PDF veröffentlicht und mit PGP\nomenclature{PGP}{Pretty Good Privacy (Verschlüsselungstechnologie)} signiert.}
Eingeschränkt trifft dies z.B. auf ein Wiki zu, weswegen die überwiegende Mehrheit der Gerichte solche Systeme für ihre laufende Rechenschaftslegung verwendet.

Die bisher leistungsfähigste Dokumentationsführung basiert auf dem verteilten Versionskontrollprogramm \emph{git},\footnote{\href{http://www.git-scm.org/}{Offizielle Website: http://www.git-scm.org/}.} was durch kryptographische Methoden auch die Dateiintegrität ausreichend sicherstellt und zudem vollständig maschinenlesbar ist.
Bei Verwendung maschinenlesbarer Urteilsdokumente (z.B. durchsuchbare PDF) ist auch eine Durchsuchbarkeit gewährleistet; einer eigenen Suchfunktion innerhalb der Software bedarf es dazu nicht.
Die Software selbst ist in der Adminsitration simpel gehalten und besitzt kaum Anforderungen an das Hosting-System.
Dabei bietet sie permanente Links auf das Urteil, sowie den Listeneintrag mit den obigen Informationen und bspw. die bereits erwähnten Vorteile der kryptographischen Dateiintegritätsprüfung und die Maschinenlesbarkeit.
Die Software ist quelloffen unter freier Lizenz und wurde auf Github\footnote{Github selbst ist ein kommerzielles Projekt. Das ändert nichts an der Lizenz der Software. Sie ist daher ausdrücklich \emph{nicht} als proprietär zu bewerten.} entwickelt.\footnote{\href{https://github.com/Bundesschiedsgericht/BSG}{https://github.com/Bundesschiedsgericht/BSG}.}

In keinster Weise für die Rechenschaftslegung geeignet ist ein Blogsystem.
Die Zuordnung einzelner Veröffentlichungen zu einem Verfahren gelingt kaum; die Integrität der Veröffentlichungen ist im Vergleich zu den geschilderten Alternativen kaum darstellbar.
Hinsichtlich der Lesbarkeit ist festzuhalten, dass ein Blog nicht in demselben Maße auslesbar ist, wie es ein git-repository oder auch eine Wikiseite ist -- letztere bieten Einblick in die Datenquellen; das Blog nicht.

\subsection{Arbeitsbericht}
\label{Dokumentation:Rechenschaftslegung:Arbeitsbericht}
\index[idx]{Arbeitsbericht}
Der Arbeitsbericht des Schiedsgerichts wird dem Parteitag vorgelegt, der es (neu) wählt.

\subsubsection{Inhalt}
\label{Dokumentation:Rechenschaftslegung:Arbeitsbericht:Inhalt}
An den Inhalt dieses Arbeitsberichts stellt die SGO höhere Anforderungen:
Gemäß \S~15 Abs.~3~SGO\index[paridx]{SGO!15@\S~15!3@Abs.~3} soll er die Fälle der Amtsperiode inklusive der jeweiligen Urteile kurz darstellen.
Das trifft den Umfang, der sich bei der laufenden Rechenschaftslegung (s.o.~\ref{Dokumentation:Rechenschaftslegung:Laufend}) als gängige Praxis eingestellt hat.
Der dort genannte Umfang ist deswegen nicht nur legitimiert (da die entsprechenden Daten ohnehin veröffentlicht würden); die Erstellung des Arbeitsberichts verläuft so \enquote{nebenbei} über die gesamte Amtszeit und wird dadurch erleichtert.
Wird der laufenden Berichtspflicht im bereits geschilderten Umfang gefolgt, so können die Fälle der Amtszeit aus der laufend aktualisierten Übersicht in den Arbeitsbericht einfach übertragen werden.

Darüber hinaus sollte der Arbeitsbericht aber auch weitere Informationen enthalten, um aus sich heraus verständlich zu sein:
Zunächst ist die Zusammensetzung des Schiedsgerichts bei seiner Wahl und alle weiteren Veränderungen (durch Rücktritte etc.) eine wichtige Information.
Die Auflistung der Zusammensetzung der Spruchkörper in einzelnen Verfahren kann unterbleiben, da dies aus den Urteilen ersichtlich wird.
Zumindest die Anzahl der Sitzungen und Verhandlungen des Schiedsgerichts sollte aufgeführt werden, ggf. sollte ein Hinweis auf Protokollierung erfolgen.
Hieran kann der Parteitag die Aktivität, aber auch den Arbeitsaufwand des Schiedsgerichts ablesen.
Das ist insbesondere für Bewerber um das Richteramt eine relevante Information.
Zuletzt sollten auch weitere Aktivitäten des Schiedsgerichts im Arbeitsbericht nicht fehlen.
Hier sind v.a. Weiterbildungen und vergleichbare Veranstaltungen zu nennen.

Verfügt das Schiedsgericht über ein eigenes Budget, ist eine Rechenschaftslegung darüber im Arbeitsbericht empfehlenswert.\index[idx]{Budget}

\subsubsection{Darstellung nichtöffentlicher Verfahren}
\label{Dokumentation:Rechenschaftslegung:Arbeitsbericht:nichtöffentlicheVerfahren}
\index[idx]{Verschlusssachen}
Schließlich bietet der Arbeitsbericht die Möglichkeit, dem Problem der nicht-öffentlichen Rechtsprechung Herr zu werden (s.o.~\ref{Dokumentation:Urteile:Nichtöffentlich}).
Zwar sind nichtöffentliche Verfahren auch vom Gericht vertraulich zu behandeln, \S~9 Abs.~4 S.~2~SGO.\index[paridx]{SGO!9@\S~9!4@Abs.~4}
Das Urteil darf nur eingeschränkt veröffentlicht werden, \S~12 Abs.~8 S.~4~SGO,\index[paridx]{SGO!12@\S~12!8@Abs.~8} und öffentliche Stellungnahmen sind vollständig unzulässig, \S~15 Abs.~2 S.~2~SGO.\index[paridx]{SGO!15@\S~15!2@Abs.~2}
Demgegenüber sollen aber alle Fälle der Amtsperiode im Arbeitsbericht kurz dargestellt werden, \S~15 Abs.~3~SGO.\index[paridx]{SGO!15@\S~15!3@Abs.~3}

Die Lösung besteht in der Führung einer eigenen Rubrik für nichtöffentliche Verfahren im Arbeitsbericht, die lediglich rechtliche Erwägungen ohne nachvollziehbaren Bezug zu einem einzelnen Verfahren enthält.
Im Stile von \enquote{Leitsätzen}\index[idx]{Leitsatz} können hier die Kerninhalte der Rechtsprechung nichtöffentlicher Verfahren zusammenfassend veröffentlicht werden.
Dabei muss sichergestellt werden, dass ein unbeteiligter Dritter keine unmittelbaren Rückschlüsse auf ein einzelnes Verfahren ziehen kann.
Insbesondere Daten, Aktenzeichen oder Schilderungen von Details oder handelnden Personen des Sachverhalts, sowie ggf. Gliederungsnamen etc. dürfen sich daher nicht in der Schilderung wiederfinden.
Weiterhin müssen bei der geschilderten Verfahrensweise mindestens zwei nichtöffentliche Verfahren vorliegen, da ansonsten eine unmittelbare Zuordnung zu einem einzelnen Verfahren möglich ist.

Sollte nur ein einziges Verfahren vorliegen, liegt die Lösung in einer entsprechenden Notiz für das nachfolgende Schiedsgericht und einem Hinweis im Arbeitsbericht, dass die rechtlich bedeutsamen Aspekte des nichtöffentlichen Verfahrens in den nächsten Arbeitsberichten erörtert werden sollen, sobald mindestens ein weiteres nichtöffentliches Verfahren vorliegt.

Alternativ kann auch eine gesammelte Veröffentlichung erfolgen.
Soweit sich die Gerichte untereinander darauf einigen, wäre auch eine zentrale Veröffentlichung solcher Rechtsprechungsinhalte denkbar, bspw. im Arbeitsbericht des Bundesschiedsgerichts.
Diese Vorgehensweise sollte dann durch gemeinsame (bzw. zmd.\nomenclature{zmd.}{zumindest} gleichlautende) Bestimmungen der Geschäftsordnung\index[idx]{Geschäftsordnung} der beteiligten Gerichte geregelt sein.

\chapterbib
% \end{refsection}


% \begin{refsection}
\chapterpreamble{Die Schiedsgerichtsordnung sieht ein Verfahren in zwei Instanzen vor: Der Eingangsinstanz und der Berufungsinstanz. Das Bundesschiedsgericht ist Eingangsinstanz für Verfahren, die sich gegen die Bundespartei oder ihre Organe richten. Ein Landesschiedsgericht ist Eingangsinstanz für Verfahren, die sich gegen den Landesverband oder eines seiner Organe richten. Bei Einsprüchen gegen Ordnungsmaßnahmen ist das Gericht niedrigster Ordnung, in der der Antragsteller seinen Wohnsitz hat, zuständig. In allen sonstigen Fällen ist das niedrigste Gericht, in dessen örtlicher Zuständigkeit der Antragsgegner sich befindet, zuständig. Die Berufungsinstanz wird nach Abschluss des erstinstanzlichen Verfahrens durch Einlegen der Berufung zuständig. Sie führt das Verfahren als weitere Tatsacheninstanz, d.h. erhebt erneut Beweis und gewährt den Parteien erneut vollumfassend rechtliches Gehör. Es hat sich jedoch die Möglichkeit der Rückverweisung bei Rechtsfehlern etabliert.}

\chapter{Der Instanzenzug}
%\blindtext[1]

%\chapterbib
% \end{refsection}


\appendix


\onecolumn
\printbibliography[heading=bibliography]
\printnomenclature

\printindex[jurisdiction]


\end{document}