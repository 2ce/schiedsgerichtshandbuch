\documentclass{sghandbuch}

\newcommand*{\documentversion}{Milestone 1}
\author{Simon Gauseweg \and Florian Zumkeller-Quast}
\title{Schiedsgerichtshandbuch\\
für die Piratenpartei Deutschland und Untgliederungen\\
\hfill\break
\hfill\break
\normalsize{\documentversion}}

\usepackage{blindtext}
\addbibresource{sghandbuch.bib}
\newindex[Urteilsverzeichnis]{ruling}
\newindex[Stichwortverzeichnis]{idx}
\newindex[Paragraphenverzeichnis]{paridx}
\makenomenclature
% \usepackage{refsection}

\begin{document}

\frontmatter
\pagenumbering{Roman}

\maketitle
\thispagestyle{empty}		% Seitennummerierung für Titelseite unterdrücken

% Widmung (Zitat)
\onecolumn			% Einspaltenlayout anschalten
\thispagestyle{empty}		% Seitennummerierung eine Seite unterdrücken
\hfill				% Zentrierung 
\vfill				% Zentrierung
\begin{quote}
\centering			% Zentrierung
\itshape
\Huge
Eine Meinung zu haben gehört zum Berufsbild eines Richters.

\par
\begin{flushright}
\Large
\normalfont
-- (unbekannt, Hans-Jürgen Papier zugeschrieben)
\end{flushright}
\end{quote}
\hfill				% Zentrierung
\vfill				% Zentrierung
\clearpage			% Seitenende
\twocolumn			% Wieder ins Zweitspaltenlayout wechselns

\renewcommand{\contentsname}{Inhaltsübersicht}
\pdfbookmark{\contentsname}{toc}
\tableofcontents

% \begin{refsection}
\chapterpreamble{Ein Schiedsgerichtshandbuch ist geeignet, erforderlich und auch angemessen, die Qualität der Rechtsprechung innerhalb der Parteigerichte der Piratenpartei zu verbessern.}

\chapter{Einleitung}
%\blindtext[1]
%\blindtext[1]\footcites{BSG1415HS}{BSG1515HS}{Loewisch1985}
%\blindtext[1]
%\blindtext[1]\footcites{BSG1415HS}{LSGBB147}{Risse1985}
%\blindtext[1]
%\blindtext[1]\footcites{BSG1715HS}{LSGBB147}
%\blindtext[1]

%\chapterbib
% \printshortbibliography[title={Urteile}, type=jurisdiction,heading=chapterbibliography]
% \printshortbibliography[title={Andere Literatur}, nottype=jurisdiction,heading=chapterbibliography]
% \end{refsection}

\maincontent
\mainmatter

% \begin{refsection}
\chapterpreamble{Die Organisation der Schiedsgerichte bestimmt sich nach ihrer jeweiligen Geschäftsordnung. Diese muss bestimmten Anforderungen der Schiedsgerichtsordnung genügen. Ansonsten sollte vor allem die Zweckmäßigkeit der Regelungen und auch der verwendeten Infrastruktur (Mailinglisten, Ticketsysteme etc.) im Vordergrund stehen.  Die Richter sollten gleichen Zugriff auf die Gerichtsakten haben. Die Ersatzrichter sollten von Anfang an in die Prozesse des Gerichts eingewiesen werden, um ein Nachrücken so reibungslos wie möglich gestalten zu können. Es empfiehlt sich, dass sich das Gericht regelmäßig fortbildet.}

\chapter{Innere Organisation eines Schiedsgerichts}

%\section{Personalsachen}
%\subsection{Vorsitz, Richter, Ersatzrichter}
%\subsection{Aus- und Fortbildungsmaßnahmen}
%z.B. dieses Handbuch, Schulungsveranstaltungen, SG-Koordinations-ML, Lektüre der Urteile anderer Schiedsgerichte

%\section{Infrastruktur}

%\section{Haushalt}

%\section{Geschäftsordnung}


%\chapterbib
% \end{refsection}

% \begin{refsection}
\chapterpreamble{\enquote{\emph{Borodin:}~\enquote{Gestatten die das?}\\ \emph{Ramius:}~\enquote{Joah…}\\ \emph{Borodin:}~\enquote{Ohne Papiere?}\\ \emph{Ramius:}~\enquote{Ohne Papiere!}}\\ --- Jagd auf Roter Oktober (US, 1990)}

\chapter{Die Anrufung}
\label{Anrufung}
Ein Schiedsgericht wird nur auf Anrufung tätig schreibt die SGO in \S~8 Abs.~1 S.~1 vor.
Damit ist die Anrufung der zentrale Punkt für ein Schiedsgericht, an dem jede Tätigkeitkeit anknüpft.

Aber auch der Umgang mit einer Anrufung ist streng formal geregelt.
Das soll es den Gerichten erleichtern, gewissen Standardfrage nicht immer noch im Nachhinein stellen zu müssen, weil sie immer auftauchen.

Grundlegend sieht der Ablauf der zu prüfenden Punkte dabei wie folgt aus:
\begin{enumerate}[label=\Roman*.]
% \itemsep0em 
\item \textbf{Antragsteller}: Ist klar genannt, \emph{wer} die Klage erheben will? (\S~8 Abs.~3 Nr. 1 SGO\index[paridx]{SGO!8@\S~8!3@Abs.~3})
\item \textbf{Antragsgegner}: Ist klar genannt, gegen \emph{wen} sich die Klage richten soll? (\S~8 Abs.~3 Nr.~2 SGO\index[paridx]{SGO!8@\S~8!3@Abs.~3})
\item Sind Antragsteller und Antragsgegner \textbf{parteifähig}? (\S~8 Abs.~1 S.~1 SGO\index[paridx]{SGO!8@\S~8!1@Abs.~1})
\item Ist jeweils eine zulässige \emph{Anschrift} angegeben? (\S~8 Abs.~3 Nr.~1, 2 SGO\index[paridx]{SGO!8@\S~8!3@Abs.~3})
\item Sagt der Antragsteller klar, \emph{was} genau er erreichen will? (\S~8 Abs.~3 Nr.~3 SGO\index[paridx]{SGO!8@\S~8!3@Abs.~3})
\item Ist das, was der Antragsteller will, etwas, das ihm nach Satzung, Parteiengesetz oder sonstigen mitgliedschaftlichen oder organschaftlichen Rechten \emph{möglicherweise zustehen könnte}? (\S~8 Abs.~1 S.~1 SGO\index[paridx]{SGO!8@\S~8!1@Abs.~1})
\item Nennt der Antragsteller \textbf{Gründe}, warum das gewünschte ihm zustehen sollte? (\S~8 Abs.~3 Nr.~4 SGO\index[paridx]{SGO!8@\S~8!3@Abs.~3})
\item Ist die \textbf{Form} eingehalten? (\S~8 Abs.~3 SGO\index[paridx]{SGO!8@\S~8!3@Abs.~3})
\item Ist die \textbf{Frist} eingehalten worden? (\S~8 Abs.~4 SGO\index[paridx]{SGO!8@\S~8!4@Abs.~4})
\item Ist das angerufene Schiedsgericht \textbf{zuständig}? (\S~8 Abs.~5 SGO\index[paridx]{SGO!8@\S~8!5@Abs.~5})
\item Hat eine \textbf{erfolglose Schlichtung} stattgefunden oder ist diese \textbf{entbehrlich}? (\S~7 SGO\index[paridx]{SGO!7@\S~7})
\end{enumerate}

\section{Statthaftigkeit}
\label{Anrufung:Statthaftigkeit}
Die grundlegendsten Anforderungen an eine Anrufung werden unter dem Begriff Statthaftigkeit zusammengefasst.
Eine Klage ist grundsätzlich nur statthaft, wenn diese Anforderungen erfüllt sind.

\subsection{Antragsteller}
\label{Anrufung:Statthaftigkeit:Antragsteller}
Das allererste dieser Kriterien ist die Benennung eines \index[idx]{Antragsteller|textbf} Antragstelllers bzw. einer Antragstellerin, andernorts auch Kläger bzw. Klägerin genannt.
Die SGO kennt allerdings den Begriff Kläger bzw. Klägerin nicht, sondern nur den Begriff Antragsteller bzw. Antragstellerin, entsprechend wird auch dieser Begriff hier verwendet.
Antragsteller bzw. Antragstellerin ist typischerweise der oder die Anrufende selbst.
In einigen Fällen, typischerweise im Falle einer Anrufung durch einen Vorstand oder ein anderes Organ, kann der oder die Anrufende auch Vertreter bzw. Vertreterin des Antragstellers bzw. der Antragstellerin sein.
In jedem Fall muss aus der Anrufung klar identifizierbar hervorgehen, wer nun mit der Anrufung Rechte bzw. Ansprüche vor dem Schiedsgericht gegen eine andere Person gelten machen will.
Dieser jemand ist der Antragsteller bzw. die Antragstellerin.

In den meisten Fällen dürfte dies ein Mitglied sein, in seltenen Fällen kann es auch ein einzelner Amtsträger oder eine einzelne Amtsträgerin sein, noch häufiger dürfte es ein Organ als Gesamtes sein.
Typischerweise gibt es lediglich vier Fälle, in denen die Anrufung durch ein Organ erfolgt:
\begin{enumerate}[label=\arabic*.)]
% \itemsep0em 
\item Parteiausschlussverfahren\index[idx]{Parteiausschluss}
\item Gliederungsordnungsmaßnahmeneinsprüche\index[idx]{Ordnungsmassnahme@Ordnungsma""snahme!Gliederungs-}
\item Gliederungskompetenzstreitigkeiten
\item Organhandlungsfähigkeitsstreitigkeiten
\end{enumerate}
Der letzte Fall, die Organhandlungsfähigkeitsstreitigkeiten sind strenggenommen sogar nur ein Unterfall der Gliederungskompetenzstreitigkeiten, aber dazu an entsprechender Stelle mehr.

Aus dem Rahmen fallen natürlich auch immer die Berufungs- und Beschwerdeanrufungen im Instanzenzug und die Widerspruchsanrufung im einstweiligen Rechtsschutz, weil hier die Anrufung typischerweise von der in der ursächlichen Schiedsgerichtsentscheidung unterlegenen Partei ausgeht.

\subsection{Antragsgegner}
\label{Anrufung:Statthaftigkeit:Antragsgegner}
Der Antragsgegner\index[idx]{Antragsgegner|textbf} oder die Antragsgegnerin ist ebenso notwendig und muss ebenso klar und eindeutig aus der Anrufung entnehmbar sein. Bezüglich der Vertretung gilt hier dasselbe wie bei dem \index[idx]{Antragsteller} Antragsteller bzw. der Antragstellerin.

Bei allen Verfahren nach SGO handelt es sicht um sogenannte kontradiktorische Verfahren.
Das bedeutet, dass das Verfahren Streit von zwei Parteien um gegenläufige Interessen ausgestaltet ist.
In den meisten Fällen ist der Antragsgegner bzw. die Antragsgegnerin daher ein Organ, in wenigen Fällen -- primär bei gerichtlich verhängten Ordnungsmaßnahmen wie dem Parteiausschluss -- aber auch ein einzelnes Mitglied.
Wichtig dabei ist aber: Es gibt nach SGO keine Verfahren von Mitgliedern gegen andere Mitglieder.
Schon fraglich ist, ob ein einzelner Amtsträger bzw. eine Amträgerin gegen einen anderen Amträgräger bzw. eine andere
Amtsträgerin einzeln vorgehen kann, dazu gab es bisher noch nie einen Anlass, das zu entscheiden.
Theoretisch ist es aber nicht komplett ausgeschlossen.
Möglich ist in jedemfall das vorgehen einer Amtsträgerin bzw. eines Amtsträgers gegen das restliche Organ, eines Organs  gegen einzelne Amtsträger und Amtsträgerinnen sowie das vorgehen von Mitgliedern gegen Organe und andersherum.
Wichtig ist daher der Grundsatz: Aus dem Antragsteller oder der Antragstellerin ergibt sich zwangsweise ein eingeschränkter Kreis möglicher Antragsgegner und Antragsgegnerinnen.

\subsection{Parteifähigkeit}
\label{Anrufung:Statthaftigkeit:Parteifaehigkeit}
Gerade wude es schon angesprochen: Antragsteller bzw. Antragstellerin und Antragsgegner bzw. Antragsgegnerin sind die \index[idx]{Streitpartei}Parteien des Prozesses.
Nach SGO sind Mitglieder und Organe explizit dazu berechtigt, ein Schiedsgericht anzurufen, wenn sie sich in ihren Rechten verletzt fühlen, \S~8 Abs.~1 S.~2 SGO\index[paridx]{SGO!8@\S~8!1@Abs.~1}.
Das bedeutet, dass diese auch zwingend Partei in einem Schiedsgerichtsverfahren sein können.
Ferner muss die Parteifähigkeit\index[idx]{Parteifähigkeit} auch Amtsträgern als spezieller Art von Mitgliedern zugestanden werden.
Ob man dabei davon ausgeht, dass Amtsträger eine eigene Kategorie von Partei darstellen oder lediglich Mitglider sind, deren Rechte aufgrund einer Wahl zeitlich beschränkt verändert wurden, ist eigentlich eine akademische Debatte.
Angesichts dessen, dass Amtsträger aber in die Rechte anderer Amtsträger eingreifen können und dann im Sinne eines effektiven Rechtsschutzes entweder das Recht des anderen Amtsträgers zu einem Organrecht erklärt werden muss oder aber eine Klage von Mitglied gegen Mitglied zugelassen werden muss, spricht doch einiges dafür, Amtrsträger als eigene, nicht explizit genannte,  aber doch erfasste Kategorie innerhalbt der Parteifähigkeit zu sehen.

Brisanz gewinnt die Einordnung der Amtsträger als eigene parteifähige Kategorie bei der Klage von Mitgliedern gegen Amtsträgern, was dadurch nicht mehr von der nicht erlaubten Klage von Mitglied gegen Mitglied vor dem Schiedsgericht ausgeschlossen ist.
Allerdings müsste dann eine Klagebefugnis vom Mitglied vorliegen, typischerweise ist die Rechtsverletzung aber nicht dem  einzelnen Amtsträger sondern dem ganzen Organ zuzurechnen genauso wie sich Ansprüche aus der Mitgliedschaft typischerweise gegen das zuständige Organ richten. 
Daher kann fast davon ausgegangen werden, dass es nicht zu einer solchen Klagekonstellation kommen kann bzw. falls doch, dass diese dann gerade nötig ist, um einen effektien Rechtsschutz für Mitglieder zu gewähren.

Fraglich ist zudem ob die Partei bzw. die einzelnen Gliederungen Parteifähig im Sinne der SGO sind.
Während dies für Prozesse vor staatlichen Gerichten aufgrund der dortigen Regelungen zur Parteifähigkeit\footnote{Vgl. etwa \S~50 ZPO\nomenclature{ZPO}{Zivilprozessordnung}\index[paridx]{ZPO!50@\S~50}.} grundsätzlich anzunehmen ist, spielt die Partei bzw. Gliederung als solche im internen Rechtsstreit eine untergeordnete Rolle, da die Handlung immer zumindest einem Organ oder zumindest Amtsträger zugeordnet werden kann.
Daher ist auch das Bundesschiedsgericht in seiner Rechtsprechung dazu übergegangen, die Parteifähigkeit der Partei und ihrer Gliederungen nur zu bejahen, wenn eine derartige Zurechnung nicht möglich ist.\footnote{Vgl. \cite[S. 4]{BSG1614HS}; Diese Rechsprechung fortführend \cite[S. 2]{BSG3815HS} m.w.N.}
Die Partei und die Gliederungen kommen also nur als subsidiärer Antragsgegner bzw. Antragsgegnerin in Betracht.

Bisher gibt es keine bekannten Fallgruppen, für die dies generell der Fall ist.
Auch die Aufstellungsversammlungen, die zwar nicht zwingend Organe im Sinne des \S~9 BS\index[paridx]{BS!9@\S~9} sind, werden jedenfalls mindestens als Organe im Sinne des \S~8 Abs.~1 S.~1 SGO\index[paridx]{SGO!8@\S~8!1@Abs.~1} anzusehen sein.
Dies liegt einerseits nahe, da schon \S~8 Abs.~7 SGO\index[paridx]{SGO!8@\S~8!7@Abs.~7} den Schiedsgerichten, die oft Organe im Sinne des \S~9 BS\index[paridx]{BS!9@\S~9} sind, die Parteifähigkeit abspricht und somit klar ist, dass die beiden Satzungsnormen unterschiedliche Organbegriffe haben, und andererseits auch das Parteiengesetz selbst schon unterschiedlichen Organbegriffe kennt und \S~9 BS\index[paridx]{BS!9@\S~9} dort den engeren Begriff gemäß \S~8 Abs.~2 PartG\index[paridx]{PartG!8@\S~8!2@Abs.~2} abbildet.
\S~8 Abs.~1 S.~1 SGO\index[paridx]{SGO!8@\S~8!1@Abs.~1} greift also einen weiter gefassten Organbegriff auf, der grundsätzlich jede formalisierte Gruppe, die für und im Namen der Partei handelt, umfasst und somit auch Aufstellungsversammlungen miteinschließt.
Die Regelung für die Vertretung von Verfahrensparteien in \S~9 Abs.~3 S.~2 SGO\index[paridx]{SGO!9@\S~9!3@Abs.~3} stützt diese Deutung, da diese schon nicht mehr von anderen Organen spricht, sondern von Mitgliederversammlungen, die gerade nicht zwingend Organe im Sinne des engeren Begriffes sein müssen.
Und die Aufstellungsversammlungen der Piratenpartei sind in der Regel Mitgliederversammlungen, sodass diese Norm in der Regel anwendbar ist.

\subsection{Anschrift}
\label{Anrufung:Statthaftigkeit:Anschrift}
Der Antragsteller muss seine eigene Anschrift sowie die des Antragsgegners angeben.
Die Anschrift ist notwendig, damit das Schiedsgericht im Verfahren mit den Parteien in Kontakt treten kann.
Auch wenn die Kommunikation per E-Mail im Verfahren üblich ist, ist es hier notwendig, dass eine postalische Adresse genannt wird.\footnote{Ausführlich dazu \cite[S. 3]{BSG20130715} m.w.N.}
So gibt es zwar keinen Anspruch mehr in der SGO auf eine Urteilszusendung in Schriftform, die die Kenntnis einer postalischen Adresse früher zwingend machte, jedoch ist die Anschrift nach wie vor zur Individualisierung und damit zur Mitgliedsverifikation durch das Schiedsgericht erforderlich. Auch im Fall, dass eine Person mittels einer lediglich vom Verfahrensgegner genannten E-Mailadresse nicht erreichbar ist, ist eine zusätzlicher Kontaktversuch per Brief erforderlich, um die Rechte der nicht erreichbaren Verfahrenspartei zu sichern.

Da es allerdings nicht die Erforderlichkeit, den Ansprüchen einer ladungsfähigen Adresse genügen zu müssen, ist eine Postfachanschrift ausreichend und erfüllt die Anforderungen der Schiedsgerichtsordnung.\footnote{Mit Begründung etwa \cites[3]{BSG20130715}[S.~4 Rn~16]{LSGBB145}{LSGBB134}. Lediglich bestätigend etwa \cites[7]{LSGBB133}[S.~10~f. Rn~48]{LSGBB146}.}
Dies gilt nicht nur für den Fall, dass die Anrufung durch ein Mitglied erfolgt,\footnote{So aber \cites[S.~1]{LSGBB134}.} sondern gerade auch für den Fall, dass die Anrufung durch ein Organ oder eine Gliederung erfolgt.\footnote{\cites[S.~2~f.]{BSG20131230}.}

Gerade aber der Grund der Mitgliedsverifikation könnte auch gegen die Zulässigkeit einer Postfachanschrift zur Erfüllung von \S~8 Abs.~3 Nr.~1, 2 SGO\index[paridx]{SGO!8@\S~8!3@Abs.~3} sprechen.
Wie das Landesschiedsgericht Brandenburg zutreffend ausführt, hängt das davon ab, ob im Mitgliederverzeichnis ebenfalls Postfachanschriften geführt werden können.\footnote{\cites[S.~4 Rn~16]{LSGBB145}.}
Dies sollte aber gerade nicht der Fall sein, da die Anschrift im Mitgliederverzeichnis regelmäßig zur Feststellung der Stimmberechtigung bei Aufstellungsversammlungen nach den Wahlgesetzen benutzt wird und dafür auch notwendig ist.
Ist aber die Angabe eine Postfachanschrift im Mitgliederverzeichnis unzulässig, kann darüber gerade keine Identitätsverifikation erfolgen.
Die Postfachanschrift erfüllt ihren Zweck nach \S~8 Abs.~3 Nr.~1, 2 SGO\index[paridx]{SGO!8@\S~8!3@Abs.~3} dann gerade nicht mehr und erfüllt daher diese Anforderung nicht mehr.

Die weiteren Kontaktdaten gemäß \S~8 Abs.~3 Nr.~1 SGO\index[paridx]{SGO!8@\S~8!3@Abs.~3} umfassen Daten wie eine E-Mailadresse. Diese ist aus prozessökonomischen Gründen sinnvoll, da ein Information per E-Mail die Verfahrensparteien schneller erreicht und diese auch schneller reagieren können. Das ganz das Verfahren insgesamt erheblich beschleunigen und macht die direkte Versendung in Kopie zudem einfacher.

Dass die Anschrift unbekannt ist, kann auch nie ein Problem darstellen. Verfahren von Mitglied gegen Mitglied sind schon gar nicht vorgesehen (siehe \ref{Anrufung:Statthaftigkeit:Antragsgegner}), der Partei ist die Anschrift eines Mitglieds aus der Mitgliederdatenbank bekannt, die Anschrift der Partei und der Gliederungen sind allgemein bekannt, die Organe haben ihre Anschrift für gewöhnlich bei der Gliederung, oder, falls diese abweicht, ist sie vom Organ bekannt gemacht.

\subsection{Anträge}
\label{Anrufung:Statthaftigkeit:Antraege}
Ebenfalls zwingend erforderlich ist es, dass der Antragssteller mit der Anrufung bereits Anträge stellt.
Diese Anträge müssen nach \S~8 Abs.~3 Nr.~3 SGO\index[paridx]{SGO!8@\S~8!3@Abs.~3} klar und eindeutig sein.
Diese innere Dopplung der Formulierung zeigt schon, dass es dem Satzungsgeber darauf ankam, dass der Antragsteller explizit sagen muss, was exakt er mit der Anrufung erreichen will.
Das Schiedgericht ist daher nicht verpflichtet, zu spekulieren und Vermutungen darüber anzustellen, was der Antragsteller gewollt haben könnte. 
Wenn der Antragsteller das nicht in seiner Anrufung klar erkennbar niederschreibt, dann ist die Anrufung unvollständig und daher nicht statthaft.

Die Erfordernis der genauen Anträge dient auch dazu, die genaue Klageart festzustellen, da diese Einfluss auf das weitere Verfahren haben kann.
Zudem dienst diese Anforderung dem Zweck, sicherzustellen, dass das Schiedsgericht weiß, in welche Richtung es seine Amtsermittlungspflicht gemäß \S~10 Abs.~1 S.~1 Hs.~1 SGO\index[paridx]{SGO!10@\S~10!1@Abs.~1} ausüben muss, um alle relevanten Fragen beantworten zu können, die sich für eine Entscheidung in der Sache stellen.

Bei der Prüfung, ob Anträge gestellt wurden, ist also eine sehr strenger Maßstab anzulegen. Wird nicht klar und eindeutig spezifiziert, was der Antragsteller vom Antragsgegner will, liegt schon gar kein Antrag vor.
Ob der Antrag erfüllbar ist und rechtmäßig gestellt wurde, wird allerdings hier noch nicht entschieden.
Insofern ist es auch nur eine sehr formale und eingschränkte Prüfung, ob einer oder mehrere Anträge gestellt wurden.

Und natürlich reicht entgegen der Pluralformulierung von \S~8 Abs.~3 Nr.~3 SGO\index[paridx]{SGO!8@\S~8!3@Abs.~3} ein einzelner Antrag auch aus. Es ist nicht Sinn und Zweck der Regelung, dass ein Antragsteller, der nur eine einzelne Sache erreichen will, einen absolut unnötigen zusätzlichen Antrag stellt und das Schiedsgericht so mit Mehrarbeit belastet oder gar schon überhaupt nicht klagen darf, nur weil er nur eine Sache erreichen will.

\subsection{Begründung und Umstände}
\label{Anrufung:Statthaftigkeit:Gruende}
Die letzte Anforderung des \S~8 Abs.~3 SGO\index[paridx]{SGO!8@\S~8!3@Abs.~3} sind eine Darstellung der Umstände und Schilderung einer Begründung der Anträge.
Damit soll dem Schiedsgericht ein Anhaltspunkt für seine Amtsermittlung des Sachverhaltes nach \S~10 Abs.~1 S.~1 Hs.~1 SGO\index[paridx]{SGO!10@\S~10!1@Abs.~1} gegeben werden und dem Antragsgegner eine Möglichkeit, auf die Klage zu erwiedern.
Der Sinn und Zweck dieser Anforderung ist also, einen Startpunkt für das Verfahren haben, und den Streit der Parteien einsortieren zu können.
Entsprechend sind hieran keine hohen inhaltlichen Anforderungen zu stellen.
Solange ein Geschehen geschildert wird und irgendwelche Argumente angeführt werden, warum diese oder jene rechtliche Bewertung dafür zu gelten habe, ist regelmäßig anzunehmen, dass diese Punkt erfüllt ist.
Andernfalls würde \S~8 Abs.~3 Nr.~4 SGO\index[paridx]{SGO!8@\S~8!3@Abs.~3} eine Vorwegnahme der eigentlichen Beurteilung des Verfahrens darstellen, dass das Gericht erst nach ausführlicher Beschäftigung mit der Sache selbst durch die Durchführung des Verfahrens tätigen soll.

\section{Weitere Kriterien}
\label{Anrufung:Kriterien}
Neben der Statthaftigkeit gibt es noch weitere Kriterien, die für eine erfolgreiche Anrufung erfüllt sein müssen.
Dies sind unechte Zulässigkeitskriterien, da sie die Klage unzulässig machen, aber nicht erst im Verfahren mit dem Urteil beschieden werden, sondern schon mit Eröffnung.
Die Satzung nennt diese Kriterien auch Zulässigkeitskriterien der Eröffnung, vgl. \S~8 Abs.~6 Satz~4 SGO\index[paridx]{SGO!8@\S~8!6@Abs.~6}.

\subsection{Antragsbefugnis}
\label{Anrufung:Kriterien:Antragsbefugnis}
Die Notwendigkeit der Antragsbefugnis geht aus \S~8 Abs.~1 SGO\index[paridx]{SGO!8@\S~8!1@Abs.~1} hervor und soll sogenannte \emph{Popularklagen} verhinden.
Eine Antragsbefugnis liegt dann vor, wenn der Antragsteller selbst in seinen eigenen Rechten aus der Mitgliedschaft oder seinem sonstigen innerparteilichen Status verletzt ist oder ein Anspruch aus dieser Position nicht erfüllt wird.
Die Abgrenzung von eigenen Rechten und eigenen Ansprüchen ist nicht immer einfach, aber auch nicht zwingend nötig, da Alternativ eines von beides ausreicht.
Entsprechend den anderen nicht rein förmlichen Anrufungskriterien ist hier kein hoher inhaltlicher Anspruch zu stellen.
Daher muss für die Eröffnung lediglich dargelegt werden, dass ein solches Recht oder ein solcher Anspruch verletzt sein könnte.
Eine tiefere Prüfung erfolgt bei der Entscheidung über Eröffnung nicht, sondern im Zweifelsfall immer erst im laufenden Verfahren, da auch ergänzender Parteivortrag und Ermittlungen des Gerichts Einfluss auf diese Entscheidung haben und es daher einer ausführlichen Wertung durch das Schiedsgericht im Laufe des Verfahrens bedarf.
Hat das Schiedgericht von Anfang an Zweifel an der in der Anrufung dargeleten Begründung der Antragsbefugnis, ist das Verfahren dennoch zu eröffnen. Es bietet sich dann aber an, der anrufenden Partei einen richterlichen Hinweis zu geben, dass diese ihre Begründung diesbezüglich erweitern sollte.
Das geht auch aus \S~8 Abs.~5, 6 Satz~1 SGO\index[paridx]{SGO!8@\S~8!5@Abs.~5} hervor. Dort wird bestimmt, wann das Verfahren zu eröffnen ist. Nämlich immer dann, wenn das angerufene Schiedsgericht zuständig ist und die Anrufung korrekt erfolgte.
Korrekte Anrufung kann aber nur formelle Kriterien meinen, eine weitergehende, inhaltliche Prüfung zu diesem Zeitpunkt ist daher unzulässig.
Die Antragsbefugnis ist damit ein echtes Zulässigkeitskriterium und erst nach durchgeführtem Verfahren im Endurteil zu bescheiden, an die Anrufung selbst darf daher nur die Anforderung gestellt werden, dass die Möglichkeit einer solchen Antragsbefugnis in zumindest nicht gänzlich unplausibler Weise geltend gemacht wird.

\subsection{Form}
\label{Anrufung:Kriterien:Form}
Die Anforderungen an die Form der Anrufung sind sehr gering.
Es wird in \S~8 Abs.~3 SGO\index[paridx]{SGO!8@\S~8!3@Abs.~3} lediglich die Textform gefordert.
Im Gegensatz zur Schriftform, deren Erfüllung und Nichterfüllung bei untergesetzlich vorgeschriebener Formerfordernis sehr strittig ist und auch in der Schiedsgerichtsbarkeit der Piratenpartei schon zu Entscheidungen führte,\footcite{LSGBB146} ist die Textform ausreichend klar und eindeutig in \S~126b BGB\index[paridx]{BGB!126b@\S~126b} definiert und eine Formlockerung in \S~127 BGB\index[paridx]{BGB!127@\S~127} nicht vorgesehen.
Es reicht daher ein Schreiben, das mit dem Namen unterzeichnet ist und das in einer Form, die zur visuellen Widergabe des Inhalts geeignet ist (also auch z.B. eine E-Mail).

\subsection{Frist}
\label{Anrufung:Kriterien:Frist}
Die Frist für eine Anrufung soll sicherstellen, dass Streitigkeiten schnell geklärt werden.
Damit soll verhindert werden, dass auf Basis einer möglicherweise rechtswidrigen Handlung weitere Dinge passieren und eine Klage, die erst nach langer Zeit eingereicht wird, unerwartet alles mitreißen kann.
Kurz gesagt: Nach einer gewissen Zeit sollen alle möglicherweise Betroffenen auf einen eingetretenen Zustand vertrauen können.
Daher darf nach Ablauf dieser Frist nicht mehr geklagt werden und eine Anrufung, die bei Fristende noch nicht vollständig beim angerufenen Schiedsgericht eingegangen ist, ist grundsätzlich abzuweisen.\footnote{So die ständige Rechtsprechung des Bundesschiedsgerichts, statt vieler: \cites{BSG2315HS}{BSG20130227}.}

Dabei kennt die SGO unterschiedliche Anrufungsfristen.
\begin{enumerate}
\item Die Standardanrufungsfrist von 2 Monaten, \S~8 Abs.~4 Satz~1 SGO,\index[paridx]{SGO!8@\S~8!4@Abs.~4}
\item Die Anrufungsfrist in Einspruchsverfahren gegen Ordnungsmaßnahmen von 14 Tagen, \S~8 Abs.~4 Satz~2 SGO,\index[paridx]{SGO!8@\S~8!4@Abs.~4}
\item Die Anrufungsfrist in Parteiausschlussverfahren, \S~8 Abs.~4 Satz~3 SGO,\index[paridx]{SGO!8@\S~8!4@Abs.~4}
\item Die Berufungsfrist gegen ein Urteil mit korrekter Rechtsmittelbelehrung, \S~13 Abs.~2 Satz~1 SGO,\index[paridx]{SGO!13@\S~13!2@Abs.~2}
\item Die Berufungfrist gegen ein Urteil ohne korrekte Rechtsmittelbelehrung, \S~13 Abs.~2 Satz~4 SGO.\index[paridx]{SGO!13@\S~13!2@Abs.~2}
\end{enumerate}

Die Ausnahme stellt hier die Anrufungsfrist im Parteiausschlussverfahren dar.
Diese ist keine formelle Frist wie die anderen Fristen dar, sondern ist eine wertende, materielle Präklusionsfrist.
Das bedeutet, dass über die fristgerechte Einbringung der Vorwürfe nicht schon mit Eröffnung zu bescheiden ist, sondern erst im Urteil nach durchlaufenem Verfahren.
Dies soll einerseits das Parteimitglied dagegen schützen, dass es sich gegen alte Vorwürfe schützen muss und andererseits ein Parteiausschlussverfahren auch dann noch ermöglichen, wenn einzelne Handlungen schon länger zurück liegen, aber Teil eines Großen und Ganzen sind, das einen einzigen, fortlaufende Gesamtverstoß gegen Satzung, Grundsätze oder Ordnung der Partei mit schwerem Schaden darstellt.\footnote{Zur Zulässigkeit des Parteiausschlusses aufgrund eines solchen Gesamtverstoßes \cite[S.~19]{LSGHE20130624}\nomenclature{LSG~HE}{Landesschiedsgericht Hessen} m.w.N.\nomenclature{m.w.N.}{mit weiteren Nachweisen} und später bestätigend \cite{BSG20131028}.}
Alle anderen genannten Fristen sind dagegen echte formelle Fristen die schon die Eröffnung eines Verfahrens verhindern sollen.

Die Fristberechnung folgt dabei den \SSS~186 ff. BGB\index[paridx]{BGB!186@\S~186}.
Der Fristanfang wird nach \S~187 Abs.~1~BGB\index[paridx]{BGB!187@\S~187 Abs.~1} bestimmt, das Fristende nach \S~188 BGB\index[paridx]{BGB!188@\S~188}.
Das ist auch der Grund, warum in der Schiedsgerichtsordnung eine Ladungsfrist für (fern-)mündliche Verhandlungen in \S~10 Abs.~5 Satz~2 SGO\index[paridx]{SGO!10@\S~10!5@Abs.~5} auf 13 Tage gesetzt ist und nicht etwa 14 Tage oder zwei Wochen.
So kann das Schiedsgericht in seiner Sitzung einen Verhandlungstermin beschließen und wenn es am selben Tag noch zur Verhandlung lädt, kann diese am selben Wochentag in der zweiten Woche danach stattfinden und nicht erst in der dritten Woche danach.

Besonders geachtet werden sollte dabei vor allem auf \S~193 BGB\index[paridx]{BGB!193@\S~193} geachtet werden: Endet die Frist an einem Samstag, Sonntag oder gesetzlichen Feiertag, dann verlängert sich die Frist entsprechend.
Eine Fristende an einem Samstag um 24 Uhr wird somit zu einem Fristende am Montag um 24 Uhr (vorausgesetzt, dass der Montag kein Feiertag ist, dann wäre das Ende am Dienstag um 23 Uhr usw.).
Dabei müssen vor allem das Bundesschiedsgericht, aber auch für Landesschiedsgerichte, die verwiesene Verfahren behandeln, im Auge behalten, dass jedes Bundesland eigene Feiertagsgesetze hat.

\subsection{Zuständigkeit}
\label{Anrufung:Kriterien:Zustaendigkeit}
Über die Zuständigkeit hat das Gericht bereits mit Anrufung zu entscheiden, \S~8 Abs.~5 SGO\index[paridx]{SGO!8@\S~8!5@Abs.~5}.
Dies hat den Hintergrund, dass die SGO anders als die staatliche Gerichtsbarkeit, keine Verweisung vom unzuständigen, angerufenen Gericht an das zuständige Gericht vorsieht.
Auch sind die Zuständigkeitskriterien der SGO nach \S~6 SGO\index[paridx]{SGO!6@\S~6} wesentlich einfacher als etwa die der ZPO.
So ist es bereits aus den Daten der Anrufung ersichtlich, ob ein Gericht zuständig ist.

\subsubsection{Grundlegendes}
\label{Anrufung:Kriterien:Zustaendigkeit:Grundlegendes}
Dabei wird instanziell bei den untersten Gerichten begonnen, \S~6 Abs.~1 SGO\index[paridx]{SGO!6@\S~6!1@Abs.~1}.
Das sind grundsätzlich die Landesschiedsgerichte.
Allerdings kann der Satzungsgeber auf Landesebene die Einrichtung von niedrigeren Schiedsgerichten in der Landessatzung für den jeweiligen Landesverband erlauben, \S~2 Abs.~1 Satz~2 SGO\index[paridx]{SGO!2@\S~2!1@Abs.~1}.
Ist das passiert, ist das niederigste Schiedsgericht der entsprechenden Regelung zu entnehmen.
Bisher ist das allerdings in keinem Landesverband passiert.

\S~6 Abs.~2 SGO\index[paridx]{SGO!6@\S~6!2@Abs.~2} bestimmt dann noch, dass sie die Zuständigkeit nach der Verbandszugehörigkeit des Antragsgegners richtet.
Verbandszugehörige sind die Mitglieder eines Verbandes, das können nach \S~2 Abs.~1 Satz~2 PartG\index[paridx]{PartG!2@\S~2!1@Abs.~1} nur natürliche Personen sein.
Dabei wird in \S~6 Abs.~2 SGO nochmal klargestellt, was sich aus \S~8 Abs.~5 SGO,\index[paridx]{SGO!8@\S~8!5@Abs.~5} der die Entscheidung über die Zuständigkeit mit der Eröffnung als Entscheidung über die Korrektheit der Anrufung zusammenlegt, ergibt: Der relevante Zeitpunkt für die Entscheidung ist immer die mitgliedschaftliche Zuordnung zum Zeitpunkt der Anrufung.
Das wiederrum hängt auch damit zusammen, dass es keine Verweisung zwischen den einzelnen Gerichten gibt und es der anrufenden Streitpartei nicht zugemutet werden soll, dass ein etwaiger Gliederungswechsel oder z.B. eine Verschmelzung von Gliederungen das schon angelaufene Verfahren ungültig macht, weil es im Nachinein unzulässig wurde.

\subsubsection{Zuständigkeit bei Verfahren gegen ein Organ}
\label{Anrufung:Kriterien:Zustaendigkeit:Organ}
Organe haben -- juristisch streng genommen -- keine Mitgliedschaft in einem Verband, sondern sind Teil eines Verbandes und handeln für diesen.
Man könnte zwar eine Mitgliedschaft eines Organs in einem Verband im allgemeinen Sprachgebrauch konstruieren für den Verband, dessen Teil sie sind.
Aber auch in diesem Fall leitet sich daraus keine Mitgliedschaft in den höheren Verbänden ab, da \S~2 Abs.~1 Satz~2 PartG\index[paridx]{PartG!2@\S~2!1@Abs.~1} ganz eindeutig die Mitgliedschaft auf natürliche Personen limitiert.
Daher lässt würde die Sonderzuständigkeitsregelung nach \S~6 Abs.~3 SGO\index[paridx]{SGO!6@\S~6!3@Abs.~3}, die die Zuständigkeit des Bundesschiedsgerichts für Verfahren gegen Bundesorgane sowie die Zuständigkeit der Landesschiedsgerichte für Verfahren gegen Landesorgane anordnet, die Zuständigkeit für Verfahren gegen Organe von Verbänden unterhalb der Landesebene offen.
Das ist eine offensichtliche planwidrige Regelungslücke, da es nicht gewollt sein kann, dass Verfahren gegen untergeordnete Gliederungen unmöglich wird, nurweil das zuständige Gericht nicht determiniert werden kann, zumal die Einrichtung von Schiedsgerichten unterhalb der Landesebene in \S~2 Abs.~1 Satz~2 SGO\index[paridx]{SGO!2@\S~2!1@Abs.~1} explizit erlaubt wird.
Deswegen drängt sich eine analoge Anwendung der Zuständigkeitsregelungen für Organe nach \S~6 Abs.~3 SGO\index[paridx]{SGO!6@\S~6!3@Abs.~3} auf: Das niedrigste Schiedsgericht auf gleicher oder höherer Verbandsebene wie das Organ, das Antragsgegner ist, ist nach SGO zuständig für Verfahren gegen ebendieses Organ.

\subsubsection{Zuständigkeit bei Disziplinarverfahren}
\label{Anrufung:Kriterien:Zustaendigkeit:Disziplinarverfahren}
Für die Verfahren mit Disziplinarcharakter, also Verfahren über Einsprüche gegen Ordnungsmaßnahmen und Parteiausschlussverfahren, ist die Zuständigkeit nochmal gesondert in \S~6 Abs.~4 SGO\index[paridx]{SGO!6@\S~6!4@Abs.~4} geregelt.
Die etwaige Einrichtung von Schiedsgerichten unterhalb der Landesebene soll nichts am Verfahrensablauf ändern und das Verfahren etwa durch eine Instanz mehr im Zweifel in die Länge ziehen und gleichzeitig soll in jedem Fall der für Parteiausschlussverfahren in \S~10 Abs.~5 Satz~2 PartG\index[paridx]{PartG!10@\S~10!5@Abs.~5} vorgeschriebene zweizügige Instanzenzug für alle Mitglieder gleich aussehen. Im selben Zug wird eine nicht notwendige zweite Instanz auch für Einsprüche gegen Ordnungsmaßnahmen garantiert.

Durch die exklusive Formulierung sperrt \S~6 Abs.~4 SGO\index[paridx]{SGO!6@\S~6!4@Abs.~4}.
Dadurch könnte man aus der Präsensformulierung des Teilsatzes \enquote{Landesverband […], bei dem der Betroffene Mitglied ist.} schließen, dass ein anderer Zeitpunkt für die Entscheidung über die Zuständigkeit relevant wird.
In Betracht kommen hier sowohl die frühere Zeitpunkt des Beschlusses des Organs über die Verhängung der Ordnungsmaßnahme oder der Einreichung eines Parteiausschlussantrages und der frühere Zeitpunkt der vorgerichtlichen Anhörung des Mitglieds im Disziplinarverfahren als auch der spätere Zeitpunkt der tatsächlichen Entscheidung über die Eröffnung.
Ein späterer Zeitpunkt käme wegen \S~8 Abs.~5 SGO\index[paridx]{SGO!8@\S~8!5@Abs.~5} definitiv nicht mehr in Betracht, eine Flucht aus einem laufenden Parteiausschlussverfahren in einen anderen Verband ist daher definitiv nicht möglich.
Grundsätzlich bietet es sich aus den schon oben geschilderten, sich aus \S~8 Abs.~5 SGO\index[paridx]{SGO!8@\S~8!5@Abs.~5}

Problematisch wird dies allerdings in den Fällen, in denen zwischen Start des Disziplinarverfahrens durch Anhörung, dem Beschluss darüber und der Schiedsgerichtsanrufung ein Verbandswechsel des Betroffenen erfolgt.
Im Falle einer Ordnungsmaßnahme würde dann ein verbandsfremdes Schiedsgericht über die Handlung eines Verbandsorgans entscheiden und auch Ermessenskontrolle über dieses Organ ausüben, im Falle eines Parteiausschlusses wäre dann auch noch die Frage, ob ein schon beschlossener und zum Beschlusszeitpunkt zulässiger, aber noch nicht einreichter Parteiausschlussantrag seine Zulässigkeit verliert, zu entscheiden.

Für die Frage der verbandsfremden Ermessenskontrolle muss allerdins bedacht werden, dass in beiden Fällen das letzinstanzlich zuständige Bundesschiedsgericht das letzte Wort hätte.
Daher scheint ein außeinanderfallen von Verbandszugehörigkeit des ersintanzlich zuständigen Schiedsgerichts und des verhängenden Organs schlussendlich doch unproblematisch, zumal die gesonderte Regelung durch \S~6 Abs.~4 SGO\index[paridx]{SGO!6@\S~6!4@Abs.~4} wohl auch das mitbezweckt hat.

Für den Parteiausschluss wird das wohl davon abhängen, wie man ihn im Vergleich zu den anderen Ordnungsmaßnahmen einsortiert. Wird er einfach nur als stärkste Ordnungsmaßnahme einsortiert, auf die das Prozedere der Ordnungsmaßnahmen nach \SSS~14, 6 Abs.~1 BS\index[paridx]{BS!14@\S~14}\index[paridx]{BS!6@\S~6!1@Abs.~1} voll anwendbar ist\footnote{So wohl die jüngste Ansicht des Bundesschiedsgerichtes in \cite{BSG3615HS} und das Sondervotum des Bundesschiedsrichters Markus Gerstel in \cite{BSG20131005}.}, muss sich der Betroffene wohl damit abfinden, dass er ab Anhörung  bzw. spätestens ab dem Beschluss des Organs über die Einreichung einen Verbandswechsel der Zulässigkeit des Verfahrens nicht mehr entgegenhalten kann, um auch hier eine Gleichheit mit den anderen Ordnungsmaßnahmen zu erreichen.
Denn wenn das betroffene Mitglied  von dem Plan für Parteiaausschlussverfahren wusste und deswegen den Verband wechselt, ist es nicht schutzwürdig, ganz unabhängig davon, ob der Wechsel im Innenverhältnis der Gliederungen nach \S~3 Abs.~2a BS\index[paridx]{SGO!3@\S~3!2@Abs.~2}
Folgt man hingegen der in \cite{BSG20131005} aufgezeigten Rechtssprechungslinie und sieht die Anhörung durch das einleitende Organ im Parteiausschlussverfahren als nicht verpflichtend an, so ergibt sich kein Grund, warum das Mitglied einen solchen Beschluss gegen sich gelten lassen muss, von dem es im Zweifel noch nicht einmal etwas wusste.
Das Vertrauen darauf, dass grundsätzlich nur die Organe der Gliederungen, in denen man tatsächlich aktuell Mitglied ist,  Disziplinarmaßnahmen gegen das Mitglied unternehmen, ist dann schutzwürdig und der relevante Zeitpunkt für die Entscheidung über die Zuständigkeit dürfte wieder mit der Anrufung beim Schiedsgericht zusammenfallen.\footnote{Vgl. wie schon oben \cite[S. 9]{BSG115HS}, unter II.1.c.c).}

\subsection{Schlichtung}
\label{Anrufung:Kriterien:Schlichtung}
Grundsätzlich ist nach \S~7 Abs.~1 SGO\index[paridx]{SGO!7@\S~7!1@Abs.~1} ein Schlichtungsversuch vor der Anrufung der Schiedsgerichtsbarkeit erforderlich.
Damit ist die Schlichtung wie auch die Zuständigkeit (siehe \ref{Anrufung:Kriterien:Zustaendigkeit}) ein besonderes \emph{unechtes Zulässigkeitskriterium}: Diese Kriterien würden in den staatlichen Gerichtsordnungen als \emph{echte Zulässigkeitskriterien} von nicht bloß formaler Natur zu einem Unzulässigkeitsurteil führen. Nach der SGO aber führen sie zu einer Nichteröffnung und sind entsprechend zum Zeitpunkt der Eröffnung abschließend zu beurteilen.

Fehlt es an einem erfolglosen Schlichtungsversuch und ist dieser auch nicht entbehrlich, fehlt der anrufenden Streitpartei das Rechtsschutzinteresse und die Klage ist als unzulässig mit einer Nichteröffnung abzuweisen.

\subsubsection{Ablauf}
\label{Anrufung:Kriterien:Schlichtung:Ablauf}
Die Schlichtung wird von den Parteien eigenständig durchgeführt, \S~7 Abs.~2 Satz~1 SGO\index[paridx]{SGO!7@\S~7!2@Abs.~2}.
Daher ist der Ablauf nicht im Verantwortungs- oder Tätigkeitsbereich der Parteischiedsgerichte.
Bei tieferer Betrachtung macht das auch Sinn:
Ein Schiedsgericht, dass schon einen Schlichtungsversuch durchführt, muss im Zuge ebendieser Tätigkeit Vorschläge zur Einigung machen und sich so zumindest schon teilweise zum konkreten Sachverhalt positionieren.
Dadurch könnte die gewünschte Neutralität eines Schiedsgerichtes verloren gehen.
Gerade deswegen dürfen auch Richter, die schon als Schlichter aktiv waren, nicht als Richter im Folgeverfahren teilnehmen (vgl. auch \ref{Zusammensetzung:Spruchkoerper:Befangenheitsvermutung:Nr8}).

Es wäre auch mit der Funktion eines mediativen Schlichters, den \S~7 SGO\index[paridx]{SGO!7@\S~7} vorsieht, nicht vereinbar, wenn der Schlichter zuerst unverbindliche Vorschläge machen würde und diese im späteren Verfahren den Parteien als Richter per Urteil aufzwingen könnte.

\subsubsection{Auswirkungen}
\label{Anrufung:Kriterien:Schlichtung:Auswirkungen}
Die erfolglose Schlichtung ist Anrufungsvoraussetzung.
Da eine Schlichtung auch Zeit in Anspruch nehmen kann, wird die Anrufungsfrist (vgl. \ref{Anrufung:Kriterien:Frist}) in der Zeit, in der eine Schlichtung versucht wird, gehemmt, \S~8 Abs.~4 Satz~4 SGO\index[paridx]{SGO!8@\S~8!4@Abs.~4}.

Das eröffnende Schiedsgericht muss also feststellen, ob ein erfolgloser Schlichtungsversuch unternommen wurde.
Eine Schlichtung ist das ernsthafte Bemühen, auf konstruktivem Wege eine gütliche Einigung in einem Streit zu erreichen.\footnote{Vgl. Kommentierung des \S~7 SGO von Benjamin Siggel, \href{https://wiki.piratenpartei.de/Bundesschiedsgericht/Schlichtung}{https://wiki.piratenpartei.de/Bundesschiedsgericht/Schlichtung}.}
Daher kann eine fehlgeschlagene Schlichtung bereits dann gegeben sein, wenn sich eine Partei der Schlichtung konsequent verweigert oder diese immer wieder verzögert und so einen effektiven Rechtsschutz der Gegenpartei verhindert.
Das Schlichtungsverfahren soll kein Hindernis für den Rechtsschutz sein sondern ein weniger invasives Verfahren, in dem die Details zur Beilegung des Streits den Streitparteien überlassen werden und diese nicht von Dritten, dem Schiedsgericht, aufgezwungen werden.

\subsubsection{Entbehrlickeit}
\label{Anrufung:Kriterien:Schlichtung:Entbehrlickeit}
Die Schlichtung ist nach \S~7 Abs.~3 Satz 1 SGO\index[paridx]{SGO!7@\S~7!3@Abs.~3} entbehrlich, wenn es sich um Disziplinarverfahren (Einspruch gegen eine Ordnungsmaßnahme, Parteiausschluss) handelt, es ein Rechsmittelverfahren ist oder ein Fall der Eilbedürftigkeit des Verfahrens, die Aussichtslosigkeit oder des Scheiterns der Schlichtung vorliegt.

Für Disziplinarverfahren ist anzunehmen, dass eine Schlichtung zwingend scheitert. Daher macht eine Pflicht zum Schlichtungsversuch hier ebensowenig Sinn wie in Rechtsmittelverfahren, die sich auf ein Schiedsgerichtsverfahren beziehen. 

Für Anrufungen im einstweiligen Rechtsschutz entfällt die Schlichtungserfordernis grundsätzlich nach  \S~7 Abs.~3 Satz 1 Var.~4 SGO\index[paridx]{SGO!7@\S~7!3@Abs.~3}.\footnote{\cites[3]{BSG20131210}{BSG3014HS}.}
Andere Fälle der Eilbedürftigkeit dürften kaum eintreten.

Die Aussichtslosigkeit oder das Scheitern der Schlichtung unterscheiden sich lediglich dadurch, ob bereits ein Schlichtungsversuch unternommen wurde, oder aber eben dieser gar nicht unternommen wurde, weil er schon gar keine Aussicht auf Erfolg hat. Dabei unterliegt die Wertung des Einzelfalls dem Ermessen der Richter. Aussichtlos dürfte eine Schlichtung immer dann sein, wenn eine Partei in Bezug auf einen konkreten Streit schon von vorn herein ankündigt, für eine Schlichtung nicht zur Verfügung zu stehen, oder wenn es um reine Rechtsfragen wie im Falle des \S~14 Abs.~1 Satz~1 Var.~2 PartG\index[paridx]{PartG!14@\S~14!1@Abs.~1} geht.

Die Entscheidung eines Schiedsgerichts, ob ein solcher Fall vorliegt, ist nach \S~7 Abs.~3 Satz 2 SGO\index[paridx]{SGO!7@\S~7!3@Abs.~3} nicht anfechtbar.
Grundsätzlich sind hier zwei Auslegungen denkbar.

Erstens eine weite Auslegung:
Jede Entscheidung über Satz~1 ist unanfechtbar.
Eine Nichteröffnungsbeschwerde könnte sich dann nicht gegen die von einem erstinstanzlichen Schiedsgericht festgestellte Erforderlichkeit der Schlichtung in diesem Verfahren richten.

Und zweitens eine enge Auslegung:
Nur die positive festgestellte Entbehrlichkeit der Schlichtung ist unanfechtbar.
Eine Berufung gegen das Urteil im eröffneten Verfahren könnte sich dann nicht mehr darauf berufen, dass das Verfahren wegen eines erforderlichen, aber nicht erfolgten Schlichtungsversuches gar nicht hätte eröffnet werden dürfen.
Das Bundesschiedsgericht hat in seiner Rechtsprechung bisher die enge Auslegung angewendet,\footnote{Vgl. etwa \cites{BSG914H1}{BSG3014HS}.} ohne dies näher zu begründen.

Das ist auch korrekt:
Die Schlichtungserfordernis soll lediglich in den Fällen, in denen eine Schlichtung besser geeignet ist, den Rechtsfrieden wiederherzustellen, dieser den Vorrang geben und nicht jeglichen Rechtsschutz verhindern.
Daher ist die Hürde nicht zu hoch zu setzen, wenn das Eingangsgericht fälschlich das Vorliegen dieses Verfahrenserfordernisses verneint, ist dagegen Rechtsschutz zu gewähren.
Der Sinn und Zweck des Vorrangs des Schlichtungserfordernisses ist nicht gefährdet, wenn das Rechtsmittelgericht dessen Entbehrlichkeit feststellt und das Verfahren eröffnet.
Das Gebot des effektiven Rechtsschutzes stützt also diese enge Auslegung. Der Sinn und Zweck der Nichtanfechtbarkeit besteht also lediglich in dem Vertrauensschutz in das Bestehen des Verfahrenserfordernisses nach Eröffnung, damit nicht nachträglich ein Verfahrenserfordernis entzogen werden kann.

\section{Bescheidung der Eröffnung}
\label{Anrufung:Beschluss}
Nach der Prüfung aller dargelegten Kriterien ist über die Eröffnung zu bescheiden.
Dabei ist relevant, sich für jedes Kritrium und seine Bestimmungen einzeln anzusehen, um festzustellen, welche Anforderungen für die Eröffnung vorliegen müssen und welche Kriterien erst für das Endurteil vorliegen müssen.
Grundsätzlich gilt dabei:
Die Fakten müssen vollständig zutreffen, aber die Informationen über diese Fakten können in einigen Fällen noch während des Verfahrens nachgeliefert werden.
Und zwar ist ein solcher Nachtrag von Fakten immer möglich, wenn es Teil der Verhandlung ist, ebendiese Informationen zu ermitteln.
Das ist damit auch das Unterscheidungskriterium zwischen unechten Zulässigkeitskriterien, die bereits mit Eröffnung zu entscheiden sind, und echten Zulässigkeitskriterien, die erst im Prozessurteil beschieden werden müssen.\footnote{Vergleiche zu echten Zulässigkeitskriterien die Ausführungen im Abschnitt Zulässigkeit, S.~\pageref{Zulaessigkeit}.}
Das Musterbeispiel für einen solchen Nachtrag dürften Informationen über Sachverhalt und Begründung der Klage sein, deren Ermittlung ja gerade Haupttätigkeit im Verfahren ist.
Mit anderen Worten:
Alle Entscheidungen, die zwangsläufig auf Ermittlung und Wertungen des Gerichts und dem Vortrag der Streitparteien beruhen, sind im Zweifel echte Zulässigkeitsvorraussetzungen und erst mit dem Endurteil zu bescheiden.

\subsection{Nachbesserung}
\label{Anrufung:Beschluss:Nachbesserung}
Ist mindestens ein Kriterium nicht erfüllt, weil eine odere mehrere notwendige Angaben ganz oder teilweise fehlen, sollte der anrufenden Partei zunächst die Möglichkeit der Nachbesserung gegeben werden.
Die anrufende Partei sollte dabei möglichst genau auf die fehlenden Angaben hingewiesen werden.
Allerdings ist besonders auf die Frist zu achten.
Eine im Zeitpunkt der Verfristung unvollständige Anrufung ist unheilbar verfristet.
Daher ist eine Anrufung, die fristgerecht, aber unvollständig einging, und deren Anrufungsfrist vor einer Antwort durch das Gericht abgelaufen ist, nicht mit einer Nachbesserungsaufforderung zu beantworten, sondern mit einem Nichteröffnungsbeschluss.
Eine Ausnahme besteht dann, wenn sich die Frist aus dem Vorgetragenen nicht berechnen lässt.
Das ist typischerweise der Fall, wenn eine schlichtungspflichtige Anrufung vorliegt, aber Angaben zur Schlichtung und ihrem Umfang v.a. in zeitlicher Hinsicht fehlen.
In diesem Fall kann wegen der Hemmungswirkung der Schlichtung die Frist nicht berechnet werden, daher bietet sich hier zunächst eine Nachbesserungsaufforderung an.

Die Nachbesserung ist in der Schiedsgerichtsordnung nicht erwähnt.
Dennoch sollten Schiedsgerichte der anrufenden Partei die Möglichkeit dazu geben, sofern die Klage noch nicht verfristet ist.
Ob es gar eine Pflicht für die Schiedsgerichte gibt, Gelegenheit zur Nachbesserung zu geben, ist unklar.
Das Landesschiedsgericht Bayern nimmmt dies jedenfalls dann an, wenn innerhalb der regulären Anrufungsfrist noch genügend Zeit verblieben ist,\footnote{So in \cite[S.~2~f.]{LSGBYB413U},\nomenclature{LSG~BY}{Landesschiedsgericht Bayern (Bayerisches Landesschiedsgericht)} in der Bestätigung der Entscheidung durch das Bundesschiedsgericht mit \cite{BSG314HA} weiter thematisiert.} das Landesschiedsgericht Brandenburg nimmt dies wohl selbst dann an, wenn nur noch wenig Restfrist verblieben ist und setzt dann eine die Anrufungsfrist verlängernde Nachfrist.\footnote{So wohl jedenfalls in \cites[S.~7]{LSGBB133}{LSGBB134}.}
Das Bundesschiedsgericht hat zumindest die Erfordernis einer die Anrufungsfrist verlängernden Nachfristsetzung verneint,\footnote{\cites[S.~2]{BSG2315HS}[S.~2]{BSG20130227}.} und auch ansonsten die verpflichtende Erfordernis einer Nachbesserungsaufforderung  nie angenommen und zumindest dann verneint, wenn eine Fristberechnung dem Schiedsgericht nicht möglich ist und es durch die Nachbesserungsaufforderung, die eine die Anrufungsfrist ersetzende neue Frist setzt, einer ansonsten verfristeten Anrufung zum Erfolg verhelfen könnte.\footnote{\cites{BSG3915HS}[S.~2]{BSG215HS}}

\subsection{Nichteröffnungsbeschluss}
\label{Anrufung:Beschluss:Nichteroeffnung}
Wenn eine Partei nicht erfolgreich nachbessert bleibt keine andere Option als die Anrufung abzuweisen.
Die Abweisung muss begründet erfolgen, \S~8 Abs.~8 Satz~2 SGO\index[paridx]{SGO!8@\S~8!6@Abs.~6}

Es sollten also genau erläutert werden, warum das Verfahren nicht eröffnet wird.
Gab es mehrere Gründe, sollten auch alle alternativen Gründe angeführt werden.
Einerseits hilft das dem Rechtsschutzschuchenden, beim nächsten Mal etwas nicht zu übersehen, und zudem hilft das für den Fall, dass eine Nichteröffnungsbeschwerde beim Rechtsmittelgericht eingelegt wird, allen Beteiligten, das Verfahren schnell zu bearbeiten.

\subsection{Eröffnungsbeschluss}
\label{Anrufung:Beschluss:Eroeffnung}
Sind alle Kriterien erfüllt und die Anrufung daher korrekt im Sinne des \S~8 Abs.~5 SGO\index[paridx]{SGO!8@\S~8!5@Abs.~5}, ist das Verfahren zu eröffnen.

Im Eröffnungsbeschluss sind nach \S~9 SGO\index[paridx]{SGO!9@\S~9} einige Mitteilungen an die Parteien zu machen.
Den Parteien soll mitgeteilt werden:
\begin{enumerate}
\item Das Datum des Eröffnungsbeschlusses.
\item Das Aktenzeichen.
\item Die komplette Besetzung des Spruchkörpers.
\item Bei Änderungen gegenüber der Standardbesetzung die Gründe dafür.
\item Eine Kopie des Anrufungsschreibens.
\item Die Information, das ein Verfahrensvertreter bestellt werden kann, bzw. im Falle, dass eine Verfahrenspartei ein Organ ist, dass sie dies tun muss.
\item Eine Aufforderung zur Stellungnahme an beide Verfahrensparteien zum Verfahren.
\item Die Ladung zur (fern-)mündlichen Verhandlung, sofern sie schon steht mit dem Hinweis, dass auch in Abwesenheit Verhandelt werden kann, \S~10 Abs.~5 Satz~4 SGO\index[paridx]{SGO!10@\S~10!5@Abs.~5}.
\item Der Hinweis, dass die Parteien das Recht haben Richter abzulehen mit Hinweis auf die Präklusion nach \S~5 Abs.~2 Satz~4 SGO\index[paridx]{SGO!5@\S~5!2@Abs.~2}.
\item Gründe, die Richter nach \S~5 Abs.~2 Satz~3 SGO\index[paridx]{SGO!5@\S~5!2@Abs.~2} anzeigen müssen.
\item Nur bei Verfahren über eine Ordnungsmaßnahme oder einen Parteiausschluss: Frage an das Mitglied, ob es ein nichtöffentliches Verfahren wünscht.
\end{enumerate}

\subsection{Folgenden eines Eröffnungsbeschlusses}
\label{Anrufung:Beschluss:EroeffnungFolgen}
Die Schiedsgerichtsordnung startet das Verfahren mit dem Eröffnungsbeschluss.
Das heißt, ab diesem Moment entsteht ein Prozessrechtsverhältnis zwischen dem eröffnenden Schiedsgericht und den Verfahrensparteien nach den Verfahrensregeln der Schiedsgerichtsordnung.
Dieses Verfahren kann grundsätzlich nur durch ein Urteil abgeschlossen werden, \S~12 Abs.~1 SGO\index[paridx]{SGO!12@\S~12!1@Abs.~1}.\footnote{So explizit auch schon \cite[S.~2]{BSG20131204}.}
Das heißt auch, dass eine andere Beendigung dieses Prozessrechtsverhältnisses nicht zulässig ist.
Ausnahmen stellen hier selbständige Verfahren im einstweiligen Rechtsschutz vor,\footnote{Ausführlich zu Eigenständigkeit der Verfahren im einstweiligen Rechtsschutz \cite[S.~3]{LSGBB145} mit Verweis auf \cites[S.~4]{LSGHE20140423II} und die in \cites{BSG41114ES}{BSG3314EA} offensichtlich zum Ausdruck kommende Praxis des Bundesschiedsgerichtes, bestätigt von \cite[S.~2~f.]{BSG4214ESWiderspruch}.} die mit der einstweiligen Anordnung begründet werden, da es dort keine Eröffnung gibt, und mit dem Widerspruchsurteil oder der Verfristung seiner Beantragung beendet werden.

Allerdings gab es in der Rechtsprechung bisher schon Abweichungen von diesem Schema.
Das Bundesschiedsgericht ist in jüngster Zeit von Rücknehmbarkeit eines Eröffnungsbeschlusses ausgegangen, wenn seine Anforderungen nicht mehr erfüllt sind oder es noch nie waren und wendet dahingehend wohl \S~10 Abs.~1 SGO analog an.\footnote{So etwa \cite{BSGPP100127862}.}
Gerade im Prozessrecht mit seinen feinen Regelungen des Verfahrensablaufs ist es äußerst problematisch und begründungsbedürftig, wenn aus dem Vorhandensein der Regelung eine Kompetenz zu einem exakt gegenlaufenden Beschluss abgeleitet wird.
Zudem hat das Bundesschiedsgericht hier die tatsächliche Antragsbefugnis fehlerhaft für ein Statthaftigkeitskriterium gehalten und sie nicht, wie es korrekt wäre, als echtes Zulässigkeitskriterium behandelt (vgl. \ref{Anrufung:Kriterien:Antragsbefugnis}).

In älterer Rechtsprechung war das Bundesschiedgericht daher der Überzeugung, dass eine solche Rücknahme nicht möglich ist, es sei denn, dass der Eröffnungsbeschluss schon explizit unter einem Vorbehalt bestehender aus einer mit einer Frist versehenen Nachbesserungsaufforderung getroffen wurde und damit selbst bedingt war.\footnote{Vgl. \cite[S.~2]{BSG20130715} zu \cite{LSGNRW2013011},\nomenclature{LSG~NRW}{Landesschiedsgericht Nordrhein-Westfalen} in welchem das Bundesschiedsgericht einen solchen Vorbehalt erfüllt sah.}

Unter Anbetracht des Wortlauts der Schiedsgerichtsordnung ist die neuere Ansicht des Bundesschiedsgerichtes nicht haltbar.
Es bedarf schon gar keiner Analogie, da es schlicht an der notwendigen planwidrigen Regelungslücke fehlt, die aber Vorraussetzung jeder Analogie ist.
Es gibt auch bei einem Eröffnungsbeschluss, der trotz Nichterfüllung oder bei Wegfall der Anrufungskriterien oder in Unzuständigkeit getroffen wurde, die Möglichkeit, das Verfahren und damit das Prozessrechtsverhältnis wegen Unzulässigkeit mittels Urteil zu beenden.

\section{Sonderfall Verfahrensverweisung}
\label{Anrufung:Verweisung}
Einen Sonderfall der Anrufung stellt die Verfahrensverweisung nach \S~6 Abs.~5 SGO\index[paridx]{SGO!6@\S~6!5@Abs.~5} dar. Hier kommt es auf den Zustand des Verfahrens vor Verweisung an.

Ist das Verfahren vor Verweisung noch nicht eröffnet gewesen, muss das Zielgericht über die Eröffnung entscheiden, als wäre es selbst angerufen.
Dabei muss so vorgegangen werden, als wären die Anrufung statt an das Ursprungsgericht direkt an das Zielgericht gegangen.
Dies ist insbesondere für die Berechnung der Anrufungsfristen relevant.
Allerdings ist eine Verzögerung durch das Ursprungsgericht nicht der anrufenden Streitpartei zulasten zu legen.

Ist das Verfahren bereits vor Verweisung eröffnet gewesen, ist das Zielgericht an diese Entscheidung gebunden und muss das Verfahren entsprechend fortsetzen.
Es sollte trotzdem analog zum Eröffnungsbeschluss ein Übernahmebeschluss getroffen werden und die Verfahrensübernahme den Streitparteien mitgeteilt werden.
In dieser Übernahmemitteilung sollten den Parteien alle Mitteilungen gemacht werden, die sonst im Eröffnungsbeschluss getätigt werden (vgl. \ref{Anrufung:Beschluss:Eroeffnung}), um ein ordentliches Verfahren zu gewähren
Die Vorschriften über die Eröffnung sind dabei mangels eigener Regelung für die Verweisung analog anzuwenden.

Eine besondere Aufmerksamkeit gilt fehlerhaften Verweisungsbeschlüssen:
Diese sind gültig und mangels Rechtsmittelfähigkeit\footnote{\S~13 Abs.~6 Satz~1 SGO\index[paridx]{SGO!13@\S~13!6@Abs.~6} erlaubt nur in den Fällen, in denen es explizit durch die SGO vorgesehen ist, eine Beschwerde. Vgl. hierzu \cites{BSG201305151}{BSG201306071}.}  auch nicht im Instanzenzug \textbf{isoliert angreifbar}.
Ein fehlerhafter Verweisungsbeschluss ist daher bestandskräftig und somit rechtswirksam und begründet die Zuständigkeit des Zielgerichtes.\footnote{Vgl. dazu etwa \cite[5]{LSGBB147}.}
Allerdings dürfte diese Rechtsfrage auch nie zur Entscheidung vor der ordentlichen Gerichtsbarkeit landen, da das Bundesschiedsgericht immer letzinstanzlich zu entscheiden hat.
Da das Bundesschiedsgericht aber eine vollständige Tatsacheninstanz ist und eine eigene Entscheidung trifft, ist ein derartiger Fehler in der ersten Instanz für die ordentliche Gerichtsbarkeit nicht mehr relevant, da diese nur die ausschlaggebende Entscheidung des Bundesschiedsgerichtes zu berücksichtigen hat, das als solches auch die Instanzengarantie im Sinne des \S~10 Abs.~5 Satz~2 SGO\index[paridx]{SGO!10@\S~10!5@Abs.~5} erfüllt.
Die vor der ordentlichen Gerichtsbarkeit gerügte Beschwer bezüglich der Fehlerhaftigkeit des gesetzlichen Richters kann sich daher nur auf die Besetzung des Bundesschiedsgerichtes beziehen und nicht auf einen wegen falscher Verweisung fehlerhaften gesetzlichen Richter in der ersten Instanz.


\chapterbib
% \end{refsection}


% \begin{refsection}
\chapterpreamble{Hier sollten ggf. erstmal die Dreizeiler rein so wie dieser Text.\newline\blindtext}

\chapter{Zusammensetzung des Schiedsgerichts}
\blindtext[1]
\section{Ablehnung von Richtern}
\blindtext[5]
\section{Richterurlaub und -abwesenheit}
\blindtext[5]

\chapterbib
% \end{refsection}


% \begin{refsection}
\chapterpreamble{Hier sollten ggf. erstmal die Dreizeiler rein so wie dieser Text.\newline\blindtext}

\chapter{Verfahrensablauf und Verfahrensführung}
\blindtext[1]
\section{Grundsätze zu unterschiedlichen Klage- bzw Verfahrensarten}
\blindtext[5]
\section{Details zu unterschiedlichen Klage- bzw Verfahrensarten}
\blindtext[5]
\section{Vertretung im Verfahren}
\blindtext[5]
\section{Analoge Anwendung anderer Verfahrensordnungen}
\blindtext[5]

\chapterbib
% \end{refsection}


% \begin{refsection}
\chapterpreamble{Der einstweilige Rechtsschutz hat den Zweck, Rechte des Antragstellers vorläufig zu sichern. Zentral sind hier das sog. „Eilbedürfnis“ und das „Sicherungsinteresse“, §~11~Abs.~2~SGO. Obwohl einstweilige Anordnungen regelmäßig dazu dienen dürften, den status quo bis zur Entscheidung in der Hauptsache zu sichern, können sie auch ohne eine Klage in der Hauptsache beantragt werden. Eine einstweilige Anordnung darf die Hauptsache nicht vorwegnehmen, d.h. es darf nicht über den Umweg einer einstweiligen Anordnunge zu einem „kurzen Prozess“ kommen, in dem die schon Hauptsache am reduzierten Beweismaß der Glaubhaftmachuung entschieden wird.}

\chapter{Der einstweilige Rechtsschutz}
%\blindtext[1]

%\chapterbib
% \end{refsection}


% \begin{refsection}
\chapterpreamble{Zulässig ist eine Klage nur dann, wenn die Anrufung erfolgreich war, eine vorhergehende Schlichtung erfolgreich oder entbehrlich war. Form und Frist der Anrufung sind bereits vor Eröffnung zu prüfen, sind im Urteil jedoch noch einmal darzulegen. Die Klage ist gegn die Partei als Rechtsträger zu richten, solange nicht das jeweilige Organ von der Schiedsgerichtsordnung explizit als Klagegegner vorgesehen ist. Klagen gegen Einzelmitglieder, losgelöst von ihrer Funktion, sieht die SGO nicht vor.}

\chapter{Zulässigkeit der Klagebehren}
%\blindtext[1]
%\section{Klagegegner (Richtige Auswahl, Vielzahl von Gegnern etc.)}
%\blindtext[5]
%\section{Grundlagen Beweiswürdigung}
%\blindtext[5]

%\chapterbib
% \end{refsection}


% \begin{refsection}
\chapterpreamble{Juristische Auslegungsmethoden sind (in dieser, absteigender Reihenfolge) die Auslegung nach dem Wortlaut (Wortlautauslegung), nach der Systematik (systematische Auslegung), der Entstehungsgeschichte der fraglichen Reglung (historische Auslegung) und dem Sinn der fraglichen Regelung (teleologische Auslegung). Die historische Auslegung, d.h. die Entstehungsgeschichte einer Rechtsvorschrift kann für Vorschriften aus Parteisatzungen nur im Ausnahmefall herangezogen werden. Grundsätzlich gilt für jede Auslegung: Die Grenze ist der Wortlaut! Behandeln zwei anwendbare Bestimmungen den selben Regelungsgegenstand und setzen sie unterschiedliche Rechtsfolgen (Kollision), so gelten die folgenden „Vorfahrtsregeln“: Die neuere Regel überschreibt die ältere Regel, die speziellere Regel drängt sich vor die allgemeinere Regel und die höhere Regel steht über der niedrigeren Regel.}

\chapter{Grundlagen der Normenauslegung}
%\blindtext[1]
%\section{Höherangiges Recht und Fortgeschrittene Normenauslegung}
%\blindtext[5]

%\chapterbib
% \end{refsection}


% \begin{refsection}
\chapterpreamble{Grundsätzlich kann eine Klage darauf gerichtet sein, etwas außer Kraft zu setzen (Anfechtungsklage), jemanden zu verpflichten (Verpflichtungsklage), oder das Bestehen oder Nichtbestehen eines Rechtsverhältnisses festzustellen (Feststellungsklage). Die Feststellungsklage tritt hinter eine zulässige Anfechtung oder Verpflichtung immer zurück, sie ist nur subsidiär zulässig. Ist eine Sache erledigt, d.h. hat sich der reale Sachverhalt so verändert, dass die Klage keine Änderung in der Sache mehr herbeiführen kann, ist das Verfahren in der Hauptsache erledigt. Nur in Fällen, in denen ein besonderes Fortsetzungsinteresse besteht (etwa zur Rehabilitierung) kann das Verfahren als Fortsetzungsfeststellungsklage weitergeführt werden.}

\chapter{Verfahrensarten}
%\section{Anfechtung von Mitgliederversammlungen}
%\blindtext[1]
%\section{Gliederungsstreitigkeiten}
%\blindtext[1]
%\section{Grundlagen Ordnungsmaßnahmen}
%\blindtext[1]
%\subsection{Gliederungsordnungsmaßnahmen}
%\blindtext[5]
%\subsection{Individualordnungsmaßnahmen}
%\blindtext[5]
%\section{Normenkontrolle}
%\blindtext[1]

%\chapterbib
% \end{refsection}

% \begin{refsection}
\chapterpreamble{Das Urteil ist das Ergebnis des Prozesses. Es sollte follguht™ sein und Rechtsfrieden schaffen.}

\chapter{Urteilsaufbau}
%\blindtext[1]
%\section{Grundlagen}
%\blindtext[1]
%\section{Das Rubrum}
%\blindtext[1]
%\section{Tenorierung}
%\blindtext[5]
%\section{Sachverhalt}
%\blindtext[5]
%\section{Entscheidungsgründe}
%\blindtext[5]
%\section{Die Rechtsmittelbelehrung}
%\blindtext[1]

%\chapterbib
% \end{refsection}


% \begin{refsection}
\chapterpreamble{Dokumentiert (vgl. \S~14~SGO\index[paridx]{SGO!14@\S~14}) werden Verfahrensakten und Urteile.
Die Verfahrensakte umfasst sämtliche verfahrensrelevante Kommunikation des Gerichts mit den Parteien (und umgekehrt).
Das umfasst auch Aktennotizen, die sich auf den Verfahrensverlauf beziehen (z.B. bei telefonisch gestellten Anträgen o.ä.).
Interne Kommunikation des Gerichts, auch wertende Notizen einzelner Mitglieder des Gremiums, gehören nicht zur Verfahrensakte.
Werden während Verfahren Tonaufzeichnungen angefertigt, sind diese zu löschen, sobald die Parteien das daraus angefertigte Protokoll erhalten und einen Monat keinen Widerspruch erhoben haben, \S~14 Abs.~3~SGO\index[paridx]{SGO!14@\S~14!3@Abs.~3}.
Die Verfahrensakte ist fünf Jahre nach Abschluss des Verfahrens aufzubewahren, Urteile unbefristet, \S~14~Abs.~5~SGO\index[paridx]{SGO!14@\S~14!5@Abs.~5}.}

% ToDo: Aktenführung (formal): Was gehört in die Akte, wie baut man sie auf, Formate etc.

\chapter{Dokumentation und Rechenschaftslegung}
\label{Dokumentation}
Die Dokumentationspflichten der Schiedsgerichte sind in \S~14~SGO\index[paridx]{SGO!14@\S~14} geregelt.
Diese Dokumentation ist für den Gebrauch im Verfahren bestimmt; bezieht sich also auf die Verfahrensakte.
Dies dient einerseits dazu, das Verfahren zu erleichtern (vgl.\nomenclature{vgl.}{vergleiche} auch \S~10 Abs.~1 S.~3~SGO),\index[paridx]{SGO!10@\S~10!1@Abs.~1} andererseits auch zur Verwendung durch eine Rechtsmittelinstanz.
Darüber hinaus dient die Dokumentation auch dem längerfristigen Nachweis, \S~14 Abs.~5~SGO\index[paridx]{SGO!14@\S~14!5@Abs.~5}.

In \S~15~SGO\index[paridx]{SGO!15@\S~15} schließlich ist die Rechenschaftslegung nach außen geregelt.
Die Veröffentlichungsrechte und -pflichten sind die Begrenzung der ansonsten für die Mitglieder von Schiedsgerichten geltenden Verschwiegenheitspflicht (\S~2 Abs.~4~SGO).\index[paridx]{SGO!2@\S~2!4@Abs.~4}\index[idx]{Verschwiegenheitspflicht}
Da Dokumentation für den internen Gebrauch und die Rechenschaftslegung zur Veröffentlichung eng miteinander verflochten sind, werden sie hier gemeinsam behandelt.

\section{Verfahrensakte}
\label{Dokumentation:Akte}
\index[idx]{Akte!Inhalt}
\index[idx]{Verfahrensakte|see{Akte}}
Der Umfang der Verfahrensakte wird von \S~14 Abs.~2~SGO\index[paridx]{SGO!14@\S~14!2@Abs.~2} definiert (\Zitat{Verlaufsprotokolle von Anhörungen und Verhandlungen, alle für das Verfahren relevanten Schriftstücke und das Urteil}).
Da es sich hierbei ausschließlich um Texte handelt -- von Tonaufzeichnungen sind Protokolle anzufertigen, \S~14 Abs.~3 S.~2~SGO\index[paridx]{SGO!14@\S~14!3@Abs.~3} -- empfiehlt es sich, die verschiedenen Quellen in eine einzelne Datei (bspw.\nomenclature{bspw.}{beispielsweise} im PDF\nomenclature{PDF}{Portable Document Format}) zusammenzufassen.

Als \enquote{Schriftstücke} i.S.d.\nomenclature{i.S.d.}{im Sinne des} \S~14 Abs.~2~SGO sind auch sämtliche Aktennotizen (z.B.\nomenclature{z.B.}{zum Beispiel} zu (fern-) mündlich gestellten Anträgen etc.\nomenclature{etc.}{et cetera} zu verstehen.
Nicht zur Akte gehören hingegen die Beratungen der Richter:
Da das Abstimmverhalten nicht mitzuteilen ist, \S~12 Abs.~3 S.~4~SGO\index[paridx]{SGO!12@\S~12!3@Abs.~3} und die Richter allgemein Verschwiegenheit zu wahren haben, \S~2 Abs.~4~SGO,\index[paridx]{SGO!2@\S~2!4@Abs.~4} sind auch Beratungen der Richter untereinander nicht Teil der Akte.
Stattdessen umfasst der Begriff der \enquote{relevanten Schriftstücke} (u.a.)\nomenclature{u.a.}{unter anderem} jedweden (Schrift-) Verkehr zwischen Gericht und den Parteien, sowie verfahrensleitende Beschlüsse, dienstliche Stellungnahmen der Richter (bspw. zur Besorgnis der Befangenheit) und die Notizen über Spruchkörperveränderungen.

Die Akte dient dazu, ein Verfahren lückenlos nachvollziehbar zu halten, ohne dabei auf andere Quellen (z.B. Beteiligte und Zeugen) angewiesen zu sein.
Ebenso wichtig wie die Vollständigkeit der Akte (s.o.) ist daher auch ein chronologischer Ablauf:
Fristen und Termine sind auch innerhalb eines Verfahrens von Bedeutung und können für die Bewertung von einzelnem Vorbringen wichtig sein.
Die Wiedergabe der Vorkommnisse im Verfahren muss daher so erfolgen, dass der zeitliche Ablauf auf Anhieb erkennbar ist.
Es reicht nicht aus, dass der zeitliche Zusammenhang aus dem Inhalt der Akte ersichtlich wird (z.B. wenn in einem Schriftsatz ein Datum genannt wird oder sich auf einen anderen Schriftsatz, der mit Datum bekannt ist, bezogen wird).
Stattdessen muss bereits durch die Abfolge der Dokumente in der Akte ersichtlich werden, in welcher Reihenfolge sie bei Gericht eingingen.
Soweit die Akte aus einzelnen Dateien besteht, empfiehlt sich der Übersicht halber daher die durchgängige Benennung nach eindeutigen Zeitstempeln\footnote{Bspw. im Format \emph{JJJJMMDD}.} als Präfix.

Die Verfahrensakte ist nach Abschluss des Verfahrens 5~Jahre aufzubewahren, \S~14 Abs.~5 S.~1~SGO.\index[paridx]{SGO!14@\S~14!5@Abs.~5}
Das ist auch elektronisch möglich, bspw. im ohnehin verwendeten Ticket-System oder als konsolidiertes PDF.
Die Akte ist gegen unbefugten Zugriff zu sichern; neben dem jeweils amtierenden Gericht haben nur die jeweiligen Verfahrensparteien das Recht auf Akteneinsicht, \S~14 Abs.~4~SGO.\index[paridx]{SGO!14@\S~14!4@Abs.~4}
Nach Ablauf der Aufbewahrungsfrist ist die Akte (mit Ausnahme des Urteils, \S~14 Abs.~5 S.~2~SGO),\index[paridx]{SGO!14@\S~14!5@Abs.~5} zu vernichten bzw. zu löschen.

\section{Protokolle}
\label{Dokumentation:Protokolle}
In Bezug auf Protokolle fallen die Dokumentations- und Berichtspflichten der Schiedsgerichte unterschiedlich aus:
Es kommt darauf an, ob es sich um Protokolle von (fern-) mündlichen Verhandlungen handelt (Verhandlungsprotokolle), oder um solche von den übrigen Sitzungen des Gerichts (Sitzungsprotokolle).

\subsection{Verhandlungsprotokolle}
\label{Dokumentation:Protokolle:Verhandlungsprotokolle}
Die Protokolle von Verhandlungen sind Teil der Verfahrensakte, \S~14 Abs.~2~SGO.\index[paridx]{SGO!14@\S~14!2@Abs.~2}
Die SGO sieht dabei explizit \emph{Verlaufsprotokolle} vor:
Solche sind umfangreicher als bloße \emph{Ergebnisprotokolle}, da sie auch den Inhalt des jeweiligen Vorbringens, d.h. ggf.\nomenclature{ggf.}{gegebenenfalls} auch der rechtlichen Diskussion widergeben.
Sie sind allerdings weniger umfangreich als \emph{Wortprotokolle} und geben im Gegensatz zu diesen das Vorbringen der Beteiligten nur zusammenfassend und in indirekter Rede wider, anstatt den genauen Wortlaut aufzuführen.

Auch die Verschriftlichung von Tonaufzeichnungen muss nur als Verlaufsprotokoll erfolgen.
Soweit das Gericht wörtliche Widergabe der Aufzeichnung im Protokoll wünscht, ist es zulässig, eine solche anzufertigen und damit über die Anforderungen der Satzung hinauszugehen.
Ein Anspruch der Parteien darauf besteht indes nicht.

\subsection{Sitzungsprotokolle}
\label{Dokumentation:Protokolle:Sitzungsprotokolle}
Die Protokollierung von Sitzungen des Schiedsgerichts, in denen nicht mit den Beteiligten verhandelt, sondern lediglich das Verfahren -- oder Administratives -- beraten wird, ist in der SGO ungeregelt.
Da keine Pflicht besteht, kann eine Protokollierung nicht verbindlich verlangt werden.
Es besteht allerdings auch kein Verbot.
Genauer:
Die Verschwiegenheitspflicht verbietet eine Protokollierung nicht:
Es steht dem Gericht im Rahmen seiner inneren Organisationsfreiheit frei, diese Sitzungen nach eigenem Ermessen zu protokollieren.
Lediglich die Veröffentlichung ist reglementiert.

Hier bietet sich eine Veröffentlichung im Rahmen der regelmäßigen Berichtspflicht des Schiedsgerichts nach \S~15 Abs.~1~SGO\index[paridx]{SGO!15@\S~15!1@Abs.~1} und bzw.\nomenclature{bzw.}{beziehungsweise} oder im Arbeitsbericht an.
In beiden Fällen sieht die SGO jeweils nur Mindestanforderungen für die Veröffentlichung vor, über die die Schiedsgerichte in gewissem Umfang hinausgehen dürfen.

\section{Beschlüsse}
\label{Dokumentation:Beschlüsse}
Im Gegensatz zum \emph{Urteil} ist der Beschluss in der SGO nicht eigens geregelt.
Dabei ist bereits der Begriff selbst nicht eindeutig belegt:
Einerseits ist ein Beschluss die formelle Entschließung eines Organs durch seine Mitglieder; im Rahmen der (staatlichen) Gerichtsbarkeit werden als \emph{Beschluss} auch Entscheidungen bezeichnet, die -- im Gegensatz zu \emph{Urteilen} -- nicht nach mündlicher Verhandlung ergehen.
Dieser Nomenklatur folgend wäre die Mehrzahl der \enquote{Urteile} im Rahmen der Schiedsgerichtsbarkeit der Piratenpartei wohl als \enquote{Beschluss} zu bezeichnen.%Verweis auf Urteilskapitel; dort noch nachbessern!

Tatsächlich kennt auch die SGO Beschlüsse, die ein Verfahren abschließen, ebenso allerdings auch Beschlüsse, die das Verfahren lediglich vorantreiben und ggf. in eine bestimmte Richtung lenken (sog. \emph{verfahrensleitende} Beschlüsse).
Alle Beschlüsse müssen jedenfalls als Teil der jeweiligen Verfahrensakte dokumentiert werden.
Unterschiedlich ausfallen muss jedoch die Bewertung bezüglich der Veröffentlichung (und damit auch Aufbewahrung) je nach Kategorie, der der Beschluss angehört.

\subsection{Verfahrensleitende Beschlüsse}
\label{Dokumentation:Beschlüsse:Verfahrensleitend}
Verfahrensleitende Beschlüsse sind Entscheidungen des Gerichts, die sich auf den Verfahrensfortgang beziehen, ohne das Verfahren zu beenden.
Diese lassen sich wiederum untergliedern in rechtsmittelfähige und unanfechtbare Beschlüsse.

\subsubsection{Rechtsmittelfähige Beschlüsse}
\label{Dokumentation:Beschlüsse:Verfahrensleitend:Rechtsmittelfähig}
Gegen einen rechtsmittelfähigen Beschluss können die Beteiligten auch aus dem Verfahren heraus ein Rechtsmittel zur nächsten Instanz erheben (vgl. auch \S~13 Abs.~6~SGO).\index[paridx]{SGO!13@\S~13!6@Abs.~6}
Der einzige rechtsmittelfähige, aber gleichzeitig nicht auch das Verfahren (zmd. am jeweiligen  Gericht) beendende Beschluss ist der Beschluss, der die Ablehnung eines Richters durch eine Streitpartei für unbegründet erklärt, \S~5 Abs.~6 S.~2, 3~SGO.\index[paridx]{SGO!5@\S~5!6@Abs.~6}
Da das Verfahren am ursprünglichen Gericht weiterläuft und die Entscheidung im Urteil ohnehin im Rahmen der Prozessgeschichte zumindest zu erwähnen ist, kann eine Veröffentlichung hier unterbleiben.
Zudem besteht für die Beteiligten die Möglichkeit des Rechtsmittelgebrauchs, die ihrerseits eine zu veröffentlichende Entscheidung (dann des Obergerichts) nach sich zöge. 

\subsubsection{Unanfechtbare Beschlüsse}
\label{Dokumentation:Beschlüsse:Verfahrensleitend:Unanfechtbar}
\index[idx]{Beschluss!Unanfechtbarkeit}
Nicht anfechtbar sind bspw. die Eröffnung eines Verfahrens gemäß \SSS~8 Abs.~6 S.~1, 9 Abs.~1 S.~1~SGO,\index[paridx]{SGO!8@\S~8!6@Abs.~6}\index[paridx]{SGO!9@\S~9!1@Abs.~1} der Ausschluss eines Mitglieds des Spruchkörpers wegen der Besorgnis der Befangenheit (\S~5 Abs.~6 S.~1~SGO\index[paridx]{SGO!5@\S~5!6@Abs.~6}), sowie die Anordnungen der Schiedsgerichte zu Fristen, über Beweisanträge (vgl. \S~10 Abs.~1 S.~2~SGO)\index[paridx]{SGO!10@\S~10!2@Abs.~2} und alle sonstigen Beschlüsse, gegen die nicht explizit ein Rechtsmittel vorgesehen ist.\footnote{Vgl. \S~13 Abs.~6~SGO\index[paridx]{SGO!13@\S~13!6@Abs.~6}, ebenso \cite[5]{LSGBB147}.}

Unanfechtbare Beschlüsse dürften in der Regel der Verfahrensakte zuzuordnen sein.
Eine Veröffentlichung muss daher nicht zwingend erfolgen.
Allerdings ist ihr Inhalt zuweilen für das Urteil von Bedeutung.
Insbesondere betrifft das Entscheidungen zur Besetzung (vgl. auch \ref{Zusammensetzung:Spruchkoerper:Befangenheitsbesorgnis}).
Soweit ein Beschluss gesondert begründet wird, diese Begründung sich aber nicht unbedingt im Urteil wiederfinden soll (bspw. um den Urteilstext nicht unnötig aufzublähen oder zu verkomplizieren), kann es ratsam sein, auch unanfechtbare, verfahrensleitende Beschlüsse zu veröffentlichen.

Soweit eine Veröffentlichung erfolgt, wäre prinzipiell die fünfjährige Aufbewahrungsfrist nach \S~14 Abs.~5 S.~1~SGO\index[paridx]{SGO!14@\S~14!5@Abs.~5} einschlägig.
Allerdings ersetzt eine Veröffentlichung in diesen Fällen häufig die Aufnahme der entsprechenden -- für das Verfahren durchaus bedeutsame -- Tatsachen und Erwägungen.
Eine unbegrenzte Aufbewahrung entsprechend den Urteilen (\S~14 Abs.~5 S.~2~SGO\index[paridx]{SGO!14@\S~14!5@Abs.~5}) ist dann zulässig, da insbesondere rechtliche Erwägungen für die Parteiöffentlichkeit relevant sind.

% Sonderfall Verweisung an den Senat des BSG → oder ist das verfahrensbeendend? Mal BSG-Praxis angucken!

\subsection{Verfahrensabschließende Beschlüsse}
\label{Dokumentation:Beschlüsse:Verfahrensabschließend}
Aus der SGO ergeben sich allerdings eine erhebliche Anzahl an Beschlüssen, die ein Verfahren beenden (können).
Dies sind:
\begin{enumerate}
\item Die Übertragung eines Verfahrens an den Senat des Bundesschiedsgerichts, \S~3 Abs.~11 S.~7~SGO,\index[paridx]{SGO!3@\S~3!11@Abs.~11}
\item die Verweisung eines Verfahrens wegen Handlungsunfähigkeit des Gerichts, \S~6 Abs.~5~SGO,\index[paridx]{SGO!6@\S~6!5@Abs.~5}
\item die Übernahme eines Verfahrens durch das Obergericht wegen Verfahrensverzögerung, \S~10 Abs.~9 S.~5~SGO,\index[paridx]{SGO!10@\S~10!9@Abs.~9}
\item die Ablehnung der Verfahrenseröffnung (Nichteröffnung),\index[idx]{Nichteröffnung} \S~8 Abs.~6~SGO,\index[paridx]{SGO!8@\S~8!6@Abs.~6}
\item Erlass oder Ablehnung einer einstweiligen Anordnung (da hiermit das Verfahren im einstweiligen Rechtsschutz zunächst beendet wird), \S~11 Abs.~1, 7~SGO.\index[paridx]{SGO!11@\S~11!1@Abs.~1}\index[paridx]{SGO!11@\S~11!7@Abs.~7}
\end{enumerate}

Es ist naheliegend, diese Beschlüsse aufgrund ihrer verfahrensbeendenden Wirkung wie Urteile zu behandeln und die auf Urteile anzuwendenden Vorschriften ebenfalls (ggf. analog) anzuwenden.\footnote{In Bezug auf einstweilige Anordnungen ist dies bereits durch die SGO vorgeschrieben, \S~11 Abs.~7~SGO.\index[paridx]{SGO!11@\S~11!7@Abs.~7}}

\section{Urteile}
\label{Dokumentation:Urteile}
Der Aufbau der Urteile ist bereits ausführlich behandelt worden (vgl. Kapitel~\ref{Urteilsaufbau} ab S.~\pageref{Urteilsaufbau}).
Diese Originalfassung muss in Papierform vorliegen (\Zitat{schriftlich}, \Zitat{von allen Richtern unterschrieben}, \S~12 Abs.~7~SGO\index[paridx]{SGO!12@\S~12!7@Abs.~7}) und darf keine Schwärzungen enthalten.
Für sie gelten daher besondere Aufbewahrungsbestimmungen (vgl.~\ref{Dokumentation:Aufbewahrung}).

Anders verhält es sich mit der für die Veröffentlichung bestimmte (\S~12 Abs.~8~SGO)\index[paridx]{SGO!12@\S~12!8@Abs.~8} Fassung.
Diese Fassung ist einerseits vom Schiedsgericht selbst zu veröffentlichen, andererseits ist eine Kopie dem Bundesschiedsgericht zuzuleiten, \S~12 Abs.~9 S.~1~SGO.
Die zur Veröffentlichung bzw. zur Weiterleitung an das Bundesschiedsgericht bestimmte Fassung muss in unterschiedlichem Umfang geschwärzt werden.

\subsection{Pseudonymisierung}
\label{Dokumentation:Urteile:Pseudonymisierung}
\index[idx]{Pseudonymisierung}
Ist das Verfahren öffentlich, so wird das Urteil insgesamt veröffentlicht, \S~12 Abs.~8 S.~1~SGO.\index[paridx]{SGO!12@\S~12!8@Abs.~8}
Zum Schutz der Persönlichkeitsrechte aller Beteiligten soll allerdings der Rückschluss auf ihre Identität erschwert werden.
Daher sind ihre Namen zu pseudonymisieren.

Pseudonymisieren ist das Ersetzen des Namens und anderer Identifikationsmerkmale durch ein Kennzeichen zu dem Zweck, die Bestimmung des Betroffenen auszuschließen oder wesentlich zu erschweren (\S~3 Abs.~6a~BDSG).\index[paridx]{BDSG!3@\S~3!6a@Abs.~6a}\nomenclature{BDSG}{Bundesdatenschutzgesetz}
Zu beachten ist, dass laut Satzung eine Pflicht zur Pseudonymisierung lediglich für die Namen besteht.
Davon sind zwar auch sämtliche Arten Nicknames umfasst, aber eben -- im Gegensatz zur Definition aus dem BDSG -- keine sonstigen Merkmale der Person.
Die Bestimmung der Identität der Beteiligten soll also erschwert werden; verpflichtet, sie tatsächlich unmöglich zu machen, ist das Schiedsgericht nicht.
Im Interesse der Beteiligten können aber auch andere Merkmale, die zur einfachen Identifizierung einer Person dienen können, pseudonymisiert werden.
Insbesondere gilt dies für Anschrift und Kontaktdaten von Individuen, die zwar nicht von Satzung wegen, aber aus Gründen des Datenschutzes zu schwärzen sind.
Aus der Praxis der Schiedsgerichte hat sich ergeben, dass sie gleich dem Namen zu behandeln sind. 

Ausdrücklich ausgenommen von der Pseudonymisierungspflicht sind gemäß \S~12 Abs.~8 S.~2~SGO\index[paridx]{SGO!12@\S~12!8Abs.~8} lediglich die Namen von Gliederungen und die Namen der Richter in ihrer Funktion.
Es soll also aus den Urteilen auch für die Öffentlichkeit stets hervorgehen, welche Gliederung in welcher Weise beteiligt war.
Auch die Bezeichnung von Organen (z.B. Vorstand, Kreisparteitag, etc.) darf nicht geschwärzt werden, da es sich hierbei nicht um Personen, sondern eben um Organe handelt.
Soweit aber Personen für die Gliederungen (oder deren Organe) handeln, sind diese wiederum zu pseudonymisieren -- selbst wenn sie in Funktion (bspw. als Vorstand oder als Prozessvertretung) handeln.
Kein Recht auf Pseudonymisierung in einem Urteil haben lediglich Richter, soweit sie als Richter auftreten und handeln.
Sobald ein Richter lediglich als Mitglied der Partei oder sonst außerhalb seiner Funktion als Richter im Urteil benannt wird, wäre sein Name ebenfalls zu pseudonymisieren.

Die Vorschrift wurde in der Vergangenheit von den Schiedsgerichten sehr eng ausgelegt bzw. fast lax gehandhabt:
So wurden bspw. Anwälte, die als Prozessvertreter auftreten, namentlich und mit Kanzleianschrift benannt oder vom streitgegenständlichen Geschehen betroffene Bundesvorstandsmitglieder namentlich im Sachverhalt aufgeführt.\footnote{Die Urteile liegen den Verfassern vor; auf einen Nachweis wurde aus offensichtlichen Gründen bewusst verzichtet.}

Da die Personennamen nicht anonymisiert, sondern lediglich pseudonymisiert werden müssen, ist die Verwendung von personenbezogenen Schlüsseln zulässig.
Das bedeutet, dass innerhalb eines Urteils (aber nicht darüber hinaus!) die Personen jeweils wiedererkennbar sein dürfen.
Hier bietet sich an, die Personen alphabetisch fortlaufend mit Buchstaben zu benennen (bspw. Zeugen A, B und C), oder aber Kürzel ihrer Funktion nach (bspw. Landesvorstandsmitglied~L, Prozessvertreter~V, Zeuge~Z) einzuführen.
Eine bloße Abkürzung von Namen auf den ersten Buchstaben hingegen sollte unterbleiben, da hierbei die Identifizierung zumindest innerhalb der Piratenpartei zu einfach möglich wäre.

\subsection{Nichtöffentliche Verfahren}
\label{Dokumentation:Urteile:Nichtöffentlich}
\index[idx]{Verschlusssachen}
Ist das Verfahren nichtöffentlich, so wird lediglich der Tenor veröffentlicht, \S~12 Abs.~8 S.~3~SGO.\index[paridx]{SGO!12@\S~12!8@Abs.~8}
Die Bestimmung ist -- auch in Ansehung der Formulierung des \S~12 Abs.~3 S.~1~SGO\index[paridx]{SGO!12@\S~12!Abs.~3} -- dergestalt auszulegen, dass das Rubrum zum \enquote{Tenor} gehört (vgl.~\ref{Urteilsaufbau:Rubrum}~f.).\index[idx]{Tenor}

Effektiv wird das Urteil wie ein Urteil eines öffentlichen Verfahrens geschwärzt bzw. pseudonymisiert (s.o. \ref{Dokumentation:Urteile:Pseudonymisierung}), allerdings vor Schilderung des Sachverhalts und der Entscheidungsgründe \enquote{abgeschnitten}.

Diese Bestimmung dient dem Schutz der Persönlichkeitsrechte des \enquote{disziplinierten} Mitglieds.
Sie birgt allerdings das Problem, dass in der Praxis nur ein kleiner Teil der (Individual-) Ordnungsmaßnahmen veröffentlicht werden:
Der Beschluss zum Nichtöffentlichen Verfahren ist in diesen Fällen eine gebundene Entscheidung, \S~10 Abs.~7 S.~2~SGO.\index[paridx]{SGO!10@\S~10!7@Abs.~7}
Insbesondere, da auf diese Möglichkeit im Eröffnungsbeschluss hinzuweisen ist, \S~9 Abs.~4 S.~1~SGO,\index[paridx]{SGO!9@\S~9!4@Abs.~4} sind viele Ordnungsmaßnahmeverfahren nichtöffentlich.
Für die Parteiöffentlichkeit, insbesondere aber auch Vorstände niedrigerer Untergliederungen mit geringerem Aufkommen an Ordnungsmaßnahmen und auch ortsfremde Schiedsgerichte besteht daher das Problem, dass insbesondere bestätigte Ordnungsmaßnahmen und erfolgreiche Anträge auf Parteiausschlussverfahren inhaltlich nicht nachvollziehbar sind.
Dies erschwert die Etablierung von Beurteilungsmaßstäben, welches Verhalten disziplinarwürdig ist, und welches nicht.

Eine Möglichkeit, die in solchen Verfahren aufgeworfenen Rechtsfragen dennoch öffentlich zu diskutieren, besteht im Rahmen des Arbeitsberichts (vgl.~\ref{Dokumentation:Rechenschaftslegung:Arbeitsbericht:nichtöffentlicheVerfahren}).

\section{Öffentliche Mitteilungen}
\label{Dokumentation:Veröffentlichungen}
Abgesehen von den zu veröffentlichenden Urteilen und Protokollen treten die Schiedsgerichte eher selten in Kommunikation mit anderen Organen oder der Gesamtpartei.
Ausnahmen hiervon sind vor allem die Bekanntmachung von Beeinflussungsversuchen, sowie die Stellungnahmen zu laufenden Verfahren.

\subsection{Bekanntmachung von Beeinflussungsversuchen}
\label{Dokumentation:Veröffentlichungen:Beeinflussungen}
\index[idx]{Beeinflussungsversuch}
Gemäß \S~2 Abs.~5~SGO\index[paridx]{SGO!2@\S~2!5@Abs.~5} sind die Schiedsgerichte verpflichtet, Versuche der Beeinflussung eines Verfahrens öffentlich bekannt zu machen.
Was eine Beeinflussung ist, liegt letztlich im Ermessen des Schiedsgerichts.
Insbesondere der Versuch, dem Schiedsgericht Weisungen zu erteilen (im Innenverhältnis der Partei durch \S~2 Abs.~2~SGO,\index[paridx]{SGO!2@\S~2!2@Abs.~2} im Außenverhältnis durch \S~14 Abs.~2 S.~4~PartG\index[paridx]{PartG!14@\S~14!2@Abs.~2} verboten) stellt eine solche veröffentlichungspflichtige Tatsache dar.
Obwohl der Wortlaut nahelegt, dass sich die Vorschrift ausschließlich auf die Beeinflussung von bestimmten Verfahren bezieht, ist aufgrund der besonderen (auch gesetzlich geschützten) Bedeutung der Unabhängigkeit der Schiedsgerichte anzunehmen, dass sich die Norm auf jedwede Beeinflussung des Organs Schiedsgericht oder der Richter in ihrer Funktion bezieht.
Es ist dabei prinzipiell nicht maßgeblich, ob die Quelle der Beeinflussung inner- oder außerhalb der Partei liegt.

Die Öffentlichkeit, der der Beeinflussungsversuch bekannt gemacht werden soll, ist die Parteiöffentlichkeit.
Eine Verlautbarung über eine geeignete Mailingliste (bspw. die Aktiven-Liste des Landes, ggf. sogar des Bundesverbandes) ist daher zur Bekanntmachung prinzipiell ausreichend.

Eine explizite Aufbewahrungsfplichtfür die Bekanntmachung von Beeinflussungsversuchen ergibt sich aus der SGO nicht.
Um allerdings die Unabhängigkeit der Schiedsgerichte zu schützen und aus dem Prinzip der Transparenz heraus bietet es sich an, die Verlautbarung auch im Rahmen der allgemeinen Dokumentation des Schiedsgerichts abrufbar zu halten.

Bezüglich der Dokumentation und Aufbewahrung bietet sich daher eine Behandlung entsprechend den Urteilen an (wobei die Aufbewahrung einer unterschriebenen Fassung wohl unterbleiben kann).
Zum Schutze der Persönlichkeitsrechte der Beteiligten sollte eine angemessene Pseudonymisierung stets in Betracht gezogen werden.

\subsection{Stellungnahmen zu laufenden Verfahren}
\label{Dokumentation:Veröffentlichungen:Stellungnahmen}
\index[idx]{Stellungnahme}
Das Gericht kann zu laufenden Verfahren öffentliche Stellungnahmen abgeben, \S~15 Abs.~2 S.~1~SGO.\index[paridx]{SGO!15@\S~15!2@Abs.~2}
Voraussetzung dafür ist, dass das Verfahren öffentlich ist (\S~15 Abs.2 S.~2~SGO)\index[paridx]{SGO!15@\S~15!2@Abs.~2} und dass das Schiedsgericht ein erhebliches parteiöffentliches Interesse feststellt.

Die \emph{Parteiöffentlichkeit}\index[idx]{Parteiöffentlichkeit} beschreibt den öffentlichen Raum, den die Mitglieder der Partei gemeinsam bilden.
Für ein \enquote{parteiöffentliches Interesse} sind Umstände außerhalb der Piratenpartei also nicht von Belang.
Ein solches Interesse kann einerseits \enquote{von Seiten} der Parteiöffentlichkeit bestehen, oder aber \enquote{für} sie:
Im ersteren Fall wird das Verfahren bereits von einem (erheblichen) Teil der Parteiöffentlichkeit verfolgt.
Dies lässt sich bspw. an Diskussionen über das Verfahren ablesen, die an für die Parteiöffentlichkeit exponiert wahrnehmbarer Stelle stattfinden, oder auch an direkten Nachfragen an das Gericht.
Im zweiten Fall ist eine tatsächliche Kenntnisnahme durch die Parteiöffentlichkeit unerheblich.
Hier ist nicht maßgeblich, dass von Seiten eines erheblichen Teils der Parteiöffentlichkeit ein Interesse an Verfahrensdetails besteht, sondern, dass dieses Interesse bestehen sollte.

Das Gericht kann zu Stellungnahmen nicht verpflichtet werden.
Ebenso liegt der Umfang der Stellungnahme ausschließlich im Ermessen des Schiedsgerichts.
Entsprechend den Vorschriften zum Urteil sollte pseudonymisiert werden.
Das Gericht sollte besonderes Augenmerk darauf richten, die Veröffentlichung neutral zu halten, um keiner Partei einen Anlass zur Besorgnis der Befangenheit (vgl.~\ref{Zusammensetzung:Spruchkoerper:Befangenheitsbesorgnis}) zu geben.

Da Stellungnahmen nur zu laufenden Verfahren zulässig sind, erlischt das Recht prinzipiell mit Abschluss des Verfahrens.
In der entsprechenden Satzungsbestimmung aber ein Verbot von Korrekturen oder sachgerechten Ergänzungen bereits veröffentlichter Stellungnahmen zu sehen, widerspräche jedoch dem Sinn der Vorschrift, die Parteiöffentlichkeit sachgerecht zu unterrichten.
Korrekturen müssen daher immer, Ergänzungen in engen Grenzen ebenfalls zulässig sein, soweit das Gericht eine Stellungnahme veröffentlicht hat.
Zu beachten ist:
Das Urteil darf in solchen, nachträglichen Veröffentlichungen, nicht berührt werden.

Bezüglich der Dokumentation und Aufbewahrung bietet sich eine Behandlung entsprechend der Bekanntmachungen von Beeinflussungsversuchen an:
Stellungnahmen sind als \enquote{relevantes Schriftstück} i.S.d. \S~14 Abs.~2~SGO\index[paridx]{SGO!14@\S~14!2@Abs.~2} zugleich Teil der Verfahrensakte.\index[idx]{Akte!Inhalt}
Während der Aufbewahrungsfrist bietet sich daher eine Aufbewahrung mit der Verfahrensakte an.
Da die Stellungnahme aber auch veröffentlicht wurde, sollte sie unbegrenzt verfügbar gehalten werden; ein Interesse an einer Depublizierung ist der SGO nicht zu entnehmen.

\section{Aufbewahrung der Akten}
\label{Dokumentation:Aufbewahrung}
Die Aufbewahrung der Akten bestimmt sich nach der Geschäftsordnung des Schiedsgerichts, \S~2 Abs.~6 S.~2~SGO.\index[paridx]{SGO!2@\S~2!6@Abs.~6}
Sie sollte möglichst zweckmäßig erfolgen.
Eine zentrale, elektronische Verwaltung, auf die das gesamte Gericht zugreifen kann, ist daher sinnvoll.
Hinsichtlich der Erstellung von Backups etc. ist ein Austausch mit den Technikverantwortlichen der Gliederung empfehlenswert.

Auf Anfrage können möglicherweise die Datenschutzbeauftragten in der Piratenpartei Hilfestellungen zum Datenschutz geben.

\subsection{Laufende Verfahren}
Für laufende Verfahren ist bedeutsam, dass der gesamte Spruchkörper einfachen Zugriff auf die gesamte Verfahrensakte nehmen kann.
Ebenso ist wichtig, dass sie zu jedem Zeitpunkt insgesamt an die Beteiligten gesendet werden kann, um deren Recht auf Akteneinsicht gewährleisten zu können.

Am einfachsten lässt sich dies durch ein geeignetes Fallbearbeitungssystem\footnote{Auch \emph{Issue-Tracking-System}, \href{https://de.wikipedia.org/wiki/Issue-Tracking-System}{Wikipedia (de): Issue-Tracking-System}.} gewährleisten.
Auch (in nicht abschließender Aufzählung) \enquote{Cloud}-Systeme, ein gemeinsam genutzer FTP\nomenclature{FTP}{File Transfer Protocol}-Server oder Etherpad bzw. Piratenpad Teampads sind hierzu geeignet.
Unabhängig von der Software bzw. dem genutzten Protokoll ist wichtig, dass die administrativen Rechte auf dem jeweiligen Server in der Hand der Piratenpartei liegen.
Dienste Dritter dürfen nicht genutzt werden.
Dass die Nutzung der in den Akten enthaltenen Daten ausschließlich beim Schiedsgericht liegen darf, versteht sich von selbst.

\subsection{Abgeschlossene Verfahren}
Mit Abschluss des Verfahrens gelten für die Verfahrensakte und das Urteil (und ggf. weitere Beschlüsse, s.o.~\nomenclature{s.o.}{siehe oben}\ref{Dokumentation:Beschlüsse}) unterschiedliche Bestimmungen.

\subsubsection{Verfahrensakten}
Soweit ein elektronisches Aufbewahrungssystem genutzt wird, können die Akten über die Aufbewahrungsfrist darin verbleiben.

In dem Falle, dass die Geschäftsordnung des Gerichts eine sofortige Löschung der elektronischen Akte festlegt, ist sie auf einem Datenträger oder in gedruckter Form (in jedem Falle aber vollständig) beim Gericht zu hinterlegen.
Hierfür bietet sich eine gegen unbefugten Zugriff gesicherte Aufbewahrung in der Geschäftsstelle der jeweiligen Gliederung an.
Zweckmäßig ist dann die Aufbewahrung im verschlossenen Umschlag mit außen angebrachtem Verfallsdatum.

Soweit es während des Verfahrens mehrere Speicherorte für die Akte gab (bspw. jeweils bei den mit dem Verfahren befassten Mitgliedern des Spruchkörpers), sollten diese nach Verfahrensabschluss auf ein einzelnes Archiv der Akte reduziert werden.
Nach Ablauf der fünfjährigen Aufbewahrungsfrist (\S~14 Abs.~5~SGO)\index[paridx]{SGO!14@\S~14!5@Abs.~5} ist die Akte an all ihren Speicherorten zu vernichten.

\subsubsection{Urteile}
Urteile sind unbefristet aufzubewahren, \S~14 Abs.~5 S.~2~SGO.\index[paridx]{SGO!14@\S~14!5@Abs.~5}
Dies bezieht sich insbesondere auf die schriftliche, von allen (am Verfahren beteiligten) Richtern unterschriebene Originalfassung nach \S~12 Abs.~7~SGO.\index[paridx]{SGO!12@\S~12!7@Abs.~7}
Diese sollten zentral, bspw. in der Geschäftsstelle der Gliederung, aufbewahrt werden.
Sie müssen gegen den Zugriff Unbefugter gesichert sein (bspw. Verschluss in einem Schrank o.ä.\nomenclature{o.ä.}{oder ähnliche(r/s)}).

Eine getrennte Aufbewahrung der Urteile ist hingegen nicht notwendig.
Einfaches Abheften (ggf. in intransparenter Hülle) in einem Ordner ist ausreichend.
Zugriffsrechte bestehen nur für die Beteiligten und für das Gericht.

\subsection{Sonstige Akten}
Die Aufbewahrung sonstiger Akten liegt im Ermessen des Gerichts.
Sie regelt sich nach der Geschäftsordnung oder aber der im Gericht gängigen Praxis.
Es gelten lediglich die allgemein üblichen Vorschriften (Datenschutz etc.).

Auch der Zugang zu sonstigen Akten des Gerichts ist grundsätzlich zu beschränken, da sie von der Verschwiegenheitspflicht aus \S~2 Abs.~4~SGO\index[paridx]{SGO!2@\S~2!4@Abs.~4} umfasst sind.
Ausnahmen hiervon sind möglich.
Als Faustformel kann gelten:
\Zitat{Was veröffentlicht ist, bleibt veröffentlicht.}
Insbesondere alle im Rahmen der Rechenschaftslegung erfolgenden Veröffentlichungen, sowie Bekanntmachungen und Stellungnahmen (s.o.~\ref{Dokumentation:Veröffentlichungen}) sollten daher nicht depubliziert werden.

\section{Rechenschaftslegung}
\label{Dokumentation:Rechenschaftslegung}
Das Gericht legt für seine Arbeit öffentlich Rechenschaft.\index[idx]{Rechenschaftspflicht}
Es unterliegt dabei einerseits einer laufenden Berichts- und Veröffentlichungspflicht, andererseits einer (zusammenfassenden) Berichtspflicht dem das Gericht wählenden Parteitag gegenüber.

\subsection{Laufende Berichtspflicht}
\label{Dokumentation:Rechenschaftslegung:Laufend}
Das Gericht soll während der Amtszeit regelmäßig berichten; gemäß \S~15 Abs.~1~SGO\index[paridx]{SGO!15@\S~15!1@Abs.~1} insbesondere über die Zahl der Fälle.
Allerdings hat das Gericht auch Urteile zu veröffentlichen, \S~12 Abs.~8 S.~1~SGO.\index[paridx]{SGO!12@\S~12!8@Abs.~8}
Als Praxis aller Schiedsgerichte hat sich daher eingebürgert, dieser Berichtspflicht durch die Führung eines regelmäßig aktualisierten, öffentlichen Verfahrensverzeichnisses nachzukommen.
Diese sind häufig im Wiki der jeweiligen Gliederung angelegt.

Der Satzungsbestimmung genügt dabei dem Wortlaut nach die Nennung der Anzahl der anhängigen, sowie der der (in der laufenden Amtsperiode) abgeschlossenen Verfahren.
Die Praxis der Gerichte geht darüber hinaus und listet die einzelnen Verfahren zumindest mit Aktenzeichen, dem Datum der Verfahrenseröffnung (ggf. auch der Anrufung), den veröffentlichten Beschlüssen (s.o.~\ref{Dokumentation:Beschlüsse}), und einer kurzen Zusammenfassung von Sachverhalt und ggf. der Verfahrensgeschichte auf.

Technisch ist die Veröffentlichung innerhalb eines elektronischen Systems sinnvoll, das sowohl für Menschen, als auch für Maschinen lesbar ist.
Insbesondere die Urteile sollten als eigenständige Dokumente verfasst und nicht lediglich in einer Datenbank abgelegt sein.
Dies erleichtert die Nachvollziehbarkeit der Dateiintegrität, bspw. über Prüfsummen oder sogar Signaturen.\footnote{Das Bundesschiedsgerichts bspw. hatte zeitweise die Urteile im PDF veröffentlicht und mit PGP\nomenclature{PGP}{Pretty Good Privacy (Verschlüsselungstechnologie)} signiert.}
Eingeschränkt trifft dies z.B. auf ein Wiki zu, weswegen die überwiegende Mehrheit der Gerichte solche Systeme für ihre laufende Rechenschaftslegung verwendet.

Die bisher leistungsfähigste Dokumentationsführung basiert auf dem verteilten Versionskontrollprogramm \emph{git},\footnote{\href{http://www.git-scm.org/}{Offizielle Website: http://www.git-scm.org/}.} was durch kryptographische Methoden auch die Dateiintegrität ausreichend sicherstellt und zudem vollständig maschinenlesbar ist.
Bei Verwendung maschinenlesbarer Urteilsdokumente (z.B. durchsuchbare PDF) ist auch eine Durchsuchbarkeit gewährleistet; einer eigenen Suchfunktion innerhalb der Software bedarf es dazu nicht.
Die Software selbst ist in der Adminsitration simpel gehalten und besitzt kaum Anforderungen an das Hosting-System.
Dabei bietet sie permanente Links auf das Urteil, sowie den Listeneintrag mit den obigen Informationen und bspw. die bereits erwähnten Vorteile der kryptographischen Dateiintegritätsprüfung und die Maschinenlesbarkeit.
Die Software ist quelloffen unter freier Lizenz und wurde auf Github\footnote{Github selbst ist ein kommerzielles Projekt. Das ändert nichts an der Lizenz der Software. Sie ist daher ausdrücklich \emph{nicht} als proprietär zu bewerten.} entwickelt.\footnote{\href{https://github.com/Bundesschiedsgericht/BSG}{https://github.com/Bundesschiedsgericht/BSG}.}

In keinster Weise für die Rechenschaftslegung geeignet ist ein Blogsystem.
Die Zuordnung einzelner Veröffentlichungen zu einem Verfahren gelingt kaum; die Integrität der Veröffentlichungen ist im Vergleich zu den geschilderten Alternativen kaum darstellbar.
Hinsichtlich der Lesbarkeit ist festzuhalten, dass ein Blog nicht in demselben Maße auslesbar ist, wie es ein git-repository oder auch eine Wikiseite ist -- letztere bieten Einblick in die Datenquellen; das Blog nicht.

\subsection{Arbeitsbericht}
\label{Dokumentation:Rechenschaftslegung:Arbeitsbericht}
\index[idx]{Arbeitsbericht}
Der Arbeitsbericht des Schiedsgerichts wird dem Parteitag vorgelegt, der es (neu) wählt.

\subsubsection{Inhalt}
\label{Dokumentation:Rechenschaftslegung:Arbeitsbericht:Inhalt}
An den Inhalt dieses Arbeitsberichts stellt die SGO höhere Anforderungen:
Gemäß \S~15 Abs.~3~SGO\index[paridx]{SGO!15@\S~15!3@Abs.~3} soll er die Fälle der Amtsperiode inklusive der jeweiligen Urteile kurz darstellen.
Das trifft den Umfang, der sich bei der laufenden Rechenschaftslegung (s.o.~\ref{Dokumentation:Rechenschaftslegung:Laufend}) als gängige Praxis eingestellt hat.
Der dort genannte Umfang ist deswegen nicht nur legitimiert (da die entsprechenden Daten ohnehin veröffentlicht würden); die Erstellung des Arbeitsberichts verläuft so \enquote{nebenbei} über die gesamte Amtszeit und wird dadurch erleichtert.
Wird der laufenden Berichtspflicht im bereits geschilderten Umfang gefolgt, so können die Fälle der Amtszeit aus der laufend aktualisierten Übersicht in den Arbeitsbericht einfach übertragen werden.

Darüber hinaus sollte der Arbeitsbericht aber auch weitere Informationen enthalten, um aus sich heraus verständlich zu sein:
Zunächst ist die Zusammensetzung des Schiedsgerichts bei seiner Wahl und alle weiteren Veränderungen (durch Rücktritte etc.) eine wichtige Information.
Die Auflistung der Zusammensetzung der Spruchkörper in einzelnen Verfahren kann unterbleiben, da dies aus den Urteilen ersichtlich wird.
Zumindest die Anzahl der Sitzungen und Verhandlungen des Schiedsgerichts sollte aufgeführt werden, ggf. sollte ein Hinweis auf Protokollierung erfolgen.
Hieran kann der Parteitag die Aktivität, aber auch den Arbeitsaufwand des Schiedsgerichts ablesen.
Das ist insbesondere für Bewerber um das Richteramt eine relevante Information.
Zuletzt sollten auch weitere Aktivitäten des Schiedsgerichts im Arbeitsbericht nicht fehlen.
Hier sind v.a. Weiterbildungen und vergleichbare Veranstaltungen zu nennen.

Verfügt das Schiedsgericht über ein eigenes Budget, ist eine Rechenschaftslegung darüber im Arbeitsbericht empfehlenswert.\index[idx]{Budget}

\subsubsection{Darstellung nichtöffentlicher Verfahren}
\label{Dokumentation:Rechenschaftslegung:Arbeitsbericht:nichtöffentlicheVerfahren}
\index[idx]{Verschlusssachen}
Schließlich bietet der Arbeitsbericht die Möglichkeit, dem Problem der nicht-öffentlichen Rechtsprechung Herr zu werden (s.o.~\ref{Dokumentation:Urteile:Nichtöffentlich}).
Zwar sind nichtöffentliche Verfahren auch vom Gericht vertraulich zu behandeln, \S~9 Abs.~4 S.~2~SGO.\index[paridx]{SGO!9@\S~9!4@Abs.~4}
Das Urteil darf nur eingeschränkt veröffentlicht werden, \S~12 Abs.~8 S.~4~SGO,\index[paridx]{SGO!12@\S~12!8@Abs.~8} und öffentliche Stellungnahmen sind vollständig unzulässig, \S~15 Abs.~2 S.~2~SGO.\index[paridx]{SGO!15@\S~15!2@Abs.~2}
Demgegenüber sollen aber alle Fälle der Amtsperiode im Arbeitsbericht kurz dargestellt werden, \S~15 Abs.~3~SGO.\index[paridx]{SGO!15@\S~15!3@Abs.~3}

Die Lösung besteht in der Führung einer eigenen Rubrik für nichtöffentliche Verfahren im Arbeitsbericht, die lediglich rechtliche Erwägungen ohne nachvollziehbaren Bezug zu einem einzelnen Verfahren enthält.
Im Stile von \enquote{Leitsätzen}\index[idx]{Leitsatz} können hier die Kerninhalte der Rechtsprechung nichtöffentlicher Verfahren zusammenfassend veröffentlicht werden.
Dabei muss sichergestellt werden, dass ein unbeteiligter Dritter keine unmittelbaren Rückschlüsse auf ein einzelnes Verfahren ziehen kann.
Insbesondere Daten, Aktenzeichen oder Schilderungen von Details oder handelnden Personen des Sachverhalts, sowie ggf. Gliederungsnamen etc. dürfen sich daher nicht in der Schilderung wiederfinden.
Weiterhin müssen bei der geschilderten Verfahrensweise mindestens zwei nichtöffentliche Verfahren vorliegen, da ansonsten eine unmittelbare Zuordnung zu einem einzelnen Verfahren möglich ist.

Sollte nur ein einziges Verfahren vorliegen, liegt die Lösung in einer entsprechenden Notiz für das nachfolgende Schiedsgericht und einem Hinweis im Arbeitsbericht, dass die rechtlich bedeutsamen Aspekte des nichtöffentlichen Verfahrens in den nächsten Arbeitsberichten erörtert werden sollen, sobald mindestens ein weiteres nichtöffentliches Verfahren vorliegt.

Alternativ kann auch eine gesammelte Veröffentlichung erfolgen.
Soweit sich die Gerichte untereinander darauf einigen, wäre auch eine zentrale Veröffentlichung solcher Rechtsprechungsinhalte denkbar, bspw. im Arbeitsbericht des Bundesschiedsgerichts.
Diese Vorgehensweise sollte dann durch gemeinsame (bzw. zmd.\nomenclature{zmd.}{zumindest} gleichlautende) Bestimmungen der Geschäftsordnung\index[idx]{Geschäftsordnung} der beteiligten Gerichte geregelt sein.

\chapterbib
% \end{refsection}


% \begin{refsection}
\chapterpreamble{Die Schiedsgerichtsordnung sieht ein Verfahren in zwei Instanzen vor: Der Eingangsinstanz und der Berufungsinstanz. Das Bundesschiedsgericht ist Eingangsinstanz für Verfahren, die sich gegen die Bundespartei oder ihre Organe richten. Ein Landesschiedsgericht ist Eingangsinstanz für Verfahren, die sich gegen den Landesverband oder eines seiner Organe richten. Bei Einsprüchen gegen Ordnungsmaßnahmen ist das Gericht niedrigster Ordnung, in der der Antragsteller seinen Wohnsitz hat, zuständig. In allen sonstigen Fällen ist das niedrigste Gericht, in dessen örtlicher Zuständigkeit der Antragsgegner sich befindet, zuständig. Die Berufungsinstanz wird nach Abschluss des erstinstanzlichen Verfahrens durch Einlegen der Berufung zuständig. Sie führt das Verfahren als weitere Tatsacheninstanz, d.h. erhebt erneut Beweis und gewährt den Parteien erneut vollumfassend rechtliches Gehör. Es hat sich jedoch die Möglichkeit der Rückverweisung bei Rechtsfehlern etabliert.}

\chapter{Der Instanzenzug}

\begin{figure}[h!]
\caption{a train with a jet engine underneath (© Randall Munroe, xkcd.com)}
\centering
	\includegraphics[width=0.5\textwidth]{train_loop_jet_bottom.png}
\end{figure}
%\blindtext[1]

%\chapterbib
% \end{refsection}


\backmatter
\pagenumbering{roman}
\appendix
\onecolumn

\printbibliography[heading=bibliography,title={Rechtsprechungsverzeichnis}, type=ruling]
\printbibliography[heading=bibliography,title={Literaturverzeichnis},nottype=ruling]

\printnomenclature

\setindexpreamble[ruling]{Ein Verzeichnis aller für dieses Handbuch relevanten Urteile und der Textstellen, an denen über deren Inhalt geschrieben wird.}
\printindex[ruling]

\setindexpreamble[idx]{Ein ganz normales Sachregister zu wichtigeren und weniger wichtigeren Stichworten.}
\printindex[idx]

\setindexpreamble[paridx]{Ein Paragraphenregister zum schnellen Auffinden von relevanten Textstellen in diesem Handbuch zu bestimmenten Paragraphen, um es als effektives Nachschlagwerk ähnlich einem Kommentar nutzen zu können.}
\printindex[paridx]

\end{document}
